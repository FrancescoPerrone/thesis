\thispagestyle{empty}
\begin{center}
	
	\vspace*{3cm}
	
	\textit{There is a traveller who reaches a crossroads at the hour\\
		when the world withdraws into itself.}
	
	\vspace{0.5cm}
	
\end{center}
	
	He studies the signposts as if they held the logic of direction.  
	The boards are clean, the words exact, but the air is heavy with a  
	silence that seems older than the road. A thin wind rises, carrying  
	with it the odour of something distant—woodsmoke, or perhaps the  
	memory of it. He cannot tell.
	
	He believes he chooses by reading; but already his gaze has shifted  
	toward the darker path, drawn by a murmur he cannot name. A shape in  
	the periphery—almost a figure, almost a shadow—tilts the balance  
	without ever declaring itself. The light changes, and with it the  
	weight of each possibility.
	
	He hesitates, though he is unaware of the reason. The stones cool  
	beneath his feet. Something in the air—presence, or its simulation— 
	presses lightly against his decision. He steps, not toward the sign  
	he had resolved to follow, but toward the path shaped by these quiet,  
	unclaimed forces.
	
	Later he will recall the moment and speak of deliberation, of judgement, of intention:
	
	\textit{- I reasoned! \\ - I deliberated... \\ - I chose.}
	
	But it was the quiet pressures of the world—the unseen gradients of light, sound, warmth, and presence—that shaped his path.
	
	And the signs?
	They were there long before he arrived, and they remain long after he has gone.
	Yet it is the field through which he walked that carried him forward.
	


	\vspace*{1cm}

\textit{Francesco Perrone}

\newpage

	\vspace*{3cm}
\begin{flushright}
	\textit{Al mio compagno d'avventure, Francesco.}
\end{flushright}

\vspace{0.5cm}
\newpage
