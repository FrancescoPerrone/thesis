\chapter{Abstract}

Moral behaviour emerges not from isolated cognitive modules or explicit reasoning, but from a structurally rich evaluative field shaped by attention, affect, social meaning, and dispositional architecture. This thesis develops and defends a field-theoretic account of moral cognition grounded in empirical evidence, formal topology, and philosophical analysis. It argues that artificial agents—particularly those with humanoid morphology—interact with this field in ways that classical Machine Ethics has systematically overlooked.

To test this, a controlled experiment examined how a humanoid robot (NAO) modulates prosocial donation under a strong moral cue (the Watching-Eye paradigm). Bayesian and regression models reveal a robust attenuation effect: participants donated less in the robot’s presence, despite identical moral affordances. Personality- and cognitive-style measures (EQ, SQ, BFI-10) were used to derive three latent evaluative ecologies, each with distinct affective and structural properties. Yet all ecologies exhibited the same directional displacement. The robot did not influence moral principles; it altered the evaluative field through which those principles acquire behavioural force.

This finding supports a structural interpretation of moral cognition: synthetic presence acts as a perturbation operator that suppresses salience, dampens affective resonance, and disrupts justificatory and attentional pathways. The result exposes a critical limitation of top-down Machine Ethics and opens a new direction for Computational Morality—shifting focus from rule encoding to the dynamics of moral environments.

The thesis concludes that artificial agents, even without agency or intent, function as moral modifiers: their perceptual salience and ontological ambiguity reshape the architecture of human moral appraisal. Moral behaviour is thus field-dependent, and synthetic presence deforms that field. This establishes a new methodological foundation for the ethical and empirical study of artificial systems, grounded in evaluative topology, Levels of Abstraction, and the dynamics of moral cognition.