\newglossaryentry{actuality}{  sort={actuality}, name={Actuality},   description={In a general philosophical context, "actuality" refers to the state of being real, existing, or occurring in the present. It represents the state of existing in fact or in reality, typically contrasted with the concept of mere potential or possible existence. It often corresponds with the fulfilled or accomplished state of an object or situation, as opposed to its potential state. \newline In the context of Aristotle's philosophy, "actuality" (often translated from the Greek word "entelecheia") is used to signify the realisation or fulfilment of an entity's inherent essence or purpose. It refers to an entity in its most complete form or optimal state of being, a state in which it has achieved its intrinsic purpose and actualised its potential.} } 

\newglossaryentry{potentiality}{ sort={potentiality}, name={Potentiality},   description={In a general philosophical context, "potentiality" refers to the latent capabilities or possibilities inherent in an entity that have not yet been actualised. It signifies the state of being capable of becoming or developing into a certain condition or performing a certain action, even if these are not yet manifest. Potentiality is typically contrasted with actuality, emphasising the future possibilities rather than the current state. \newline In Aristotle's philosophy, the concept of "potentiality" signifies the inherent capacity within an entity to change, develop, or become something else. It encapsulates the possibilities for an object or being that are yet to be realised, or the "potential" for transformation that is inherent in its nature.} } 

\newglossaryentry{humanoid}{
	sort={humanoid},
  name={Humanoid},
  description={The term "humanoid" typically refers to an entity, often a robot or a fictional character, which possesses physical characteristics, behaviors, or qualities similar to those of humans. These might include bipedal locomotion, manipulative abilities, facial features, or cognitive capacities resembling human thought processes. The term is used extensively in fields such as robotics and artificial intelligence, where humanoid robots serve various practical and research purposes, ranging from service robots to platforms for studying human cognition and social interaction. \newline From a philosophical perspective, humanoid entities raise intriguing questions about the nature of personhood, consciousness, and ethical considerations regarding artificial lifeforms. They challenge traditional notions of what it means to be human and blur the boundaries between humans and machines. \newline In psychology, humanoids offer a tool for understanding human cognition, behaviour, and social interaction. They can serve as experimental apparatuses in social, cognitive, and developmental psychology. However, the creation of highly humanoid entities also brings forth psychological considerations related to human-robot interaction, such as the 'uncanny valley' phenomenon, where highly human-like robots may elicit feelings of eeriness or discomfort.}
}

\newglossaryentry{practical_behaviour}{
	sort={practical},
  name={practical behaviour},
  description={While \textit{practical behaviour} is a term widely used across philosophy and psychology, it's challenging to create an exhaustive chronological definition because the term does not correspond to a singular theory or concept that has evolved over time in a linear fashion. Instead, it has been interpreted and applied differently depending on the context, theoretical framework, or school of thought. \textbf{Practical behaviour in philosophy}: in Philosophy, \textit{practical behaviour} has been interpreted in various ways across different philosophical schools of thought. 1) Aristotelian Philosophy: In Aristotelian philosophy, practical behaviour is associated with "praxis" or action guided by moral virtue aimed at the good life. Practical wisdom ("phronesis") is crucial here as it guides one's decisions and actions in accordance with moral virtue. 2) Kantian Philosophy: Immanuel Kant distinguished between theoretical reason (used to understand the natural world) and practical reason (used to govern behaviour and moral decision-making). For Kant, practical behaviour is guided by the categorical imperative, an absolute moral law. 3) Pragmatism: In the late 19th and early 20th century, the pragmatists (like William James and John Dewey) viewed practical behaviour as action informed by the effects that such behaviour would bring about. \textbf{Practical Behaviour in Psychology}: In Psychology, practical behaviour has been understood as observable actions and reactions to stimuli in behaviourism, deeply intertwined with internal cognitive processes during the cognitive revolution, and as a complex interplay of cognitive processes, emotional states, individual traits, and environmental influences in contemporary psychology. 1) Behaviourism (Early 20th Century): In the behaviourist approach, pioneered by John Watson and B.F. Skinner, practical behaviour is understood in terms of observable actions and reactions to stimuli, often studied through conditioning processes. 2) Cognitive Revolution (Mid-20th Century): With the cognitive revolution, practical behaviour started to be seen as deeply intertwined with internal cognitive processes like problem-solving, decision-making, and planning. 3) Social-Cognitive Theory (Late 20th Century): Albert Bandura's social-cognitive theory emphasised the role of observational learning, self-efficacy, and goal setting in practical behaviour. \textbf{Modern times}: today, in ethics and action theory, practical behaviour typically refers to behaviour guided by practical reason, that is, reason concerned with action and decision-making. This involves deliberation about means and ends, moral obligations, and the values at stake in different courses of action. Similarly in Contemporary Psychology, practical behaviour is understood as a complex interplay of cognitive processes, emotional states, individual traits, and environmental influences. It is typically studied in context-specific terms, such as health behaviour, consumer behaviour, or prosocial behaviour.}
}




