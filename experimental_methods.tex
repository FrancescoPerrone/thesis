\chapter{Experimental Methods}
\thispagestyle{pprintTitle}


% Adjusting epigraph settings
\setlength\epigraphwidth{.8\textwidth}
\setlength\epigraphrule{0pt}
\renewcommand{\epigraphflush}{flushleft}
\renewcommand{\sourceflush}{flushright}

% Setting the font and spacing for the epigraph
%\epigraph{\itshape \setstretch{1.2}But one thing is the thought, another thing is the deed, and another thing is the idea of the deed. The wheel of causality doth not roll between them.}{\small{Friedrich Nietzsche, \textit{Thus Spoke Zarathustra} (1883)}}
%
%
During the past decade, new emerging technologies have caused profound changes in the way we communicate and interact~\cite{Pantic2014a}. Some of these changes have affected certain aspects of human behaviour and caused psychiatric disorders~\cite{Xerxa2023}. These technologies have fundamentally altered how we connect with others, potentially exacerbating feelings of loneliness despite increased opportunities for connection. The role that modern technologies—such as mobile communications, digital interaction platforms, and interactive humanoid robots might play in shaping these dynamics is critical, influencing not only interpersonal communications but also moral decision-making in complex social settings~\cite{Allcott2020, Auxier2021, Bail2021, Dwyer2020, Vosoughi2018}. Furthermore, technologies that increase interactive opportunities may not necessarily enhance the quality or \textit{ethical dimensions} of those interactions, which are crucial in scenarios involving moral choices~\cite{Sharkey2010, Vallor2016, Lin2012, Bryson2010}. The constant presence of interactive technologies can lead to a reshaping of social norms and behaviours, which might lead to more engaged or more detached human responses depending on the context and implementation~\cite{Misra, Turkle}.

Foundational insights from studies such as \cite{Xerxa2023} set the stage for a deeper exploration into how contemporary communication technologies, particularly humanoid robots, might amplify or mitigate these effects by altering the quality and nature of social interactions in both visible and subtle ways.
%

This work presents experiments based on the Watching Eye effect, the tendency of people to behave more honestly or more pro-socially when they have the impression of being observed. In particular, the experiments of this work show that the presence of a robot is associated to a lower tendency to donate to a charity despite the presence of a Watching Eye stimulus (the picture of a child portrayed on the brochure of a Non-Governmental Organization providing medical care in poor countries). The tendency to donate was measured in terms of actually donated money and the results show that people donate roughly one and half times as much  when there are no robots (a statistically significant difference). This suggests that, while not necessarily being involved in moral decisions, robots can still be associated to changes in the way people (possibly users) make decisions involving a moral dimension.

%
The \emph{Watching Eye} effect is the tendency of people to behave more honestly or more pro-socially when they feel observed~\cite{Oda2015}, whether such a feeling results from the presence of pictures depicting eyes~\cite{Atran2004}, from the belief in a supernatural being that can see everything~\cite{Bering2005,Shariff2007}, or from any other factors. The goal of this article is to investigate the interplay between the Watching Eye effect and the presence of humanoid robots, a technology expected to play an increasingly more important role in everyday life. In particular, the experiments of this work show that there is an association between the presence of a robot and the observable consequences of the Watching Eye effect.

\section{The Influence of Observational Presence on Human Behavior: Experimental Insights from Human-Robot Interactions}