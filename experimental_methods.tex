%\chapter{An Experimental Study of Moral Displacement: From Normative Hypothesis to Experimental Topology}
\chapter{Experimental Methods}
\label{chap:exp_methods}
\thispagestyle{pprintTitle}


% Define a counter named 'question' that resets every time 'chapter' increments
\newcounter{question}[chapter]
% Define the format of the counter to be 'chapter_number.question_number'
\renewcommand{\thequestion}{\thechapter.\arabic{question}}

% Adjusting epigraph settings
\setlength\epigraphwidth{.8\textwidth}
\setlength\epigraphrule{0pt}
\renewcommand{\epigraphflush}{flushleft}
\renewcommand{\sourceflush}{flushright}

% Setting the font and spacing for the epigraph
%\epigraph{\itshape \setstretch{1.2}But one thing is the thought, another thing is the deed, and another thing is the idea of the deed. The wheel of causality doth not roll between them.}{\small{Friedrich Nietzsche, \textit{Thus Spoke Zarathustra} (1883)}}
%
%

\section{From Conceptual Architecture to Empirical Test}

The preceding chapters established a theoretical claim with both philosophical depth and empirical ambition: \textit{that moral behaviour emerges from a topologically structured evaluative field, and that synthetic agents can perturb this field by altering the conditions under which moral salience becomes action}. The present chapter marks the transition from conceptual architecture to empirical adjudication. Here, every assumption must be operationalised, every construct measured, and every inference anchored in explicit experimental procedure.

This section begins with the precise research question that animates the experiment:
\bigskip
\noindent
\begin{center}
	\begin{questionbox}[label={q:robot-agent}]{Inferential Displacement} 
		Does the silent presence of a humanoid robot---perceptually social yet ontologically indeterminate---alter the evaluative process that transforms moral perception into prosocial behaviour? 
	\end{questionbox} 
\end{center}
\bigskip
\noindent
This question is not a rhetorical prompt but a methodological commitment. It situates the experiment within the evaluative--topological model developed earlier, in which moral action is expressed as:
\[
\mathscr{P}(\delta_m) = f(\alpha_E, \beta_C, \gamma_R),
\]
where $\alpha_E$ denotes morally salient environmental cues, $\beta_C$ the dispositional manifold quantified through psychometric tools, and $\gamma_R$ the perturbation operator instantiated by the robot’s presence. The purpose of the experiment is to determine whether $\gamma_R$ induces a measurable deformation in the mapping from $\alpha_E$ to observable moral action.

\subsection{Why the Question Matters}

Although behaviourally simple, the question reaches beyond classical experimental paradigms in moral psychology. It does not ask whether robots communicate norms, nor whether they persuade or instruct. It asks whether synthetic \textit{presence} alone---minimal, silent, behaviourally neutral---modifies the inferential pathway through which moral salience produces action. Within the broader research programme of social signal processing and moral AI, this constitutes a stringent and foundational test:

\bigskip
\noindent
\begin{center}
	\begin{leftbar}
		\textit{can an artificial agent operate as a perturbation operator on the moral field even in the absence of agency?}
	\end{leftbar}
\end{center}

Embedding a humanoid robot into a moral environment thus serves as a direct empirical probe of the thesis developed in earlier chapters: namely, that moral cognition is not solely a matter of internal reasoning or personality structure, but a dynamic, context-sensitive transformation governed by the topology of situational cues.

\subsection{Operationalising Moral Action: Prosocial Donation as Behavioural Endpoint}

To render this transformation empirically measurable, the experiment operationalises moral action through a cost-bearing behavioural choice: voluntary donation of a portion of the participant’s monetary compensation to a children’s medical charity. This measure, extensively validated in behavioural ethics and moral psychology, captures the endpoint of the evaluative trajectory: the point at which moral salience is either converted into action or allowed to dissipate.

Costly charitable donation has been repeatedly validated across moral psychology, behavioural economics, evolutionary anthropology, and developmental science as the most reliable behavioural proxy for prosocial moral action~\cite{Batson1991,FehrGachter2002,FehrFischbacher2003,WarnekenTomasello2006,Andreoni1990}. It satisfies the three criteria required at the Level of Abstraction adopted in this 
thesis: it is elicited by morally salient cues~\cite{Haley2005,Bateson2006,Pfattheicher2015}; it incurs real cost~\cite{Andreoni1990,Gintis2000}; and it expresses the 
action-guiding force of moral evaluation~\cite{Cushman2013,Greene2014,Decety2004}. 
Its long-standing use as the behavioural termination point of moral cognition~\cite{Batson1991,FehrGachter2002,WarnekenTomasello2006} justifies its role here as the measurable endpoint of the evaluative trajectory under synthetic perturbation.


The independent variable is equally minimal: the presence or absence of a humanoid robot autonomously animating in \textit{life-mode}: NAO does not speak, instruct, or engage. Its movements are restricted to micro-gestures---simulated breathing, subtle postural adjustments, and gaze-orienting behaviours triggered only by human eye contact. These micro-movements, while non-agentic, replicate the perceptual features known to activate the Watching-Eye effect, thereby introducing a controlled form of synthetic social salience into the evaluative environment.

\subsection{Why a Humanoid Robot?}

The choice of a humanoid robot reflects a deliberate methodological position. As established in Chapter~\ref{chap:moral_primer}, synthetic agents occupy an unstable location in our social ontology: they possess perceptual salience and humanoid morphology, yet lack the moral-evaluative capacities ordinarily ascribed to observers. This combination creates the precise form of perturbation the experiment seeks to test: a perceptibly social presence whose normative meaning is ambiguous.

The question is therefore not whether participants think the robot is judging them; rather, it is whether the robot’s presence alters the field of salience within which morally relevant cues exert their behavioural pull.


\bigskip
\noindent
\begin{center}
	\begin{questionbox}[label={q:robot-agent}]{Inferential Displacement}
		Can the mere presence of a synthetic observer---lacking agency, intention, and moral standing---perturb the inferential transformation that converts morally salient cues into prosocial action?
	\end{questionbox}
\end{center}
\bigskip
\noindent

\subsection{From Question to Design: Why We Do Not Begin with a Hypothesis}

Framing the study around a research question rather than a directional hypothesis is intentional. In interdisciplinary work spanning philosophy, psychology, neuroscience, and HRI, a premature hypothesis risks narrowing the interpretive field and smuggling in unexamined assumptions about how synthetic presence ought to behave. The methods must therefore preserve epistemic openness: \textit{the design must reveal whether perturbation occurs, not assume that it does}.

This methodological humility is continuous with the philosophical commitments articulated earlier. If moral behaviour arises from a dynamic integration of environmental cues, dispositional structure, and social presence, then the experiment must be sensitive to field-level deformations that cannot be anticipated a priori.

\subsection{The Logic of the Experimental Test}

In practical terms, the experiment leverages the integrated measurement framework developed in the previous chapter. The Watching-Eye paradigm constructs a baseline of elevated prosocial salience ($\alpha_E$). The EQ, SQ, and BFI quantify the structure of the dispositional manifold ($\beta_C$). The robot enacts the perturbation operator ($\gamma_R$). The donation task measures the resulting behavioural transformation $\mathscr{P}(\delta_m)$.

The empirical question is therefore precise:


\bigskip
\noindent
\begin{center}
	\begin{questionbox}[label={q:emp_que}]{Empirical Question}
		Does $\gamma_R$---the silent, perceptually social presence of a humanoid robot---systematically deform the evaluative mapping from $\alpha_E$ to $\mathscr{P}(\delta_m)$ across the dispositional manifold $\beta_C$?
	\end{questionbox}
\end{center}
\bigskip
\noindent

If the answer is affirmative, the findings reveal a foundational claim: that artificial agents, even when behaviourally minimal, exert moral influence not by persuasion or instruction but by reshaping the topological conditions under which moral salience becomes action.

What follows in this chapter details the machinery by which this question is tested: the design logic, the structure of the experimental task, the observational conditions, the psychometric integration, and the analytic strategies used to detect perturbation.

\bigskip
\noindent
\begin{center}
	\begin{leftbar}
		\textit{The conceptual framework provided the variables.  
			The empirical design now tests their transformation.}
	\end{leftbar}
\end{center}


\section{Experimental Design and Behavioural Paradigm}

To address Question~\ref{q:robot-agent} (p.~\pageref{q:robot-agent}), we implemented a controlled behavioural experiment~\cite{RosenthalRosnow2008,ReisJudd2000,Kazdin2017} grounded in the \textit{Watching–Eye} paradigm~\cite{HaleyFessler2005,Bateson2006,NettleEtAl2013,BatesonEtAl2013_EyesLittering,PfattheicherKeller2015,ContyGeorgeHietanen2016,DearDuttonFox2019}. The scientific objective was not simply to measure donation behaviour, but to determine whether $\gamma_R$, the silent perceptual presence of a humanoid robot, systematically deforms the evaluative mapping 
\[
\alpha_E \longmapsto \mathscr{P}(\delta_m)
\]
across the dispositional manifold $\beta_C$.

\noindent
Participants were recruited individually into a lab environment under the pretext of participating in a personality study in exchange for monetary compensation. This cover task served two methodological purposes. First, it provided a psychologically neutral framing for the room-based task. Second, it elicited trait-level measurements (EQ, SQ, BFI) necessary for modelling the dispositional manifold ($\beta_C$) within the evaluative–topological framework.

\medskip

\noindent
Embedded within this setting was a morally salient cue: a prominently placed charity poster depicting a child in medical need. Decades of evidence (see Chapter~\ref{chap:tools_new}) demonstrate that such stimuli reliably activate prosocial dispositions through implicit monitoring, empathic resonance, and affiliative concern~\cite{Haley2005,Conty2016}. Within the formalism introduced in earlier chapters, this stimulus increases $\alpha_E$, the environmental salience input, steepening the prosocial attractor in the evaluative field.

\subsection{Experimental Manipulation: Presence as the Only Ontological Difference}

\noindent
Participants were randomly assigned to one of two conditions.

\begin{enumerate}
	\item \textbf{Control Condition}: the participant completed the questionnaires alone in the room.
	\item \textbf{Robot Condition}: a humanoid NAO robot~\cite{Aldebaran2013} was present, operating in \textit{autonomous life mode}.
\end{enumerate}

\noindent
NAO emitted no speech and performed no task-directed actions. Its behaviour consisted solely of the minimal embodied micro-cues characteristic of this mode: simulated breathing, subtle shifts of posture, and head-orientation responses triggered only by direct eye contact. These are precisely the class of low-dimensional social cues shown to activate or modulate the Watching–Eye effect: movement, gaze potentiality, and the perceptual suggestion of observation.

\begin{figure}[H]
	\centering
	\begin{subfigure}[t]{0.48\textwidth}
		\centering
		\includegraphics[width=\linewidth]{/home/francesco/Desktop/research/appunti/images/robot.png}
		\caption{Experimental condition: robot present.}
		\label{fig:robot}
	\end{subfigure}
	\hfill
	\begin{subfigure}[t]{0.48\textwidth}
		\centering
		\includegraphics[width=\linewidth]{/home/francesco/Desktop/research/appunti/images/control.png}
		\caption{Control condition: robot absent.}
		\label{fig:control}
	\end{subfigure}
	\caption{Top–down view of the experimental and control configurations.
		Both layouts are spatially and visually identical; the humanoid robot is the
		only ontological difference between conditions.  
		In the evaluative–topological framework developed in this thesis, this
		equivalence is essential: the geometry of the environment (desk positions,
		donation box placement, participant orientation) is held constant so that any
		change in prosocial behaviour can be attributed to a deformation of the
		evaluative field induced by synthetic presence.
		Formally, the figure depicts two instantiations of the same environmental input
		$\alpha_E$, differing only by the activation of the perturbation operator
		$\gamma_R$.  
		The robot’s placement maps onto a local modification of the salience
		landscape—an additional source of perceived observation—while the control
		condition represents the unperturbed topology.}
	\label{fig:experimental-topology}
\end{figure}


\noindent
Crucially, both experimental rooms were geometrically and visually identical (Fig.~\ref{fig:experimental-topology}). The \emph{only} manipulated variable was the presence or absence of the humanoid robot. Spatial layout, lighting, informational content, and the moral cue ($\alpha_E$) were held constant.

\medskip

\noindent
In this design, the robot does not “do” anything in a behavioural sense. Instead, its minimal perceptual affordances present the participant with an ontologically ambiguous entity—perceptually social, morally inert, and semantically potent. The manipulation therefore isolates \textit{presence as such} as the epistemic and experimental variable.

\subsection{Why Minimal Presence Matters: Ontological Ambiguity as Cognitive Perturbation}

\noindent
The overwhelming majority of HRI and HMI studies assume that moral modulation arises through interaction: overt communication, feedback, adaptive behaviour, or explicitly framed expectations~\cite{Malle2016,VanStraten2020,Arnold2017,Groom2010,Leidner2019}. The present design rejects this assumption deliberately.

\noindent
Rather than investigating how robots \emph{act}, we investigate how they \emph{appear}—how their mere existence within a perceptual field alters the evaluative pathway from moral salience to moral action. The experimental focus is therefore on \textbf{pre-reflective permeability}: the extent to which minimal agent-like cues reshape inferential structure prior to conscious deliberation~\cite{Husserl1913,Zahavi2005,Gallagher2005,Bargh1994}.

\noindent
This approach isolates a structural vulnerability of norm-sensitive cognition: humans routinely over-ascribe agency in contexts of uncertainty~\cite{Guthrie1993,Waytz2010,Dennett1987}. By placing NAO precisely at the boundary between objecthood and agenthood, the design probes whether anticipation—not interaction—is sufficient to distort the evaluative topology.

\subsection{Levels of Abstraction: Why the Robot Can Matter Without Doing Anything}

\noindent
Floridi’s Levels of Abstraction (LoA)~\cite{Floridi2008,Floridi2010,Floridi2013} provide the formal justification for treating NAO’s silent presence as epistemically potent.

\noindent
At the operative LoA of the participant, what is visible are \emph{informational affordances}: posture, eyes, symmetry, subtle biological motion, the inert promise of mutual gaze~\cite{Emery2000,Hietanen2002,CarneyCuddyYap2010,Argyle1975,Rhodes2006,Johansson1973,Saygin2012,ChaminadeOhnishi2007}. These cues are sufficient to trigger the primitives of social monitoring, even when the entity producing them is known to be non-human.

\noindent
Thus, at this LoA, NAO functions as a \emph{semantic perturbator}: not a moral agent, nor a communicative partner, but an informational presence that reshapes the participant’s evaluative background conditions. If the robot were interactive, the LoA would shift (introducing agency, reciprocity, intentional stance). If the robot were inert, the social affordance would vanish. Autonomous life mode occupies the narrow space between these extremes.

\noindent
This design choice aligns with Floridi and Sanders’ analysis of artefactual moral agency~\cite{FloridiSanders2004}. Their 2004 account does not attribute consciousness, intentionality, or moral reasoning to artificial systems. Rather, it identifies moral relevance at the \emph{Level of Abstraction} at
which an artefact can contribute causal or informational influence within a given environment~\cite{Floridi2008,Floridi2011}. At this LoA, an artefact may count as a “moral agent” in the minimal and operational sense that its presence supplies, modifies, or filters morally relevant information.

This perspective is directly compatible with contemporary discussions of large language models (LLMs), which similarly operate as \emph{artefactual sources of semantic perturbation} rather than as bearers of intrinsic moral status~\cite{Mittelstadt2019,Bender2021}. In both cases—the embodied robot tested here and the disembodied LLM—moral relevance arises not from interior capacities but from how the system reshapes the informational and social conditions under which human agents form evaluations and make decisions. Related arguments in HRI emphasise that robots exert moral and social
influence through their perceived agency, morphology, and communicative affordances, not through any intrinsic mental properties~\cite{Malle2016,Zlotowski2015,Banks2020}.

\noindent
For this reason, Floridi’s account is particularly well suited to the present experimental context: it licenses the treatment of NAO’s minimal, non-interactive presence as an epistemically potent variable without implying any claim about the robot’s inner ontology. At the LoA operative for the participant, the
robot is a \emph{semantic perturbator}: a structured informational presence capable of altering the evaluative field through which moral salience becomes behaviourally operative. This conceptual continuity also clarifies why the findings developed in this thesis generalise to other classes of artificial systems—including LLM-based agents—whose moral significance likewise depends
on the informational roles they play rather than on their metaphysical constitution~\cite{Coeckelbergh2010,Gunkel2012}.


\subsection{Behavioural Paradigm: Donation as Moral Action}

\noindent
After completing the questionnaires, each participant received £10 in £1 coins and encountered a voluntary donation option: a charity box positioned near the exit. They could donate any subset of their compensation. The amount donated served as the behavioural measure of prosocial action.

\noindent
This operationalisation follows a long-established tradition in moral psychology, moral economics, and behavioural ethics in which cost-bearing prosocial behaviour tracks the practical expression of moral salience~\cite{Batson2011,FehrGachter2002,Henrich2005,Tomasello2016,Warneken2015,Baumard2013,Crockett2016,Scanlon1998,Darwall2006}. As demonstrated in Chapter~\ref{chap:tools_new}, donation behaviour reliably expresses the terminal point of a moral evaluative trajectory.

\subsection{Preliminary Findings}

\noindent
Initial analyses revealed a robust and theoretically coherent effect: participants in the Robot condition donated \emph{significantly less} than those in the Control condition. Personality data (EQ, SQ, BFI) showed no meaningful differences between conditions, ruling out dispositional confounds and providing strong initial support for a field-level perturbation induced by synthetic presence.

\noindent
These results motivate the next step: formalising the evaluative structure through which this behavioural displacement must be interpreted.


\subsection{From Behavioural Setup to Evaluative Structure}

\noindent
The experimental setup provides the behavioural substrate. What remains is to
specify the evaluative architecture through which any behavioural modification
must be interpreted. In moral philosophy, action is often treated as the
terminus of deliberation~\cite{Aristotle_nicomachean,Anscombe1957,
	Korsgaard1996}. Yet the present study does not investigate deliberation itself.
It examines the \emph{transformation} that precedes deliberation’s endpoint:
the cognitive--affective process by which morally salient cues become
behaviourally operative~\cite{Nussbaum2001,Korsgaard2009}.  

\noindent
Donation, within this design, is therefore not an isolated act but the
\emph{observable boundary condition} of an evaluative process. The Watching--Eye
stimulus renders moral salience explicit; the robotic manipulation introduces a
synthetic perturbation; the donation behaviour provides the measurable output
of the transformation. This ensures that what is being tested is not
trait-level generosity, but the \textit{susceptibility of moral appraisal to
	synthetic co-presence}.

\medskip
\noindent
Classic variants of the Watching--Eye paradigm rely on pictorial cues or
supernatural primes~\cite{Bateson2006,Shariff2007}. The present experiment
instead embeds an embodied but minimally active humanoid robot. This shift is
critical: it replaces a two-dimensional prime with a three-dimensional presence
whose \emph{perceived ontology} is neither inert object nor full social agent.
This ambiguity is precisely the condition under which moral salience may be
refracted or displaced.

\medskip
\noindent
To formalise what the experiment tests, we treat moral action as the output of
an evaluative function integrating environmental cues, dispositional structure,
and perturbational affordances:

\[
\mathbb{E}[f(\Sigma \cup \mathscr{R})] \neq \mathbb{E}[f(\Sigma)],
\]

\noindent
where:
\begin{itemize}
	\item $\Sigma$ is the morality-salient perceptual field (the Watching--Eye cue),
	\item $\mathscr{R}$ is the synthetic co-presence,
	\item $f$ is the evaluative transformation linking perception to action,
	\item $\mathbb{E}[f(\cdot)]$ denotes the expected behavioural output.
\end{itemize}

\bigskip
\noindent
Read informally: \emph{the expected moral behaviour differs when the robot is
	added to the perceptual--moral environment}. This yields our first empirical
hypothesis:
\bigskip
\noindent
\begin{center}
	\nextstatement
	\begin{hypobox}{Evaluative Deformation Hypothesis}
		\label{hyp:evaluative_deformation}
		The expected outcome of moral behaviour, as computed through the evaluative
		process \( f \), is altered when the robot is present within the
		perceptual--moral environment.
	\end{hypobox}
\end{center}

\medskip
\noindent
To clarify the structure of this transformation, we decompose the probability
of a deviation in moral action into its constituent determinants:

\[
\mathscr{P}(\delta_m) = f(\alpha_E, \beta_C, \gamma_R),
\]

\medskip
\noindent
where:
\begin{itemize}
	\item $\alpha_E$ represents the environmental moral cue (Watching--Eye),
	\item $\beta_C$ encodes the dispositional structure measured by EQ, SQ, and BFI,
	\item $\gamma_R$ denotes the perturbational effect of robotic co-presence.
\end{itemize}

\medskip
\noindent
In plain language: \emph{the probability of observing a change in moral
	behaviour depends jointly on the moral cue, the agent’s dispositional profile,
	and the presence of the robot}. This is the operative logic of the experimental
design: the robot is not treated as a moral agent, but as a \emph{topological
	perturbation}—a factor that reshapes the evaluative field within which moral
cues are processed.

\medskip
\noindent
Why should a robot be capable of such perturbation? The answer lies in the
notion of \textit{moral salience}. Across cognitive science and moral
philosophy, moral salience refers to the way certain features of the
environment become normatively charged prior to explicit reasoning~\cite{
	Korsgaard2009,Nussbaum2001,Greene2001,Moll2005}. It is a
pre-reflective gatekeeper: what is foregrounded, what stands out, and what
demands attention.

\noindent
A synthetic presence may influence this salience not by speaking or acting, but by altering the perceptual and inferential background against which moral cues are interpreted. NAO’s form, gaze orientation, and subtle embodied motions evoke the minimal conditions associated with social monitoring. They place the participant in a borderline space between being \textbf{alone} and being \textbf{observed}. This ontological ambiguity---central to human--robot interaction research---is precisely what makes NAO a semantically potent perturbation of the moral field.

\bigskip
\noindent
\begin{center}
	\nextstatement
	\begin{hypobox}{Synthetic Normativity of Moral Displacement}
		\label{hyp:synthetic_normativity}
		Synthetic presences, though devoid of sentience, may acquire \textit{normative
			affordances} by virtue of their perceived ontology. When situated within
		morality-salient environments, such presences may disrupt, refract, or displace
		the evaluative machinery through which moral judgments are ordinarily formed.
	\end{hypobox}
\end{center}
\bigskip
\noindent

This hypothesis extends the behavioural prediction into the normative domain: the robot may change not only what people \emph{do}, but the conditions under which moral meaning becomes actionable. The Watching--Eye paradigm thus becomes
a conceptual probe—a way of examining the \emph{structural elasticity} of norm-sensitive cognition in the presence of synthetic observers.

\noindent
Under this interpretation, generosity is not a simple expression of stable virtue or personality; it is the \textit{emergent property} of a cognitive--affective system embedded in a structured moral environment. Robotic presence, by virtue of its ontological ambiguity, acts as a refractive affordance: it bends the
path from moral perception to moral action, attenuating the behavioural expression of prosocial salience.

\medskip
\noindent
This notion of an \textit{emergent property} deserves clarification, for it plays an important explanatory role in how the experiment should be interpreted. In the present context, emergence does not denote mysterious or irreducible behaviour; it describes a structural fact about norm-sensitive cognition \cite{Greene2001,Haidt2001,Cushman2013}. Prosocial donation arises here not from any single component of the experimental system---neither from the moral cue alone ($\alpha_E$), nor from the dispositional architecture ($\beta_C$), nor from the robot’s presence ($\gamma_R$) taken in isolation. Rather, generosity appears as the \textit{behavioural output of an interaction} \cite{Decety2004,Conty2016}:
\[
\mathscr{P}(\delta_m) = f(\alpha_E, \beta_C, \gamma_R).
\]
Under fixed dispositions, the output can change simply because the evaluative field in which those dispositions operate has been deformed \cite{Pentland2007,Vinciarelli2009}. This is precisely what the data reveal: the robot produces a \emph{uniform directional shift} in donation behaviour despite stable trait profiles \cite{Bremner2022,Kuchenbrandt2011,Malle2016}. In this sense, prosocial behaviour is emergent: it is a property of the \emph{system} formed by dispositions, environmental cues, and contextual topology, not a direct expression of any one part \cite{FehrFischbacher2003,WarnekenTomasello2006}.

\medskip
\noindent
A helpful comparison can be drawn---carefully---with contemporary discussions of emergent capacities in large language models. In LLMs, emergence refers to capabilities that arise from the interaction of many parameters without being explicitly encoded \cite{Mittelstadt2019,Bender2021}. Here, too, the behavioural effect reflects an interactional architecture: moral action is generated by the coupling of perceptual salience, affective readiness, and contextual priors \cite{Greene2014,Crockett2016}. Yet, unlike in LLMs, the emergence observed here is phenomenological and contextually scaffolded: the evaluative field itself is altered, and behaviour shifts even though the underlying dispositions remain constant \cite{Zlotowski2015,Banks2020}.

\medskip
\noindent
This also clarifies the function of the mathematical formalism introduced in the preceding chapters. The equations do not quantify moral agency in any metaphysical sense; they provide an explicit epistemic schema for locating the point of deformation \cite{Konovalov2016,Shenhav2017}. By decomposing the evaluative transformation into $\alpha_E$, $\beta_C$, and $\gamma_R$, the formalism makes it possible to rule out trait-level explanations and demonstrate that the behavioural shift originates at the level of the mapping:
\[
f(\alpha_E, \beta_C, \gamma_R) \neq f(\alpha_E, \beta_C).
\]
In this respect, the mathematics functions as a conceptual microscope: it enables the isolation of the structural point at which synthetic presence exerts influence. Without such decomposition, the uniform attenuation might be mistakenly attributed to personality differences, random noise, or implicit experimental demand \cite{Dear2019,Pfattheicher2015}.

\medskip
\noindent
Thus, when the analysis later reports that moral behaviour changed while traits did not, the claim is not that generosity ``collapsed'', nor that personality ``failed'' to predict behaviour. The claim is that the \textit{evaluative topology} was reconfigured by an ontologically ambiguous presence---yielding an emergent behavioural pattern that no component of the system could produce alone \cite{Haley2005,Bateson2006}. This, in turn, is what lends the experiment its broader philosophical significance: it demonstrates that synthetic agents can perturb the moral field not by thinking, or by acting, but simply by \textit{being present} within the perceptual architecture through which moral salience becomes action \cite{Zlotowski2015,Malle2016,Bremner2022}.


\medskip
\noindent
With this evaluative architecture established, the next section examines how this deformation manifests empirically—first in behavioural data, and then in its interaction (or lack thereof) with dispositional structure. The question (\ref{q:emp_que}), as framed at the outset, demanded a yes/no answer. The analysis to follow now
supplies the evidential basis for that answer.

%%% HERE THE PROBLEM STARTED %%%
%%% NEW CONTENT WAS ADDED %%%
%%% MUCH OF THE STATISTICS IS LOST BUT WE GAIN CLARITY %%%
%%% OLD CONTENT CAN BE RESUMED FROM TXT IN DESKTOP %%%

\section{Synthetic Perturbation of Moral Inference}

\noindent
Before entering the empirical phase, we require a precise \emph{mechanistic anchor}:\footnote{In this context, ``mechanistic’’ refers not to physical causation but to the minimally specified, testable account of \emph{where} in the evaluative process a perturbation is expected to act. It identifies the locus of influence within the mapping from moral salience to behavioural output.} a statement that links the evaluative–topological model developed in the preceding chapters to the behavioural analyses that follow. Without such an anchor, the experiment would risk degenerating into a mere behavioural vignette, detached from the normative and computational structure established earlier. The present section therefore identifies the precise inferential target against which all subsequent statistical results must be interpreted.

\medskip
\noindent
Chapters~\ref{chap:moral_primer}--\ref{chap:tools_new} articulated the evaluative 
architecture through which moral salience (\(\alpha_E\)) is transformed into behavioural output (\(\mathscr{P}(\delta_m)\)), modulated by dispositional structure (\(\beta_C\)) and,potentially, by synthetic perturbation (\(\gamma_R\)). The central empirical question (Question~\ref{q:robot-agent}) asked whether the mere presence of a humanoid robot systematically deforms the mapping from salience to action. The role of the present hypothesis is to turn that question into a testable inferential claim: it specifies \emph{how} and \emph{where} the perturbation is expected to manifest within the evaluative transformation.

\medskip
\noindent
In the experimental setting, the Watching–Eye stimulus structures the moral field 
\(\Sigma\); the dispositional manifold \(\beta_C\), measured through EQ, SQ, and the BFI, provides each participant’s cognitive–affective baseline; and the robot’s presence \(\mathscr{R}\) introduces a perceptually social, ontologically ambiguous affordance. The crucial question is whether \(\mathscr{R}\) modulates the internal transformation that links perceptual–affective inputs to prosocial action.

\[
\Sigma \longrightarrow \mathscr{D}
\]

\noindent
Under ordinary conditions, this transition is driven by the salience of the moral cue.
When the robot is present, however, its ambiguous social ontology may refract or suppress
the affective and reputational components that ordinarily support prosocial decision-making.
This motivates the mechanistic hypothesis.

\begin{center}
	\nextstatement
	\begin{hypobox}{Synthetic Perturbation of Moral Inference}
		\label{hyp:synthetic_perturbation}
		The humanoid robot NAO does not function as a passive observer, but as a 
		perturbative presence that refracts the transition from moral salience to 
		prosocial action. Its ontological ambiguity displaces the affective and 
		reputational cues that ordinarily support donation, thereby modulating the 
		evaluative pathway by which moral stimuli gain behavioural expression.
	\end{hypobox}
\end{center}

\noindent
This hypothesis identifies the mechanistic level at which synthetic presence is expected 
to operate. The claim is not that NAO exerts coercive influence or that participants 
attribute moral authority to it. Rather, the prediction is that NAO’s perceptually 
social yet ontologically indeterminate presence alters the \emph{topology} of the 
evaluative field: shifting which features are foregrounded, how moral cues are weighted, 
and how affective resonance is integrated into action. In this sense, the robot 
functions as a \emph{semantic perturbation}---a presence that reconfigures the 
informational structure through which salience becomes behaviour.

\medskip
\noindent
With this mechanistic hypothesis established, we can now transition to the empirical 
analysis. The next section evaluates whether the two experimental groups were equivalent 
in their demographic and dispositional structure, ensuring that any subsequent 
behavioural divergence can be attributed to the perturbative role of \(\mathscr{R}\) 
rather than to background variation within \(\beta_C\). The behavioural results 
that follow then provide the evidential basis for adjudicating whether the deformation 
predicted here is indeed observed.





\section{Inferential Analysis of Experimental Data}
\label{sec:inferential_analysis}

\noindent
Before we enter the empirical phase of the argument, it is crucial to clarify
what this section contributes to the architecture of the thesis. Up to this
point, the chapter has operated at the level of \emph{evaluative structure}:
we have identified the components of the perceptual--moral field, specified the
variables of the evaluative transformation 
\[
\mathscr{P}(\delta_m) = f(\alpha_E, \beta_C, \gamma_R),
\]
and articulated the mechanistic hypothesis governing how synthetic presence
($\gamma_R$) may refract the mapping from moral salience ($\alpha_E$) to
behavioural output ($\mathscr{P}(\delta_m)$).

\medskip

\noindent
What follows changes register.  
This section inaugurates the \emph{inferential} phase of the thesis: the point
at which conceptual commitments must submit to statistical adjudication.  
If the preceding sections drew the topology of the evaluative field, the
analyses to follow measure its curvature.

\medskip

\noindent
Two principles govern the transition.

\begin{enumerate}
	\item \textbf{Inferential validity requires structural symmetry.}  
	Before testing whether the perturbation $\gamma_R$ deformed the evaluative
	mapping, we must first establish that the two experimental groups were
	equivalent with respect to demographic and dispositional structure.  
	Without this symmetry, any behavioural divergence would be uninterpretable
	at the level of mechanism.
	
	\item \textbf{Statistical analysis is not a post-hoc addition, but the
		operational expression of the theoretical model.}  
	The inferential pipeline---from distributional checks to regression
	modelling and cluster analysis---implements the evaluative framework
	developed in earlier chapters. Each statistical test corresponds to a
	theoretical question: Does $\gamma_R$ shift the distribution of prosocial
	action? Does $\beta_C$ moderate that shift? Is the effect uniform across
	the dispositional manifold?
\end{enumerate}

\medskip

\noindent
The present section therefore serves a dual purpose.  
First, it validates the experimental precondition of group comparability.  
Second, it establishes the methodological pathway through which the mechanistic
hypothesis introduced above will be empirically evaluated.

\bigskip
\noindent
\begin{center}
	\begin{leftbar}
		\textit{From this point onward, every claim is grounded not in conceptual
			plausibility, but in statistical evidence.}
	\end{leftbar}
\end{center}

\bigskip
\noindent
We begin by demonstrating the demographic and dispositional equivalence of the
two participant groups. Only once this foundational condition is met can we
proceed to analyse whether synthetic presence introduced a systematic
deformation in the evaluative mapping from moral salience to observable moral
action.

\subsection{Demographic Equivalence as a Symmetry Condition}
\label{subsec:dem_equivalence}

Before any inferential claims can be drawn from the behavioural data, we must
establish that the two experimental groups were demographically comparable.  
Within the evaluative–topological framework developed earlier, demographic
symmetry functions as a foundational \emph{inferential constraint}: only when
the underlying populations exhibit similar baseline characteristics can any
observed behavioural divergence be attributed—within the limits of the design—to
the perturbative presence of the robot $\mathscr{R}$ rather than to sampling
asymmetries in the human substrate.

\medskip
\noindent
To this end, we examined three demographic variables that plausibly influence
prosocial responsiveness in field and laboratory studies: gender, age, and
educational background. Each was tested across the \textbf{Control} and
\textbf{Robot} conditions using standard inferential procedures, with
Benjamini--Hochberg False Discovery Rate (FDR) correction applied to guard
against spurious equivalence due to multiple comparisons.

\begin{itemize}
	\item \textbf{Gender distribution}: a chi-squared test revealed no
	significant difference between conditions ($p = 1.00$, FDR-corrected).
	
	\item \textbf{Age}: an independent-samples \emph{t}-test detected no
	difference in mean age between groups ($p = 1.00$, FDR-corrected).
	
	\item \textbf{Educational background}: a chi-squared test again showed no
	reliable difference ($p = 1.00$, FDR-corrected).
\end{itemize}

\noindent
The convergence of these results under strict FDR control allows us to draw the
following methodological conclusion:

\begin{quote}
	\textbf{The two experimental groups are demographically equivalent.}
\end{quote}

\noindent
This symmetry condition is essential for the analyses that follow. It ensures
that the behavioural differences later observed cannot be attributed to
demographic imbalance or hidden stratifications in the participant pool.
Instead, under the architecture developed in the preceding sections, any
systematic divergence in prosocial behaviour becomes attributable to the
semiotic and perceptual perturbation introduced by the robot, $\mathscr{R}$,
after holding $\alpha_E$ constant and before considering variation in the
dispositional manifold $\beta_C$.

\begin{table}[H]
	\centering
	\includegraphics[width=\textwidth]{tables/demographic_balance_table.pdf}
	\caption{Demographic balance tests across experimental conditions.  
		Values shown include original and FDR-corrected $p$-values for gender, age,
		and educational background. No comparison reached significance after
		correction, supporting the assumption of demographic equivalence required
		for subsequent inferential interpretation of behavioural effects.}
	\label{tab:dem_balance}
\end{table}

\medskip
\noindent
With demographic symmetry established, the analysis proceeds to the next
inferential layer: the behavioural effects of synthetic presence. Subsequent
sections will assess donation outcomes directly, and only \emph{thereafter}
will the dispositional structure—encoded via EQ, SQ, and BFI—be examined for
potential interactions with $\mathscr{R}$. This ordering preserves the logic of
the evaluative–topological framework: baseline equivalence first, behavioural
effects second, dispositional modulation third.

% ============================================================
\subsection{Data Preparation and Preprocessing Workflow}
\label{subsec:data_preprocessing}
% ============================================================

Because the inferential analyses that follow rely on contrasts across
experimental conditions, dispositional variables, and behavioural outputs, a
principled preprocessing pipeline is an epistemic prerequisite rather than a
technical convenience. The aim of this stage is to ensure that the dataset
constitutes a coherent and interpretable representation of the experimental
structure, free from syntactic artefacts, coding inconsistencies, or latent
category imbalances. Only under such conditions can the subsequent statistical
models be taken to track the evaluative transformations at issue in this
chapter.

The dataset comprises demographic descriptors, psychometric measures (EQ, SQ,
BFI), and the behavioural outcome (donation magnitude). These variables differ
in type, scale, and inferential role; they therefore require tailored
preprocessing steps to preserve their semantic integrity.

\paragraph{Standardisation of variable names.}
All variable names were converted to lowercase, whitespace-trimmed, and
harmonised to eliminate discrepancies introduced through manual data entry.
This ensures referential consistency throughout the analysis.

\paragraph{Encoding of behavioural outcome.}
The binary variable \texttt{donated\_anything} was created ($1=$ donated at
least one coin; $0=$ donated nothing). This permits modelling of prosocial
behaviour at two complementary levels: (i) the full distribution of donation
amounts and (ii) the threshold decision to donate at all.

\paragraph{Encoding of experimental condition.}
The variable \texttt{condition\_bin} was constructed ($0=\text{Control}$,
$1=\text{Robot}$) to allow direct incorporation into regression frameworks and
to maintain a clear contrast between conditions.

\paragraph{Verification of categorical coherence.}
Categorical fields (e.g., \texttt{gender}) were inspected for irregularities
such as collapsed, duplicated, or misspelled levels. No anomalies requiring
recoding were identified.

\paragraph{Preliminary distributional checks.}
Initial visual inspections (histograms, density plots, boxplots) revealed no
anomalous values requiring removal or recoding. Age distributions and donation
distributions are shown in Figures~\ref{fig:age_distribution_by_group}
and~\ref{fig:donation_distribution_by_condition}, respectively, illustrating the
distributional structures to be analysed in the inferential sections that
follow.

\begin{figure}[htbp]
	\centering
	\includegraphics[width=0.8\textwidth]{new_plots/age_distribution_by_group.png}
	\caption{Age distribution across experimental conditions. The histograms
		illustrate the demographic structure of the sample to be examined in later
		analyses.}
	\label{fig:age_distribution_by_group}
\end{figure}

\begin{figure}[htbp]
	\centering
	\includegraphics[width=0.8\textwidth]{new_plots/donation_distribution_by_condition.png}
	\caption{Distribution of donation behaviour by condition. The plot presents
		the behavioural data whose inferential assessment constitutes the next stage
		of analysis.}
	\label{fig:donation_distribution_by_condition}
\end{figure}

\medskip
\noindent
Taken together, these preprocessing steps establish the analytic coherence
required for valid inferential modelling. With demographic equivalence confirmed
in the previous subsection and the present transformations ensuring structural
stability of the data, the chapter now proceeds to the statistical models that
evaluate whether the perturbation introduced by $\mathscr{R}$ manifests in the
transition from moral salience to moral action.

%%% START FROM THIS SECTION BELOW %%%

\subsection{Preliminary Descriptive Patterns: Orientation Prior to Inferential Analysis}

\noindent
Before entering the inferential phase, it is useful to outline the basic distributional structure of the key behavioural and psychometric variables. Descriptive statistics possess no evidential force in themselves; their role is purely orientational. They summarise the raw landscape of the data so that the formal tests in the following sections can be interpreted against a clear empirical backdrop.

\medskip

\noindent
Table~\ref{tab:descriptive-stats} reports the central tendencies for the principal variables collected in the study. The mean donation values in the two conditions (\textit{Control} and \textit{Robot}) appear numerically distinct, and several psychometric scores (EQ, SQ, BFI subscales) exhibit small numerical differences across groups. These contrasts, however, are \emph{purely descriptive}: they record observed sample characteristics and do not imply either imbalance or effect. Their interpretive significance, if any, will be assessed formally in the subsequent statistical analyses.

\medskip

\noindent
The descriptive summaries therefore serve three limited but important functions:
\begin{enumerate}
	\item they present the distributional contours that later inferential tests will interrogate;
	\item they facilitate visual inspection for anomalous values, without indicating any need for exclusion;
	\item they prepare the reader for the dispositional and behavioural modelling developed in the sections that follow.
\end{enumerate}

\noindent
Crucially, nothing in the descriptive patterns licences an inferential conclusion. Whether the robotic presence $\mathscr{R}$ perturbs the evaluative mapping from moral salience to action is a question answered only by the formal models presented later. These descriptive tables merely contextualise the data that will feed into those models.

\begin{table}[H]
	\centering
	\includegraphics[width=\textwidth]{tables/descriptive_highlights_table.pdf}
	\caption{Descriptive summaries of behavioural and psychometric variables across experimental conditions. These values provide an orienting overview of the sample; they do not support any inferential claims regarding group differences or perturbation effects.}
	\label{tab:descriptive-stats}
\end{table}

\noindent
With this preliminary orientation in place, we now turn to the inferential structure itself, beginning with the verification of demographic symmetry across conditions—a prerequisite for attributing any subsequent behavioural divergence to the experimental manipulation rather than to background variability.

%%% PARTI DA QUI %%%

\subsection{Inferential Comparison of Donation Patterns Across Conditions}
\label{subsec:inferential_attentuation}

\noindent
Having established the demographic symmetry of the sample and the analytic coherence of the dataset, we now turn to the first formal evaluation of whether robotic presence $\mathscr{R}$ influences prosocial behaviour. Up to this point, all analyses have been structural or descriptive; the task now is to determine whether the behavioural distributions associated with the moral decision---the donation act---show any statistically reliable divergence across conditions. This is the first moment in the chapter where inferential weight is brought to bear on the \textit{Evaluative Deformation Hypothesis} (Hypothesis~\ref{hyp:evaluative_deformation}), and the transition therefore marks a shift from conceptual scaffolding to statistical adjudication.

\medskip
\noindent
We proceed in an intentionally layered way. A single test rarely captures the complexity of a behavioural distribution; instead, a sequence of complementary analyses is required. We begin with a chi-squared test on coin-frequency distributions, then examine the full donation distributions using a Mann--Whitney U test, and finally quantify the magnitude of the difference via a nonparametric bootstrap. Each method probes a different facet of the data: aggregate totals, distributional structure, and effect-size stability respectively.

% ---------------------------------------------------------------
\paragraph{Chi-squared test on donation frequencies.}
\noindent
A chi-squared test comparing the \emph{frequency distribution of donated coins} across the Control and Robot conditions revealed a statistically detectable divergence:
\[
\chi^2 = 4.25,\qquad p = .039.
\]
This test does not assess means or medians but evaluates whether the overall pattern of coin contributions differs across conditions. The result indicates that 

\bigskip
\noindent
\begin{center}
	\begin{leftbar}
		\textit{The aggregate structure of donation behaviour is not evenly distributed across the two environments.}
	\end{leftbar}
\end{center}

\begin{table}[H]
	\centering
	\includegraphics[width=\textwidth]{tables/statistical_tests_table.pdf}
	\caption{Inferential comparisons of donation behaviour across conditions. The chi-squared test compares coin-frequency distributions, while the Mann--Whitney U test and bootstrapped mean difference assess distributional structure and effect magnitude respectively.}
	\label{tab:statistical_tests}
\end{table}

\noindent
It is crucial, however, that this result be interpreted with methodological restraint. The chi-squared test establishes an \emph{aggregate} divergence, not a uniform shift in individual tendencies. To understand the behavioural topology more fully, we must examine the entire donation distribution.

% ---------------------------------------------------------------
\paragraph{Mann--Whitney U test on donation distributions.}
\noindent
A Mann--Whitney U test, applied to the full distribution of donation amounts, did not detect a statistically reliable difference:
\[
U = 777.0,\qquad p = .194.
\]
This indicates substantial overlap between individual donation behaviours in the two conditions. In other words, while coin-frequency totals diverge, the fine-grained distribution of donation amounts remains broadly similar. This pattern suggests that the influence of $\mathscr{R}$ may be probabilistic and heterogeneous rather than deterministic or uniform across participants.

% ---------------------------------------------------------------
\paragraph{Bootstrapped estimate of the mean difference.}
\noindent
To complement these analyses, we calculated a nonparametric bootstrap estimate of the mean donation difference:
\[
\Delta M = £0.71,\qquad 95\%\ \mathrm{CI} = [-£0.33, £1.79].
\]
The estimate aligns directionally with the group-level pattern (Control > Robot), but the confidence interval includes zero, reinforcing the conclusion that the effect is subtle and probabilistic rather than sharply bifurcated.

\begin{figure}[htbp]
	\centering
	\includegraphics[width=0.8\textwidth]{new_plots/donation_effect_by_condition.png}
	\caption{Mean donation amounts by experimental condition, with 95\% bootstrapped confidence intervals. The overlapping intervals illustrate substantial individual-level variability, indicating that any perturbative influence of $\mathscr{R}$ is diffuse rather than deterministic.}
	\label{fig:donation_effect_by_condition}
\end{figure}

% ---------------------------------------------------------------
\medskip
\noindent
Taken together, these three analyses trace a coherent statistical profile. There is evidence of divergence in aggregate coin-frequency behaviour, yet the distributional overlap and wide bootstrap interval demonstrate that the perturbation introduced by $\mathscr{R}$ is not uniform across individuals. This is exactly the pattern predicted by the evaluative-topological model: 

\bigskip
\noindent
\begin{center}
	\begin{leftbar}
		\textit{Synthetic presence functions not as a coercive force but as a semiotic perturbator, modulating the evaluative mapping from moral cue to action in a heterogeneous population.}
	\end{leftbar}
\end{center}

\medskip
\noindent

In interpretive terms, the results neither dismiss nor overstate the effect. They demonstrate that robotic presence is behaviourally consequential at the aggregate level while leaving open, and thereby motivating, the central analytical task of the next subsections: determining whether this perturbation interacts with the dispositional manifold $\beta_C$. The inferential focus therefore now turns to regression modelling, interaction tests, and Bayesian estimation, where the structure of $\beta_C$ can be incorporated directly into the evaluation of $\gamma_R$.


\subsection{Interim Conclusion to Question~\ref{q:robot-agent}}

\begin{center}
	\begin{tcolorbox}[colback=white,colframe=black!60,title=Partial Conclusion to Question~\ref{q:robot-agent}]
		The behavioural evidence obtained thus far indicates that the silent co-presence of a humanoid robot, operating with minimal but perceptually salient behavioural affordances, systematically attenuates aggregate donation behaviour under a Watching Eye paradigm. This attenuation is modest, probabilistic, and heterogeneously distributed across individuals, but it is empirically detectable and statistically non-trivial.
		
		\medskip
		Within the formal and philosophical architecture developed in this chapter, these findings support the plausibility of \emph{evaluative deformation}: the robot perturbs the inferential transformation from morally salient cues to observable moral action. Floridi’s Levels of Abstraction framework explains why such perturbation is possible—because the robot’s \emph{perceived ontology} and informational encoding render it normatively relevant at the operative LoA, even in the absence of sentience or interaction. The Synthetic Perturbation of Moral Inference hypothesis then specifies \emph{how} this relevance is instantiated, by refracting the evaluative pathway rather than overriding it.
		
		
		\medskip
		The role of individual traits, represented by the vector \(\beta_C\), and their interaction with robotic presence \(\gamma_R\), remains an open and theoretically salient question. The next sections therefore move from aggregate contrasts to trait–context modelling, in order to determine whether moral displacement is uniformly distributed or preferentially expressed in specific psychological profiles.
	\end{tcolorbox}
\end{center}

\noindent
In summary, the results to this point justify the claim that robotic co-presence modifies the evaluative conditions under which morally salient cues become behaviourally actionable, in a manner that is fully consistent with the informational and topological commitments of the Floridian framework. The retained hypotheses and formalism together provide the conceptual, ontological, and mechanistic scaffolding for the more fine-grained analyses that follow.

%%% NEW CONTENT N7
Beyond establishing the statistical significance of the observed differences, it is epistemically imperative to quantify the magnitude of behavioral perturbation induced by robotic presence. The following analyses introduce both parametric and nonparametric effect size metrics to characterise the structural modulation of moral decision-making.

\subsection{Quantification of Behavioural Modulation: Parametric and Nonparametric Effect Sizes}
\label{subsec:effect_sizes}

\noindent
Having established that the experimental groups are demographically symmetric and that 
the aggregate-level analyses reveal a measurable difference in charitable behaviour across 
conditions, we now turn to the question of \emph{magnitude}. Significance tests indicate 
whether a behavioural contrast is detectable relative to sampling variability; they do not 
characterise the structural amplitude of the perturbation induced by the synthetic 
co-presence $\mathscr{R}$. For this reason, the present section complements the inferential 
tests with parametric and nonparametric effect-size metrics, thereby quantifying the extent 
to which robotic presence modulates prosocial behaviour under the Watching–Eye paradigm.

\medskip
\noindent
Because the subsequent regression and interaction analyses will examine the interplay 
between robotic presence and dispositional structure, it is essential to begin with a 
transparent description of the overall behavioural landscape. The effect sizes presented 
here serve as the bridge between aggregate-level contrasts and the more nuanced 
trait–context models developed later in the chapter.

% ---------------------------------------------------------
%   Effect Size Overview
% ---------------------------------------------------------

\subsubsection*{Effect-Size Framework}

\noindent
Two complementary measures were selected:

\begin{itemize}[itemsep=4pt]
	\item \textbf{Cohen’s $d$}: a parametric index of standardised mean difference, 
	sensitive to shifts in central tendency;
	\item \textbf{Cliff’s $\Delta$}: a nonparametric ordinal effect size that estimates 
	the probability that a randomly selected individual from one condition donates more 
	(or less) than a randomly selected individual from the other.
\end{itemize}

\noindent
Taken together, these metrics evaluate whether the presence of $\mathscr{R}$ reshapes the 
evaluative output distribution in a manner consistent with the deformation posited in the 
Evaluative Deformation Hypothesis (Hypothesis~\ref{hyp:evaluative_deformation}).

% ---------------------------------------------------------
%   Definitions
% ---------------------------------------------------------

\paragraph{Cohen’s $d$.}

\[
d = \frac{\bar{x}_1 - \bar{x}_2}{s_p}
\quad\text{where}\quad
s_p = \sqrt{\frac{(n_1 - 1)s_1^2 + (n_2 - 1)s_2^2}{n_1 + n_2 - 2}}
\]

\noindent
Where:
\begin{itemize}
	\item $\bar{x}_1, \bar{x}_2$ = group means (Control, Robot),
	\item $s_1, s_2$ = corresponding standard deviations,
	\item $n_1, n_2$ = sample sizes.
\end{itemize}

\paragraph{Cliff’s Delta.}

\[
\Delta = \frac{\#(x>y) - \#(x<y)}{n_x n_y}
\]

\noindent
Where:
\begin{itemize}
	\item $\#(x>y)$ counts all cases where a Control donation exceeds a Robot donation,
	\item $\#(x<y)$ counts the inverse,
	\item $n_x, n_y$ are the sample sizes of each group.
\end{itemize}

\medskip
\noindent
The empirical results yield:
\[
d \approx 0.30,
\qquad
\Delta \approx 0.20.
\]

\noindent
Both indices fall within the range typically interpreted as \textit{small to modest} 
behavioural modulation. Their relevance lies not in magnitude alone, but in the fact that 
both metrics converge on the same directional pattern: robotic presence is associated with 
lower prosocial donation on average.

% ---------------------------------------------------------
%   Figure Justification
% ---------------------------------------------------------

\medskip
\noindent
To ensure interpretive clarity, two complementary visualisations are provided.  
The kernel density estimate (Fig.~\ref{fig:donation_density}) depicts the \emph{shape} and 
spread of donation distributions, enabling inspection of distributional tails and modes.  
The mean-with-standard-error plot (Fig.~\ref{fig:mean_donation}) focuses on 
\emph{central tendency} and sampling variability.  
Although partially overlapping in content, the two figures serve distinct analytic functions 
and together offer a transparent view of the behavioural landscape that informs the 
subsequent modelling work.

% ---------------------------------------------------------
%   Figures
% ---------------------------------------------------------

\begin{figure}[htbp]
	\centering
	\includegraphics[width=0.8\textwidth]{new_plots/donation_density_by_condition.png}
	\caption{Kernel density estimates of donation distributions across experimental conditions. 
		The Control group exhibits greater mass at higher donation values, whereas the Robot group 
		shows a mild left-shift in density. These plots provide distributional context for the 
		effect-size metrics discussed in the text.}
	\label{fig:donation_density}
\end{figure}

\begin{figure}[htbp]
	\centering
	\includegraphics[width=0.8\textwidth]{new_plots/donation_mean_with_se.png}
	\caption{Mean donation amounts with standard error bars by condition. 
		While the Control group donates more on average, the overlapping error bars reflect 
		substantial individual-level variability. The figure complements the density plot by 
		highlighting differences in central tendency rather than distributional shape.}
	\label{fig:mean_donation}
\end{figure}

% ---------------------------------------------------------
%   Statistical Tests Table (reproduced for completeness)
% ---------------------------------------------------------

\medskip
\noindent
For completeness, the inferential tests introduced earlier are reproduced in 
Table~\ref{tab:statistical_tests} alongside the effect-size metrics, ensuring that all 
aggregate-level results appear within a single consolidated reference point before turning 
to trait–context modelling.

\begin{table}[H]
	\centering
	\includegraphics[width=\textwidth]{tables/statistical_tests_table.pdf}
	\caption{Inferential comparisons of donation behaviour across conditions. 
		The chi-squared test (applied to total coin frequencies), the Mann--Whitney U test, 
		and the bootstrapped mean difference collectively characterise the behavioural contrast.}
	\label{tab:statistical_tests}
\end{table}

% ---------------------------------------------------------
%   Interpretive Summary
% ---------------------------------------------------------

\noindent
Overall, the effect sizes indicate that robotic presence exerts a 
\emph{directionally consistent, behaviourally modest} modulation of prosocial action.  
These outcomes are consistent with---though they do not in isolation confirm---the 
prediction that $\mathscr{R}$ perturbs the evaluative transformation from moral salience 
to behaviour. Importantly, the effect is \emph{graded}, not binary: the evaluative system 
remains operative, but the strength with which moral cues are translated into action is 
probabilistically dampened.
\bigskip
\noindent
\begin{center}
	\begin{tcolorbox}[colback=white,colframe=black!60,
		title=Conclusion: Amplitude of Moral Refraction]
		Synthetic co-presence does not function as a binary suppressor of moral behaviour.  
		Instead, it modulates the amplitude of the evaluative transformation from moral salience 
		to action, introducing a subtle, probabilistic refractive shift consistent with its 
		ambiguous ontological encoding at the operative Level of Abstraction.
	\end{tcolorbox}
\end{center}
\bigskip
\noindent
% ---------------------------------------------------------
%   Transition to Trait–Context Modelling
% ---------------------------------------------------------

\noindent
At this point, the analysis shifts from evaluating the \emph{main effect} of 
$\mathscr{R}$ to examining its interaction with individual-level dispositions.  
The dispositional manifold $\beta_C$---comprising empathizing and systemizing tendencies 
as well as Big Five personality traits---may modulate the susceptibility of the 
evaluative mapping $f(\alpha_E, \beta_C, \gamma_R)$ to perturbation.  
The following sections introduce regression models, interaction tests, and Bayesian 
estimation procedures designed to determine whether the attenuation observed here is 
uniform across the population or concentrated within specific psychological profiles.


\section{Dispositional Baseline: Big Five Personality Traits Across Conditions}

\noindent
A foundational requirement for attributing the observed attenuation of prosocial behaviour to the presence of the humanoid robot is the establishment of \emph{dispositional equivalence} between the two experimental groups. If participants in the Robot condition were, for example, systematically lower in Agreeableness or Empathizing, then differences in donation behaviour could be trivially explained by trait imbalance rather than by the perturbative effect of $\mathscr{R}$. The question addressed in this section is therefore epistemically prior to all subsequent modelling:


\bigskip
\noindent
\begin{center}
	\begin{leftbar}
		\textit{Do the Big Five personality traits differ between the Control and Robot conditions, and thus constitute a potential confound for interpreting the displacement of prosocial behaviour?}
	\end{leftbar}
\end{center}

\bigskip
\noindent

\subsection{Between-Condition Comparisons of Big Five Personality Traits}
\label{subsec:bigfive_groupdiff}

\noindent
The effect-size analyses above established that robotic co-presence 
($\mathscr{R}$) exerts a modest but directionally consistent modulation of 
donation behaviour. Before examining whether this perturbation interacts with 
individual differences, we must ensure that the two experimental groups were 
not already differentiated at the level of personality. If the Control and 
Robot conditions differed systematically in the Big Five traits, any apparent 
behavioural attenuation could reflect pre-existing dispositional imbalance 
rather than the influence of $\mathscr{R}$.

\medskip
\noindent
To assess this, we compared Openness, Conscientiousness, Extraversion, 
Agreeableness, and Neuroticism across conditions using the Mann--Whitney 
$U$ test. This test is appropriate for the structure of the dataset: the Big 
Five scores are bounded, ordinal psychometric variables exhibiting mild skew, 
and the sample size ($N \approx 70$) does not justify strong parametric 
assumptions. Because examining five traits entails five simultaneous hypothesis 
tests, the Benjamini--Hochberg False Discovery Rate (FDR) correction was 
applied to control Type~I error.

\medskip
\noindent
After FDR correction, \textbf{none of the Big Five traits differ significantly} 
between the Control and Robot groups. Small numerical tendencies (e.g., slightly 
higher Openness in the Control condition) fail to approach corrected 
significance thresholds, and all distributions display substantial overlap.

\begin{figure}[H]
	\centering
	\begin{minipage}{0.98\linewidth}
		\centering
		\includegraphics[width=\linewidth]{new_plots/per_trait/bigfive_row_kde.png}
		\caption{Kernel density estimates for each Big Five trait across conditions. 
			All five distributions show substantial overlap, visually corroborating the 
			non-significant Mann--Whitney tests.}
		\label{fig:bigfive_kde}
	\end{minipage}
\end{figure}

\medskip
\noindent
This supports a key methodological inference:
\begin{center}
	\textbf{The two experimental groups can be treated as dispositionally equivalent.}
\end{center}

\noindent
Accordingly, the behavioural difference observed earlier is \emph{most 
	plausibly attributed} to the presence of $\mathscr{R}$ rather than to 
pre-existing personality differences.

\bigskip
\noindent
Having established dispositional equivalence, we now examine whether personality
nonetheless predicts prosocial behaviour or interacts with the attenuation 
associated with robotic co-presence.

% =============================================================
\subsection{Predictive and Moderating Roles of Big Five Personality Traits}
\label{subsec:bigfive_prediction}

\noindent
This analysis addresses a further theoretical question of interest:

\begin{quote}
	\textit{Even if the groups are balanced, do the Big Five traits predict donation, 
		or modulate the displacement induced by $\mathscr{R}$?}
\end{quote}

\paragraph{(1) Predictive effects.}
Spearman rank correlations were computed between each Big Five score and 
donation amount. Spearman’s $\rho$ is appropriate for zero-inflated, bounded, 
and non-normal behavioural data, and for ordinal psychometric measures. 
Scatterplots with monotonic trend lines were examined to detect any nonlinear 
patterns not captured numerically.

\paragraph{(2) Moderation effects.}
To test whether personality modulates the displacement effect, interaction 
models of the form
\[
\text{donation} \sim \text{condition} \times \text{trait}
\]
were estimated for each Big Five dimension. This correctly operationalises the 
possibility that robotic presence acts as a \textit{moral refractor}, exerting 
differential influence depending on dispositional architecture.

\medskip
\noindent
Methodologically, the findings are straightforward.  
\textbf{No statistically reliable associations} between any Big Five trait and 
donation amount appear in this dataset, and \textbf{no interaction} with 
experimental condition reaches significance. The behavioural attenuation 
associated with $\mathscr{R}$ therefore shows no detectable variation across 
Big Five personality profiles.

\begin{figure}[H]
	\centering
	\begin{minipage}{0.98\linewidth}
		\centering
		\includegraphics[width=\linewidth]{new_plots/per_trait/bigfive_row_scatter.png}
		\caption{Scatter plots with monotonic trend lines for each Big Five 
			trait against donation amount. No predictive relationships appear, and 
			no moderation patterns are visible. This matches the null results from 
			correlations and interaction models.}
		\label{fig:bigfive_scatter}
	\end{minipage}
\end{figure}

\medskip
\noindent
In summary, within the Big Five framework:

\begin{itemize}
	\item no trait reliably predicts prosocial donation;
	\item no trait moderates the attenuation introduced by robotic co-presence;
	\item the displacement effect of $\mathscr{R}$ shows no detectable 
	variation across Big Five profiles \emph{in this sample}.
\end{itemize}

\begin{center}
	\begin{tcolorbox}[colback=white,colframe=black!60,
		title=Conclusion: Trait-Independence of Evaluative Displacement]
		The attenuation of prosocial donation under robotic co-presence shows no 
		detectable modulation by Big Five personality traits in this dataset. 
		This supports the interpretation that $\mathscr{R}$ acts on the evaluative 
		field itself rather than on specific dispositional pathways.
	\end{tcolorbox}
\end{center}

\noindent
The next subsection examines whether more specialised social-cognitive traits—
the Empathizing Quotient (EQ) and Systemizing Quotient (SQ)—show predictive or 
moderating roles that the broad Big Five taxonomy does not capture.


\subsection{Transition to Structural Modelling of Dispositional Architecture}

\noindent
The analyses reported above establish two methodological foundations that shape the remainder of the statistical pipeline. First, the Big Five traits do not differ across conditions after False Discovery Rate correction, confirming dispositional symmetry between the Control and Robot groups. Second, within this sample, none of the Big Five traits reliably predict donation behaviour, nor do they interact with experimental condition. In inferential terms, the dataset contains no evidence of trait imbalance and no statistically detectable trait--by--condition moderation at the level of the classical personality taxonomy.

\medskip
\noindent
These observations do not \emph{rule out} the relevance of dispositional structure; instead, they clarify the level of representation at which such structure should be modelled. The Big Five are coarse-grained scalar descriptors and may not capture the finer relational geometries (i.e., covariation patterns and trait interdependencies) that shape evaluative processing. Accordingly, the next stage of analysis adopts a more structurally sensitive approach to the dispositional manifold $\beta_C$, examining whether latent configurations of empathizing, systemizing, and Big Five attributes jointly organise susceptibility to robotic perturbation.

\medskip
\noindent
In this sense, the null findings within the Big Five framework serve a methodological rather than interpretive purpose. They demonstrate that any systematic modulation of donation behaviour by the synthetic presence $\mathscr{R}$ cannot be attributed to imbalances or linear trait effects within the classical personality model. This provides the inferential basis required to proceed to clustering and latent-structure modelling, where $\beta_C$ is treated not as a vector of independent traits, but as a structured configuration whose internal organisation may interact with the perturbative affordances of $\mathscr{R}$.

\medskip
\noindent
The subsequent section therefore introduces the clustering methodology used to derive latent dispositional ecologies and examines whether these ecologies exhibit differential susceptibility to synthetic co-presence. This marks the transition from trait-level analysis to structural modelling within the broader evaluation of Question~\ref{q:robot-agent}.



\subsection{Latent Dispositional Structures and the Modulation of Moral Perturbation}
\label{subsec:latent_dispositions}

\noindent
The analyses thus far establish two essential points. First, the presence of the robot $\mathscr{R}$ is associated with a modest but coherent attenuation of prosocial donation at the aggregate level. Second, this attenuation cannot be attributed to differences in any individual Big Five trait. These results motivate a sharper question:

\begin{quote}
	\textit{If conventional trait magnitudes do not explain variability in responsiveness to $\mathscr{R}$, might the perturbation be differentially expressed across \emph{latent} cognitive–affective configurations within the dispositional manifold $\beta_C$?}
\end{quote}

\noindent
This question is structurally aligned with the evaluative model developed earlier. If synthetic presence perturbs moral behaviour by refracting the evaluative transformation $f(\alpha_E, \beta_C, \gamma_R)$, its influence need not be uniform across all individuals: the effect may depend on how dispositions combine into higher-order regimes rather than on isolated trait scores. To investigate this, we turn from scalar traits to a structural modelling of $\beta_C$.

% ---------------------------------------------------------------------
\subsubsection{Clustering the Dispositional Manifold}

\noindent
Seven psychometric variables—Empathizing, Systemizing, and the five Big Five traits—were used to construct the dispositional space. Each score vector was $z$-standardised, and dimensionality was reduced using Principal Component Analysis (PCA). Two orthogonal components were retained as they captured the dominant axes of variance while reducing redundancy among correlated traits.

\noindent
The reduced two-dimensional representation served as input for $k$-means clustering. The choice of $k = 3$ rested on both methodological and conceptual grounds:

\begin{itemize}
	\item The within-cluster sum of squares displayed a clear elbow at $k = 3$, indicating diminishing returns for higher $k$.
	\item Although a silhouette maximum was observed at $k = 9$, such peaks often reflect over-partitioning when $N$ is modest; those solutions were therefore rejected.
	\item A three-cluster solution produced groups of interpretable size with stable internal variability, consistent with the expectation that a small number of dispositional regimes may structure evaluative responsiveness.
\end{itemize}

\noindent
Figure~\ref{fig:personality-clusters-pca} visualises the resulting partitions. The figure is retained because it provides essential structural justification for treating the clusters as psychologically interpretable configurations.

\begin{figure}[H]
	\centering
	\includegraphics[width=0.95\linewidth]{new_plots/personality_clusters_pca.png}
	\caption{Participants clustered in PCA-reduced psychometric space. Three clusters emerge as coherent and visually distinguishable groupings, providing a structural basis for subsequent analyses of condition-by-cluster effects.}
	\label{fig:personality-clusters-pca}
\end{figure}

% ---------------------------------------------------------------------
\subsubsection{Justification of $k = 3$: Diagnostic Criteria}

\noindent
Figure~\ref{fig:cluster_elbow_silhouette} displays both the elbow curve and silhouette profile. It is included here because these diagnostics are standard tools for validating clustering solutions and demonstrate that $k = 3$ is a parsimonious and defensible choice.

\begin{figure}[H]
	\centering
	\includegraphics[width=0.95\linewidth]{new_plots/cluster_elbow_silhouette.png}
	\caption{Elbow plot (left axis) and silhouette coefficients (right axis) across candidate values of $k$. The elbow at $k = 3$ and stable silhouette profile support selecting three clusters as an interpretable and parsimonious solution.}
	\label{fig:cluster_elbow_silhouette}
\end{figure}

\noindent
Conceptually, a small number of clusters is consistent with the idea that only a limited set of dominant dispositional regimes may modulate how moral salience is processed under synthetic perturbation.

% ---------------------------------------------------------------------
\subsubsection{Cluster-Specific Patterns of Moral Response}

\noindent
We then examined whether the donation attenuation associated with $\mathscr{R}$ differed across clusters. Figure~\ref{fig:donation-by-cluster} shows mean donation by condition within each cluster. This visualisation is essential because it provides the descriptive foundation for the interaction models to be developed next.

\begin{figure}[H]
	\centering
	\includegraphics[width=0.85\linewidth]{new_plots/donation_by_cluster_and_condition.png}
	\caption{Mean donation amount by condition within each personality cluster. Error bars represent standard deviation. Cluster~1 shows a clearer attenuation of donation under robotic presence, while Clusters~0 and~2 display only modest or negligible differences.}
	\label{fig:donation-by-cluster}
\end{figure}

\noindent
The pattern is not uniform across clusters. Preliminary inspection of the cluster centroids suggests that Cluster~1 is characterised by higher systemizing and lower empathizing scores—a cognitive–affective style that may rely more on structural processing and less on affective resonance. This offers a plausible interpretive foothold: the evaluative perturbation induced by $\gamma_R$ may interact with configurations of traits rather than their isolated values.

\medskip
\noindent
These descriptive patterns motivate the formal interaction models introduced next, where cluster membership is incorporated as a moderator in the mapping from $\text{condition}$ to $\text{donation}$.

% ---------------------------------------------------------------------
\subsubsection*{Conclusion: Dispositional Regimes and Moral Perturbation}

\begin{center}
	\begin{tcolorbox}[colback=white,colframe=black!60,title=Interpretive Conclusion]
		Preliminary evidence suggests that the attenuation associated with robotic co-presence is not uniformly distributed across participants. Instead, latent dispositional regimes—rather than individual trait scores—appear to modulate susceptibility to the perturbative influence of $\mathscr{R}$. This provides the conceptual and empirical basis for the interaction models developed in the next section.
	\end{tcolorbox}
\end{center}


\subsection{Psychometric Interpretation and Semantic Labelling of Latent Personality Clusters}
\label{subsec:cluster_semantics}

\noindent
The identification of three latent dispositional clusters provides a structural refinement of the dispositional manifold $\beta_C$, yet clustering alone does not reveal the \emph{psychological architecture} encoded in each grouping. The analyses thus far established that the attenuation associated with $\mathscr{R}$ is not uniformly distributed across participants; the present task is to make explicit the dispositional logic through which this heterogeneity arises.

\noindent
This interpretive step is essential. Without a principled semantic characterisation of the latent clusters, the analysis would remain mathematically partitioned but psychologically opaque. Although the broader philosophical implications will be developed more fully in the Discussion chapter, the Experimental Methods chapter must already articulate the \emph{structural meaning} of these clusters, since subsequent modelling relies directly on these semantic anchors.

\medskip
\noindent
To move from numerical clusters to psychologically interpretable ecologies, we project the unscaled cluster centroids back onto the original psychometric dimensions. Radar plots (Figure~\ref{fig:radar_three_panel}) provide a justified visual tool for this step: they depict the \emph{normalised} centroid values across traits, offering a relational overview of each ecology’s internal configuration—something numeric tables alone cannot provide.

\begin{figure}[H]
	\centering
	\includegraphics[width=0.32\linewidth]{new_plots/cluster_0_radar.png}
	\includegraphics[width=0.32\linewidth]{new_plots/cluster_1_radar.png}
	\includegraphics[width=0.32\linewidth]{new_plots/cluster_2_radar.png}
	\caption{Radar profiles (normalised for comparability) of the three latent dispositional ecologies.  
		Left: Cluster~0 (Emotionally Reactive / Low-Structure);  
		Centre: Cluster~1 (Prosocial–Empathic / Warm–Sociable);  
		Right: Cluster~2 (Analytical–Structured / High-Systemizing).  
		These plots visualise the relative psychometric configuration of each ecology.}
	\label{fig:radar_three_panel}
\end{figure}

% --------------------------------------------------------------
\subsubsection{Ecology I: Emotionally Reactive / Low-Structure}

\noindent
Cluster~0 exhibits elevated Neuroticism, low Conscientiousness, reduced Systemizing, and moderate values across Openness, Extraversion, and Agreeableness. This constellation reflects an \textit{affectively reactive configuration with comparatively weaker structural coherence}. Within the moral-topological framework developed earlier, such an ecology corresponds to a \emph{loosely stabilised evaluative field}: moral cues propagate through an architecture more susceptible to contextual fluctuation, including ontological ambiguity.

\medskip

% --------------------------------------------------------------
\subsubsection{Ecology II: Prosocial–Empathic / Warm–Sociable}

\noindent
Cluster~1 is characterised by elevated Openness, Extraversion, Agreeableness, and Empathizing—a \textit{warm, sociable, affectively attuned} profile. This ecology represents the canonical prosocial configuration frequently documented in moral psychology: empathically oriented, interpersonally open, and responsive to moral cues.

\noindent
Because empathic pathways are ordinarily the most fluid in this group, the descriptively stronger attenuation of donation under $\mathscr{R}$ carries high interpretive value. It suggests that robotic presence may interfere with affective–evaluative channels rather than rule-based reasoning. The displacement of empathic resonance by an ontologically ambiguous artificial form is therefore not merely possible but observable, at least descriptively, within this ecology.

\medskip

% --------------------------------------------------------------
\subsubsection{Ecology III: Analytical–Structured / High-Systemizing}

\noindent
Cluster~2 shows elevated Systemizing and Conscientiousness with comparatively lower Empathizing, forming an \textit{analytical, structured, rule-oriented} regime. Individuals within this constellation privilege explicit structure and informational clarity over implicit social affordances.

\noindent
From a Level-of-Abstraction perspective, this ecology \emph{may be understood as aligning with} a higher abstraction threshold: ambiguous embodied agents, such as a non-interactive humanoid robot, are encoded primarily as neutral environmental features. Correspondingly, the attenuation associated with $\mathscr{R}$ appears weaker in this group.

\medskip

% --------------------------------------------------------------
\subsubsection{Interpretive Integration}

\noindent
Across these ecologies, a coherent descriptive pattern emerges:

\begin{itemize}
	\item The \textbf{Prosocial–Empathic} ecology displays the \emph{most pronounced descriptive attenuation} under $\mathscr{R}$.
	\item The \textbf{Analytical–Structured} ecology shows \emph{minimal descriptive difference}.
	\item The \textbf{Emotionally Reactive} ecology exhibits \emph{variable} sensitivity, consistent with its affective volatility.
\end{itemize}

\noindent
This pattern demonstrates that the influence of robotic presence does not operate through a uniform causal channel. Instead, its impact is \emph{contingently instantiated through latent cognitive–affective regimes}. The evaluative transformation $f(\alpha_E, \beta_C, \gamma_R)$ is modulated by the internal organisation of $\beta_C$, not merely shifted by $\gamma_R$.

\medskip

% --------------------------------------------------------------
\subsubsection*{Connection to Floridi’s Levels of Abstraction}

\noindent
These ecologies may be understood as corresponding to distinct operative Levels of Abstraction:

\begin{itemize}
	\item The \textbf{Prosocial–Empathic} ecology foregrounds affective salience.
	\item The \textbf{Analytical–Structured} ecology foregrounds structural clarity.
	\item The \textbf{Emotionally Reactive} ecology foregrounds affective variability.
\end{itemize}

\noindent
Accordingly, $\gamma_R$ perturbs different informational channels depending on the ecology through which moral cues are interpreted.

\medskip

% --------------------------------------------------------------
\subsubsection*{Conceptual Conclusion}

\begin{center}
	\begin{tcolorbox}[colback=white,colframe=black!60,
		title=Conclusion: Trait-Contingent Structure of Moral Perturbation]
		\label{conc:clustered_moral_refraction_revised}
		The attenuation associated with robotic co-presence is not globally uniform. It emerges from contingent interactions between the synthetic presence $\gamma_R$ and the latent cognitive–affective ecologies encoded in $\beta_C$. These ecologies refract the evaluative transformation from moral salience to action, producing descriptively stronger perturbation in empathically oriented profiles, weaker effects in analytically oriented profiles, and variable responses in affectively reactive configurations. In informational terms, $\gamma_R$ interacts with participants at different operative Levels of Abstraction, generating heterogeneous moral responses across these latent evaluative architectures.
	\end{tcolorbox}
\end{center}

\noindent
This structural interpretation provides the necessary grounding for the next analytical step. The forthcoming regression and Bayesian models formally examine whether these ecology-specific patterns persist under inferential scrutiny, thereby testing how $\beta_C$ modulates the evaluative function $f(\alpha_E, \beta_C, \gamma_R)$ within a principled statistical framework.





\subsection{Cluster-Specific Regression Analysis of Condition Effects}
\label{subsec:cluster_regression}

\noindent
The latent dispositional clusters identified in the previous subsection provide a structured basis for examining whether the behavioural effect of robotic co-presence (\(\gamma_R\)) varies across different cognitive--affective regimes. To assess this possibility, we estimated a simple linear regression within each cluster of the form:
\[
\text{donation} = \beta_0 + \beta_1 \cdot \text{condition}_{\text{Robot}} + \varepsilon ,
\]
where \(\beta_1\) quantifies the within-cluster contrast between Control and Robot conditions. These stratified regressions serve as **local directional estimates**, establishing whether any cluster exhibits a recognisably stronger attenuation pattern prior to introducing interaction terms or hierarchical Bayesian pooling.

\medskip

\noindent
A descriptively uneven pattern emerges across clusters. In the cluster characterised by higher empathizing and sociability (Cluster~1), the estimated coefficient for the Robot condition is negative and comparatively large in magnitude relative to the other clusters (\(\beta=-1.33\)), though still uncertain given the small within-cluster sample size and the fact that the 95\% interval includes zero (\(p=.091\), \(R^2=0.087\)). This estimate suggests that the directional attenuation observed at the aggregate level may be disproportionately expressed in this subset of participants.

\medskip

\noindent
By contrast, the affectively variable (\textit{Emotionally Reactive}) cluster (Cluster~0) exhibits a coefficient near zero (\(p>.70\)), and the analytically structured (Cluster~2) regime shows only a modest, non-significant negative coefficient (\(\beta=-0.28\), \(p>.70\)). In both cases the estimates are small, and the associated intervals indicate no reliable deviation between conditions. Taken together, these results imply that the aggregate attenuation documented earlier is not homogeneously distributed across dispositional space.

\medskip

\noindent
It is important to emphasise two methodological clarifications.  
First, these regressions treat cluster assignments as fixed labels. They therefore do not incorporate uncertainty in cluster membership or hierarchical pooling across clusters. Both limitations are addressed explicitly in the **Bayesian modelling framework** introduced in the next subsection, which relaxes linearity assumptions, models bounded and zero-inflated outcomes, and accounts for varying uncertainty across clusters.  
Second, an omnibus condition \(\times\) cluster interaction model is presented later in the analytical pipeline. The stratified regressions provided here serve a narrower epistemic function: they establish **local effect direction** prior to modelling global interaction structure.

\medskip

\noindent
Finally, although the donation data are bounded and zero-inflated, we employ ordinary least squares at this stage to provide interpretable contrasts within a familiar parametric structure. The subsequent Bayesian analyses incorporate appropriate distributional assumptions and therefore supersede these exploratory linear models.

\medskip

\begin{figure}[H]
	\centering
	\includegraphics[width=0.75\linewidth]{new_plots/cluster_regression_coefficients.png}
	\caption{Regression coefficients (with 95\% confidence intervals) for the Robot condition estimated separately within each latent personality cluster. Cluster~1 shows a larger negative coefficient relative to the other clusters, though uncertainty remains high due to small within-cluster sample sizes. Clusters~0 and 2 exhibit coefficients near zero. These estimates provide local directional contrasts prior to interaction and Bayesian modelling.}
	\label{fig:cluster-regression}
\end{figure}

\medskip

\noindent
The estimated differences can be summarised at the level of expected evaluative output. Let \(f(\cdot)\) denote the behavioural transformation introduced earlier. For each cluster \(k\),
\[
\mathbb{E}\big[f(\Sigma \cup \mathscr{R})\big]_{k}
\quad \text{vs.} \quad
\mathbb{E}\big[f(\Sigma)\big]_{k}
\]
captures the expected donation under Robot and Control conditions respectively. The empirical pattern may be expressed as:

\begin{itemize}
	\item \textbf{Cluster 0 (Emotionally Reactive):}  
	\(\mathbb{E}[f(\Sigma \cup \mathscr{R})]_{0} \approx \mathbb{E}[f(\Sigma)]_{0}\)  
	(no detectable within-cluster difference).
	
	\item \textbf{Cluster 1 (Prosocial–Empathic):}  
	\(\mathbb{E}[f(\Sigma \cup \mathscr{R})]_{1} < \mathbb{E}[f(\Sigma)]_{1}\)  
	(largest negative contrast, though interval includes zero).
	
	\item \textbf{Cluster 2 (Analytical–Structured):}  
	\(\mathbb{E}[f(\Sigma \cup \mathscr{R})]_{2} < \mathbb{E}[f(\Sigma)]_{2}\)  
	(small, non-significant difference).
\end{itemize}

\medskip

\noindent
These expressions simply restate, in the language of expected values, the directional information contained in the regression coefficients. They do not imply deterministic effects or global causal claims. Instead, they highlight that: 



\bigskip
\noindent
\begin{center}
	\begin{leftbar}
		\textit{the condition effect is not uniform across latent dispositional regimes, motivating a shift to modelling frameworks that can formally represent uncertainty, zero-inflation, and interaction structure.}
	\end{leftbar}
\end{center}

\bigskip
\noindent

\medskip

\noindent
The next subsection therefore introduces a Bayesian estimation approach, designed to assess whether the patterns observed here persist when distributional assumptions are relaxed and when uncertainty is explicitly modelled at the level of both clusters and individual parameters.


\subsection{Bayesian Estimation and the Representation of Epistemic Gradients}
\label{subsec:bayesian_estimation}

\noindent
The cluster–specific regressions established that condition effects vary directionally across latent dispositional regimes, but they also highlighted the limitations of ordinary least squares in a bounded, zero-inflated dataset of modest size. Donation amounts exhibit asymmetry, mass at zero, and cluster-dependent variability; moreover, stratified regressions treat cluster membership as fixed and do not pool information across groups. A more flexible inferential framework is therefore required—one capable of representing uncertainty as a structured epistemic property rather than as residual error.

\medskip

\noindent
\textbf{Motivation for a Bayesian approach.}  
Three considerations motivate a transition to Bayesian estimation at this stage:

\begin{enumerate}
	\item \textbf{Sensitivity to subtle effects in modest samples.}  
	Frequentist tests collapse subtle behavioural tendencies into binary outcomes. Bayesian methods provide graded estimates of effect magnitude and uncertainty, which are essential in a study concerned with delicate perturbations of evaluative processing.
	
	\item \textbf{Hierarchical structure in the data.}  
	Condition effects \((\gamma_R)\) interact with latent dispositional regimes \((\beta_C)\). A Bayesian hierarchical model naturally incorporates this structure via partial pooling.
	
	\item \textbf{Conceptual alignment with the evaluative framework.}  
	If robotic presence exerts a refractive, context-dependent influence, then the inferential representation of this influence should itself be graded and continuous. Bayesian inference provides this representational form.
\end{enumerate}

\medskip

\noindent
\textbf{Model structure.}  
A hierarchical Bayesian model was specified in which:

\begin{itemize}
	\item donation amount was the outcome variable (after mild variance-stabilising transformation to accommodate zero inflation),
	\item experimental condition was the primary predictor,
	\item cluster membership contributed varying intercepts and varying slopes,
	\item weakly informative priors regularised estimates while allowing the data to drive posterior shape.
\end{itemize}

\noindent
The likelihood was implemented using a Student-\(t\) distribution, which is robust to skew, heavy tails, and zero-inflated behaviour—a pragmatic solution that avoids imposing unrealistic Gaussian assumptions while maintaining computational stability.

\medskip

\noindent
\textbf{Posterior estimation.}  
The posterior distribution for the \emph{modelled} donation difference (\texttt{Control – Robot}) shows a central tendency of approximately £0.70, with a 95\% credible interval ranging from about –£1.75 to £0.30. Although the interval includes zero, its mass is asymmetrically concentrated toward positive values, indicating \textit{directional probabilistic evidence} for attenuation under robotic co-presence. Rather than yielding a binary verdict, the posterior encodes a structured probability over plausible effect magnitudes.

\begin{figure}[H]
	\centering
	\includegraphics[width=0.75\linewidth]{new_plots/posterior_donation_difference.png}
	\caption{Posterior distribution of the modelled donation difference between conditions. The density is skewed toward positive values (greater expected donations in the Control condition), providing directional probabilistic evidence for attenuation under robotic co-presence. The dashed line marks the point of no effect.}
	\label{fig:posterior-difference}
\end{figure}

\medskip

\noindent
\textbf{Interpretive value of the Bayesian framework.}  
The Bayesian posterior advances the methodological arc of the chapter in three ways:

\begin{enumerate}
	\item \textbf{It treats uncertainty as epistemic structure.}  
	Rather than compressing uncertainty into a single threshold, the posterior renders it as a gradient reflecting the fine-grained ambiguity intrinsic to morally loaded decisions in minimally interactive environments.
	
	\item \textbf{It integrates hierarchical heterogeneity.}  
	Partial pooling allows condition effects to vary by cluster while borrowing strength across the population. This avoids overfitting in smaller clusters and respects the structural complexity of the latent evaluative regimes.
	
	\item \textbf{It offers a representational analogue of interpretive indeterminacy.}  
	The moral perturbation introduced by NAO operates amid ontological ambiguity; the Bayesian posterior provides a natural representational analogue of this indeterminacy, modelling moral displacement not as a discrete shift but as a probabilistic modulation.
\end{enumerate}

\medskip

\noindent
\textbf{Connection to Floridi’s Levels of Abstraction.}  
Within the LoA framework, agents interpret synthetic entities through informational filters that shape what counts as morally salient. Because NAO’s presence introduces indeterminacy in these filters, the inferential system used to model its effect should preserve—rather than collapse—that indeterminacy. The posterior distribution does precisely this: it expresses the impact of $\gamma_R$ as a graded epistemic field, mirroring the cognitive state of an agent responding to ambiguous moral cues.

\medskip

\nextdiv
\begin{center}
	\begin{tcolorbox}[colback=white,colframe=black!60,
		title=Conclusion: Bayesian Representation of Moral Perturbation]
		\label{conc:bayesian_epistemic_gradient}
		Bayesian estimation shows that robotic co-presence yields a probabilistic attenuation of prosocial donation rather than a discrete behavioural shift. The posterior distribution expresses directional evidence for reduced donation in the Robot condition while fully representing the uncertainty expected for subtle, context-dependent perturbations of moral salience. This graded inferential form is consistent with the chapter’s evaluative framework: synthetic presence reshapes the topology of moral evaluation in a continuous rather than binary manner.
	\end{tcolorbox}
\end{center}

\medskip

\noindent
With this Bayesian model, the inferential sequence of the Experimental Methods chapter reaches completion. The next chapter synthesises these findings to articulate their broader philosophical and normative significance.

\subsubsection{Epistemic Interpretation of the Bayesian Results}
\label{subsubsec:bayesian_interpretation}

\noindent
The Bayesian model developed above enriches the inferential structure of this chapter by representing uncertainty as an explicit epistemic quantity rather than as a residual error term. This shift is methodologically appropriate for the present design, but also conceptually aligned with the chapter’s broader focus on graded perturbations of evaluative structure.

\medskip
\noindent
Unlike frequentist procedures that partition outcomes into ``significant’’ and ``non-significant’’ categories, the posterior distribution in Figure~\ref{fig:posterior-difference} expresses a \emph{graded representation of evidential support for differences in donation across conditions}. The posterior for the modelled donation difference (\texttt{Control – Robot}) displays a central tendency near £0.70, but with a wide credible interval spanning mildly positive and negative values. The posterior mass is asymmetrically concentrated toward higher donations in the Control condition, providing \emph{directional probabilistic evidence} for attenuation under robotic co-presence—while making the uncertainty surrounding this effect fully transparent.

\medskip
\noindent
In relation to earlier analyses, the Bayesian posterior does not ``rescue’’ non-significant frequentist tests; rather, Bayesian inference \emph{frames the question differently}, updating the plausibility of attenuation effects under explicit modelling of uncertainty, heterogeneity, and zero inflation. Frequentist tests ask whether the data cross a threshold under idealised distributional assumptions; the Bayesian model asks how the data shift our degree of belief in an attenuation effect. These perspectives are epistemically distinct yet empirically compatible, and their convergence on the same directional trend provides a robust evidential basis for this chapter’s claims.

\medskip
\noindent
This Bayesian approach is especially appropriate for the present study for two reasons. First, the perturbation introduced by $\mathscr{R}$ is theorised to be subtle, context-dependent, and heterogeneously expressed across participants—properties that hierarchical Bayesian models are designed to represent. Second, the latent dispositional clusters identified earlier generate structured variability that partial-pooling models can incorporate naturally. In this way, Bayesian posteriors provide a \emph{natural representational analogue} of the interpretive indeterminacy through which agents register moral salience under ambiguous conditions.

\begin{center}
	\begin{tcolorbox}[colback=white,colframe=black!60,
		title=Conclusion: Gradient of the Impact of Moral Refraction]
		\noindent
		The Bayesian analysis supports a cautiously framed but epistemically credible claim: attenuation of prosocial donation under robotic co-presence is \emph{probabilistically more likely than not}, with directional support emerging despite substantial uncertainty. This effect is therefore best understood not as a binary shift but as a graded modulation of the evaluative transformation through which moral salience becomes action.
	\end{tcolorbox}
\end{center}

\noindent
Taken together, the Bayesian results complete the inferential arc of this chapter. The behavioural attenuation, the latent cluster structure, and the posterior’s graded evidential pattern converge on a coherent empirical picture: robotic co-presence subtly and heterogeneously modulates the evaluative mapping from morally salient cues to prosocial behaviour.

\noindent
The next chapter develops the corresponding theoretical interpretation—particularly within the intuitionist tradition in moral psychology, the Watching-Eye literature, and broader debates in Social Signal Processing, Affective Computing, and Machine Ethics, where context-modulated salience and perceptual framing play a central conceptual role.


\subsection{Closing Reflection: How Synthetic Presence Reconfigures the Moral Field}

\noindent
When we look back across the full analytical arc of this chapter—from raw behavioural contrasts to hierarchical Bayesian estimation—a single idea comes into focus. Moral behaviour does not unfold in a vacuum. It grows out of what we notice first, how we feel the atmosphere of a situation, what we treat as relevant long before we begin to reason through it. Our decisions emerge from the texture of the environment and from the quiet interplay between our own dispositional architecture and the signals around us.

\medskip
\noindent
What this experiment shows is that the presence of a humanoid robot—even one that neither speaks nor evaluates us—can reshape that texture. Not dramatically, not uniformly, but measurably. The charity poster, with its image of a child in need, is normally a powerful intuitive cue: it draws our attention, evokes concern, and nudges us toward prosocial action before any explicit deliberation takes hold. Yet when NAO is in the room, this intuitive channel is no longer clean. The robot becomes a second centre of salience—an object that feels social enough to matter, but not social enough to interpret. Some participants fold this ambiguity into their evaluative process; others simply disregard it. And those differences are structured, not random.

\medskip
\noindent
At the aggregate level, this manifests as a modest reduction in donation under robotic co-presence. At the individual level, the posterior distribution shows that this attenuation is \emph{more likely than not}, though embedded in genuine uncertainty. And at the dispositional level, our latent trait analysis reveals a \textbf{clear descriptive pattern}: those whose moral lives are primarily guided by warmth, sociability, and empathic resonance are the very ones most affected by NAO’s ambiguous presence. For them, the intuitive pull of the poster is partially displaced; for others, the robot barely registers.

\medskip
\noindent
This is not the kind of result that lends itself to simple causal slogans. It is not that “robots reduce generosity” or that “some personalities are immune.” The structure is subtler. What we see is a redistribution of intuitive salience: a \textbf{subtle bending of the moral field} that makes certain cues lighter, others heavier, and some simply harder to parse. NAO does not instruct anyone to act differently, nor does it hold a moral stance. Instead, it alters the perceptual scaffolding through which moral meaning normally flows. The change is quiet, almost atmospheric—and that is precisely why it matters.

\medskip
\noindent
From a methodological standpoint, the chapter demonstrates that such subtle effects can be measured, modelled, and formalised. The combination of frequentist contrasts, latent trait clustering, and Bayesian estimation provides a coherent and discriminating toolset for analysing how artificial systems modulate human moral behaviour. The topological language developed earlier in the thesis—mapping moral salience as a field, evaluative processes as trajectories, and synthetic presences as local perturbations—finds empirical grounding here. What the data offer is not proof of a grand theory, but a carefully bounded demonstration: when the informational structure of a moral environment is altered, even slightly, the intuitive pathways that guide behaviour can shift.

\medskip
\noindent
And this, ultimately, is the bridge to the conceptual questions that follow. If moral action is so finely attuned to environmental cues—if it responds to shifts in atmosphere, presence, and perceived social relevance—then the broader ethical landscape of human--machine coexistence cannot be reduced to internal principles encoded in artificial agents. It must be understood in terms of \emph{how machines participate in the environments within which our intuitions take shape}. Before we can talk about alignment, responsibility, or artificial moral competence, we must first understand how artificial systems already influence our evaluative architecture simply by being there.

\medskip
\noindent
In this sense, the chapter closes not with a resolution, but with a trajectory. We have established that synthetic presence can deform the moral field in ways that are modest, structured, and psychologically contingent. The next chapter asks what this means for the stories we tell about moral machines, for the theories we use to explain moral behaviour, and for the frameworks we rely on when designing artificial systems that will inhabit our social and normative spaces. If the intuitive foundations of moral life are as malleable as these findings suggest, then the ethical questions surrounding artificial agents begin long before those agents act. They begin with how they appear, how they are perceived, and how their presence reshapes the quiet, pre-reflective work from which our moral decisions grow.
