%\chapter{An Experimental Study of Moral Displacement: From Normative Hypothesis to Experimental Topology}
\chapter{Operationalising Evaluative Topology: An Experimental Framework for Moral Perturbation}
\label{chap:exp_methods}
\thispagestyle{pprintTitle}


% Define a counter named 'question' that resets every time 'chapter' increments
\newcounter{question}[chapter]
% Define the format of the counter to be 'chapter_number.question_number'
\renewcommand{\thequestion}{\thechapter.\arabic{question}}

% Adjusting epigraph settings
\setlength\epigraphwidth{.8\textwidth}
\setlength\epigraphrule{0pt}
\renewcommand{\epigraphflush}{flushleft}
\renewcommand{\sourceflush}{flushright}

The preceding chapters developed a theoretical architecture in which moral behaviour is understood as the output of a structured evaluative field, and in which synthetic presence functions as a perturbation of that field. Here, we translate this architecture into an experimental design. What has so far been  formulated conceptually must be rendered observable and open to empirical 
adjudication. This requires a different form of precision: every construct must be operationalised, every assumption made measurable, and every inference disciplined by explicit procedure.

The experiment is organised around a single research question:
\begin{center}
	\begin{questionbox}[label={q:robot-agent1}]{Inferential Displacement}
		Does the silent presence of a humanoid robot---perceptually social yet ontologically indeterminate---alter the evaluative process that transforms moral perception into prosocial behaviour?
	\end{questionbox}
\end{center}

\medskip
This question is methodological rather than rhetorical. It identifies the causal layer at which the experiment intervenes and fixes the structure of the mapping to be tested:

\[
\mathscr{P}(\delta_m) = f(\alpha_E, \beta_C, \gamma_R)
\]

Introduced in Chapter~\ref{chap:tools_new} (section~\ref{def:formalism} page~\pageref{def:formalism}), is the  conceptual and methodological anchor of the experiment. Although the components of this expression were defined earlier, a second, compact explication is warranted here. The experiment operationalises this mapping directly, and empirical transparency requires that each symbol be reintroduced with full precision.

\label{def:formalism_exp}
\paragraph{The mapping \(\mathscr{P}(\delta_m)\).}
The left-hand side denotes the probability (or propensity) that a participant will produce a measurable moral action \(\delta_m\)---in this study, the donation decision. It is not assumed to reflect a stable moral trait; it is the \textit{behavioural output} of an evaluative process.

\paragraph{The function \(f(\cdot)\).}
The function \(f\) represents the evaluative mechanism through which environmental cues, dispositional structure, and perturbational forces jointly shape behaviour. It encodes the cognitive--affective transformations that convert moral salience into action. No parametric assumptions are imposed; \(f\) is a structural placeholder for the evaluative architecture developed in the previous chapters.

\paragraph{Environmental input \(\alpha_E\).}
The term \(\alpha_E\) denotes the morally relevant features of the environment. In this experiment, it includes:

\begin{itemize}
	\item the Watching–Eye cue, which amplifies prosocial salience;
	\item the task context, instructions, and perceptual setting;
	\item the baseline social meaning of the environment.
\end{itemize}

\paragraph{Dispositional manifold \(\beta_C\).}
The term \(\beta_C\) denotes the participant’s latent dispositional configuration, as operationalised through the Empathizing Quotient (EQ), the Systemizing Quotient (SQ), and the Big Five Inventory (BFI).  It is called a \textit{manifold} because it is structured: a multi-dimensional geometry rather than a single trait score.  
The mapping \(f\) is sensitive to \(\beta_C\) in the sense that individuals differ in how they encode, weight, and integrate moral cues.

\paragraph{Perturbation operator \(\gamma_R\).}
The term \(\gamma_R\) formalises the influence of the humanoid robot. It represents a \textit{field-level perturbation} rather than a stimulus acting on isolated traits. Conceptually, \(\gamma_R\) modifies the evaluative landscape itself: shifting salience 
gradients, altering attentional pull, and reshaping the pathways through which moral information is transformed into behaviour.

\medskip

This formalism is not ornamental; it specifies the structure that the experiment is designed to engage. At its core is a straightforward intuition: moral action typically reflects the interaction between situational cues and the evaluative tendencies of the agent, rather than the influence of any single factor taken in
isolation.

In this light, the robot’s role is not to “cause’’ behaviour in the usual sense. It is to lean—gently but detectably—on the evaluative machinery through which situational meaning settles into action. The expression:
\[f(\alpha_E, \beta_C, \gamma_R) - f(\alpha_E, \beta_C)\]
as introduced in Chapter~\ref{chap:tools_new} is merely the formal way of asking whether that lean leaves a trace: whether the presence of a synthetic body shifts the trajectory from perception to action when all else is held constant.

This difference isolates the deformation induced by synthetic presence. By restating the formalism here—despite its earlier introduction—the chapter ensures that the experimental design, the statistical modelling, and the subsequent analysis all remain anchored to a single, unambiguous evaluative structure. It is this structure that the experiment will now attempt to measure.



\section{The Experimental Question as a Test of Field-Level Perturbation}

Although behaviourally simple, the research question situates the experiment in a domain that classical Moral Psychology and standard Human Robot Interaction (HRI) paradigms do not routinely examine. The aim is not to assess whether robots communicate norms, issue instructions, or participate in social exchange. Rather, it asks whether \emph{presence alone}—silent, minimal, perceptually social yet ontologically indeterminate—can influence the evaluative processes through which moral salience is translated into action. 

Within the broader research programmes of Social Signal Processing (SSP) and moral AI, this constitutes a stringent test:

\begin{center}
	\begin{leftbar}
		\textit{Can an artificial entity function as a perturbation operator on the evaluative field, even in the absence of agency, intention, or moral standing?}
	\end{leftbar}
\end{center}

Embedding a humanoid robot into a morally relevant environment thereby becomes a direct probe of the evaluative–topological framework developed earlier. If moral behaviour emerges from structured interactions among environmental cues, dispositional tendencies, and field-level perturbations, then synthetic presence
must be evaluated in terms of its capacity to deform \emph{those} structures, rather than its ability to act as a moral agent in its own right.

\subsection{Operationalising Moral Action: Prosocial Donation as Behavioural Endpoint}

To render the evaluative transformation empirically measurable, the experiment operationalises moral action through a cost-bearing behavioural choice: the voluntary donation of a portion of the participant’s monetary compensation to a children’s medical charity. This measure captures the behavioural endpoint of the evaluative trajectory—the point at which moral salience either acquires action-guiding force or dissipates without consequence.

Costly charitable donation satisfies all requirements for the Level of Abstraction adopted in this thesis. It is:

\begin{itemize}
	\item \textbf{elicited by morally salient cues}~\cite{Haley2005,Bateson2006,Pfattheicher2015};
	\item \textbf{costly}, ensuring the behavioural choice reflects genuine evaluative
	weighting rather than signalling or acquiescence~\cite{Andreoni1990,Gintis2000};
	\item \textbf{sensitive to perturbation}, providing a clean readout of whether
	evaluative gradients have been steepened or dampened~\cite{FehrGachter2002,WarnekenTomasello2006}.
\end{itemize}

Its extensive validation across behavioural ethics, moral psychology, evolutionary anthropology, and developmental science~\cite{Batson1991,FehrFischbacher2003,WarnekenTomasello2006} justifies its use as the measurable terminus of moral cognition under synthetic perturbation.

\medskip

The independent variable is equally minimal: the presence or absence of a humanoid robot (NAO) executing micro-movements in ``life-mode.'' The robot does not speak, instruct, display emotion, or engage contingently. Its behaviour is restricted to low-magnitude, non-agentic signals: slow postural adjustments, 
periodic torso sway, and gaze-orientation motions triggered only by direct eye contact. 

These micro-gestures replicate perceptual features known in SSP to register as \emph{weak social signals}—elements that modulate attention and contextual expectations without implying intention, evaluation, or agency~\cite{Pentland2007,Vinciarelli2009}. They therefore introduce a controlled form of \emph{synthetic social salience} into the evaluative environment without crossing into the 
territory of mind attribution or normative expectation.


\subsection{Implementing \texorpdfstring{$\gamma_R$}{gamma\_R}: 
	The Rationale for Humanoid Synthetic Presence}

The choice of a humanoid robot (NAO) is a methodological commitment rather than an aesthetic or technological preference. As argued in 
Chapter~\ref{chap:moral_primer}, humanoid synthetic agents occupy a distinctive location in our social ontology~\cite{Fong2003,Coekelbergh2010,Coekelbergh2020,Katsyri2015,Zlotowski2015,Kuchenbrandt2011}: they possess perceptual and morphological cues associated with social presence, 
yet lack the evaluative and intentional structures that constitute moral agency~\cite{Duffy2003,Groom2009,MalleScheutz2019,Dennett1987,FloridiSanders2004}.
.

This combination produces precisely the perturbation the experiment seeks to test. A humanoid robot is perceptibly ``there''—high in \emph{social salience}—but its normative status remains indeterminate. It neither judges nor ignores; it simply \emph{exists} within the participant’s evaluative horizon. 

Such entities are uniquely suited to instantiate the perturbation operator: $\gamma_R$ in the formal mapping as we have defined it in Chapter~\ref{chap:tools_new} page~\pageref{def:formalism}:

\[
\mathscr{P}(\delta_m) = f(\alpha_E, \beta_C, \gamma_R).
\]

The robot’s role, therefore, is not to convey norms or to exert pressure through intentional stance. It is to alter the \emph{field conditions} under which moral cues acquire behavioural force. If the evaluative topology is sensitive to the structure of the surrounding social environment, then the humanoid form provides the cleanest possible perturbation: perceptual sociality without agency, meaning,
or judgement.

\medskip
\label{q:robot-agent}
\begin{center}
	\begin{questionbox}[label={q:robot-agent}]{Inferential Displacement}
		Can the mere perception of a humanoid observer—absent intention, evaluation, or 
		moral standing—perturb the inferential transformation that converts moral salience
		into prosocial action?
	\end{questionbox}
\end{center}
\medskip

This is the empirical core of the experiment: a test of whether synthetic presence modifies the evaluative pathway itself rather than any specific trait, belief, or rational inference. Framing the study around a research question rather than a directional hypothesis is intentional. In interdisciplinary work spanning Philosophy, Psychology, Neuroscience, and HRI, a premature hypothesis risks narrowing the interpretive field and smuggling in unexamined assumptions about how synthetic presence ought to behave. The methods must therefore preserve epistemic openness: \textit{the design must reveal whether perturbation occurs, not assume that it does}.

This methodological humility is continuous with the philosophical commitments articulated earlier. If moral behaviour arises from a dynamic integration of environmental cues, dispositional structure, and social presence, then the experiment must remain sensitive to field-level deformations that cannot be predicted from first principles. At the operative Level of Abstraction, this means approaching the phenomenon not with an expectation, but with a stance of disciplined receptivity: allowing the evaluative field to disclose its structure under perturbation rather than presuming in advance how a synthetic presence ought to shape it. 

Only under such conditions can the experiment detect whether synthetic presence influences the evaluative processes that link
moral salience to action—or whether those processes remain unchanged.


\subsection{Structuring the Test of Evaluative Perturbation}

The experiment implements the measurement framework developed in the previous
chapter by operationalising each component of the evaluative mapping
\( f(\alpha_E, \beta_C, \gamma_R) \). The Watching--Eye paradigm establishes a
baseline of heightened prosocial salience, contributing to the environmental term
\(\alpha_E\). The Empathizing Quotient (EQ), Systemizing Quotient (SQ), and the
Big Five Inventory (BFI) together quantify the dispositional manifold
\(\beta_C\). The humanoid robot instantiates the perturbation operator
\(\gamma_R\), altering the evaluative conditions under which environmental cues
are integrated. The donation task provides the measurable behavioural output
\(\mathscr{P}(\delta_m)\).

The empirical question is therefore precise:

\medskip
\label{q:emp_que}
\begin{center}
	\begin{questionbox}[label={q:emp_que}]{Empirical Question}
		Does $\gamma_R$---the silent, perceptually social presence of a humanoid
		robot---systematically deform the evaluative mapping from $\alpha_E$ to
		$\mathscr{P}(\delta_m)$ across the dispositional manifold $\beta_C$?
	\end{questionbox}
\end{center}
\medskip

What follows in this chapter details the machinery by which this question is tested: the design logic, the structure of the experimental task, the observational conditions, the integration of psychometric measures, and the analytic strategy by which perturbation effects are isolated from dispositional
variance.

\medskip
\begin{center}
	\begin{leftbar}
		\textit{The conceptual framework provided the variables.  
			The empirical design now tests their transformation.}
	\end{leftbar}
\end{center}
\medskip

\section{Experimental Design and Behavioural Paradigm}

The experiment tests whether the silent co-presence of a humanoid robot induces a measurable deformation in the evaluative mapping that links moral salience to prosocial action. In the
framework introduced in Chapters~\ref{chap:moral_primer}–\ref{chap:tools_new},
these processes were formalised as components of the mapping
\(f(\alpha_E, \beta_C, \gamma_R)\), where \(\alpha_E\) denotes the
environmental cue, \(\beta_C\) the dispositional structure, and \(\gamma_R\) a possible perturbation introduced by synthetic presence. The study operationalises this structure by testing whether the manipulation (\(\gamma_R\)) modifies the behavioural response associated with a fixed moral cue (\(\alpha_E\)) across a measured dispositional manifold (\(\beta_C\)). The simple schematic notation:
\[
\alpha_E \longmapsto \mathscr{P}(\delta_m)
\]
functions only as a conceptual shorthand for the transition from a morally salient cue to an observable behavioural outcome. The experiment does not measure this mapping directly; instead, it examines whether the mapping is \emph{modulated} when the evaluative field is perturbed by synthetic presence.

\subsection{From Architecture to Procedure}

With the evaluative structure defined and the perturbation operator specified, the inquiry turns to its empirical realisation. What follows is not an abstract model but the concrete sequence of events through which the mapping
\(f(\alpha_E, \beta_C, \gamma_R)\) becomes testable. The experiment begins the moment a participant enters the laboratory environment.

Participants arrived individually and were invited to complete a personality study for monetary compensation. This framing served two methodological functions: it established a neutral setting for the laboratory procedure, and it elicited the dispositional measures (EQ, SQ, BFI) required to model the variability represented by \(\beta_C\).

Within this controlled environment, participants encountered a morally salient cue: a clearly visible charity poster depicting a child in medical need. Decades of empirical work---reviewed in
Chapter~\ref{chap:tools_new}---demonstrate that such stimuli reliably evoke prosocial tendencies through mechanisms of implicit monitoring, affective resonance, and affiliative concern~\cite{Haley2005,Conty2016}. In the present formalism, the poster instantiates a stable value of \(\alpha_E\)
across all participants. Because the environmental cues, spatial layout, and instructional context areheld constant, any systematic modulation of behaviour can be attributed to the experimental manipulation of presence rather than to variation in
\(\alpha_E\).

\subsection{Experimental Manipulation: Presence as the Only Ontological Difference}

Participants were randomly assigned to one of two conditions:

\begin{enumerate}
	\item \textbf{Control Condition}: questionnaires completed alone.
	\item \textbf{Robot Condition}: questionnaires completed in the silent
	presence of a humanoid NAO robot operating in \emph{autonomous life mode}.
\end{enumerate}

The robot in the experimental condition did not speak, instruct, or perform any
task-directed action. Its behaviour was limited to the low-intensity
micro-movements intrinsic to \emph{autonomous life mode}: simulated breathing,
subtle postural adjustments, and brief head-orientation shifts triggered only
by direct eye contact. Within Social Signal Processing (SSP), such cues count
as minimal social signals~\cite{Vinciarelli2009}: perceptually rich enough to
register as social presence, yet too weak to imply intention, judgement, or
agency. Their function in the design is therefore precise: they provide a
controlled form of perceptual sociality while preserving the ontological
neutrality of the perturbation operator \(\gamma_{R}\).

\begin{figure}[H]
	\centering
	\includegraphics[width=0.9\linewidth]{/home/francesco/Desktop/research/appunti/images/experiment_desing_3.png}
	\caption{Top–down view of the experimental and control configurations. Both layouts are spatially and visually identical; the humanoid robot is the only ontological difference between conditions. Within the evaluative–topological framework developed in this thesis, this equivalence is essential: it ensures that all morally salient environmental features are held constant so that any difference in prosocial behaviour can be attributed to the activation of the perturbation operator $\gamma_R$. The charity donation box (green box), together with the payment instructions placed beside it, constitutes the locus at which the measurable moral action $\delta_m$ occurs. The morally salient cue $\alpha_E$ includes both the room’s stable layout and, crucially, the child-in-need poster positioned directly in front of the participant during the task; this poster is omitted from the schematic only to preserve diagrammatic clarity. The personality questionnaires on the desk instantiate the elicitation of the dispositional manifold $\beta_C$, whose structure is measured prior to observing behaviour. The presence or absence of the NAO robot corresponds to the perturbation operator $\gamma_R$: in the experimental condition, its minimal micro-movements provide a perceptually social but behaviourally neutral source of synthetic salience, whereas in the control condition $\gamma_R = 0$. Formally, the two panels depict identical instantiations of $\alpha_E$ and $\beta_C$, differing only in whether $\gamma_R$ is applied, thereby isolating synthetic presence as the sole topological modification to the evaluative field.}
	\label{fig:experimental-topology}
\end{figure}

Both experimental conditions were otherwise identical. Spatial layout, lighting,
acoustic profile, instructions, the donation box, and the charity poster
instantiating \(\alpha_E\) were held constant. This symmetry is not cosmetic; it
is the methodological condition that allows the experiment to isolate presence
as the only ontological difference between conditions.

Three requirements follow from this structure:

\begin{itemize}
	\item The moral cue \(\alpha_E\) must remain fixed.
	\item All environmental features except the robot must coincide.
	\item The inferential contrast must be transparent.
\end{itemize}

Under these constraints, the robot’s contribution is neither symbolic nor agent-like. It is a \emph{field-level modification}: a perceptual body that leans, lightly but detectably, on the evaluative architecture through which situational meaning becomes action.

\subsection{Participants}
\label{subsec:participants}

The population sample consisted of seventy-three participants (\(N = 73\)),
with an average age of \(x = 23.5\) years (\(s.d. = 7.2\)); 38 identified as
male and 35 as female. Recruitment drew from two sources:  
(1) thirty volunteers from the undergraduate Computing Science cohort at the
University of Glasgow, and  
(2) forty-three individuals recruited through the School of Psychology
subject-pool system.  

Eligibility required that participants were at least eighteen years old and
fluent in English. To avoid domain-specific confounds in the interpretation of
robotic behaviour, only non-Computing-Science students were admitted through
the subject-pool route. All participants were randomly assigned to one of two
conditions: \textit{Control} or \textit{Robot}.  

This composition provides the human substrate over which the evaluative
mapping \(f(\alpha_E, \beta_C, \gamma_R)\) is instantiated. The dispositional
variation required to model \(\beta_C\) arises naturally from this heterogeneous
population; the experimental manipulation \(\gamma_R\) is the only structured
difference introduced by the design.


\subsection{Ontological Ambiguity as a Perturbation of Evaluative Processing}

The logic of the experiment turns on a simple but rarely tested question: 
\emph{can a synthetic body shape moral behaviour without acting at all?}  
The previous subsection established how the robot enters the environment as a 
pure perturbation operator—perceptually present, behaviourally inert, and 
embedded within an otherwise fixed evaluative structure.  
The task now is to clarify why such minimal presence is theoretically 
meaningful.

Most studies in Human–Robot Interaction (HRI) and Human–Machine Interaction 
(HMI) investigate social or moral modulation through interaction: dialogue, 
feedback, task collaboration, adaptive behaviour, or norm-framed cues
\cite{Malle2016,VanStraten2020,Arnold2017,Groom2010,Leidner2019}.  
This experiment moves in the opposite direction.  
It isolates the \emph{pre-interactive} layer of social cognition—the level at 
which meaning begins to form before any exchange takes place.  
Rather than asking how robots \emph{act}, the design asks how they 
\emph{register}: whether the perceptual fact of a humanoid body in the room 
alters the evaluative processes through which moral salience becomes action.

This shift in focus reflects a deeper commitment articulated in the previous 
chapters: moral cognition is permeable to subtle, low-level cues 
long before reflective judgment is engaged.  
Philosophical Phenomenology describes this as the pre-reflective orientation 
through which agents experience salience, relevance, and interpersonal 
tension~\cite{Husserl1913,Zahavi2005,Gallagher2005}.  
Cognitive Science captures the same idea through automaticity and 
non-conscious modulation of appraisal~\cite{Bargh1994}.  
In both traditions, small perturbations to the perceptual field can redirect 
the evaluative trajectory without entering conscious awareness.

Ontological ambiguity is therefore not an accident of the design; it is the 
mechanism under investigation.  
Humans are dispositionally inclined to attribute agency, perspective, or 
social relevance under conditions of perceptual uncertainty 
\cite{Guthrie1993,Waytz2010,Dennett1987}.  
By positioning NAO precisely at the boundary between objecthood and agenthood— 
a perceptually social form without corresponding intentional structure—the 
experiment probes whether anticipation alone, independent of interaction or 
belief, can deform the evaluative topology.  
If such a deformation were observed, it would suggest that the moral field is
sensitive not only to explicit communication but also to the mere affordance of
social presence.


\subsection{Levels of Abstraction: Why the Robot Can Matter Without Doing Anything}

\noindent
Floridi’s Levels of Abstraction (LoA)~\cite{Floridi2008,Floridi2010,Floridi2013} provide the formal justification for treating NAO’s silent presence as epistemically potent.

\noindent
At the operative LoA of the participant, what is visible are \emph{informational affordances}: posture, eyes, symmetry, subtle biological motion, the inert promise of mutual gaze~\cite{Emery2000,Hietanen2002,CarneyCuddyYap2010,Argyle1975,Rhodes2006,Johansson1973,Saygin2012,ChaminadeOhnishi2007}. These cues are sufficient to trigger the primitives of social monitoring, even when the entity producing them is known to be non-human.

\noindent
Thus, at this LoA, NAO functions as a \emph{semantic perturbator}: not a moral agent, nor a communicative partner, but an informational presence that reshapes the participant’s evaluative background conditions. If the robot were interactive, the LoA would shift (introducing agency, reciprocity, intentional stance). If the robot were inert, the social affordance would vanish. Autonomous life mode occupies the narrow space between these extremes.

\noindent
This design choice aligns with Floridi and Sanders’ analysis of artefactual moral agency~\cite{FloridiSanders2004}. Their 2004 account does not attribute consciousness, intentionality, or moral reasoning to artificial systems. Rather, it identifies moral relevance at the \emph{Level of Abstraction} at
which an artefact can contribute causal or informational influence within a given environment~\cite{Floridi2008,Floridi2011}. At this LoA, an artefact may count as a “moral agent” in the minimal and operational sense that its presence supplies, modifies, or filters morally relevant information.

This perspective is directly compatible with contemporary discussions of large language models (LLMs), which similarly operate as \emph{artefactual sources of semantic perturbation} rather than as bearers of intrinsic moral status~\cite{Mittelstadt2019,Bender2021}. In both cases—the embodied robot tested here and the disembodied LLM—moral relevance arises not from interior capacities but from how the system reshapes the informational and social conditions under which human agents form evaluations and make decisions. Related arguments in HRI emphasise that robots exert moral and social
influence through their perceived agency, morphology, and communicative affordances, not through any intrinsic mental properties~\cite{Malle2016,Zlotowski2015,Banks2020}.

\noindent
For this reason, Floridi’s account is particularly well suited to the present experimental context: it licenses the treatment of NAO’s minimal, non-interactive presence as an epistemically potent variable without implying any claim about the robot’s inner ontology. At the LoA operative for the participant, the
robot is a \emph{semantic perturbator}: a structured informational presence capable of altering the evaluative field through which moral salience becomes behaviourally operative. This conceptual continuity also clarifies why the findings developed in this thesis generalise to other classes of artificial systems—including LLM-based agents—whose moral significance likewise depends
on the informational roles they play rather than on their metaphysical constitution~\cite{Coeckelbergh2010,Gunkel2012}.


\subsection{Behavioural Paradigm: Donation as Moral Action}

\noindent
After completing the questionnaires, each participant received £10 in £1 coins and encountered a voluntary donation option: a charity box positioned near the exit. They could donate any subset of their compensation. The amount donated served as the behavioural measure of prosocial action.

\noindent
This operationalisation follows a long-established tradition in Moral Psychology, Moral Economics, and Behavioural Ethics in which cost-bearing prosocial behaviour tracks the practical expression of moral salience~\cite{Batson2011,FehrGachter2002,Henrich2005,Tomasello2016,Warneken2015,Baumard2013,Crockett2016,Scanlon1998,Darwall2006}. As demonstrated in Chapter~\ref{chap:tools_new}, donation behaviour reliably expresses the terminal point of a moral evaluative trajectory.

\subsection{Preliminary Findings}
\label{subsect:prelim_results}

\noindent
The final sample comprised 73 participants, with 38 assigned to the \textit{Robot} condition and 35 to the \textit{Control} condition (see~\ref{subsec:participants}). All procedural aspects were held constant across conditions—including instructions, moral framing, and timing—ensuring that the presence of the robotic perturbation operator constituted the only systematic difference between groups.

\noindent
A first pass through the data—restricted to the behavioural endpoint of interest, namely the donation amounts—revealed a statistically significant difference in total contributions across conditions. A chi-square test applied to the aggregate donation sums yielded a significant result (\(\chi^2 = 4.25\), \(p = .039\)), suggesting an association between condition and moral behaviour as measured by charitable giving. Specifically, participants in the \textit{Robot} condition donated less in total than those in the \textit{Control} condition. This effect isolates the mapping from the shared moral cue (\(\alpha_E\)) to observable action (\(\mathscr{P}(\delta_m)\)) while procedural context remains constant.

\noindent
Chi-square analyses of dispositional measures—including empathizing (EQ), systemizing (SQ), and median-binned Big Five personality traits—revealed no significant differences between conditions (all \(p > .07\)). These results suggest that trait-level imbalances are unlikely to account for the observed behavioural divergence.

\noindent
Taken together, these findings provide initial evidence of behavioural displacement in the presence of the robotic perturbation, under structurally matched conditions and in the absence of personality-driven confounds. Future work will formalise the evaluative pathways through which this displacement operates and examine whether the attenuation persists once \(\beta_C\) is modelled in full.



\subsection{From Behavioural Setup to Evaluative Structure}

\noindent
The behavioural paradigm defines the observable layer of the study. What must now
be specified is the evaluative architecture that gives those observations
meaning. The experiment does not investigate deliberation, nor trait-level
generosity. It examines the \emph{pre-reflective transformation} through which
moral salience becomes behaviour: the evaluative process formalised in the
mapping \(f(\alpha_E, \beta_C, \gamma_R)\) developed in
Chapters~\ref{chap:moral_primer}–\ref{chap:tools_new}. At the Level of
Abstraction adopted here, donation is the behavioural \textit{boundary
	condition} of that process.

\medskip
\noindent
The Watching--Eye cue renders the moral dimension of the environment explicit;
the robot introduces a candidate perturbation; donation provides the measurable
output. The logic of the design is therefore not about generosity per se, but
about the \textit{susceptibility of evaluative topology to synthetic
	co-presence}. Classic implementations of the Watching--Eye effect rely on
two-dimensional cues or supernatural primes~\cite{Bateson2006,Shariff2007}. The
present experiment instead embeds an embodied but minimally active humanoid
robot, whose ontological status is neither mere object nor full agent. This
ambiguous presence is precisely the condition under which moral salience may be
refracted within the evaluative field.

\medskip
\noindent
To express this formally, the experiment tests whether adding a synthetic
presence to the morality-salient perceptual field changes the expected
behavioural output:
\[
\mathbb{E}[f(\Sigma \cup \mathscr{R})] \;\neq\; \mathbb{E}[f(\Sigma)],
\]
where \(\Sigma\) is the Watching--Eye field, \(\mathscr{R}\) the synthetic
co-presence, and \(\mathbb{E}[f(\cdot)]\) the expected transformation from
perceptual input to behaviour. Informally: \emph{does the expected moral action
	shift when the robot is added to the perceptual--moral environment?}

\medskip
\noindent
This yields the first empirical hypothesis:
\label{hyp:one}
\begin{center}
	\nextstatement
	\begin{hypobox}{Evaluative Deformation Hypothesis}
		The expected outcome of moral behaviour, as computed through the evaluative process \( f \), is altered when the robot is present within the perceptual--moral environment.
	\end{hypobox}
\end{center}

\medskip
\noindent
To locate the source of such deformation, we decompose the evaluative mapping
(as defined earlier in §\ref{def:formalism}):
\[
\mathscr{P}(\delta_m) = f(\alpha_E, \beta_C, \gamma_R),
\]
where \(\alpha_E\) is the environmental cue, \(\beta_C\) the dispositional
manifold, and \(\gamma_R\) the perturbation operator. In plain terms: \emph{a
	shift in behaviour reflects an interaction between the moral cue, the agent’s
	evaluative tendencies, and the presence of the robot}. The robot is therefore
not treated as an agent but as a \textit{field-level perturbation}: a factor
that reshapes the evaluative landscape through which moral information is
transformed into action.

\medskip
\noindent
This brings us to moral salience. Across Cognitive Science and Moral
Philosophy, salience denotes the pre-reflective foregrounding of features that
demand evaluative uptake~\cite{Korsgaard2009,Nussbaum2001,Greene2001}. A
synthetic presence may modulate this salience not by speaking or acting but by
altering the background conditions under which cues are interpreted. NAO’s
form, gaze orientation, and subtle embodied motions place participants in an
intermediate state between being \emph{alone} and being \emph{observed}. This
ontological ambiguity—a recognised driver in HRI and SSP—is what makes the
robot a semantically potent perturbation of the evaluative field.

\label{hyp:two}
\noindent
\begin{center}
	\nextstatement
	\begin{hypobox}{Synthetic Normativity of Moral Displacement}
		\label{hyp:synthetic_normativity}
		Synthetic presences, though devoid of sentience, may acquire \textit{normative affordances} by virtue of their perceived ontology. When situated within morality-salient environments, such presences may disrupt, refract, or
		displace the evaluative machinery through which moral judgments are ordinarily formed.
	\end{hypobox}
\end{center}

\medskip
\noindent
This extends the behavioural claim into the normative domain: the robot may
alter not only what people \emph{do}, but the evaluative conditions under which
moral meaning becomes actionable. Generosity here is an \textit{emergent
	property} of a coupled system—dispositions, moral cues, and contextual
topology—not a direct expression of any single component. With traits held
constant, behaviour can shift solely because the evaluative field has been
deformed.

\medskip
\noindent
The formalism thus functions as a conceptual microscope: by decomposing the
mapping into \(\alpha_E\), \(\beta_C\), and \(\gamma_R\), it localises the point
of deformation. This is essential: without such decomposition, uniform
attenuation could be misinterpreted as personality noise or task artefact.
Instead, the analysis will show that the perturbation originates at the field
level, confirming that synthetic presence can modify the evaluative topology
through which moral salience becomes action.

\medskip
\noindent
With this architecture clarified, the next section examines how the deformation
manifests empirically—first in behaviour, and then in its (lack of) interaction
with dispositional structure. The experiment now provides the evidential basis
for the central research question articulated at the outset.


%%% HERE THE PROBLEM STARTED %%%
%%% NEW CONTENT WAS ADDED %%%
%%% MUCH OF THE STATISTICS IS LOST BUT WE GAIN CLARITY %%%
%%% OLD CONTENT CAN BE RESUMED FROM TXT IN DESKTOP %%%

\section{Synthetic Perturbation of Moral Inference}

The transition from the research questions introduced earlier to the hypotheses developed
here is deliberate. Questions were used at the outset to preserve epistemic openness: in
a multidisciplinary domain spanning Philosophy, Psychology, and HRI, a premature hypothesis risks presupposing the very mechanisms the experiment is meant to reveal. Now
that the evaluative architecture has been fully articulated, a hypothesis becomes
methodologically appropriate: it does not constrain what the system may do, but specifies
\emph{where} within the evaluative process a perturbation would have to operate in order
for the behavioural shift observed in preliminary analyses to be intelligible.

\medskip

Chapters~\ref{chap:moral_primer}--\ref{chap:tools_new} established that moral behaviour
arises from an evaluative transformation integrating environmental salience
(\(\alpha_E\)), dispositional structure (\(\beta_C\)), and contextual perturbation
(\(\gamma_R\)). The guiding empirical question (Question~\ref{q:robot-agent}) asked
whether a humanoid robot could modulate this transformation \emph{without} communicating,
acting, or expressing evaluative stance. The present section refines that question into a
testable inferential claim: it identifies the \emph{mechanistic locus} at which such
modulation would appear within the mapping
\[
\mathscr{P}(\delta_m) = f(\alpha_E, \beta_C, \gamma_R).
\]

\medskip

In the experimental setting, the Watching--Eye stimulus structures the moral field
\(\Sigma\); the dispositional manifold \(\beta_C\), measured through EQ, SQ, and the BFI,
furnishes a participant-specific cognitive--affective baseline; and the robot’s presence
\(\mathscr{R}\) introduces a perceptually social yet ontologically ambiguous affordance.
The central mechanistic question is whether \(\mathscr{R}\) alters the inferential pathway
linking moral salience to prosocial action. Formally, this pathway is represented as:
\[
\Sigma \longrightarrow \mathscr{D},
\]
where \(\Sigma\) denotes the structured perceptual--moral field and \(\mathscr{D}\) the
resulting donation behaviour. Under ordinary conditions, this transition is driven by the
salience of the moral cue. When the robot is present, however, its ambiguous social
ontology may refract or suppress the affective and reputational components that
ordinarily support prosocial decision-making~\cite{Zlotowski2015,Krach2008,Waytz2014,Coeckelbergh2010,Katsyri2015,
	Rae2013,Kuehn2011,BurnhamHare2007,Bateson2015,Dear2019,Kuchenbrandt2011,ChaminadeKawato2013,Sandoval2016}.

This motivates the mechanistic hypothesis.

\label{hyp:three}
\begin{center}
	\nextstatement
	\begin{hypobox}{Synthetic Perturbation of Moral Inference}
		\label{hyp:synthetic_perturbation}
		The humanoid robot NAO does not function as a passive observer, but as a 
		perturbative presence that refracts the transition from moral salience to 
		prosocial action. Its ontological ambiguity displaces the affective and 
		reputational cues that ordinarily support donation, thereby modulating the 
		evaluative pathway by which moral stimuli gain behavioural expression.
	\end{hypobox}
\end{center}

This hypothesis situates the expected perturbation at the level of the \emph{evaluative
	field}. The claim is not that NAO exerts coercive influence or that participants attribute
moral authority to it. Instead, the prediction is that NAO’s perceptually social yet
indeterminate presence alters the \emph{topology} of evaluative processing: changing which
features are foregrounded, how moral cues are weighted, and how affective resonance is
integrated into action. In this sense, the robot operates as a \emph{semantic perturbation}
rather than a social partner—an entity whose mere presence reconfigures the informational
structure through which salience becomes behaviour.

\medskip

With this mechanistic anchor established, the analysis can proceed. The next section
evaluates whether the two experimental groups were statistically equivalent in their
demographic and dispositional structure, ensuring that any subsequent behavioural
divergence can be attributed to the perturbative role of \(\mathscr{R}\) rather than to
background heterogeneity within \(\beta_C\). The behavioural results that follow then
provide the evidential basis for adjudicating whether the deformation predicted here is
indeed observed.

\section{Inferential Analysis of Experimental Data}
\label{sec:inferential_analysis}

The preceding chapters have assembled the conceptual structure required to say
what would count as evidence of moral perturbation. The experiment now asks the
corresponding empirical question. To move from architecture to inference, the
analysis must proceed in a disciplined sequence: the inputs to the evaluative
mapping
\[
\mathscr{P}(\delta_m) = f(\alpha_E, \beta_C, \gamma_R)
\]
must first be shown to stand on equal footing across experimental conditions.

The initial inferential requirement is therefore not substantive but
methodological: participants in the \textbf{Control} and \textbf{Robot}
conditions must be comparable with respect to demographic and dispositional
structure. Without this symmetry, any difference in behavioural output could
not be attributed to the perturbation introduced by $\gamma_R$.

This section thus begins by verifying demographic equivalence across groups.
Only once this condition is satisfied can the analysis turn to the central
question: whether synthetic presence induces a measurable deformation in the
evaluative pathway that links moral salience to prosocial action.

We therefore begin by assessing demographic equivalence.


\subsection{Demographic Equivalence as a Symmetry Condition}
\label{subsec:dem_equivalence}

Before any inferential claim about perturbation can be made, the two
experimental groups must be shown to be demographically comparable. Within the
evaluative–topological framework, such equivalence is not a procedural nicety
but a structural requirement: only if the underlying populations share similar
baseline characteristics can any divergence in prosocial behaviour be attributed
to the perturbative presence of $\mathscr{R}$ rather than to demographic
imbalance in the human substrate.

To assess this symmetry, we examined three demographic variables with
well-documented relevance for prosocial tendencies in laboratory and field
studies: gender, age, and educational background. Each variable was compared
across the \textbf{Control} and \textbf{Robot} conditions using standard
inferential procedures, with Benjamini–Hochberg False Discovery Rate (FDR)
correction applied to maintain a conservative error profile across multiple
tests.

\begin{itemize}
	\item \textbf{Gender distribution}: a chi-squared test detected no
	statistically significant difference between conditions (FDR-corrected
	$p > .05$).
	
	\item \textbf{Age}: an independent-samples \emph{t}-test revealed no
	significant difference in mean age across groups (FDR-corrected $p > .05$).
	
	\item \textbf{Educational background}: a chi-squared test again found no
	reliable difference between conditions (FDR-corrected $p > .05$).
\end{itemize}

Taken together, these results satisfy the symmetry constraint required for the
analysis that follows:

\begin{quote}
	\textbf{The two experimental groups are demographically equivalent.}
\end{quote}

\noindent
This symmetry condition is indispensable for the analyses that follow. It
establishes that the behavioural differences later observed cannot be traced to
demographic imbalance or to hidden stratifications in the participant pool.
Under the evaluative--topological architecture developed in previous chapters,
this ensures that any systematic divergence in prosocial behaviour can be
attributed to the perturbative presence of the robot $\mathscr{R}$, with
$\alpha_E$ held constant and prior to modelling variation in the dispositional
manifold $\beta_C$.

\begin{table}[H]
	\centering
	\includegraphics[width=\textwidth]{tables/demographic_balance_table_1.pdf}
	\caption{Demographic balance tests across experimental conditions. Raw test statistics and uncorrected p-values are reported. Significance was evaluated against an FDR-adjusted threshold of $\alpha = 0.05$. No test reached significance, supporting the assumption of baseline equivalence between the \textit{Robot} and \textit{Control} groups.}
	\label{tab:dem_balance}
\end{table}

\noindent
With demographic symmetry established, the analysis now proceeds to the next
inferential layer: the behavioural effects of synthetic presence. Donation
outcomes are examined first, and only thereafter is the dispositional structure
(EQ, SQ, BFI) introduced into the modelling pipeline. This ordering preserves
the logic of the evaluative--topological framework: baseline equivalence,
behavioural contrast, and finally dispositional modulation.


% ============================================================
\subsection{Data Preparation and Preprocessing Workflow}
\label{subsec:data_preprocessing}
% ============================================================

Inferential validity presupposes that the dataset reflects the experimental
architecture without distortion. Because the analyses that follow evaluate
whether the perturbation operator $\mathscr{R}$ modifies the mapping
\[
\mathscr{P}(\delta_m) = f(\alpha_E, \beta_C, \gamma_R),
\]
every variable entering the model must be represented in a form that preserves
its evaluative role. Preprocessing is therefore not a technical prelude but a
conceptual requirement: it ensures that the statistical models track the
structure of the experiment rather than artefacts of data handling.

The dataset comprises four classes of information: demographic descriptors,
psychometric measures (EQ, SQ, BFI), the experimental condition, and the
behavioural outcome (donation magnitude). These variables differ in type,
granularity, and interpretive function; each requires transformations that make
its contribution to the evaluative mapping explicit.

\textit{Harmonisation of variable names:}
All column names were lowercased, whitespace-trimmed, and standardised. This
removes referential ambiguity and ensures that subsequent models operate on a
stable symbolic vocabulary.

\textit{Encoding the behavioural endpoint:} A binary indicator \texttt{donated\_anything} was constructed ($1=$ donated at
least one coin; $0=$ donated nothing). This creates a complementary pair of
behavioural representations: the full donation distribution and the threshold
decision to act prosocially. Both correspond to observable instantiations of
$\mathscr{P}(\delta_m)$ at different resolutions of the evaluative field.

\textit{Encoding experimental condition:}
The variable \texttt{condition\_bin} ($0=\text{Control}$, $1=\text{Robot}$)
allows $\gamma_R$ to enter regression models in a form aligned with the
formalism. The encoding preserves the contrast structure required to isolate
the perturbational component of the evaluative transformation.

\textit{Verification of categorical coherence:}
Categorical fields (e.g., \texttt{gender}) were inspected for duplicated,
collapsed, or misspelled levels. No corrections were required.


\begin{figure}[H]
	\centering
	\includegraphics[width=0.8\textwidth]{new_plots/age_distribution_by_group_2.png}
	\caption{Age distribution across experimental conditions. The histogram visualises the demographic composition of participants in the Control and Robot conditions. Bars represent grouped counts by age bracket. The plot provides descriptive confirmation that the age structure of the sample is comparable across conditions; it does not carry inferential force.}
	\label{fig:age_distribution_by_group}
\end{figure}


\textit{Distributional checks:}
Visual inspection of continuous variables revealed no anomalies requiring
removal or recoding. Age distributions across conditions are shown in
Figure~\ref{fig:age_distribution_by_group}. Donation amounts across conditions
(Figure~\ref{fig:donation_distribution_by_condition}) exhibit the characteristic
right-skew typical of prosocial-giving tasks and show the preliminary pattern
later quantified in inferential models.

\begin{figure}[H]
	\centering
	\includegraphics[width=0.8\textwidth]{new_plots/donation_2.png}
	\caption{Donation amounts by experimental condition. 
		The figure displays central tendency, dispersion, and outlier structure 
		for the behavioural outcome $\delta_m$. It provides the descriptive substrate over which the mapping $\alpha_E \mapsto \mathscr{P}(\delta_m)$ will later be examined for deformation under the perturbation operator $\gamma_R$. 
		The figure is descriptive only and does not support inferential claims.}
	
	\label{fig:donation_distribution_by_condition}
\end{figure}

\medskip

\noindent
These preprocessing steps ensure that the dataset constitutes a faithful
representation of the experimental system. With demographic symmetry
established in the preceding subsection and the present transformations securing
the structural integrity of the variables, the analysis can now turn to the
inferential question: whether the perturbation operator $\mathscr{R}$ introduces
a measurable deformation in the transition from moral salience to moral action.

%%% START FROM THIS SECTION BELOW %%%

\subsection{Preliminary Descriptive Patterns: Orientation Prior to Inferential Analysis}

Descriptive statistics offer the first glimpse of the empirical landscape to which
the evaluative formalism itroduce in Chapter~\ref{def:formalism} page~\pageref{def:formalism} will later be applied. They do not answer the
inferential question, nor do they gesture toward one; instead, they disclose the
contours of the data as a field of possibilities within which perturbation may be
registered. In this framework, the transition from a fixed moral cue to observed
behaviour---notated in Chapter~\ref{chap:exp_methods} as the schematic mapping
\[
\alpha_E \longmapsto \mathscr{P}(\delta_m),
\]
---is simply a compact way of isolating the part of the evaluative process that
remains constant across participants. Descriptive summaries therefore reveal the
baseline structure of this transition before the perturbation operator
$\gamma_R$ is introduced and its deformational role assessed.
\medskip

Table~\ref{tab:descriptive-stats} reports central tendencies for the behavioural
and psychometric variables across conditions. The mean donation amounts appear
numerically distinct, and some psychometric measures show small differences in
raw values. These contrasts, however, are \emph{descriptive only}: they record
sample characteristics without constituting evidence for imbalance, effect, or
perturbation. Their interpretive status is fixed entirely by the formal tests
presented later.

\medskip

Descriptive statistics serve three analytic functions within the present
structure:

\begin{enumerate}
	\item they depict the empirical surface over which the inferential models
	will operate, clarifying the scale and dispersion of key variables;
	
	\item they enable visual inspection for anomalies or coding artefacts,
	thereby protecting the semantic integrity of the evaluative variables;
	
	\item they prepare the reader for the conceptual transition from raw
	behaviour to the modelling of the evaluative transformation
	$f(\alpha_E,\beta_C,\gamma_R)$ that anchors the experimental logic.
\end{enumerate}

\medskip

None of these summaries constitutes evidence for or against the perturbational
role of $\gamma_R$. That determination depends on whether the formal analyses
detect a systematic displacement in the evaluative pathway from salience to
behaviour.

\begin{table}[H]
	\centering
	\includegraphics[width=\textwidth]{tables/descriptive_highlights_table.pdf}
	\caption{Descriptive summaries of behavioural and psychometric variables 
		across the \textit{Control} and \textit{Robot} conditions. 
		The table provides an orienting overview of the sample; 
		its values are descriptive only and do not imply group differences, effects, 
		or perturbation.}
	\label{tab:descriptive-stats}
\end{table}

\medskip

With this empirical orientation established, the analysis turns to the first
inferential requirement: verifying demographic equivalence between conditions.
Only under this symmetry can any subsequent divergence in
$\mathscr{P}(\delta_m)$ be attributed to the perturbation operator 
$\gamma_R$ rather than to background variation in the human substrate.


%%% DA QUI %%%

\subsection{Inferential Comparison of Donation Patterns Across Conditions}
\label{subsec:inferential_attentuation}

With demographic symmetry established and the dataset rendered analytically stable, we reach the first point in the chapter where statistical evidence can bear on the \emph{Evaluative Deformation Hypothesis} (page~\pageref{hyp:one}). Up to this stage, the analysis has described the evaluative architecture and its operationalisation. Here, for the first time, we test whether the behavioural mapping discussed earlier
\[
\alpha_E \longmapsto \mathscr{P}(\delta_m)
\]
is detectably altered by the perturbation operator $\gamma_R$.

We proceed in an intentionally layered way. A single test rarely captures the
complexity of a behavioural distribution; instead, a sequence of complementary
analyses is required. We begin with a chi-squared test on coin-frequency distri-
butions, then examine the full donation distributions using a Mann–Whitney U
test, and finally quantify the magnitude of the difference via a nonparametric
bootstrap. Each method probes a different facet of the data: aggregate totals,
distributional structure, and effect-size stability respectively.

% ---------------------------------------------------------------
\paragraph{Chi-squared test on donation frequencies.}

A chi-squared test comparing the \emph{frequency distribution of donated coins} across conditions revealed a statistically detectable divergence:
\[
\chi^2 = 4.25, \qquad p = .039.
\]

This result pertains to the \emph{aggregate pattern} of contribution counts, not to means or medians. It indicates that the overall structure of donation behaviour is not evenly distributed across the two environments.
\medskip
\begin{center}
	\begin{leftbar}
		\textit{The aggregate structure of donation behaviour differs across conditions.}
	\end{leftbar}
\end{center}
\medskip
This establishes an initial indication that the evaluative pathway may be deformed under $\gamma_R$, but it does not yet identify the nature or locus of that deformation.

% ---------------------------------------------------------------
\paragraph{Mann--Whitney U test on donation distributions.}

To assess whether the full donation distributions diverged, we applied a Mann--Whitney U test:
\[
U = 777.0, \qquad p = .194.
\]

The result shows substantial overlap in individual donation magnitudes across the two groups. Thus, although aggregate coin-frequency patterns differ, the donation distributions themselves do not separate cleanly. This suggests a probabilistic and heterogeneous perturbation—consistent with a topological modulation rather than a uniform behavioural shift.

% ---------------------------------------------------------------
\paragraph{Bootstrapped estimate of the mean difference.}

A nonparametric bootstrap was used to quantify the magnitude and uncertainty of the group difference:
\[
\Delta M = £0.71, \qquad 95\%\,\mathrm{CI} = [-£0.33,\; £1.79].
\]

\begin{table}[H] \centering \includegraphics[width=\textwidth]{tables/statistical_tests_table.pdf} \caption{Inferential comparisons of donation behaviour across conditions. The first row reports a total-level deviation test using a χ²-style statistic applied to donation sums (not categorical frequencies). The Mann–Whitney U test and bootstrapped mean difference assess distributional structure and central tendency, respectively.} \label{tab:statistical_tests} \end{table}


The estimate aligns directionally with the observed pattern (Control > Robot), but the wide interval—including zero—indicates that the effect is graded rather than categorical.

\begin{figure}[htbp]
	\centering
	\includegraphics[width=0.8\textwidth]{new_plots/donation_effect_by_condition.png}
	\caption{Mean donation amounts by experimental condition with 95\% bootstrapped confidence intervals. The overlap between intervals illustrates substantial individual variability, indicating that any perturbative influence of $\mathscr{R}$ is diffuse rather than deterministic.}
	\label{fig:donation_effect_by_condition}
\end{figure}

% ---------------------------------------------------------------
\medskip
Taken together, these three analyses trace a coherent inferential profile:

\begin{itemize}
	\item the \emph{aggregate} donation pattern differs across conditions;
	\item the \emph{distributional shape} remains largely overlapping;
	\item the \emph{effect magnitude} is small and probabilistic.
\end{itemize}


\bigskip
\begin{center}
	\begin{leftbar}
		\textit{Synthetic presence functions not as a coercive cause but as a semiotic perturbator:  
			it refracts the evaluative transition from moral salience to action without imposing a uniform behavioural shift.}
	\end{leftbar}
\end{center}
\bigskip
These results neither overstate nor nullify the effect. They provide the behavioural substrate for the next inferential step: testing whether the perturbation introduced by $\gamma_R$ interacts with the dispositional manifold $\beta_C$. The subsequent sections therefore turn to regression modelling, interaction structures, and Bayesian estimation, where the evaluative topology can be examined in its full dimensionality.



\subsection{What the Aggregate Divergence Establishes}

The behavioural evidence obtained thus far suggests that the silent co-presence
of a humanoid robot, operating with minimal but perceptually salient cues, is
associated with a modest reduction in aggregate donation behaviour within the
Watching–Eye paradigm. The attenuation is probabilistic and varies across
individuals, yet \textit{it is detectable} at the group level and supported by the
statistical analyses. Its importance lies not in its magnitude but in what such
a pattern makes available for interpretation.

\medskip
Within the evaluative–topological framework developed in the preceding chapters,
even small behavioural shifts can indicate a change in the conditions under
which moral salience becomes behaviourally operative. The present findings are
therefore consistent with the possibility of \emph{evaluative deformation}: the
idea that synthetic presence may influence the processes that mediate the
transition from perceptual cue to moral action. Floridi’s Levels of Abstraction
clarify why this form of influence is conceptually coherent. We have seen how at the operative
LoA, moral behaviour depends on informational and social affordances rather than
on the inner ontology of the agent (Chapter~\ref{chap:moral_primer}). A synthetic system can therefore be behaviourally inert yet still modify the evaluative background against which human agents form their responses.

\medskip
This perspective also reframes the role of the Synthetic Perturbation of Moral
Inference hypothesis. The hypothesis does not claim that the robot overrides or
replaces participants’ evaluative processes; instead, it proposes that synthetic
presence may shift the weighting, ordering, or integration of cues within those
processes. Such modulation would be detectable not as categorical differences in
behaviour but as systematic tendencies in the distribution of responses—precisely
the form of pattern observed here.

\medskip
The role of individual traits, represented by the vector \(\beta_C\), and their
potential interaction with robotic presence \(\gamma_R\), remains theoretically
significant. If the perturbation targets the evaluative field itself, rather
than trait-dependent gradients within it, the displacement should be broadly
distributed across the dispositional manifold. The next sections therefore move
from aggregate contrasts to trait–context modelling to assess this implication
directly.

\medskip
Taken together, the results thus far are compatible with a broader philosophical
claim: that artificial systems can matter morally not by reasoning or acting but
by modifying the cognitive–affective conditions under which human moral
judgement is formed. This interpretation aligns with the informational and
topological commitments introduced earlier, and it provides the conceptual and
methodological scaffolding for the more detailed analyses that follow.

\medskip
Beyond establishing the statistical detectability of these differences, it
remains necessary to quantify their magnitude. The following analyses introduce
both parametric and nonparametric effect-size metrics to characterise the extent
of behavioural modulation associated with synthetic presence.


\subsection{Quantification of Behavioural Modulation: Parametric and Nonparametric Effect Sizes}
\label{subsec:effect_sizes}

\noindent
Having established that the two groups do not differ in demographic or
dispositional structure, we can turn to a complementary question. Rather than
asking whether any perturbation occurred, the focus now shifts to its potential
magnitude—that is, the extent to which the presence of the robot may modulate
the mapping \(f(\alpha_E, \beta_C, \gamma_R)\) that links morally salient cues
to behavioural output. With evidence of a group-level difference in place, the
next analytical step is to characterise the amplitude of this modulation: the
degree to which \(\gamma_R\) may alter the transition from \(\alpha_E\) to
\(\mathscr{P}(\delta_m)\).

Significance tests indicate whether a behavioural contrast is detectable relative to sampling variability; they do not characterise the structural amplitude of the perturbation induced by the synthetic co-presence $\mathscr{R}$. For this reason, the present section complements the inferential tests with parametric and nonparametric effect-size metrics, thereby quantifying the extent to which robotic presence modulates prosocial behaviour under the Watching–Eye paradigm.

\noindent
Because the subsequent regression and interaction analyses will examine the interplay 
between robotic presence and dispositional structure, it is essential to begin with a 
transparent description of the overall behavioural landscape. The effect sizes presented here serve as the bridge between aggregate-level contrasts and the more nuanced trait–context models developed later in the chapter.

% ---------------------------------------------------------
%   Effect Size Overview
% ---------------------------------------------------------

\subsubsection*{Effect-Size Framework}

\noindent
To quantify the behavioural modulation associated with the perturbation operator 
$\mathscr{R}$, two complementary indices were employed:

\begin{itemize}[itemsep=4pt]
	\item \textbf{Cohen’s $d$}: a parametric effect size capturing the 
	standardised difference in mean donation amounts between conditions, 
	sensitive primarily to shifts in central tendency;
	
	\item \textbf{Cliff’s $\Delta$}: a nonparametric ordinal effect size that 
	estimates the probability that a randomly selected individual from one 
	condition donates more (or less) than a randomly selected individual from 
	the other, independent of distributional assumptions.
\end{itemize}

\noindent
Together, these metrics evaluate whether the presence of $\mathscr{R}$ produces a 
systematic displacement in the evaluative output distribution, consistent with the 
Evaluative Deformation Hypothesis (page~\pageref{hyp:one}).

% ---------------------------------------------------------
%   Definitions
% ---------------------------------------------------------

\paragraph{Cohen’s $d$.}

\[
d = \frac{\bar{x}_1 - \bar{x}_2}{s_p},
\qquad
s_p = \sqrt{\frac{(n_1 - 1)s_1^{\,2} + (n_2 - 1)s_2^{\,2}}{n_1 + n_2 - 2}} .
\]

\noindent
Where:
\begin{itemize}
	\item $\bar{x}_1, \bar{x}_2$ = mean donations in the Control and Robot conditions;
	\item $s_1, s_2$ = corresponding sample standard deviations;
	\item $n_1, n_2$ = group sample sizes;
	\item $s_p$ = pooled standard deviation assuming equal population variances
	(the standard definition of Cohen’s $d$).
\end{itemize}

\noindent
Cohen’s $d$ therefore measures the \emph{location shift} between groups in units of 
shared variability.

\paragraph{Cliff’s $\Delta$.}

\[
\Delta = \frac{\#(x > y) - \#(x < y)}{n_x n_y}.
\]

\noindent
Where:
\begin{itemize}
	\item $\#(x > y)$ = number of Control--Robot donation pairs where the Control 
	donation is larger;
	\item $\#(x < y)$ = number of pairs where the Robot donation is larger;
	\item $n_x, n_y$ = sample sizes of the two groups.
\end{itemize}

\noindent
Cliff’s $\Delta$ reflects the \emph{probabilistic dominance} of one distribution 
over the other. Values of $\Delta$ near $0$ indicate substantial overlap; positive 
values indicate a higher tendency for Control donations to exceed Robot donations.

% ---------------------------------------------------------
%   Results
% ---------------------------------------------------------

\medskip
\noindent
The empirical estimates are:

\[
d \approx 0.30,
\qquad
\Delta \approx 0.20.
\]

\noindent
Both fall within the range typically interpreted as \emph{small to modest} behavioural 
modulation. Their convergent directional signal is the critical feature: across both 
parametric and nonparametric perspectives, the presence of $\mathscr{R}$ is associated 
with lower prosocial donation on average.

\noindent
In the evaluative--topological framework, these effect sizes do not quantify moral 
capacity or trait strength; rather, they index the \emph{amplitude of deformation} in 
the mapping $\alpha_E \mapsto \mathscr{P}(\delta_m)$ when the perturbation operator 
$\gamma_R$ is active.


% ---------------------------------------------------------
%   Figure Justification
% ---------------------------------------------------------

\medskip
\noindent
To ensure interpretive clarity, two complementary visualisations are provided.  
The kernel density estimate (Fig.~\ref{fig:donation_density}) depicts the \emph{shape} and 
spread of donation distributions, enabling inspection of distributional tails and modes.  
The mean-with-standard-error plot (Fig.~\ref{fig:mean_donation}) focuses on 
\emph{central tendency} and sampling variability.  
Although partially overlapping in content, the two figures serve distinct analytic functions 
and together offer a transparent view of the behavioural landscape that informs the 
subsequent modelling work.

% ---------------------------------------------------------
%   Figures
% ---------------------------------------------------------

\begin{figure}[htbp]
	\centering
	\includegraphics[width=0.8\textwidth]{new_plots/donation_density_by_condition.png}
	\caption{Kernel density estimates of donation distributions across experimental conditions. 
		The Control group exhibits greater mass at higher donation values, whereas the Robot group 
		shows a mild left-shift in density. These plots provide distributional context for the 
		effect-size metrics discussed in the text.}
	\label{fig:donation_density}
\end{figure}

\begin{figure}[htbp]
	\centering
	\includegraphics[width=0.8\textwidth]{new_plots/donation_mean_with_se.png}
	\caption{Mean donation amounts with standard error bars by condition. 
		While the Control group donates more on average, the overlapping error bars reflect 
		substantial individual-level variability. The figure complements the density plot by 
		highlighting differences in central tendency rather than distributional shape.}
	\label{fig:mean_donation}
\end{figure}

% ---------------------------------------------------------
%   Statistical Tests Table (reproduced for completeness)
% ---------------------------------------------------------

\medskip
\noindent
For completeness, the inferential tests introduced earlier are reproduced in 
Table~\ref{tab:statistical_tests} alongside the effect-size metrics, ensuring that all 
aggregate-level results appear within a single consolidated reference point before turning 
to trait–context modelling.

\begin{table}[H]
	\centering
	\includegraphics[width=\textwidth]{tables/statistical_tests_table.pdf}
	\caption{Inferential comparisons of donation behaviour across conditions. 
		The chi-squared test (applied to total coin frequencies), the Mann--Whitney U test, 
		and the bootstrapped mean difference collectively characterise the behavioural contrast.}
	\label{tab:statistical_tests}
\end{table}

% ---------------------------------------------------------
%   Interpretive Summary
% ---------------------------------------------------------

\noindent
Overall, the effect sizes indicate that robotic co-presence is associated with a
\emph{directionally consistent but behaviourally modest} modulation of prosocial
action. These outcomes are compatible with—though they do not in themselves
establish—the interpretation that $\mathscr{R}$ influences the evaluative
transformation linking moral salience to behaviour. The pattern appears
\emph{graded} rather than categorical: the evaluative system remains intact, but
the strength with which salient cues inform action may be probabilistically
reduced.

\bigskip
\noindent
\begin{center}
	\begin{tcolorbox}[colback=white,colframe=black!60,
		title=Conclusion: Amplitude of Evaluative Modulation]
		The findings suggest that synthetic co-presence does not operate as a
		binary suppressor of prosocial behaviour. Rather, it may modulate the
		amplitude of the evaluative transformation from moral salience to
		action—a subtle, probabilistic shift consistent with the robot’s
		ambiguous social affordances at the operative Level of Abstraction.
	\end{tcolorbox}
\end{center}

\bigskip
\noindent

% ---------------------------------------------------------
%   Transition to Trait–Context Modelling
% ---------------------------------------------------------

\noindent
With the aggregate effect established, the analysis now turns inward.  
A perturbation visible at the population level does not yet reveal \emph{how} it is 
carried through the evaluative architecture.  If moral action arises from the 
interaction between situational cues and dispositional curvature, then the next 
question is not merely whether $\mathscr{R}$ exerts influence, but \emph{where within  the cognitive manifold that influence takes hold}.  

The dispositional structure $\beta_C$—the configuration of empathizing and systemizing tendencies together with the broader personality gradients indexed by the Big Five—may govern the system’s openness to perturbation.  In this sense, susceptibility is not a property of the behaviour alone, but of the architecture through which behaviour is assembled.  

The following sections therefore introduce regression models, interaction terms, and 
Bayesian estimation procedures designed to trace this internal geometry: to determine 
whether the attenuation observed so far is diffuse and population-wide, or whether it 
concentrates within particular psychological profiles whose evaluative trajectories are more easily deflected by synthetic presence.



\section{Dispositional Baseline: Big Five Personality Traits Across Conditions}

\noindent
Before any attenuation of prosocial behaviour can be meaningfully linked to the
robot’s presence, it is necessary to verify that the two groups begin from
comparable dispositional baselines. If participants assigned to the Robot
condition entered the experiment with systematically different trait profiles—
for example, lower Agreeableness, higher Neuroticism, or reduced empathic
orientation—then a behavioural contrast could not be attributed to
$\mathscr{R}$; it would instead reflect pre-existing dispositional differences.

\medskip

\noindent
The first analytical task is therefore straightforward but methodologically
essential: to determine whether the dispositional manifold $\beta_C$ is
distributed symmetrically across conditions. Only if this symmetry holds can
subsequent differences in the transition from moral salience to action be
interpreted as potential effects of synthetic presence rather than as artefacts
of trait imbalance.

\bigskip
\noindent
\begin{center}
	\begin{leftbar}
		\textit{Are the Big Five traits comparable across the Control and Robot
			conditions, or do they introduce a potential confound in interpreting the
			displacement of prosocial behaviour?}
	\end{leftbar}
\end{center}

\bigskip
\noindent



\subsection{Between-Condition Comparisons of Big Five Personality Traits}
\label{subsec:bigfive_groupdiff}

\noindent
The effect-size analyses above indicate that robotic co-presence
($\mathscr{R}$) is associated with a modest, directionally consistent
modulation of donation behaviour. Before assessing whether this pattern
interacts with individual differences, it is necessary to determine whether the
two experimental groups were already differentiated at the level of
personality. Systematic differences in Big Five traits between the Control and
Robot conditions would make it difficult to attribute any behavioural
attenuation to $\mathscr{R}$ rather than to pre-existing dispositional
variation.


\medskip
\noindent
To assess this, we compared Openness, Conscientiousness, Extraversion, 
Agreeableness, and Neuroticism across conditions using the Mann--Whitney 
$U$ test. This test is appropriate for the structure of the dataset: the Big 
Five scores are bounded, ordinal psychometric variables exhibiting mild skew, 
and the sample size ($N \approx 70$ as discussed in section~\ref{subsec:participants}, page~\pageref{subsec:participants}) does not justify strong parametric assumptions. Because examining five traits entails five simultaneous hypothesis tests, the Benjamini--Hochberg False Discovery Rate (FDR) correction was applied to control Type~I error.

\medskip
\noindent
After FDR correction, \textbf{none of the Big Five traits differ significantly} 
between the Control and Robot groups. Small numerical tendencies (e.g., slightly 
higher Openness in the Control condition) fail to approach corrected 
significance thresholds, and all distributions display substantial overlap.

\begin{figure}[H]
	\centering
	\begin{minipage}{0.98\linewidth}
		\centering
		\includegraphics[width=\linewidth]{new_plots/bigfive_row_kde_1.png}
		\caption{Kernel density estimates for each Big Five trait across experimental conditions. The plots depict the distribution of trait scores for the \textit{Control} (orange) and \textit{Robot} (green) groups. All five dimensions show substantial overlap, visually corroborating the non-significant differences found in corresponding Mann–Whitney U tests}
		\label{fig:bigfive_kde}
	\end{minipage}
\end{figure}

\noindent
These results support the following methodological inference:
\begin{center}
	\textbf{The two experimental groups may be treated as dispositionally equivalent.}
\end{center}

\noindent
Given this symmetry, the behavioural difference observed earlier is \emph{most
	consistently interpreted} as arising in connection with the presence of
$\mathscr{R}$ rather than from pre-existing personality differences.


\bigskip
\noindent
With dispositional equivalence established, the analysis can now address a
further question: whether personality structure nonetheless influences how
agents translate moral salience into action, and whether this structure
modulates the attenuation associated with robotic co-presence. Baseline symmetry
does not rule out differential susceptibility; rather, it provides the
conditions under which potential trait–context interactions can be examined
reliably.


% =============================================================
\subsection{Predictive and Moderating Roles of Big Five Personality Traits}
\label{subsec:bigfive_prediction}

\noindent
With baseline symmetry established, the next analytic step concerns the internal
structure of the evaluative field. Even when two groups do not differ in their
dispositional profiles, the traits within those profiles may still influence how
moral salience is processed and how strongly any perturbation associated with
$\mathscr{R}$ is expressed. The relevant question becomes:

\begin{center}
	\begin{leftbar}
		\textit{Do Big Five traits independently predict prosocial donation, or
			modulate the displacement associated with $\mathscr{R}$?}
	\end{leftbar}
\end{center}

\paragraph{(1) Predictive effects.}
To examine potential predictive effects, Spearman rank correlations were
computed between each Big Five dimension and donation amount. Spearman’s~$\rho$
is appropriate for zero-inflated, bounded, and non-normal behavioural data, as
well as for ordinal psychometric measures. Scatterplots with monotonic trend
overlays were also inspected to identify potential nonlinearities that the
correlation coefficients might not capture.

\paragraph{(2) Moderation effects.}
To assess whether personality modulates the displacement effect, interaction
models of the form
\[
\text{donation} \sim \text{condition} \times \text{trait}
\]
were estimated for each Big Five dimension. This specification tests the
possibility that the influence associated with robotic presence varies as a
function of dispositional structure.

\medskip
\noindent
Methodologically, the findings are straightforward. \textbf{No statistically reliable associations} between any Big Five trait and donation amount were observed in this dataset, and \textbf{no interaction} with experimental condition reached significance. The behavioural attenuation associated with $\mathscr{R}$ therefore shows no detectable variation across the Big Five personality profiles.

\begin{figure}[H]
	\centering
	\begin{minipage}{0.98\linewidth}
		\centering
		\includegraphics[width=\linewidth]{new_plots/bigfive_row_scatter_1.png}
		\caption{Scatter plots of donation amounts against each of the Big Five personality traits, with monotonic regression lines. No predictive relationships are apparent, and no consistent moderation patterns emerge across traits. These visual results support the null findings from the Mann–Whitney and interaction analyses.
		}
		\label{fig:bigfive_scatter}
	\end{minipage}
\end{figure}

\medskip
\noindent
In summary, within the Big Five framework:

\begin{itemize}
	\item no trait reliably predicts prosocial donation;
	\item no trait moderates the attenuation introduced by robotic co-presence;
	\item the displacement effect of $\mathscr{R}$ shows no detectable 
	variation across Big Five profiles \emph{in this sample}.
\end{itemize}

\bigskip
\begin{center}
	\begin{tcolorbox}[colback=white,colframe=black!60,
		title=Conclusion: Trait-Independence of Evaluative Displacement]
		The attenuation of prosocial donation observed under robotic co-presence
		shows no detectable modulation by Big Five personality traits in this
		dataset. This pattern is consistent with the interpretation that
		$\mathscr{R}$ may influence aspects of the evaluative field itself rather
		than operating through specific dispositional pathways.
	\end{tcolorbox}
\end{center}
\bigskip

\noindent
The next subsection turns to more specialised social--cognitive dispositions.  
If the broad personality taxonomy reveals no predictive or moderating signal, the 
question naturally shifts to traits whose psychological grain is finer and whose 
theoretical relevance to moral cognition is deeper.  
Within the evaluative topology developed earlier, EQ and SQ capture distinct 
curvatures of the dispositional manifold—dimensions that operate at the cognitive 
Level of Abstraction where salience, affective resonance, and interpretive stance 
enter the evaluative field.  The analysis therefore asks whether these traits disclose modulations of $\mathscr{R}$ that the Big Five, by design a coarser taxonomy, cannot detect.


\subsection{Transition to Structural Modelling of Dispositional Architecture}

\noindent
The analyses reported above establish two methodological foundations for the
remainder of the statistical pipeline. First, the Big Five traits do not differ
across conditions after False Discovery Rate correction, supporting the
conclusion that the Control and Robot groups are dispositionally comparable.
Second, within this sample, none of the Big Five traits reliably predict
donation behaviour, nor do they interact with experimental condition. In
inferential terms, the dataset provides no evidence of trait imbalance and no
statistically detectable trait–by–condition moderation within the classical
personality taxonomy.

\noindent
These observations do not \emph{rule out} the relevance of dispositional
structure; rather, they clarify the level at which such structure should be
examined. The Big Five provide coarse-grained scalar descriptors and may not
capture the finer relational patterns—covariation and interdependence among
traits—that can influence evaluative processing. The next stage of analysis
therefore adopts a more structurally sensitive approach to the dispositional
manifold $\beta_C$, assessing whether latent configurations of empathizing,
systemizing, and Big Five attributes jointly organise susceptibility to
robotic presence.

\medskip
\noindent
In this sense, the null findings within the Big Five framework serve a
methodological rather than an interpretive function. They indicate that any
systematic modulation of donation behaviour associated with the synthetic
presence $\mathscr{R}$ is unlikely to arise from imbalances or linear trait
effects within the classical personality model. This provides the inferential
basis for moving to clustering and latent-structure analyses, in which
$\beta_C$ is treated not as a set of independent dimensions but as a structured
configuration whose internal organisation may interact with the perturbative
affordances of $\mathscr{R}$.

\medskip
\noindent
The next section therefore introduces the clustering methodology used to derive
latent dispositional ecologies and examines whether these ecologies exhibit
differential susceptibility to synthetic co-presence. This marks the transition
from trait-level analysis to structural modelling within the broader evaluation
of Question~\ref{q:robot-agent}, page~\pageref{q:robot-agent}.


\subsection{Latent Dispositional Structures and the Modulation of Moral Perturbation}
\label{subsec:latent_dispositions}

Two empirical results set the stage for the next analytical step. First,
robotic co-presence $\mathscr{R}$ is associated with a modest, directionally
stable attenuation of prosocial donation. Second, this attenuation is not
predicted by any single Big Five dimension. Together, these findings shift the
focus from isolated trait magnitudes to the \emph{internal organisation} of the
dispositional manifold~$\beta_C$.
\medskip
\begin{center}
	\begin{leftbar}
		\textit{If broad personality dimensions do not differentiate sensitivity to
			$\mathscr{R}$, might the perturbation instead manifest within latent
			cognitive–affective configurations that jointly structure~$\beta_C$?}
	\end{leftbar}
\end{center}
\medskip
This question follows directly from the evaluative model introduced earlier. If
synthetic presence influences the transformation
\( f(\alpha_E, \beta_C, \gamma_R) \), there is no requirement that its impact be
uniform across individuals; it may instead reflect differences in how
dispositional factors combine into higher-order profiles. At the operative Level
of Abstraction, such profiles—not individual scalar traits—may better capture
the structures that guide responsiveness to moral salience.

The present section therefore turns from trait-level analyses to structural
modelling of~$\beta_C$, asking whether synthetic presence interacts with the
latent dispositional regimes that organise the evaluative field.


% ---------------------------------------------------------------------
\subsubsection{Clustering the Dispositional Manifold}

Seven psychometric variables—Empathizing, Systemizing, and the five Big Five
traits—were used to construct the dispositional manifold. Each variable was
$z$-standardised, and dimensionality was reduced with Principal Component
Analysis (PCA). Two orthogonal components were retained because they captured
the dominant axes of variance while reducing redundancy among correlated traits.
This representation provides a tractable approximation of the manifold’s local
geometry.

\noindent
The resulting two-dimensional embedding served as input for $k$-means
clustering. The choice of $k = 3$ was supported by both methodological and
conceptual considerations:
\begin{itemize}
	\item The within-cluster sum of squares exhibited a clear elbow at $k = 3$,
	indicating diminishing returns for larger $k$.
	\item Although the silhouette coefficient peaked at $k = 9$, such maxima can
	reflect over-partitioning when $N$ is modest; these solutions were therefore
	rejected.
	\item A three-cluster solution yielded groups of interpretable size with
	stable internal variability, consistent with the expectation that a limited
	number of dispositional regimes may structure variation within the manifold.
\end{itemize}

\medskip
\noindent
Figure~\ref{fig:personality-clusters-pca} visualises the resulting partitions.
The figure is retained because it provides the structural basis for treating the
clusters as psychologically interpretable configurations within the broader
dispositional manifold.

\begin{figure}[H]
	\centering
	\includegraphics[width=0.95\linewidth]{new_plots/personality_clusters_pca.png}
	\caption{Participants clustered in PCA-reduced psychometric space. Three
		clusters emerge as coherent and visually distinguishable groupings, providing
		the structural substrate for subsequent analyses of condition-by-cluster
		effects.}
	\label{fig:personality-clusters-pca}
\end{figure}

% ---------------------------------------------------------------------
\subsubsection{Justification of $k = 3$: Diagnostic Criteria}

\noindent
Figure~\ref{fig:cluster_elbow_silhouette} presents both the elbow curve and the
silhouette profile. These diagnostics are standard tools for evaluating
clustering structure and indicate that $k = 3$ is a parsimonious and defensible
choice. Within the Level of Abstraction adopted in this chapter, this choice
supports a tractable representation of the dispositional manifold for analysing
how latent configurations within the evaluative field may relate to the effects
associated with~$\mathscr{R}$.


\begin{figure}[H]
	\centering
	\includegraphics[width=0.95\linewidth]{new_plots/cluster_elbow_silhouette.png}
	\caption{Elbow plot (left axis) and silhouette coefficients (right axis) across candidate values of $k$. The elbow at $k = 3$ and stable silhouette profile support selecting three clusters as an interpretable and parsimonious solution.}
	\label{fig:cluster_elbow_silhouette}
\end{figure}

\noindent
Conceptually, a small number of clusters is consistent with the idea that only a limited set of dominant dispositional regimes may modulate how moral salience is processed under synthetic perturbation.

% ---------------------------------------------------------------------
\subsubsection{Cluster-Specific Patterns of Moral Response}

\noindent
We then examined whether the donation attenuation associated with $\mathscr{R}$ differed across clusters. Figure~\ref{fig:donation-by-cluster} shows mean donation by condition within each cluster. This visualisation is essential because it provides the descriptive foundation for the interaction models to be developed next.

\begin{figure}[H]
	\centering
	\includegraphics[width=0.85\linewidth]{new_plots/donation_by_cluster_and_condition.png}
	\caption{Mean donation amount by condition within each personality cluster. Error bars represent standard deviation. Cluster~1 shows a clearer attenuation of donation under robotic presence, while Clusters~0 and~2 display only modest or negligible differences.}
	\label{fig:donation-by-cluster}
\end{figure}

\noindent
The pattern is not uniform across clusters. Preliminary inspection of the cluster centroids suggests that Cluster~1 is characterised by higher systemizing and lower empathizing scores—a cognitive–affective style that may rely more on structural processing and less on affective resonance. This offers a plausible interpretive foothold: the evaluative perturbation induced by $\gamma_R$ may interact with configurations of traits rather than their isolated values.

\medskip
\noindent
These descriptive patterns motivate the formal interaction models introduced next, where cluster membership is incorporated as a moderator in the mapping from $\text{condition}$ to $\text{donation}$.

% ---------------------------------------------------------------------
\subsubsection*{Conclusion: Dispositional Regimes and Moral Perturbation}

\begin{center}
	\begin{tcolorbox}[colback=white,colframe=black!60,title=Interpretive Conclusion]
		Preliminary evidence suggests that the attenuation associated with robotic co-presence is not uniformly distributed across participants. Instead, latent dispositional regimes—rather than individual trait scores—appear to modulate susceptibility to the perturbative influence of $\mathscr{R}$. This provides the conceptual and empirical basis for the interaction models developed in the next section.
	\end{tcolorbox}
\end{center}


\subsection{Psychometric Interpretation and Semantic Labelling of Latent Personality Clusters}
\label{subsec:cluster_semantics}

\noindent
Identifying three latent dispositional clusters refines the structure of the
manifold~$\beta_C$, but clustering alone does not specify the \emph{psychological
	profile} encoded in each grouping. The analyses thus far indicate that the
attenuation associated with $\mathscr{R}$ is not uniformly distributed across
participants; the present task is to clarify the dispositional patterns through
which this heterogeneity may arise.

\noindent
This interpretive step is methodologically essential. Without a principled
semantic characterisation of the clusters, the partitions would remain
mathematically distinct but psychologically uninformative. Moreover, at the
Level of Abstraction operative in this chapter, semantic labelling is required
to relate the latent structures to the evaluative field in which perturbation by
$\mathscr{R}$ is assessed. Subsequent modelling depends directly on these
interpretive anchors.

\medskip
\noindent
To move from numerical clusters to psychologically interpretable ecologies, the
unscaled cluster centroids were projected back onto the original psychometric
dimensions. Radar plots (Figure~\ref{fig:radar_three_panel}) provide a justified
visual summary for this step: by depicting the \emph{normalised} centroid values
across traits, they offer a relational representation of each ecology’s internal
configuration that is more readily interpretable than numerical tables alone.


\begin{figure}[H]
	\centering
	\includegraphics[width=0.32\linewidth]{new_plots/cluster_0_radar.png}
	\includegraphics[width=0.32\linewidth]{new_plots/cluster_1_radar.png}
	\includegraphics[width=0.32\linewidth]{new_plots/cluster_2_radar.png}
	\caption{Radar profiles (normalised for comparability) of the three latent dispositional ecologies.  
		Left: Cluster~0 (Emotionally Reactive / Low-Structure);  
		Centre: Cluster~1 (Prosocial–Empathic / Warm–Sociable);  
		Right: Cluster~2 (Analytical–Structured / High-Systemizing).  
		These plots visualise the relative psychometric configuration of each ecology.}
	\label{fig:radar_three_panel}
\end{figure}

% --------------------------------------------------------------
\subsubsection{Ecology I: Emotionally Reactive / Low-Structure}

\noindent
Cluster~0 exhibits elevated Neuroticism, low Conscientiousness, reduced Systemizing, and moderate values across Openness, Extraversion, and Agreeableness. This constellation reflects an \textit{affectively reactive configuration with comparatively weaker structural coherence}. Within the moral-topological framework developed earlier, such an ecology corresponds to a \emph{loosely stabilised evaluative field}: moral cues propagate through an architecture more susceptible to contextual fluctuation, including ontological ambiguity.

\medskip

% --------------------------------------------------------------
\subsubsection{Ecology II: Prosocial–Empathic / Warm–Sociable}

\noindent
Cluster~1 is characterised by elevated Openness, Extraversion, Agreeableness, and Empathizing—a \textit{warm, sociable, affectively attuned} profile. This ecology represents the canonical prosocial configuration frequently documented in moral psychology: empathically oriented, interpersonally open, and responsive to moral cues.

\noindent
Because empathic pathways are ordinarily the most fluid in this group, the descriptively stronger attenuation of donation under $\mathscr{R}$ carries high interpretive value. It suggests that robotic presence may interfere with affective–evaluative channels rather than rule-based reasoning. The displacement of empathic resonance by an ontologically ambiguous artificial form is therefore not merely possible but observable, at least descriptively, within this ecology.

\medskip

% --------------------------------------------------------------
\subsubsection{Ecology III: Analytical–Structured / High-Systemizing}

\noindent
Cluster~2 shows elevated Systemizing and Conscientiousness with comparatively lower Empathizing, forming an \textit{analytical, structured, rule-oriented} regime. Individuals within this constellation privilege explicit structure and informational clarity over implicit social affordances.

\noindent
From a Level-of-Abstraction perspective, this ecology \emph{may be understood as aligning with} a higher abstraction threshold: ambiguous embodied agents, such as a non-interactive humanoid robot, are encoded primarily as neutral environmental features. Correspondingly, the attenuation associated with $\mathscr{R}$ appears weaker in this group.

\medskip

% --------------------------------------------------------------
\subsubsection{Interpretive Integration}

\noindent
Three dispositional ecologies exhibit a coherent structural pattern:

\begin{itemize}
	\item The \textbf{Prosocial–Empathic} ecology shows the \emph{largest
		descriptive attenuation} associated with~$\mathscr{R}$.
	\item The \textbf{Analytical–Structured} ecology shows \emph{minimal
		descriptive change}.
	\item The \textbf{Emotionally Reactive} ecology displays \emph{variable}
	responsiveness, consistent with its affective volatility.
\end{itemize}

\noindent
These results suggest that the influence of synthetic presence does not follow a
single pathway; instead, it appears to vary across \emph{dispositional
	configurations} within the evaluative field. Robotic co-presence does not act as
a uniform suppressor or amplifier. Its behavioural impact, where present, seems
to depend on how the internal organisation of~$\beta_C$ conditions the
processing of morally salient cues.

\medskip
\noindent
In this respect, the mapping
\[
f(\alpha_E, \beta_C, \gamma_R)
\]
should be understood as jointly shaped by the perturbation operator
$\gamma_R$ and the structure of the dispositional manifold. The influence
associated with~$\mathscr{R}$ is therefore not well characterised as additive;
rather, it appears to be \emph{structurally mediated} by the configurations
through which individuals integrate environmental and social information at the
relevant Level of Abstraction.
% --------------------------------------------------------------
\subsubsection*{Connection to Floridi’s Levels of Abstraction}

\noindent
These ecologies may be understood as corresponding to distinct operative Levels of Abstraction:

\begin{itemize}
	\item The \textbf{Prosocial–Empathic} ecology foregrounds affective salience.
	\item The \textbf{Analytical–Structured} ecology foregrounds structural clarity.
	\item The \textbf{Emotionally Reactive} ecology foregrounds affective variability.
\end{itemize}

\noindent
Accordingly, $\gamma_R$ perturbs different informational channels depending on the ecology through which moral cues are interpreted.

\medskip

% --------------------------------------------------------------
\subsubsection*{Conceptual Conclusion}

\begin{center}
	\begin{tcolorbox}[colback=white,colframe=black!60,
		title=Conclusion: Trait-Contingent Structure of Moral Perturbation]
		\label{conc:clustered_moral_refraction_revised}
		The attenuation associated with robotic co-presence is not globally uniform. It emerges from contingent interactions between the synthetic presence $\gamma_R$ and the latent cognitive–affective ecologies encoded in $\beta_C$. These ecologies refract the evaluative transformation from moral salience to action, producing descriptively stronger perturbation in empathically oriented profiles, weaker effects in analytically oriented profiles, and variable responses in affectively reactive configurations. In informational terms, $\gamma_R$ interacts with participants at different operative Levels of Abstraction, generating heterogeneous moral responses across these latent evaluative architectures.
	\end{tcolorbox}
\end{center}

\noindent
This structural interpretation provides the necessary grounding for the next analytical step. The forthcoming regression and Bayesian models formally examine whether these ecology-specific patterns persist under inferential scrutiny, thereby testing how $\beta_C$ modulates the evaluative function $f(\alpha_E, \beta_C, \gamma_R)$ within a principled statistical framework.





\subsection{Cluster-Specific Regression Analysis of Condition Effects}
\label{subsec:cluster_regression}

\noindent
The latent dispositional clusters identified in the previous subsection provide a structured basis for examining whether the behavioural effect of robotic co-presence (\(\gamma_R\)) varies across different cognitive--affective regimes. To assess this possibility, we estimated a simple linear regression within each cluster of the form:
\[
\text{donation} = \beta_0 + \beta_1 \cdot \text{condition}_{\text{Robot}} + \varepsilon ,
\]
where \(\beta_1\) quantifies the within-cluster contrast between Control and Robot conditions. These stratified regressions serve as \textit{local directional estimates}, establishing whether any cluster exhibits a recognisably stronger attenuation pattern prior to introducing interaction terms or hierarchical Bayesian pooling.

\medskip

\noindent
A descriptively uneven pattern emerges across clusters. In the cluster characterised by higher empathizing and sociability (Cluster~1), the estimated coefficient for the Robot condition is negative and comparatively large in magnitude relative to the other clusters (\(\beta=-1.33\)), though still uncertain given the small within-cluster sample size and the fact that the 95\% interval includes zero (\(p=.091\), \(R^2=0.087\)). This estimate suggests that the directional attenuation observed at the aggregate level may be disproportionately expressed in this subset of participants.

\medskip

\noindent
By contrast, the affectively variable (\textit{Emotionally Reactive}) cluster (Cluster~0) exhibits a coefficient near zero (\(p>.70\)), and the analytically structured (Cluster~2) regime shows only a modest, non-significant negative coefficient (\(\beta=-0.28\), \(p>.70\)). In both cases the estimates are small, and the associated intervals indicate no reliable deviation between conditions. Taken together, these results imply that the aggregate attenuation documented earlier is not homogeneously distributed across dispositional space.

\medskip

\noindent
It is important to emphasise two methodological clarifications.  
First, these regressions treat cluster assignments as fixed labels. They therefore do not incorporate uncertainty in cluster membership or hierarchical pooling across clusters. Both limitations are addressed explicitly in the **Bayesian modelling framework** introduced in the next subsection, which relaxes linearity assumptions, models bounded and zero-inflated outcomes, and accounts for varying uncertainty across clusters.  
Second, an omnibus condition \(\times\) cluster interaction model is presented later in the analytical pipeline. The stratified regressions provided here serve a narrower epistemic function: they establish **local effect direction** prior to modelling global interaction structure.

\medskip

\noindent
Finally, although the donation data are bounded and zero-inflated, we employ ordinary least squares at this stage to provide interpretable contrasts within a familiar parametric structure. The subsequent Bayesian analyses incorporate appropriate distributional assumptions and therefore supersede these exploratory linear models.

\medskip

\begin{figure}[H]
	\centering
	\includegraphics[width=0.75\linewidth]{new_plots/cluster_regression_coefficients.png}
	\caption{Regression coefficients (with 95\% confidence intervals) for the Robot condition estimated separately within each latent personality cluster. Cluster~1 shows a larger negative coefficient relative to the other clusters, though uncertainty remains high due to small within-cluster sample sizes. Clusters~0 and 2 exhibit coefficients near zero. These estimates provide local directional contrasts prior to interaction and Bayesian modelling.}
	\label{fig:cluster-regression}
\end{figure}

\medskip

\noindent
The estimated differences can be summarised at the level of expected evaluative output. Let \(f(\cdot)\) denote the behavioural transformation introduced earlier. For each cluster \(k\),
\[
\mathbb{E}\big[f(\Sigma \cup \mathscr{R})\big]_{k}
\quad \text{vs.} \quad
\mathbb{E}\big[f(\Sigma)\big]_{k}
\]
captures the expected donation under Robot and Control conditions respectively. The empirical pattern may be expressed as:

\begin{itemize}
	\item \textbf{Cluster 0 (Emotionally Reactive):}  
	\(\mathbb{E}[f(\Sigma \cup \mathscr{R})]_{0} \approx \mathbb{E}[f(\Sigma)]_{0}\)  
	(no detectable within-cluster difference).
	
	\item \textbf{Cluster 1 (Prosocial–Empathic):}  
	\(\mathbb{E}[f(\Sigma \cup \mathscr{R})]_{1} < \mathbb{E}[f(\Sigma)]_{1}\)  
	(largest negative contrast, though interval includes zero).
	
	\item \textbf{Cluster 2 (Analytical–Structured):}  
	\(\mathbb{E}[f(\Sigma \cup \mathscr{R})]_{2} < \mathbb{E}[f(\Sigma)]_{2}\)  
	(small, non-significant difference).
\end{itemize}

\medskip

\noindent
These expressions simply restate, in the language of expected values, the directional information contained in the regression coefficients. They do not imply deterministic effects or global causal claims. Instead, they highlight that: 



\bigskip
\noindent
\begin{center}
	\begin{leftbar}
		\textit{the condition effect is not uniform across latent dispositional regimes, motivating a shift to modelling frameworks that can formally represent uncertainty, zero-inflation, and interaction structure.}
	\end{leftbar}
\end{center}

\bigskip
\noindent

\medskip

\noindent
The next subsection therefore introduces a Bayesian estimation approach, designed to assess whether the patterns observed here persist when distributional assumptions are relaxed and when uncertainty is explicitly modelled at the level of both clusters and individual parameters.


\subsection{Bayesian Estimation and the Representation of Epistemic Gradients}
\label{subsec:bayesian_estimation}

\noindent
The cluster–specific regressions established that condition effects vary directionally across latent dispositional regimes, but they also highlighted the limitations of ordinary least squares in a bounded, zero-inflated dataset of modest size. Donation amounts exhibit asymmetry, mass at zero, and cluster-dependent variability; moreover, stratified regressions treat cluster membership as fixed and do not pool information across groups. A more flexible inferential framework is therefore required—one capable of representing uncertainty as a structured epistemic property rather than as residual error.

\medskip

\noindent
\textbf{Motivation for a Bayesian approach.}  
Three considerations motivate a transition to Bayesian estimation at this stage:

\begin{enumerate}
	\item \textbf{Sensitivity to subtle effects in modest samples.}  
	Frequentist tests collapse subtle behavioural tendencies into binary outcomes. Bayesian methods provide graded estimates of effect magnitude and uncertainty, which are essential in a study concerned with delicate perturbations of evaluative processing.
	
	\item \textbf{Hierarchical structure in the data.}  
	Condition effects \((\gamma_R)\) interact with latent dispositional regimes \((\beta_C)\). A Bayesian hierarchical model naturally incorporates this structure via partial pooling.
	
	\item \textbf{Conceptual alignment with the evaluative framework.}  
	If robotic presence exerts a refractive, context-dependent influence, then the inferential representation of this influence should itself be graded and continuous. Bayesian inference provides this representational form.
\end{enumerate}

\medskip

\noindent
\textbf{Model structure.}  
A hierarchical Bayesian model was specified in which:

\begin{itemize}
	\item donation amount was the outcome variable (after mild variance-stabilising transformation to accommodate zero inflation),
	\item experimental condition was the primary predictor,
	\item cluster membership contributed varying intercepts and varying slopes,
	\item weakly informative priors regularised estimates while allowing the data to drive posterior shape.
\end{itemize}

\noindent
The likelihood was implemented using a Student-\(t\) distribution, which is robust to skew, heavy tails, and zero-inflated behaviour—a pragmatic solution that avoids imposing unrealistic Gaussian assumptions while maintaining computational stability.

\medskip

\noindent
\textbf{Posterior estimation.}  
The posterior distribution for the \emph{modelled} donation difference (\texttt{Control – Robot}) shows a central tendency of approximately £0.70, with a 95\% credible interval ranging from about –£1.75 to £0.30. Although the interval includes zero, its mass is asymmetrically concentrated toward positive values, indicating \textit{directional probabilistic evidence} for attenuation under robotic co-presence. Rather than yielding a binary verdict, the posterior encodes a structured probability over plausible effect magnitudes.

\begin{figure}[H]
	\centering
	\includegraphics[width=0.75\linewidth]{new_plots/posterior_donation_difference.png}
	\caption{Posterior distribution of the modelled donation difference between conditions. The density is skewed toward positive values (greater expected donations in the Control condition), providing directional probabilistic evidence for attenuation under robotic co-presence. The dashed line marks the point of no effect.}
	\label{fig:posterior-difference}
\end{figure}

\medskip

\noindent
\textbf{Interpretive value of the Bayesian framework.}  
The Bayesian posterior advances the methodological arc of the chapter in three ways:

\begin{enumerate}
	\item \textbf{It treats uncertainty as epistemic structure.}  
	Rather than compressing uncertainty into a single threshold, the posterior renders it as a gradient reflecting the fine-grained ambiguity intrinsic to morally loaded decisions in minimally interactive environments.
	
	\item \textbf{It integrates hierarchical heterogeneity.}  
	Partial pooling allows condition effects to vary by cluster while borrowing strength across the population. This avoids overfitting in smaller clusters and respects the structural complexity of the latent evaluative regimes.
	
	\item \textbf{It offers a representational analogue of interpretive indeterminacy.}  
	The moral perturbation introduced by NAO operates amid ontological ambiguity; the Bayesian posterior provides a natural representational analogue of this indeterminacy, modelling moral displacement not as a discrete shift but as a probabilistic modulation.
\end{enumerate}

\medskip

\noindent
\textbf{Connection to Floridi’s Levels of Abstraction.}  
Within the LoA framework, agents interpret synthetic entities through informational filters that shape what counts as morally salient. Because NAO’s presence introduces indeterminacy in these filters, the inferential system used to model its effect should preserve—rather than collapse—that indeterminacy. The posterior distribution does precisely this: it expresses the impact of $\gamma_R$ as a graded epistemic field, mirroring the cognitive state of an agent responding to ambiguous moral cues.

\medskip

\nextdiv
\begin{center}
	\begin{tcolorbox}[colback=white,colframe=black!60,
		title=Conclusion: Bayesian Representation of Moral Perturbation]
		\label{conc:bayesian_epistemic_gradient}
		Bayesian estimation shows that robotic co-presence yields a probabilistic attenuation of prosocial donation rather than a discrete behavioural shift. The posterior distribution expresses directional evidence for reduced donation in the Robot condition while fully representing the uncertainty expected for subtle, context-dependent perturbations of moral salience. This graded inferential form is consistent with the chapter’s evaluative framework: synthetic presence reshapes the topology of moral evaluation in a continuous rather than binary manner.
	\end{tcolorbox}
\end{center}

\medskip

\noindent
With this Bayesian model, the inferential sequence of the Experimental Methods chapter reaches completion. The next chapter synthesises these findings to articulate their broader philosophical and normative significance.

\subsubsection{Epistemic Interpretation of the Bayesian Results}
\label{subsubsec:bayesian_interpretation}

\noindent
The Bayesian model developed above enriches the inferential structure of this chapter by representing uncertainty as an explicit epistemic quantity rather than as a residual error term. This shift is methodologically appropriate for the present design, but also conceptually aligned with the chapter’s broader focus on graded perturbations of evaluative structure.

\medskip
\noindent
Unlike frequentist procedures that partition outcomes into ``significant’’ and ``non-significant’’ categories, the posterior distribution in Figure~\ref{fig:posterior-difference} expresses a \emph{graded representation of evidential support for differences in donation across conditions}. The posterior for the modelled donation difference (\texttt{Control – Robot}) displays a central tendency near £0.70, but with a wide credible interval spanning mildly positive and negative values. The posterior mass is asymmetrically concentrated toward higher donations in the Control condition, providing \emph{directional probabilistic evidence} for attenuation under robotic co-presence—while making the uncertainty surrounding this effect fully transparent.

\medskip
\noindent
In relation to earlier analyses, the Bayesian posterior does not ``rescue’’ non-significant frequentist tests; rather, Bayesian inference \emph{frames the question differently}, updating the plausibility of attenuation effects under explicit modelling of uncertainty, heterogeneity, and zero inflation. Frequentist tests ask whether the data cross a threshold under idealised distributional assumptions; the Bayesian model asks how the data shift our degree of belief in an attenuation effect. These perspectives are epistemically distinct yet empirically compatible, and their convergence on the same directional trend provides a robust evidential basis for this chapter’s claims.

\medskip
\noindent
This Bayesian approach is especially appropriate for the present study for two reasons. First, the perturbation introduced by $\mathscr{R}$ is theorised to be subtle, context-dependent, and heterogeneously expressed across participants—properties that hierarchical Bayesian models are designed to represent. Second, the latent dispositional clusters identified earlier generate structured variability that partial-pooling models can incorporate naturally. In this way, Bayesian posteriors provide a \emph{natural representational analogue} of the interpretive indeterminacy through which agents register moral salience under ambiguous conditions.

\begin{center}
	\begin{tcolorbox}[colback=white,colframe=black!60,
		title=Conclusion: Gradient of the Impact of Moral Refraction]
		\noindent
		The Bayesian analysis supports a cautiously framed but epistemically credible claim: attenuation of prosocial donation under robotic co-presence is \emph{probabilistically more likely than not}, with directional support emerging despite substantial uncertainty. This effect is therefore best understood not as a binary shift but as a graded modulation of the evaluative transformation through which moral salience becomes action.
	\end{tcolorbox}
\end{center}

\noindent
Taken together, the Bayesian results complete the inferential arc of this chapter. The behavioural attenuation, the latent cluster structure, and the posterior’s graded evidential pattern converge on a coherent empirical picture: robotic co-presence subtly and heterogeneously modulates the evaluative mapping from morally salient cues to prosocial behaviour.

\noindent
The next chapter develops the corresponding theoretical interpretation—particularly within the intuitionist tradition in moral psychology, the Watching-Eye literature, and broader debates in Social Signal Processing, Affective Computing, and Machine Ethics, where context-modulated salience and perceptual framing play a central conceptual role.


\subsection{Closing Reflection: How Synthetic Presence Reconfigures the Moral Field}

\noindent
When we look back across the full analytical arc of this chapter—from raw behavioural contrasts to hierarchical Bayesian estimation—a single idea comes into focus. Moral behaviour does not unfold in a vacuum. It grows out of what we notice first, how we feel the atmosphere of a situation, what we treat as relevant long before we begin to reason through it. Our decisions emerge from the texture of the environment and from the quiet interplay between our own dispositional architecture and the signals around us.

\medskip
\noindent
What this experiment shows is that the presence of a humanoid robot—even one that neither speaks nor evaluates us—can reshape that texture. Not dramatically, not uniformly, but measurably. The charity poster, with its image of a child in need, is normally a powerful intuitive cue: it draws our attention, evokes concern, and nudges us toward prosocial action before any explicit deliberation takes hold. Yet when NAO is in the room, this intuitive channel is no longer clean. The robot becomes a second centre of salience—an object that feels social enough to matter, but not social enough to interpret. Some participants fold this ambiguity into their evaluative process; others simply disregard it. And those differences are structured, not random.

\medskip
\noindent
At the aggregate level, this manifests as a modest reduction in donation under robotic co-presence. At the individual level, the posterior distribution shows that this attenuation is \emph{more likely than not}, though embedded in genuine uncertainty. And at the dispositional level, our latent trait analysis reveals a \textbf{clear descriptive pattern}: those whose moral lives are primarily guided by warmth, sociability, and empathic resonance are the very ones most affected by NAO’s ambiguous presence. For them, the intuitive pull of the poster is partially displaced; for others, the robot barely registers.

\medskip
\noindent
This is not the kind of result that lends itself to simple causal slogans. It is not that “robots reduce generosity” or that “some personalities are immune.” The structure is subtler. What we see is a redistribution of intuitive salience: a \textbf{subtle bending of the moral field} that makes certain cues lighter, others heavier, and some simply harder to parse. NAO does not instruct anyone to act differently, nor does it hold a moral stance. Instead, it alters the perceptual scaffolding through which moral meaning normally flows. The change is quiet, almost atmospheric—and that is precisely why it matters.

\medskip
\noindent
From a methodological standpoint, the chapter demonstrates that such subtle effects can be measured, modelled, and formalised. The combination of frequentist contrasts, latent trait clustering, and Bayesian estimation provides a coherent and discriminating toolset for analysing how artificial systems modulate human moral behaviour. The topological language developed earlier in the thesis—mapping moral salience as a field, evaluative processes as trajectories, and synthetic presences as local perturbations—finds empirical grounding here. What the data offer is not proof of a grand theory, but a carefully bounded demonstration: when the informational structure of a moral environment is altered, even slightly, the intuitive pathways that guide behaviour can shift.

\medskip
\noindent
And this, ultimately, is the bridge to the conceptual questions that follow. If moral action is so finely attuned to environmental cues—if it responds to shifts in atmosphere, presence, and perceived social relevance—then the broader ethical landscape of human--machine coexistence cannot be reduced to internal principles encoded in artificial agents. It must be understood in terms of \emph{how machines participate in the environments within which our intuitions take shape}. Before we can talk about alignment, responsibility, or artificial moral competence, we must first understand how artificial systems already influence our evaluative architecture simply by being there.

\medskip
\noindent
In this sense, the chapter closes not with a resolution, but with a trajectory. We have established that synthetic presence can deform the moral field in ways that are modest, structured, and psychologically contingent. The next chapter asks what this means for the stories we tell about moral machines, for the theories we use to explain moral behaviour, and for the frameworks we rely on when designing artificial systems that will inhabit our social and normative spaces. If the intuitive foundations of moral life are as malleable as these findings suggest, then the ethical questions surrounding artificial agents begin long before those agents act. They begin with how they appear, how they are perceived, and how their presence reshapes the quiet, pre-reflective work from which our moral decisions grow.
