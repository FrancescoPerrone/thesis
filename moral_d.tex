\chapter{On Moral Decision Making}
\thispagestyle{pprintTitle}


% Adjusting epigraph settings
\setlength\epigraphwidth{.8\textwidth}
\setlength\epigraphrule{0pt}
\renewcommand{\epigraphflush}{flushleft}
\renewcommand{\sourceflush}{flushright}

% Setting the font and spacing for the epigraph
\epigraph{\itshape \setstretch{1.2}But one thing is the thought, another thing is the deed, and another thing is the idea of the deed. The wheel of causality doth not roll between them.}{\small{Friedrich Nietzsche, \textit{Thus Spoke Zarathustra} (1883)}}

Analysing the concept of \textit{Moral Decision Making} in the context of predicate logic involves interpreting various linguistic elements within a logical framework. 

\begin{itemize}
	\item \textbf{The Word "Decision"}: In predicate logic, "Decision" can be a constant or a variable. 
	\begin{itemize}
		\item As a constant (for a specific decision), it might be represented as \( d \).
		\item As a variable (representing any decision), it could be denoted as \( x \), where \( x \) is a decision.
	\end{itemize}
	
	\item \textbf{The Noun Phrase "Decision Making"}: "Decision Making" can be interpreted as a function in predicate logic. 
	\begin{itemize}
		\item The function \( \text{DecisionMaking}(x) \) represents the output or consequence of making decision \( x \).
	\end{itemize}
	
	\item \textbf{The Adjective "Moral" in "Moral Decision Making"}: "Moral" is a modifier and can be viewed as a predicate.
	\begin{itemize}
		\item The predicate \( \text{Moral}(\text{DecisionMaking}(x)) \) indicates that the decision-making process of \( x \) is of a moral nature.
	\end{itemize}
\end{itemize}

A typical formula connecting these elements might be:

\[ \forall x (\text{Decision}(x) \rightarrow \text{Moral}(\text{DecisionMaking}(x))) \]

This formula can be interpreted as: "For all \( x \), if \( x \) is a decision, then the decision-making process of \( x \) is moral." It employs a universal quantifier (\( \forall \)) to express a general statement about all decisions.

In moral philosophy, these logical structures assist in defining and debating ethical theories and concepts, enabling a rigorous analysis of the nuances of moral decision-making.


The concept of \textit{Moral Decision Making} can be more accurately represented in predicate logic by considering that not all decisions are inherently moral, but rather, they become moral under certain conditions.

Consider the revised approach:

\begin{itemize}
	\item \textbf{Existential Quantification and Conditionality}: The formula should reflect that only some decisions fall under the category of moral decisions, contingent upon specific conditions being met.
\end{itemize}

A more realistic formula would be:

\[ \exists x (C(x) \rightarrow (\text{Decision}(x) \land \text{Moral}(\text{DecisionMaking}(x)))) \]

Here, \( C(x) \) represents the specific conditions under which a decision \( x \) can be considered moral. The formula is interpreted as: "There exists some decision \( x \) such that if the conditions \( C(x) \) are met, then \( x \) is a decision and the decision-making process concerning \( x \) is moral."

This formula acknowledges that morality in decision-making is not a universal attribute of all decisions, but rather a characteristic of certain decisions under specific circumstances. Identifying and analyzing these conditions \( C(x) \) is a key aspect of ethical philosophy and moral reasoning.




I want to precisely narrow down the meaning of the word \textit{Decision}, in the  	

In the discourse regarding the evolution of moral theories across philosophy and modern psychology, there emerges a nuanced interconnection where \textit{emotional} and \textit{rational} elements not only diverge but also integrate. This interconnection reveals the intricate complexities inherent in the formulation of moral models, transcending a mere dichotomy to embrace a more holistic, undefined perspective.

\begin{definition}[Rational Model]
	Moral decision making, is a cognitive process of choosing between competing moral judgments- \ie, mutually exclusive evaluations we make on what is right or wrong, good or bad, and that we use as motive, purpose and direction for our conscious, \gls{practical_behaviour}.
\end{definition}

\begin{definition}[Emotional Model]
	Moral decision-making is an emotive process, wherein individuals navigate and choose between competing moral judgments \ie, mutually exclusive evaluations we make on what is right or wrong, good or bad. This process is driven by emotional responses and intuitions, which guide and inform our conscious and practical behaviour, often preceding and shaping cognitive deliberation.
\end{definition}

Moral decision-making represents a cognitive exercise in the calculus of ethics. Within this framework of moral calculus, contemporary and classical scholars offer a spectrum of perspectives on the central role of emotional and cognitive faculties alike that incorporates both emotional and cognitive faculties.

It is worth delineating the concept of 'moral calculus' as distinct from \textit{hedonistic calculus}. While the latter term typically refers to Benthamite utility maximization, often quantified in terms of pleasure and pain, 'moral calculus' serves as a broader framework for ethical deliberation. Unlike hedonistic calculus, which is generally rooted in consequentialist traditions, moral calculus navigates the complexities of diverse ethical systems, be they deontological, virtue-based, or others.



\nextdiv
The philosophical canon profoundly integrates the conceptual distinction between emotion-driven and reason-driven moral philosophies. This differentiation has seen considerable evolution over centuries, leading to a significant impact on contemporary psychological thinking, especially in the sphere of moral psychology. The nuanced separation of emotion-driven and reason-driven moral frameworks, deeply rooted in philosophical discourse, has evolved extensively over time, culminating in its marked influence on modern psychological studies, with a particular focus on moral psychology. This journey begins with the foundational works of ancient philosophy. In this era, thinkers like Plato in his 'Republic' delineate a clear preference for reason over emotion in guiding ethical conduct. Aristotle, in his 'Nicomachean Ethics,' echoes this sentiment to some extent by emphasizing rational virtues, yet he also acknowledges the significant role emotions play in ethical existence.

Moving forward into the Enlightenment, this conceptual distinction was further crystallized. A paradigmatic figure of this era, Immanuel Kant, championed a morality firmly rooted in reason and universal maxims in his works, such as 'Critique of Pure Reason' and 'Groundwork for the Metaphysics of Morals,' thereby relegating emotions to a subsidiary role. This period marked a significant shift toward a rationalist perspective in moral philosophy.

However, this shift was met with a counterpoint in the British empirical tradition. Figures like David Hume presented a challenge to the Kantian rationalism. In his 'A Treatise of Human Nature,' Hume provocatively posited that reason is subordinate to passions, thereby anchoring moral judgments in emotional responses. This perspective from British empiricists highlighted the importance of emotions, or 'sentiments', in moral considerations, offering a contrasting view to the prevailing rationalist approach.

Moral decision making, is a cognitive process of choosing between competing moral judgments- \ie, mutually exclusive evaluations we make on what is right or wrong, good or bad, and that we use as motive, purpose and direction for our conscious, \gls{practical_behaviour}.

Moral decision-making is primarily an emotive process, wherein individuals navigate and choose between competing moral judgments \ie, mutually exclusive evaluations we make on what is right or wrong, good or bad. This process is driven by emotional responses and intuitions, which guide and inform our conscious and practical behaviour, often preceding and shaping cognitive deliberation.




\nextdiv
Hence, in the discourse of moral calculus, certain schools of thought, notably those propounded by Haidt (2012) and Greene (2007), assert with compelling vigour that the substratum of emotional faculties, rather than those of the cognitive domain, governs the architecture of ethical decision-making. This viewpoint finds a harmonic resonance in the philosophical canon, corroborated by seminal treatises such as Nussbaum's 'Upheavals of Thought' (2001) and Damasio's 'Descartes' Error' (1994).

\nextdiv
The theoretical edifice of moral calculus, while intellectually robust, gains tangible relevance when juxtaposed with empirical and phenomenological data. Bridging these domains allows for a more encompassing understanding of moral decision-making, marrying the abstract with \textit{the} concrete, \textit{the} theoretical with \textit{the} experiential.

For the empirical aspect:

Neuroscientific research offers valuable empirical insight into the machinery of moral cognition. Studies have implicated regions like the prefrontal cortex and the amygdala in the ethical decision-making process (Greene et al., 2001; Decety \& Cacioppo, 2012). These findings suggest that our 'moral calculus' may indeed have a tangible neurological substrate, grounding ethical theory in the biological realm.

For the phenomenological aspect:

Complementing these empirical observations, phenomenological accounts provide a subjective lens through which moral decision-making can be examined. Authors such as Sartre and Merleau-Ponty have explored the existential dimensions of choice, capturing the lived experience of moral deliberation (Sartre, 1943; Merleau-Ponty, 1945). 

These works serve to enrich our understanding of 'moral calculus' by infusing it with the subjective quality of human experience.

\paragraph{Emotion-Centered Models:} Some theories argue \cite{} that emotional processes, rather than cognitive ones, are at the core of moral decision making. Your definition may not adequately capture the emotive factors often considered essential. onathan Haidt's work in "The Righteous Mind" explores the role of emotional intuition in moral judgments, arguing that reasoning often follows, rather than guides, our moral intuitions~\cite{Haidt2012}

\textit{Practical behaviour} is a term widely used across philosophy and psychology, it's challenging to create an exhaustive chronological definition because the term does not correspond to a singular theory or concept that has evolved over time in a linear fashion. Instead, it has been interpreted and applied differently depending on the context, theoretical framework, or school of thought.

\nextdiv
Practical behaviour in philosophy: has been interpreted in various ways across different philosophical schools of thought. 1) In Aristotelian philosophy, practical behaviour is associated with "praxis" or action guided by moral virtue aimed at the good life. Practical wisdom ("phronesis") is crucial here as it guides one's decisions and actions in accordance with moral virtue. 2) Immanuel Kant distinguished between theoretical reason (used to understand the natural world) and practical reason (used to govern behaviour and moral decision-making). For Kant, practical behaviour is guided by the categorical imperative, an absolute moral law. 3) In the late 19th and early 20th century, the pragmatists (like William James and John Dewey) viewed practical behaviour as action informed by the effects that such behaviour would bring about. 

\nextdiv
Practical Behaviour in Psychology has been understood as observable actions and reactions to stimuli in behaviourism, deeply intertwined with internal cognitive processes during the cognitive revolution, and as a complex interplay of cognitive processes, emotional states, individual traits, and environmental influences in contemporary psychology. 1) In the behaviourist approach (Early 20th Century) pioneered by John Watson and B.F. Skinner, practical behaviour is understood in terms of observable actions and reactions to stimuli, often studied through conditioning processes. 2) With the cognitive revolution (Mid-20th Century), practical behaviour started to be seen as deeply intertwined with internal cognitive processes like problem-solving, decision-making, and planning. 3) Social-Cognitive Theory (Late 20th Century): Albert Bandura's social-cognitive theory emphasised the role of observational learning, self-efficacy, and goal setting in practical behaviour. \textbf{Modern times}: today, in ethics and action theory, practical behaviour typically refers to behaviour guided by practical reason, that is, reason concerned with action and decision-making. This involves deliberation about means and ends, moral obligations, and the values at stake in different courses of action. Similarly in Contemporary Psychology, practical behaviour is understood as a complex interplay of cognitive processes, emotional states, individual traits, and environmental influences. It is typically studied in context-specific terms, such as health behaviour, consumer behaviour, or prosocial behaviour.


This is quite an encompassing scope, but there are inevitable aspects of the broader discourse on moral decision making that we need to include in this purview. 

---
\textbf{the following needs to be integrated in the text}
---

\textcolor{orange}{The term "practical behaviour" is a broad one, encompassing a wide range of actions that an individual might take in their everyday life. These can range from simple behaviours like brushing teeth or driving to work, to more complex ones like making a significant decision about one's career or personal life."Moral behaviour," on the other hand, is a subset of practical behaviour. It refers specifically to actions that involve moral or ethical considerations. In other words, all moral behaviours are practical behaviours, but not all practical behaviours are moral behaviours. The distinguishing feature is the presence of moral or ethical considerations in the motivations, implications, or consequences of the action. So, in response to your question, the key to understanding the difference between "practical behaviour" and "moral behaviour" does indeed lie in understanding the specific meaning and implications of "behaviour" in these contexts. However, it's also crucial to consider the specific nature and context of the action itself, including the intentions behind it and its potential consequences. the key to understanding the difference between "practical behaviour" and "moral behaviour" does indeed lie in understanding the specific meaning and implications of "behaviour" in these contexts. However, it's also crucial to consider the specific nature and context of the action itself, including the intentions behind it and its potential consequences. Moral domain: A behaviour typically falls within the moral domain when it pertains to questions of right and wrong, fairness, justice, harm, and welfare. So, for instance, deciding to donate to charity falls within the moral domain because it involves considerations about the welfare of others. Playing the piano, on the other hand, would generally fall outside the moral domain because it's largely a personal interest or skill, not directly associated with the welfare or rights of others. Intention and motive: Moral behaviour often involves a level of intentionality, where the individual acts with a specific purpose or motive that is morally charged. An individual who donates to charity with the motive of helping others is engaging in moral behaviour. In contrast, an individual who plays the piano for personal enjoyment is engaging in a practical behaviour that isn't inherently moral or immoral. Consequences: The potential or actual impact of behaviour on others also plays a crucial role in determining its moral status. Behaviours with positive or negative impacts on others are often evaluated on a moral basis. \textbf{Additional notes}: 1) \textit{Interaction of factors determining behaviour:} In both philosophy and psychology, behaviour is viewed as a result of a complex interplay of multiple factors, including cognitive processes, emotional states, individual traits, and environmental influences. 1) Cognitive processes: Cognitive processes, such as perception, memory, decision-making, and problem-solving, play a critical role in practical behaviour. For example, decision-making theories, such as the Dual Process Theory, suggest that people use both intuitive (automatic, fast, and emotional) and deliberative (slow, controlled, and logical) systems in guiding their behaviours \cite{kahneman2011}. 2) Emotional states: Emotions can also guide our behaviour. The James-Lange theory of emotion suggests that our emotional experiences are a response to our bodily reactions to a stimulus. For example, we don't run away because we're afraid; instead, we're afraid because we see ourselves running\cite{james1884}. 3) Individual traits: Personality traits influence how individuals interpret and respond to their environment. The Big Five personality traits (openness, conscientiousness, extraversion, agreeableness, and neuroticism) have been linked to various behavioural outcomes. For instance, high levels of conscientiousness have been associated with better job and academic performance\cite{costa1992}. 4) Environmental influences: Social and physical environments shape behaviour. Social Cognitive Theory emphasises the reciprocal nature of this relationship: our behaviour can both influence and be influenced by our environment. For example, observational learning suggests we learn behaviours by observing others, while self-efficacy can determine how we respond to challenges\cite{bandura1986}. Modern psychology acknowledges that many behaviours are driven by processes outside of conscious awareness. For instance, implicit bias research shows that we often harbour unconscious biases that can influence our behaviour, including decision-making and interpersonal interactions\cite{greenwald2006}.Decision-making is not always rational and is often influenced by cognitive biases. For example, the 'availability heuristic' suggests that people are more likely to consider information that's easily retrievable when making decisions, which may not always lead to accurate or optimal outcomes\cite{tversky1973}.This interdisciplinary field combines psychology and economics to understand decision-making and behaviour. For example, the concept of 'nudge theory' suggests that subtle changes in how choices are presented can significantly influence decisions and behaviour, a principle that has been applied in various domains like healthcare, finance, and public policy\cite{thaler2008}. These insights suggest that our understanding of practical behaviour needs to be multifaceted, taking into account not only conscious, deliberate processes but also unconscious influences and the way cognitive biases and heuristics shape our decisions and actions. They also underscore the importance of considering the individual within their social and environmental context}

\nextdiv 
Moral decisions theories are often analysed into components features such as the model of judgment adopted- whether factual or normative, rational of affects laden-  its causes, and the ethical outset it seems to follow. All three components happen to be useful for identifying and organise methods and work done in Computational Ethics since they are easily linked to different scientific approaches adopted in the field, and their basis deeply root into both philosophical and psychological theories which have deeply inspired and implicitly shaped the objectives set for this field in the past two decades.

In particular, most modern philosophers have frequently written about the conflict between factual and normative judgments \cite{Black1972}, between reason and emotions \cite{Haidt2001}, and between normative and motivating reasoning \cite{Gert2020} three dichotomies.

\subsection{Normative Non-Ethical agents}
A moral decision is what we \textit{judged} necessary to resolve conflicts with an explicitly moral dimension via special type of judgements which often called \textit{normative} or \textit{value judgements}: responses to stimuli with a moral dimension. Normative judgements assert or deny what \textit{ought} to be the case whether or not it is \textit{actually} the case (see figure \ref{judgments}, page \pageref{judgments}), in contrast to factual judgements which assert or deny facts that \textit{are the case} or a properly justified believe.

Factual judgements assert or deny facts that \textit{are the case} or a properly justified believe, while normative judgements assert or deny what \textit{ought} to be the case whether or not it \textit{actually} is the case.

\begin{figure}[h]
    \centering
    \scalebox{0.65}{
    \includegraphics[width=\textwidth]{img/graphs/judgments.pdf}
    }
    \caption{This distinction presuppose a sufficient prior understanding of the relevant uses of \textit{is} and \textit{ought} (or \textit{should}) which will not discuss in details here. It is important to notice that, the presence of such marks as is neither a sufficient nor necessary criterion for the distinction we make, due to the striking variability of the relevant uses of the two words in every day language. For example, the sentence 'copper should be a metal' is not intended to be normative, and 'murder is evil' is not meant to be factual. Some philosophical theories claim that moral judgements lack of some desirable properties that factual statements have such as \textit{objectivity} or \textit{truth-apt}.}
    \label{judgments}
\end{figure}

\noindent
Judgements such as 'copper is a metal' \cite{Black1972} or '$2 + 2 = 4$' express what is the case because they are \textit{truth-apt} judgements which means that they are either true or false in there being some corresponding \textit{fact} which settles the question of their truth value. In simple terms, we cannot have different opinions on '$2 + 2 = 4$' because there exists a system of mathematical principles that, if accepted, makes us committed to the believe that '$2 + 2 = 4$': there are corresponding \textit{facts} that make these locutions true or false. 

\nextdiv
On the other hand, judgements such as 'innocents ought not to be punished' \cite{Black1972} or 'wrongful killing is always wrong' are judgments about what \textit{ought} to be the case but that do not have any corresponding fact that makes it true or false. Can machines grasp the difference between the two? In contrast with other more empirical judgments, moral judgements seem to have an intrinsic connection to motivation and action, for they form in us a uniquely bonding intentions to perform a behaviour, and motivate us to act in accordance with it \cite{Rosati2016}.

\nextdiv
Moor in \cite{Moor2011} was one of the fist to examine how this distinction is relevant for a predominate class of works in Machine Ethics. Moor noticed that ordinary computers are designed with a purpose in mind, they are \textit{normative agents} in the sense that they perform something on our behalf, executing rule-based instructions of which efficacy can be assessed assessed. However, ethical agents are those that perform actions with an ethical impact (positive or negative), but not by being constrained by their designers as this would not count has ethical act by the the very same definition of Ethics.


\subsection{Other}
Hence, genuine moral decisions must have as 'end-product', actions or inactions. In The Language of Morals \cite{Hare1991}, R.M. Hare, one of the leading British moral philosopher of the twentieth century, gives this clear characterisation:

\blockquote[\cite{Hare1991}]{If we were to ask of a person "What are his moral principles?" the way in which we could be most sure of a true answer would be by studying what he did. He might, to be sure, profess in his conversation all sorts of principles, which in his actions he completely disregarded; but it would be when, knowing all the relevant facts of a situation, he was faced with choices or decisions between alternative courses of action, between alternative answers to the question "What shall I do?", that he would reveal in what principles of conduct he really believed. The reason why actions are in a peculiar way revelatory of moral principles is that the function of moral principles is to guide conduct.}

\nextdiv
By the same token, the main objective of Machine Ethics is to develop implicit ethical agents that is to say, machines that have been programmed in a way that can decide on actions with an ethical impact on their environment. Machine Ethics revolves around a precise subset of decision-type, since not all decisions have a moral dimension, and therefore not all types of judgments are relevant to morality. For example, whether I should get a frosty cold drink on a hot day using my last pound is not a moral decision. Whether I should use my last pound to get a cold drink, or give it to the women begging for money, appears to be. Both are instances of decision concerned with actions, they drive \textit{goal-oriented behaviours} in which our perceptual and memory system support decisions that determine our \textit{actions} \cite{Gazzaniga2006}.

\section{Extended Types of Judgments}

Typically, judgments are classified into two primary types: factual (descriptive or necessary) and normative (prescriptive or contingent). However, these categories, though fundamental, may not fully encompass the range of judgments humans engage with. Various disciplines, including philosophy, psychology, and mathematics, suggest other classifications or subcategories. 


While factual and normative categories provide a foundational classification of judgments, the diversity and complexity of human thought suggest the utility of additional categories. The appropriateness of any specific set of categories, however, depends on the nature of the subject matter and the research questions at hand.

\begin{enumerate}
    \item \textbf{Value Judgments:} Value judgments focus on the worth, importance, or intrinsic merit of a subject. As a subset of normative judgments, they often pertain to ethical or moral dimensions. However, their unique emphasis on 'value' might warrant a separate consideration.
    \item \textbf{Aesthetic Judgments:} Aesthetic judgments concern beauty or other aesthetic attributes. Although they might be regarded as a form of value or normative judgments, the discipline of aesthetics often treats them as a distinct category due to their specialised focus.
    \item \textbf{Prudential Judgments:} Prudential judgments, often used in economics, decision theory, and practical ethics, consider what is prudent or practically wise. These judgments typically involve an interplay of both descriptive and normative elements.
    \item \textbf{Probabilistic Judgments:} Probabilistic judgments, prevalent in statistics, psychology, and decision theory, assess the likelihood or probability of a given event or condition. They often require a balance between empirical data and theoretical models.
    \item \textbf{Counterfactual Judgments:} Counterfactual judgments, commonly used in philosophy and cognitive psychology, speculate on alternate realities or conditions. These judgments often hinge on the ability to imagine and reason about hypothetical situations.
    \item \textbf{Analytic Judgments:} In Kantian philosophy, analytic judgments are those in which the predicate concept is included within the subject concept. These judgments are typically tautological and contrast with synthetic judgments.
    \item \textbf{Synthetic Judgments:} Kant also proposed synthetic judgments, wherein the predicate concept is not contained within the subject concept. They can be classified further into a priori (based on reasoning independent of experience) and a posteriori (based on experience).
\end{enumerate}


for example, the judgments made in physics, like those in other scientific disciplines, can be seen to fall into several categories depending on the specific context. Much of the work in physics involves making \textit{descriptive (factual) judgments} about the nature of the physical world. These judgments are usually based on observation and experimentation and aim to accurately describe how the world is. While less common in physics than in other fields such as ethics, \textit{prescriptive (normative) judgments} are sometimes made in the context of methodological rules about how to do physics. Physics often involves making \textit{probabilistic judgments}. In quantum mechanics, the behaviour of particles is often described in terms of probabilities rather than definite outcomes. Physicists also frequently make \textit{counterfactual judgments}, considering what would happen under different hypothetical scenarios. The distinction between \textit{analytic and synthetic judgments} is also relevant in physics. An example of an analytic judgment in physics might be a mathematical truth that holds by definition within a certain model, while a synthetic judgment might be a statement about the physical world that is supported by empirical evidence.

 The judgments made in physics, like those in other scientific disciplines, can be seen to fall into several categories depending on the specific context. Drawing upon Chalmers' work on the philosophy of science \cite{chalmers2013}, we find that much of the work in physics involves making \textit{descriptive (factual) judgments} about the nature of the physical world. These judgments are usually based on observation and experimentation and aim to accurately describe how the world is.  While less common in physics than in other fields such as ethics, \textit{prescriptive (normative) judgments} are sometimes made in the context of methodological rules about how to do physics, a concept explored by Laudan in his work on normative naturalism \cite{laudan1987}.  Physics often involves making \textit{probabilistic judgments}. In quantum mechanics, the behaviour of particles is often described in terms of probabilities rather than definite outcomes \cite{bricmont2016}.  Physicists also frequently make \textit{counterfactual judgments}, considering what would happen under different hypothetical scenarios, a concept explored in the work of Woodward \cite{woodward2007}.  The distinction between \textit{analytic and synthetic judgments} is also relevant in physics. Drawing on Bird's exploration of Kuhn's philosophy \cite{bird2000}, we see that an example of an analytic judgment in physics might be a mathematical truth that holds by definition within a certain model, while a synthetic judgment might be a statement about the physical world that is supported by empirical evidence.   In practice, many judgments in physics may involve a combination or an interplay of these types. The specific context and objectives of the work play a large role in determining which types of judgments are most relevant.


In practice, the types of judgments made in physics often involve a mixture of these categories. For instance, a descriptive judgment about the behaviour of a particle might be based on a combination of observation (a synthetic judgment) and mathematical reasoning (often involving analytic judgments). Thus, the understanding and classification of judgments in physics, like in other fields, benefit from a nuanced approach. 

In a field such as Computer Science, a discipline that often intersects with logic, mathematics, and engineering, several types of judgments can be identified. Much of the work in computer science involves making \textit{descriptive (factual) judgments}. These often take the form of specifying the behaviour of algorithms or systems, such as a judgment about the time complexity of a particular sorting algorithm. \textit{Prescriptive (normative) judgments} are also found in computer science, often relating to best practices for coding, architectural decisions in system design, or ethical considerations in AI development. \textit{Analytic judgments}, where the predicate is contained within the subject, often emerge from logical deductions that follow from the definition of a concept or operation. \textit{Synthetic judgments}, which refer to empirical findings that don't just follow from definitions, might involve observations about the performance of certain algorithms in specific contexts. Especially in areas like machine learning and algorithm analysis, computer scientists often make \textit{probabilistic judgments}, like assessing the probability of a certain outcome given a set of inputs. In troubleshooting, system design, or in planning the development process, \textit{counterfactual judgments} often play a significant role as computer scientists consider alternate scenarios or possibilities. 

The rise of fields such as AI ethics and Human-Computer Interaction (HCI) has brought attention to \textit{value judgments} in computer science. These might concern what constitutes fair treatment in an algorithm's decision-making process, for example. In practice, many judgments in computer science may involve a combination or an interplay of these types. The specific context and objectives of the work play a large role in determining which types of judgments are most relevant.

\section{A definition of judgment}

So, what is \textit{judgment}? 

A \textit{judgment} has been defined differently across various fields. From a logical and mathematical perspective, it carries specific interpretations. In formal logic, a judgment is typically understood as an assertion that a proposition is true. This idea can be represented as follows:
\[
J(P)
\]

Here, \(J\) denotes the judgment operation and \(P\) is a proposition. The entire expression, \(J(P)\), is read as "\textit{it is judged that \(P\)}".


In mathematics, a judgment can be considered akin to a function. If we think of a judgment as mapping from a set of premises to a conclusion, we can represent it in a similar way to a function:

\[
J : P \rightarrow C
\]

Here, \(J\) is the judgment, \(P\) represents the premises, and \(C\) is the conclusion. This can be understood as a judgment \(J\) mapping a set of premises \(P\) to a conclusion \(C\). Note, however, that this is a rather abstract and non-standard interpretation. Judgments in mathematics and logic are more typically represented as statements or propositions that are asserted to be true. German logician Gottlob Frege's work in the field of logic provides valuable insight into the concept of judgment. His Begriffsschrift, or concept script, was a formal language of logic devised to represent clear, logical thoughts. In Frege's system, judgments about a proposition can be symbolically expressed and manipulated.
