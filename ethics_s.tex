\section{From Moral Cognition to Ethical Theory}
\chapter{Ethical Theory in a Cognitive--Topological Framework}
\label{chap:ethics_s}
\thispagestyle{pprintTitle}


We often speak as if our moral judgements were the visible tips of ethical
theories—small verdicts issued by an inner court of principles. But this picture
can be misleading. Much of the time, moral judgement does not arrive as a
conclusion reached by explicit reasoning; it arrives as a sense of direction
(see Chapter~\ref{chap:moral_primer}). Something feels called for, something else
recedes, and action may follow before justification enters the scene. Long
before we ask whether a judgement is coherent, consistent, or defensible, it may
already have done its practical work.

This chapter begins from that observation. It treats moral judgements not as
miniature moral philosophies enacted in behaviour, but as first-order evaluative
outputs: psychologically real responses generated by the cognitive–affective
machinery through which agents register what matters in a situation. These
judgements are structured, responsive, and often reliable—but they are not
required, at this stage, to satisfy the standards later imposed by ethical
reflection.

Only once this distinction is held in view does a deeper question come into
focus. If moral judgements tend to arise upstream of reasoning, embedded in
perception, affect, and social context, then they are also exposed. They may be
stabilised, amplified, or quietly deflected by features of the environment that
never present themselves as reasons at all. Understanding moral life therefore
requires more than asking what principles we endorse. It requires asking how
evaluative force is generated in the first place—and how it might be altered
before deliberation begins.

In particular, this chapter addresses a question left deliberately open by the
empirical core of the thesis:
\bigskip
\noindent
\begin{center}
	\begin{leftbar}
		\textit{How should ethical theory be situated once moral judgement is no
			longer treated as the output of structured deliberation, but as the product
			of distributed, context-sensitive evaluative processes?}
	\end{leftbar}
\end{center}
\bigskip
\noindent
The preceding chapters analysed moral judgements as first-order evaluative
outputs: psychologically real assessments generated by the cognitive–affective
mechanisms through which agents register morally salient features of their
environment. These outputs were shown to be empirically tractable and
behaviourally expressive, yet not required to exhibit internal coherence,
principled justification, or reflective endorsement. They constitute the data of
moral life, not its justification.

Ethical theories, by contrast, operate at a different Level of Abstraction.
Rather than describing how moral judgements are generated, they articulate
second-order frameworks through which such judgements are interpreted,
evaluated, and disciplined. Deontological, consequentialist, virtue-theoretic,
sentimentalist, contractualist, particularist, and pluralist traditions do not
model cognitive mechanisms; they propose justificatory structures, normative
constraints, and value orientations aimed at rendering moral practice
intelligible and accountable.

From this perspective, the tendency of much ethical theory to treat moral
judgement as the outcome of structured deliberation is not an empirical claim
but a normative and methodological commitment. It expresses an aspiration about
moral authority—how moral judgements ought to be grounded—rather than a
description of how they ordinarily arise. The tension between this aspiration
and the empirical texture of moral cognition, made explicit earlier in the
thesis, does not invalidate ethical theory; it relocates it.

The task of this chapter is therefore neither to test ethical theories against
psychological data nor to recast them as implementable decision procedures.
Instead, ethical traditions are reconstructed here within a
cognitive–topological framework that respects their proper Level of Abstraction.
They are treated as reflective structures: patterns of constraint, value
gradients, dispositional emphasis, and affective orientation that organise how
moral phenomena are interpreted, stabilised, and justified, rather than how
they are produced in real time.

\medskip

The purpose of this chapter is therefore explicitly methodological. Its function is to attempt to reconstruct major normative frameworks at a Level of
Abstraction compatible with the cognitive–topological model developed in the
thesis, thereby clarifying how ethical theory can inform the interpretation of
empirical moral phenomena without being conflated with their psychological
generation. By treating ethical traditions as reflective structures rather than
behavioural algorithms, the chapter provides normative coordinates that can be
used to assess the experimental perturbation ethically rather than merely
descriptively, and to situate the thesis’s claims within a coherent philosophical
landscape.

This distinction is methodologically crucial for computing science, Machine
Ethics, and Human–AI interaction. Treating first-order evaluative outputs as if
they were principled commitments invites over-interpretation; designing
artificial systems around abstract norms that human agents do not reliably
instantiate risks misalignment. What is required instead is a framework that
preserves normative coherence while remaining faithful to the cognitive
architecture of moral judgement.

The sections that follow pursue this task by reconstructing major ethical
traditions at the Level of Abstraction appropriate to the aims of the thesis. In
doing so, they provide the normative coordinates required to interpret the
experimental perturbation ethically rather than merely descriptively, and to
situate the thesis’s hypotheses within both normative theory and cognitive
science.


\section{Introduction: Why Ethics Needs Psychology (and Why Computing Science Needs Both)}

\noindent
Classical ethical theory has typically approached moral judgement as the outcome of
structured deliberation: a process governed by reasons, principles, and
normatively defensible commitments~\cite{Kant1785,Mill1861,Rawls1971,Scanlon1998}. As the earlier chapters of this thesis have suggested, this picture is descriptively incomplete. In everyday moral life, action often does not follow extended reflection. It appears instead to emerge from rapid, affectively mediated evaluations shaped by perceptual salience, social context, and embodied interaction (refer to chapters~\ref{chap:lit_rev} and \ref{chap:moral_primer}). What agents \emph{ought} to do, what they \emph{say} they did, and what they \emph{actually} do can diverge in
systematic ways.

\medskip
\noindent
This divergence is not merely a problem for moral psychology; it poses a
structural challenge for ethical theory itself. \textit{If moral behaviour is generated through first-order cognitive--affective processes, then ethical frameworks
cannot be straightforwardly understood as mechanistic drivers of action}. They
operate at a different Level of Abstraction: not as causal engines, but as
second-order interpretive and justificatory structures. This distinction has
guided the experimental work of the thesis (Chapter~\ref{chap:exp_methods}). Arguably, the effects examined were not changes in articulated moral belief, but shifts in the practical expression of moral cognition under conditions of synthetic ambiguity.

\medskip
\noindent
Artificial systems are not just reasoning in the abstract; they are entering our 
environments. They are in phones, homes, classrooms, and offices. Their presence 
affects how we behave, how we interpret situations, and how we allocate attention%
~\cite{Rahwan2019,Weidinger2021,Bender2021}. Research in Social Signal Processing and Affective Computing has shown that human moral cognition is sensitive to subtle cues—gaze, posture, spatial orientation, and embodied co-presence—that structure the interaction order in which meaning is negotiated~\cite{Picard1997,Vinciarelli2009,Conty2016,Goffman1967}. When synthetic 
entities enter this order, they do not introduce new moral rules; they can reshape 
the informational and affective landscape in which moral appraisal unfolds~\cite{Breazeal2003,Coeckelbergh2010,Malle2016}.


\medskip
\noindent
The central claim motivating the experimental work follows from this
perspective: \emph{moral behaviour appears sensitive to the topology of the
	immediate perceptual--social environment}. If moral judgement arises through
intuitive appraisal and attentional weighting, then even a silent,
behaviourally neutral synthetic presence may perturb the evaluative pathway
from moral salience to action. The empirical results of the thesis provide
controlled evidence consistent with this possibility, indicating that robotic
co-presence can be associated with attenuated prosocial behaviour even in the
presence of a strong moral cue.

\medskip
\noindent
When viewed through ethical theory, this pattern has important consequences.
Ethics, understood as a second-order discipline, does not generate moral
judgements; it seeks to interpret, justify, and constrain them. Machine Ethics
has at times blurred this distinction by treating normative principles as if
they could be implemented directly as behavioural algorithms. The experimental
findings challenge this tendency. They suggest that moral influence may operate
primarily through salience modulation and environmental structuring, rather
than through explicit rule-following or deliberative reasoning alone.

\medskip
\noindent
The aim of the present chapter is therefore methodological rather than
encyclopaedic. It does not adjudicate between competing ethical theories, nor
does it propose a new normative system. Instead, it reconstructs major ethical
traditions at a Level of Abstraction compatible with the cognitive--topological
model developed in the thesis. Ethical theories are treated here as reflective
structures—patterns of constraint, value orientation, and justificatory
organisation—that inform how moral phenomena are interpreted, rather than how
they are generated.

\medskip
\noindent
By situating ethical theory, Moral Psychology, and Computing Science within a
shared conceptual framework, this section should provide normative coordinates that can be used to assess the experimental perturbation ethically rather than merely
descriptively. In doing so, it clarifies why synthetic presence matters not
because artificial systems reason morally, but because they participate in—and
can reshape—the evaluative fields through which human moral judgement takes
form.

\section{Ethical Theory as Second-Order Analysis}
\label{sec:second_order_ethics}

\noindent
The experimental results developed earlier in the thesis rest on a crucial
methodological distinction: the difference between the mechanisms that generate
moral behaviour and the frameworks used to interpret or justify it. The present
section makes that distinction explicit. Ethical theory, as it is understood
here, does not operate at the same Level of Abstraction as moral cognition
itself. Confusing these levels has repeatedly led to conceptual difficulties in
both philosophical ethics and computational modelling~\cite{Black1972,Hare1981,Hempel1965,Floridi2008,Moor2006,FloridiSanders2004,McLaren2006,Coeckelbergh2023}.

\medskip
\noindent
The analyses in this thesis treat moral behaviour as the output of a
cognitive--affective evaluative field, shaped by perceptual salience,
dispositional structure, and environmental perturbations. Ethical theory does
not explain how this field operates; rather, it reflects on how its outputs may
be understood, assessed, or constrained. This difference is not merely
terminological. It determines \textit{which claims are subject to empirical testing},
which are subject to normative evaluation, and which cannot plausibly do both.

\subsection{Ethical Reflection and the Second-Order Stance}

\noindent
First-order moral judgements arise from the cognitive architecture examined in~\ref{chap:moral_primer}. They are psychologically instantiated, context-sensitive,
and behaviourally measurable. Their structure reflects processes of perceptual
weighting, affective appraisal, intuitive heuristics, and socially mediated
meaning-making. These judgements constitute the \emph{phenomena} examined and
perturbed by the experimental work of the thesis.

\medskip
\noindent
Second-order ethical theory occupies a different position. It is reflexive
rather than generative. Its questions concern justification rather than causal
production: \emph{What counts as a reason? What makes an obligation binding? What
	norms govern responsibility and moral evaluation?} Addressing these questions
requires abstraction and generalisation, but not the direct modelling of the
psychological mechanisms that produce moral responses~\cite{Haidt2001,Greene2001,Young2012,Kohlberg1969,Narvaez2005,Kahneman2011,Baumeister2010}.

\medskip
\noindent
This distinction has a long pedigree. Sidgwick’s separation between the
psychology of moral sentiment and the ``methods'' of determining right conduct
already marked ethics as a second-order enterprise~\cite[Book~I]{SidgwickMethods}.
Analogously, contemporary epistemology distinguishes the psychology of belief
formation from the normative assessment of justification~\cite{Lemos2020}.
Ethics stands to moral judgement as epistemology stands to belief: it evaluates,
rather than generates, its object.

\medskip
\noindent
Recognising this separation is essential for the thesis as a whole. The
experimental findings do not challenge ethical theories by falsifying their
principles, nor do they vindicate any particular normative system. Instead,
they indicate that moral behaviour is patterned by topological features of the
evaluative field that ethical theory typically presupposes but does not model.
Treating ethical principles as if they were causal operators—whether in
philosophical explanation or in computational implementation—constitutes a
category error. Ethical theory provides interpretive structure, not behavioural
machinery.

\medskip
\noindent
With this second-order stance clarified, the next section turns explicitly to
Floridi’s Levels of Abstraction framework. That framework supplies the
methodological tool needed to keep cognitive mechanisms, evaluative topology,
and normative interpretation analytically distinct while allowing them to be
integrated within a single coherent account.


\subsection{Levels of Abstraction and the Proper Location of Ethical Explanation}

\noindent
The distinction between first-order moral cognition and second-order ethical
reflection becomes methodologically precise through Floridi’s framework of
\emph{Levels of Abstraction} (LoA)~\cite{Floridi2008,Floridi2010}. Any explanatory
account selects an LoA defined by its observables, its conceptual resolution, and
the kinds of questions it can meaningfully address. Ethical theory and moral
cognition do not merely occupy different LoAs; they respond to different
explanatory demands.

\medskip
\noindent
At the \textbf{cognitive LoA}, the explananda are psychologically realised
processes: perceptual salience, affective appraisal, intuitive heuristics, social
cue integration, and the temporal dynamics through which these elements shape
action. These processes are causally efficacious and, in principle, manipulable
and measurable in experimental settings. This is the LoA at which the present
thesis intervenes empirically.

\medskip
\noindent
At the \textbf{normative LoA}, by contrast, the objects of analysis are principles
of justification, conceptions of obligation, and standards of responsibility.
These are not causal operators. They organise interpretation and evaluation, not
behavioural generation. Ethical theory therefore operates as a reflective
framework: it evaluates the \emph{outputs} of moral cognition without specifying
the mechanisms by which those outputs arise.

\medskip
\noindent
Much of classical Machine Ethics implicitly collapsed these LoAs. By treating
deontic principles, utility functions, or virtue schemas as if they were
behaviour-generating mechanisms, early systems attributed causal relevance to
abstractions that are not behaviourally operative~\cite{Arkin2009,Anderson2007,Anderson2011,Bringsjord2006,Ganascia2007,Abel2016,Powers2006,Thornton2013}. From an
LoA perspective, this move is not merely incomplete but conceptually unstable:
it attempts to engineer behaviour by manipulating constructs that belong to a
justificatory rather than mechanistic register.

\medskip
\noindent
LoA discipline therefore becomes essential. Behavioural explanation requires the
cognitive LoA; normative assessment requires the ethical LoA. Neither reduces to
the other, yet neither is independent. Ethical evaluation presupposes a
psychology capable of rendering moral salience effective, while empirical
findings can constrain which normative ideals can plausibly guide action.

\medskip
\noindent
This interdependence links the present chapter to both the analysis of moral
cognition in Chapter~\ref{chap:moral_primer} and the experimental results
developed earlier in the thesis. Those chapters indicate that moral cognition is
\emph{topologically structured}: it unfolds within an evaluative field shaped by
salience gradients, affective weighting, and interpretive dynamics. Perturbations
to this field—such as the introduction of a synthetic presence—may redirect
behaviour even when explicit normative commitments remain unchanged.

\medskip
\noindent
Seen through the LoA framework, the central research question of the thesis can be
restated with greater precision:
\begin{quote}
	\emph{How do psychological mechanisms, environmental structures, and normative
		expectations jointly shape the transition from moral perception to moral
		action?}
\end{quote}

\noindent
Addressing this question requires neither collapsing ethics into psychology nor
treating ethical principles as causal drivers. It requires a representational
structure capable of linking first-order mechanisms with second-order
interpretation without conflation. The remainder of this section argues that the
notion of \textbf{evaluative topology}, developed throughout the thesis, provides
such a bridge.
\paragraph{Topological Consequences for Moral Perturbation.}

\noindent
The evaluative–topological framework developed earlier holds that moral
behaviour does not arise from isolated judgements or discrete rules, but from
movement within a dynamically structured evaluative field. Within such a field,
\emph{perturbation} has a precise meaning: modifications that alter salience
gradients, affective weighting, or attractor structure are expected to change
the distribution of behavioural trajectories available to the agent.

\noindent
From this perspective, the presence of a synthetic entity that is perceptually
social yet ontologically indeterminate is not a neutral background feature. It
can function as a \emph{topological operator}: reshaping the field within which
moral cues acquire evaluative force. Crucially, this influence does not depend on
explicit action, instruction, or normative content. It operates by reorganising
the geometry of attention, interpretation, and affective readiness through which
moral meaning becomes behaviourally operative.

\noindent
This is the theoretical motivation underlying the experimental design in
Chapter~\ref{chap:exp_methods}. By embedding a morally salient cue (the
Watching--Eye stimulus) within an evaluative field perturbed by humanoid robotic
co-presence, the experiment examined whether minimal topological deformation is
sufficient to attenuate prosocial action. The empirical results are consistent
with this possibility.

\medskip

\paragraph{Reasoned Synthesis: Positioning the Argument.}

\noindent
At this point, the architecture of the thesis can be stated compactly. Moral
judgement operates at the cognitive Level of Abstraction through affectively
responsive and socially sensitive mechanisms. Ethical theory operates at the
normative Level of Abstraction, providing justificatory structure rather than
behavioural causation. Evaluative topology links these orders by modelling how
structural features of the environment constrain and redirect the transition from
moral perception to moral action.

\noindent
This bridge is indispensable for interpreting the experimental perturbation.
Without it, one could describe \emph{what} changed under synthetic presence, but
not \emph{how that change should be understood}. Topology allows the attenuation
effect to be interpreted neither as a failure of virtue nor as a violation of
principle, but as a reconfiguration of the evaluative field within which
normative cues are processed.

\noindent
With this scaffolding in place, the task of the chapter becomes clear.

\medskip
\noindent
\begin{center}
	\begin{leftbar}
		\textit{To assess the ethical significance of the observed perturbation, the observed perturbation of moral behaviour produced by synthetic presence must be situated within the major normative traditions that articulate different conceptions of value, obligation, and moral authority.}
	\end{leftbar}
\end{center}
\medskip
\noindent 
The next section therefore reconstructs these traditions through the combined lens of Levels of Abstraction and evaluative topology, preparing the conceptual ground for ethical interpretation of the empirical findings.

\section{The Normative Landscape: Structuring Ethical Theories Through LoA and Topology}
\label{sec:normative_landscape}

\noindent
With the methodological apparatus now in place, the major normative frameworks
relevant to the interpretation of the behavioural findings can be introduced.
The aim of this section is not encyclopaedic survey but conceptual
reconstruction: each theory is presented in a form that preserves its
philosophical integrity while locating it at the appropriate Level of
Abstraction and within the evaluative–topological architecture developed across
the thesis~\cite{Floridi2008,Rawls2020,Scanlon1998,Korsgaard2009}.

\noindent
Two constraints guide this reconstruction. First, philosophical fidelity: the
theories must remain recognisably Kantian, Aristotelian, utilitarian, or
contractualist as traditionally understood~\cite{Kant1785,AristotleNE,Mill1861,SidgwickMethods,HumeTreatise,SmithTMS}. Second, integrative compatibility: they
must be articulated in a manner that allows principled engagement with empirical
accounts of moral cognition and evaluative topology developed earlier in the
thesis~\cite{Haidt2001,Greene2001,Churchland2011,Narvaez2005}.

\noindent
The goal is therefore not to catalogue doctrines, but to map the structural logic
of normativity in a way that clarifies how different ethical frameworks
constrain, interpret, and respond to the field-level perturbations associated
with synthetic presence.

% -----------------------------------------------------------
\subsection{The Three Dimensions of Normative Analysis}

\noindent
Normative theories diverge not only in their substantive moral claims, but in the
\emph{structure of normativity} they presuppose~\cite{Scanlon1998,Korsgaard2009,Wallace2012}.
To make these differences analytically tractable—and compatible with the
cognitive–topological framework developed in this thesis—it is useful to
distinguish three orthogonal dimensions along which ethical theories can be
organised. Each corresponds to a structural feature of the evaluative field and
to a distinct Level of Abstraction.

\medskip

\noindent
First, theories differ in their \textbf{source of normativity}: the locus from
which justificatory authority is taken to arise. This may be rational agency
(Kant~\cite{Kant1785}), human flourishing (Aristotle~\cite{AristotleNE}), aggregated
welfare (Mill, Sidgwick~\cite{Mill1861,SidgwickMethods}), affective sentiment (Hume,
Smith~\cite{HumeTreatise,SmithTMS}), or interpersonal justification (Scanlon~\cite{Scanlon1998}).

\noindent
Second, they differ in their \textbf{mode of evaluation}: which features of action
or character are treated as morally salient—maxims, outcomes, virtues, motives,
relationships, or context-sensitive particulars~\cite{Dancy2004,Foot1978,Rawls2020}.

\noindent
Third, they differ in their \textbf{mechanism of action-guidance}: how evaluation
is taken to become behaviour—through categorical constraints, optimisation,
trained perception, affective resonance, or justificatory equilibrium~\cite{McDowell1979,Korsgaard2009,Scanlon1998}.

\medskip

\noindent
When expressed along these dimensions, classical ethical theories can be
reconstructed as distinct \emph{evaluative topologies}: structured ways of
configuring the moral field.

\noindent
Kantian ethics imposes deontic invariants that partition the field into sharply
bounded regions of permissibility~\cite{Kant1785,Korsgaard2009}. Consequentialism
defines a gradient field over outcomes, along which action is guided by expected
welfare~\cite{Mill1861,SidgwickMethods,Railton1984}. Virtue ethics organises moral
space around dispositional attractors that shape perceptual and affective
sensitivity~\cite{AristotleNE,Foot1978,McDowell1979}. Sentimentalist theories trace
morality along affectively weighted pathways~\cite{HumeTreatise,SmithTMS,Prinz2004}.
Contractualism structures the field through justificatory equilibria among
agents~\cite{Scanlon1998,Rawls2020}. Particularism resists fixed global structure,
treating moral relevance as locally emergent~\cite{Dancy2004}.

\medskip

\noindent
This reconstruction does not collapse normative theory into Psychology. Rather,
it provides a shared representational language in which ethical theory and moral
cognition can be jointly articulated: theories that diverge in content become
comparable in structural terms—how they configure salience, constrain
trajectories, and mediate the transition from judgement to action~\cite{McDowell1979,Korsgaard2009,Railton2017}.

% -----------------------------------------------------------
\subsection{Why This Framework Matters for the Experimental Results}

\noindent
This normative–topological framework is not an abstract addendum; \textit{it is required to interpret the experimental findings ethically rather than merely descriptively.}
The behavioural question examined in this thesis—whether robotic co-presence
attenuates prosocial donation—cannot be meaningfully assessed without some
account of how moral cues acquire force within a structured evaluative
field~\cite{Haidt2001,Greene2001,Churchland2011}.

\medskip

\noindent
Three structural implications follow.

\noindent
First, moral action depends on the configuration of the evaluative field: across
normative traditions, behaviour is understood as emerging from structured
relations of salience, value, and constraint, rather than from arbitrary
choice~\cite{AristotleNE,Korsgaard2009,Scanlon1998}.

\noindent
Second, synthetic presence perturbs this field not by introducing new reasons,
but by modulating salience, attention, and affective resonance. A humanoid robot
alters the environment in which reasons may become behaviourally
operative~\cite{Pentland2007,Conty2016,Dear2019,Zlotowski2015}.

\noindent
Third, the ethical significance of such perturbations can only be assessed if
normative theories are situated within the joint discipline of Levels of
Abstraction and evaluative topology. Without this, one can report behavioural
change without being able to say what the change \emph{represents} in normative
terms.

\medskip

\noindent
The function of the present chapter is therefore methodological and
interpretive. It establishes the normative coordinates required to read the
experimental attenuation as a shift within a moral landscape rather than as a
mere statistical effect. With this scaffolding in place, the following sections
reconstruct the major normative traditions—not as implementable rule systems,
but as structured topologies of normativity embedded within human
cognitive–affective architecture.

%%%%%%%%%%%%%%%%%%%%%%%%%%%%%%%%%%%%%%%%%%%%%%%%%%%%%%%%%%%%%%%%%%%%%%%%%

\section{Deontological Structures: The Architecture of Practical Reason}
\label{sec:deontology}

\noindent
The preceding sections clarified the methodological stance of this chapter:
ethical theories are interpreted neither as psychological models nor as
behavioural algorithms, but as \emph{second-order structures} that organise moral
evaluation at a distinct Level of Abstraction. Deontological ethics provides a
particularly clear case for this approach. It is often misrepresented—especially
in Machine Ethics—as a system of action-guiding rules, when its normative force
lies elsewhere~\cite{Kant1785,Korsgaard1996,ONeill1985,Moor2006,Anderson2011}.


\medskip

\noindent
The aim of this section is therefore not to explain moral behaviour
deontologically, but to reconstruct deontic normativity in a way that preserves
its philosophical identity while making explicit how it constrains the
\emph{evaluative field} within which moral cognition unfolds. This reconstruction
is required to interpret the experimental results of the thesis without
committing a category error: the observed behavioural attenuation must be
assessed relative to deontic structure, rather than mistaken for its violation.

\medskip

\noindent
Three constraints guide the reconstruction. First, \textbf{philosophical
	fidelity}: deontological commitments must remain recognisably Kantian. Second,
\textbf{LoA discipline}: deontic principles are not treated as psychological
mechanisms or behaviour-generating procedures. Third, \textbf{topological
	embedding}: duties are represented as structural constraints on evaluative
trajectories, not as causal operators.

\medskip

\noindent
Under this lens, deontology identifies \emph{invariant structures} in the moral
field: boundaries of permissibility and prohibition that regulate justificatory
coherence. These invariants operate at a reflective LoA. They do not generate
action; they delimit which actions can be justified.

% -----------------------------------------------------------
\subsection{Rational Agency and the Form of Law}

\noindent
For Kant, moral authority arises from the structure of rational agency itself.
The categorical imperative functions as a test of \emph{justificatory form}—
whether a maxim can be willed as universal law—not as a psychological process
for producing behaviour~\cite{Kant1785,Korsgaard2009,Allison2011}. Its role is
evaluative and reflective. It belongs to a Level of Abstraction distinct from
the cognitive–affective mechanisms analysed in
Chapter~\ref{chap:moral_primer}.

\medskip

\noindent
This distinction is decisive for the present thesis. Much of classical Machine
Ethics misread universalisability as a procedural decision rule, implementing
deontic principles as if they were executable algorithms
\cite{Anderson2007,Anderson2011,Bringsjord2006,Govindarajulu2017,Ganascia2007,Arkin2009}.
But Kant did not propose that moral agents \emph{compute} categorical imperatives
in real time. Treating reflective constraints as mechanistic operators collapses
Levels of Abstraction and mistakes justification for causation
\cite{Allison2011,Korsgaard1996,Moor2023,Coeckelbergh2023}.


\noindent
From the perspective developed here, such systems do not implement deontology;
they instantiate the representational assumptions of their designers. The error
is not technical but conceptual: it assigns causal force to norms that function
as standards of assessment.

% -----------------------------------------------------------
\subsection{Deontic Invariants as Topological Constraints}

\noindent
Reconstructed through the evaluative–topological lens, deontological norms appear
not as forces that push behaviour along a gradient, but as \emph{structural
	boundaries} that give the moral field its shape. They do not pull agents toward
certain actions; they mark regions that cannot be entered without forfeiting
justificatory coherence. An evaluative trajectory may feel compelling, salient,
or even urgent, yet still terminate at a boundary it is not permitted to cross.

\medskip

\noindent
This interpretation preserves the philosophical core of deontology while
integrating it with the empirical architecture of moral cognition developed
earlier in the thesis. At the \textbf{cognitive LoA}, intuitive appraisal and
affective resonance sculpt the local terrain of salience. At the
\textbf{topological LoA}, deontic invariants function as fixed contours: they do
not vary with the agent, but they determine where movement is allowed. At the
\textbf{normative LoA}, reflective assessment evaluates whether a given trajectory
respects these contours under conditions of universal justifiability. The levels
remain distinct, yet their coordination is rendered conceptually transparent
rather than conflated.

\medskip

\noindent
Crucially, deontic constraints do not vary in strength. What varies is the
agent’s \emph{access} to them: whether these boundaries are foregrounded,
tracked, and rendered behaviourally operative within a particular evaluative
configuration. The structure holds; the pathways leading to it may narrow, blur,
or recede.

% -----------------------------------------------------------
\subsection{Deontology and Synthetic Perturbation}

\noindent
This reconstruction clarifies how the experimental findings should be interpreted
from a deontological standpoint. If duties function as invariant constraints
rather than motivational drivers, then the attenuation observed under robotic
co-presence does not constitute a failure of duty. What changes is not the norm,
but the cognitive–affective conditions under which that norm acquires practical
grip.

\medskip

\noindent
Synthetic presence might subtly alter attention, perceived sociality, and affective
tone. In doing so, it might be perturbing the evaluative field through which morally
relevant features are registered and integrated. In this scenario, the would duty remains intact; what is modulated is the route by which the agent encounters it. The boundary is unchanged, but the terrain leading to it is reshaped.

\medskip
Read in this way, the experimental results are compatible with deontological
ethics and diagnostically informative. They show how environmental and social
perturbations—here introduced by a synthetic embodied presence—could reconfigure
moral behaviour without altering normative structure. Deontology thus supplies
one indispensable dimension of the interpretive framework needed to understand
how synthetic presence reorganises the moral field.


\subsection{Deontic Invariants as Topological Constraints}

\noindent
Reconstructed through the evaluative–topological lens, deontological norms appear
not as forces that push behaviour along a gradient, but as \emph{structural
	boundaries} that give the moral field its shape. They do not pull agents toward
certain actions; they mark regions that cannot be entered without forfeiting
justificatory coherence. An evaluative trajectory may feel compelling, salient,
or even urgent, yet still terminate at a boundary it is not permitted to cross.

\medskip

\noindent
This interpretation preserves the philosophical core of deontology while
integrating it with the empirical architecture of moral cognition developed
earlier in the thesis. At the \textbf{cognitive LoA}, intuitive appraisal and
affective resonance sculpt the local terrain of salience. At the
\textbf{topological LoA}, deontic invariants function as fixed contours: they do
not vary with the agent, but they determine where movement is allowed. At the
\textbf{normative LoA}, reflective assessment evaluates whether a given trajectory
respects these contours under conditions of universal justifiability. The levels
remain distinct, yet their coordination is rendered conceptually transparent
rather than conflated.

\medskip

\noindent
Crucially, deontic constraints do not vary in strength. What varies is the
agent’s \emph{access} to them: whether these boundaries are foregrounded,
tracked, and rendered behaviourally operative within a particular evaluative
configuration. The structure holds; the pathways leading to it may narrow, blur,
or recede.

% -----------------------------------------------------------
\subsection{Deontology and Synthetic Perturbation}

\noindent
This reconstruction clarifies how the experimental findings should be interpreted
from a deontological standpoint. If duties function as invariant constraints
rather than motivational drivers, then the attenuation observed under robotic
co-presence does not constitute a failure of duty. What changes is not the norm,
but the cognitive–affective conditions under which that norm acquires practical
grip.

\medskip

\noindent
Synthetic presence subtly alters attention, perceived sociality, and affective
tone. In doing so, it perturbs the evaluative field through which morally
relevant features are registered and integrated. The duty remains intact; what
is modulated is the route by which the agent encounters it. The boundary is
unchanged, but the terrain leading to it is reshaped.

\medskip

\noindent
Read in this way, the experimental results are compatible with deontological
ethics and diagnostically informative. They show how environmental and social
perturbations—here introduced by a synthetic embodied presence—can reconfigure
moral behaviour without altering normative structure. Deontology thus supplies
one indispensable dimension of the interpretive framework needed to understand
how synthetic presence reorganises the moral field.

% -----------------------------------------------------------
\subsection{Mode of Evaluation: Maxims, Duties, and the Geometry of Permissibility}

\noindent
Deontological theories evaluate actions through the \emph{form} of their
underlying maxims and the duties that follow from rational consistency. When
expressed within the present framework, these commitments impose a distinctive
geometry on the evaluative field. Their core features can be stated
topologically:

\begin{itemize}
	\item \textbf{Invariance}: duties bind independently of context, affective
	state, or anticipated outcome.
	\item \textbf{Non-gradience}: moral permissibility is often partitioned by
	sharp boundaries rather than continuous slopes.
	\item \textbf{Symmetry}: the universal law test enforces interpersonal
	consistency across agents.
	\item \textbf{Role-relativity}: certain constraints arise only within specific
	relational or social configurations.
\end{itemize}

\noindent
Within evaluative topology, these features correspond to \emph{hard constraints}
in the moral landscape. They do not shape the gradients that drive moment-to-
moment appraisal; they delimit the space within which such appraisal may
legitimately unfold. Deontological normativity thus defines the
\emph{regulatory geometry} of moral evaluation: a stable scaffold against which
cognitive–affective dynamics play out, and relative to which the perturbations
introduced by synthetic presence acquire their ethical significance.


%%%%%%%%%%%%%%%%%%%%%%%%%%%%%%%%%%%%%%%%%%%%%%%%%%%%%%%%%%

% LAST SAFE REVIEW From here UP, NOT DOWN.

%%%%%%%%%%%%%%%%%%%%%%%%%%%%%%%%%%%%%%%%%%%%%%%%%%%%%%%%%%

% ===========================================================
% Micro-bridge into deontic mode of evaluation
% ===========================================================

\noindent
At this stage, the role of deontology within the evaluative--topological framework
should structurally clear: it fixes boundaries of permissibility without supplying
the forces that move agents within those boundaries. What remains to be explained
is how such non-generative constraints nonetheless matter for action. The
following sections address this question directly, shifting from the
\emph{geometry} of deontic normativity to its \emph{mode of influence} within
moral cognition and behaviour.

% -----------------------------------------------------------
\subsection{Mode of Evaluation: Maxims, Duties, and the Geometry of Permissibility}

\noindent
Deontological theories evaluate actions through the \emph{form} of the maxims from
which they issue and the duties that follow from rational consistency. When
expressed within the present framework, these commitments impose a distinctive
geometry on the evaluative field. Rather than shaping motivational gradients,
they define the contours within which moral movement is permitted.

\medskip

\noindent
Topologically, the central features of deontological normativity can be stated as
follows:

\begin{itemize}
	\item \textbf{Invariance}: duties bind independently of affective state,
	context, or anticipated outcome.
	\item \textbf{Non-gradience}: permissibility is often articulated through
	sharp boundaries rather than continuous slopes.
	\item \textbf{Symmetry}: the universal law test enforces consistency across
	agents and perspectives.
	\item \textbf{Role-relativity}: certain constraints arise only within specific
	relational or social configurations.
\end{itemize}

\noindent
Within evaluative topology, these features correspond to \emph{hard constraints}
in the moral landscape. They do not determine where an agent is drawn at any given
moment; they determine where an agent may go without forfeiting justificatory
coherence. Deontological normativity thus supplies the
\emph{regulatory geometry} of moral evaluation: a stable scaffold against which
cognitive--affective dynamics unfold.

% -----------------------------------------------------------
\subsection{Action-Guidance: How Normative Constraints Influence Behaviour}

\noindent
A natural question follows. If deontological principles do not describe
psychological processes, in what sense do they guide action at all?

\medskip

\noindent
Within a disciplined Levels of Abstraction framework, their influence is best
understood as indirect and temporally distributed. At the \textbf{cognitive LoA},
moment-to-moment behaviour arises from the integration of perceptual salience,
affective valuation, intuitive appraisal, and controlled modulation—processes
analysed empirically in~\ref{chap:moral_primer}. Deontic principles do not operate
at this level as causal drivers.

\medskip

\noindent
At the \textbf{normative LoA}, however, deontological commitments shape the
conditions under which agents evaluate, endorse, and regulate their own conduct.
Over time, internalised duties can function as a form of \emph{normative
	scaffolding}: they influence patterns of attention, affective sensitivity, and
inhibitory control through reflection, habituation, and self-constitution rather
than real-time computation.

\medskip

\noindent
In this sense, deontology does not run the machinery of moral cognition. It
contributes to how that machinery is calibrated across development and reflective
practice. Its role is structural rather than mechanical.

% -----------------------------------------------------------
\subsection{Deontological Normativity as Topological Invariance}

This perspective allows the core insight of the reconstruction to be stated
precisely. Within a topological model of moral cognition, deontological ethics
identifies \emph{invariants}: features of the evaluative field that are not
supposed to bend with fluctuations in affect, context, or incentive%
~\cite{Kant1785,Rawls1971,Scanlon1998,Allison2011,Korsgaard2009}.


\medskip

\noindent
Such invariants:
\begin{itemize}
	\item partition the evaluative manifold into permissible and impermissible
	regions;
	\item resist short-term deformation by situational pressures;
	\item stabilise patterns of rational endorsement over time;
	\item provide a reflective standpoint from which agents assess the legitimacy
	of their conduct.
\end{itemize}

\noindent
Accordingly, the categorical imperative is not a decision procedure but a
\emph{topological constraint}: a principle aimed at preserving global coherence
of evaluative structure rather than optimising local outcomes.

% -----------------------------------------------------------
\subsection{Why Deontology Matters for the Experimental Logic}

\noindent
This reconstruction is essential for interpreting the experimental findings
normatively rather than merely descriptively. The experiment identifies a
behavioural attenuation under robotic co-presence; deontological analysis helps
clarify what, if anything, such attenuation signifies ethically.

\medskip

\noindent
The use of a Watching-Eye stimulus is central here. As described in
Chapter~\ref{chap:exp_methods}, such cues reliably activate expectations
of accountability and reciprocity without introducing a real observer. They
operate precisely on the sensitivities that internalised deontic structures help
stabilise.

\medskip

\noindent
From this perspective, three interpretive claims follow:

\begin{enumerate}
	\item behavioural attenuation must be assessed relative to deontic
	permissibility, not merely outcome magnitude;
	\item interference with Watching-Eye effects plausibly reflects modulation of
	duty-tracking conditions rather than rejection of duty itself;
	\item deontological analysis supplies the vocabulary needed to distinguish
	morally significant distortion from ethically benign modulation.
\end{enumerate}

\noindent
Arguably, the experimental results expose some potential limitation of classical Machine Ethics. Encoding duties as behavioural algorithms presupposes that normativity operates at the same LoA as action selection. The findings instead suggest that synthetic presence acts \emph{upstream} of duty, reshaping the evaluative field in which deontic invariants become behaviourally salient.


\noindent
With deontology reconstructed as a system of topological constraints rather than
computational rules, we are now positioned to turn to consequentialism—where
normativity takes the form of graded landscapes over outcomes rather than
boundary conditions on permissibility.

%%%%%%%%%%%%%%%%%%%%%%%%%%%%%%%%%%%%%%%%%%%%%%%%%%%%%%%%%%%%%%%%%%
% Be Careful with this passage up
%%%%%%%%%%%%%%%%%%%%%%%%%%%%%%%%%%%%%%%%%%%%%%%%%%%%%%%%%%%%%%%%%
% -----------------------------------------------------------
\subsection{Conceptual Note: Gradient Fields in Consequentialist Topology}

\noindent
If deontological normativity fixes boundaries in the evaluative landscape,
consequentialism instead organises that landscape from within. Within the
evaluative--topological framework developed in this thesis, a
\emph{gradient field} denotes: 
\medskip
\noindent
\begin{center}
	\begin{leftbar}
		\textit{A moral space in which possible action--outcome configurations are ordered by scalar value—typically expected welfare or outcome-based moral worth.} (Note that the definition is ours.)
	\end{leftbar}
\end{center}
\medskip
\noindent

Structurally, what matters is not the absolute value assigned to any single point, but the \emph{directional structure} this ordering induces: movement is treated as morally preferable insofar as it tracks increases in expected value. Classical utilitarian reasoning presupposes such a structure when it assesses actions by their contribution to overall welfare~\cite{Bentham1789,Mill1861,SidgwickMethods}.

\noindent
Three features characterise consequentialist topology in this sense:

\begin{enumerate}
	\item \textbf{Scalar valuation}: actions and outcomes admit of comparative
	assessment along a common value dimension.
	\item \textbf{Directional guidance}: moral preference is expressed as
	orientation toward higher expected value rather than adherence to categorical
	constraints.
	\item \textbf{Empirical sensitivity}: the shape of the field depends on beliefs,
	evidence, and uncertainty about consequences.
\end{enumerate}

\noindent
As with deontology, these structures do \emph{not} describe the mechanisms of
real-time moral cognition. Agents do not compute gradients or integrate welfare
functions when acting. Consequentialist fields operate at the \emph{normative
	Level of Abstraction}: they articulate standards of justification under
reflective endorsement, not causal generators of behaviour. Sidgwick’s
distinction between the ``point of view of the universe’’ and ordinary moral
motivation captures this separation with particular clarity
\cite[Book~IV]{SidgwickMethods}.

\medskip

\noindent
\textbf{Interaction with moral cognition.}  
Although gradient fields belong to the normative LoA, they interact indirectly
with the evaluative machinery analysed empirically in
Chapter~\ref{chap:moral_primer}. This interaction is mediated through several
channels already familiar from the preceding chapters:

\begin{itemize}
	\item \emph{Salience}: anticipated harm or benefit tends to amplify attention
	and reconfigure local evaluative focus.
	\item \emph{Affect}: positive and negative valence often tracks
	outcome-related information, providing coarse local proxies for value
	gradients.
	\item \emph{Heuristics}: repeated exposure supports tractable rules of thumb
	(e.g.\ avoid harm, help when costs are low) that approximate gradient-following
	without explicit computation.
	\item \emph{Deliberation}: in reflective or conflicted contexts, agents may
	engage in limited comparative reasoning over expected outcomes.
\end{itemize}

\noindent
For the present thesis, one further feature is central. Because valuations depend
on perceived outcomes, consequentialist structures are especially sensitive to
perturbations of perception, attention, and social meaning. A change in the
interaction ecology—\textit{such as the introduction of a humanoid robot}—may therefore reshape the \emph{experienced} gradient field without altering the underlying normative standard.

\medskip

\noindent
Evaluative topology renders this sensitivity explicit. Behaviour is modelled not
as the execution of outcome calculations, but as movement through a dynamically
shaped field whose gradients are only indirectly approximated by affective and
attentional processes. This provides the conceptual link between
consequentialism and the empirical finding that prosocial donation is attenuated
under synthetic presence: the perturbation can be interpreted as a local
distortion of a field that ordinarily favours prosocial trajectories.

% ----------------------------------------------------------------------
\section{Consequentialist Structures: Value Gradients and the Topology of Outcomes}
\label{sec:consequentialism}

\noindent
Having reconstructed deontological ethics as a system of topological invariants
that constrain permissible trajectories, we now turn to consequentialism. The
contrast is structural. Where deontology fixes boundaries, consequentialism
supplies slopes; where duties impose limits, outcomes define directions. Moral
evaluation is organised not around what must not be done, but around how states
of the world compare~\cite{Mill1861,SidgwickMethods,Rawls1971,Norcross1997}.


\medskip

\noindent
As before, the aim is not historical survey but conceptual reconstruction. The
task is to articulate consequentialist normativity in a form that respects
Levels of Abstraction discipline while integrating with the
evaluative--topological account of moral cognition developed earlier. This
reconstruction also prepares a distinct normative lens through which the
experimental attenuation effect can be interpreted.

\subsection{The Source of Normativity: Welfare, Impartiality, and Reasons}

\noindent
Classical utilitarianism locates moral authority in the promotion of welfare.
Bentham’s felicific calculus~\cite{Bentham1789}, Mill’s qualitative distinctions~
\cite{Mill1861}, and Sidgwick’s analysis of impartiality~\cite{SidgwickMethods}
converge on the view that outcomes matter because of their contribution to
overall good, aggregated across persons. Right action, on this view, is action
that sufficiently promotes such value.

\noindent
From the perspective of Levels of Abstraction, this places consequentialist
normativity at a reflective LoA concerned with evaluating outcomes, aggregating
value, and justifying action from an impartial standpoint. These are standards of
assessment, not descriptions of psychological operation. Sidgwick is explicit
that the justificatory standpoint of utilitarianism is distinct from ordinary
motivation: consequentialism supplies a criterion of rightness, not a model of
decision-making.

\medskip

\noindent
This distinction matters for the present thesis because it marks the limit of
\textbf{computational analogy}. Outcome-based formalisms—utility functions, reward
optimisers, expected-value maximisers—are often treated as if they directly
modelled moral cognition. But such formalisms inhabit a different explanatory
order from the cognitive--affective processes analysed in
Chapter~\ref{chap:moral_primer}. Any connection between them therefore requires
explicit mediation rather than assumption.

\subsection{Mode of Evaluation: Consequences and Scalar Normativity}

\noindent
Consequentialism evaluates actions by the value of their actual or expected
outcomes. Unlike deontological constraints, which are categorical,
consequentialist evaluation is scalar: options can be better or worse by degree.
This scalar structure admits of a natural topological representation.

\medskip

\noindent
Within the evaluative--topological model, a consequentialist landscape exhibits:

\begin{itemize}
	\item \textbf{Gradience}: moral evaluation varies continuously with expected
	value.
	\item \textbf{Optimisation structure}: preferable actions correspond to
	trajectories toward higher-valued regions of the field.
	\item \textbf{Context-dependence}: the shape of the field shifts with empirical
	facts about consequences.
	\item \textbf{Impartiality}: value contributions are treated as having equal
	standing across persons.
\end{itemize}

\noindent
These features help explain why consequentialism lends itself to computational
representation. However, computational tractability should not be conflated with
cognitive realism. Human moral cognition does not optimise utility; it relies on
heuristic, affective, and context-sensitive processes that at best approximate
consequentialist ideals~\cite{Kahneman2011,Greene2001,Haidt2001,Slovic2007}.
Treating agents as literal expected-utility maximisers would repeat the same
Levels of Abstraction conflation identified earlier.

\medskip

\noindent
Read through evaluative topology, consequentialism therefore contributes a
second, complementary dimension to the interpretive framework of the thesis.
Where deontology clarifies how synthetic presence may weaken the practical grip
of invariant duties, consequentialism clarifies how the same presence may distort
the perceived gradients that ordinarily favour prosocial action. The following
sections build on this contrast as we continue mapping normative structures onto
the cognitive--topological space within which moral behaviour unfolds.

% -------------------------------------------------------------------------
\subsection{Action-Guidance in Consequentialism: From Value to Pressure}

\noindent
A familiar objection to consequentialism is that it appears to require a form of
explicit calculation that human agents rarely perform. Read through the lens of
Levels of Abstraction, this objection rests on a category mistake.
Consequentialism does not guide action by prescribing a psychological algorithm;
it guides action by exerting \emph{indirect pressure} on the evaluative topology
within which behaviour emerges.

\medskip

\noindent
At the \textbf{normative LoA}, consequentialism articulates a criterion of
rightness: actions are justified insofar as they promote expected welfare
\cite{Bentham1789,Mill1861,SidgwickMethods}. At the \textbf{cognitive LoA},
behaviour arises from the interaction of salience, affect, intuitive appraisal,
and controlled modulation, as analysed empirically in
Chapter~\ref{chap:moral_primer}. The influence of consequentialist normativity is
therefore mediated and diachronic. Over time, it may shape the evaluative field
through:

\begin{itemize}
	\item \emph{dispositional shaping}, whereby moral education tends to heighten
	sensitivity to harm and benefit;
	\item \emph{heuristic internalisation}, yielding outcome-sensitive rules of
	thumb that approximate value comparison;
	\item \emph{attentional modulation}, as anticipated consequences influence
	what is foregrounded in appraisal;
	\item \emph{deliberative correction}, when reflective comparison reweights
	competing options.
\end{itemize}

\noindent
Topologically, consequentialism does not ``run’’ the cognitive system. It alters
the relative steepness and orientation of evaluative gradients, thereby biasing
trajectories without determining them.

% -----------------------------------------------------------
\subsection{Consequentialist Topology: Moral Action as Gradient Tracking}

\noindent
Within the evaluative--topological framework, the core consequentialist intuition
admits a compact expression: morally preferable behaviour corresponds to
approximate movement along gradients of expected value. Unlike deontic
constraints, which partition the field, consequentialist normativity structures
it continuously.

\medskip

\noindent
This yields several structural features:

\begin{itemize}
	\item \textbf{Continuity}: moral improvement admits of degree rather than
	all-or-nothing transitions.
	\item \textbf{Directionality}: moral relevance depends on orientation relative
	to welfare ascent.
	\item \textbf{Trade-offs}: competing benefits and harms are represented as
	interacting gradients.
	\item \textbf{Perturbation sensitivity}: because valuation depends on perceived
	consequences, shifts in salience or social meaning may locally distort the
	gradient.
\end{itemize}

\noindent
The final feature is central for us. 
\noindent
\begin{center}
	\begin{leftbar}
		\textit{If synthetic presence alters
			attention or the social meaning of helping, the gradient that ordinarily favours prosocial action may be flattened or redirected, even when underlying normative standards remain unchanged.}
	\end{leftbar}
\end{center}

% -----------------------------------------------------------
\subsection{Consequentialism and the Experimental Logic}

\noindent
Consequentialism provides a second indispensable lens for interpreting the
experimental attenuation effect. Prosocial donation is simultaneously a
behavioural output of the cognitive architecture and an action whose outcomes are
readily ordered in welfare terms.

\medskip

\noindent
Within this frame, the experimental manipulations can be interpreted as follows:

\begin{enumerate}
	\item \textbf{The Watching--Eye cue steepens the prosocial gradient.}  
	Cues of observation tend to increase the perceived social and reputational
	value of helping, biasing evaluative trajectories toward donation
	\cite{Haley2005,Bateson2006,Dear2019}.
	
	\item \textbf{Synthetic presence may attenuate this gradient.}  
	The humanoid robot might be introducing an ambiguous social element that can absorb attention or reframe the meaning of helping, thereby weakening the perceived payoff of donation~\cite{Zlotowski2015}.
	
	\item \textbf{Consequentialist analysis classifies the shift.}  
	From a consequentialist perspective, attenuation is best read as a local
	deformation of the \textbf{perceived welfare landscape} rather than as a failure of rational computation.
\end{enumerate}

\noindent
Nothing in this reconstruction treats consequentialism as a template for machine
implementation. Contrary to classical Machine Ethics, which equated ethical design
with encoding utility functions, consequentialism here functions as a
\emph{normative perspective}: a way of interpreting how synthetic presence
perturbs the evaluative topology that ordinarily favours prosocial behaviour.

\medskip

\noindent
With deontology and consequentialism reconstructed as complementary topological
structures—constraints and gradients—we can now turn to virtue ethics. There,
normativity is located not in rules or outcomes but in cultivated dispositions
and perceptual sensitivities, adding a further dimension to the evaluative field
within which synthetic perturbation operates.

%------------------------
\section{Virtue-Theoretic Structures: Dispositions, Character Topology, and Moral Sensitivity}
\label{sec:virtue_ethics}

\noindent
Deontological invariants and consequentialist gradients describe constraints and
directions within the evaluative field, but neither specifies \emph{who} is
moving through that field. Virtue ethics supplies this missing dimension. From
Aristotle onward, virtue-theoretic accounts locate normativity in the
\emph{perceptual and dispositional architecture of the agent} rather than in
rules or outcomes alone \cite{Aristotle_nicomachean,Foot2001,Hursthouse1999,Annas2011}.
This makes virtue ethics especially salient for the present thesis, where the
experimental attenuation effect varies systematically across latent
dispositional ecologies identified in~\ref{chap:exp_methods}.

\medskip

\noindent
Reconstructed within the evaluative--topological framework, virtue ethics treats
character as a structured configuration of moral sensitivity. Long-term
habituation shapes how situations are perceived, which features stand out as
salient, and which responses appear fitting. In topological terms, virtues
correspond to \emph{stable curvature in the evaluative field}: patterns that tend
to channel appraisal and action along prosocial trajectories
\cite{McDowell1979,Foot2001}. Deficiencies of character appear as shallow,
unstable, or poorly integrated regions of the field.

\subsection{Source of Normativity: Character and Moral Perception}

\noindent
Virtue-theoretic normativity is grounded in well-formed character rather than in
external principles. Aristotle’s notion of \emph{phronesis} captures this as a
capacity for morally attuned perception: the ability to discern what matters in
concrete circumstances and to respond appropriately
\cite{Aristotle_nicomachean}. This conception aligns closely with the empirical
architecture developed earlier in the thesis. Moral judgement arises from
patterns of salience and affective appraisal that are relatively stable across
contexts while remaining responsive to situational nuance.

\medskip

\noindent
Within evaluative topology, a \textit{virtuous} agent’s field is characterised by:
(i) deep attractor basins associated with benevolence and fairness,
(ii) well-shaped gradients that bias appraisal toward appropriate action, and
(iii) relative robustness under minor perturbations. These are not formal
mechanisms but structural regularities in how moral meaning is perceived and
weighted.

\subsection{Dispositions as Topological Structure}

\noindent
The thesis has already introduced the dispositional parameter
$\beta_C$, representing an agent’s latent trait configuration, and its role in
modulating evaluative dynamics. Here, virtue ethics supplies a normative
interpretation of that parameter. Differences in $\beta_C$ correspond to
differences in the \emph{geometry} of the evaluative field.

\medskip

\noindent
Empirically, participants clustered into coherent dispositional profiles in the
space indexed by $\beta_C$~(Chapter \ref{chap:exp_methods}). Interpreted
virtue-theoretically, these profiles can be read as reflecting different forms of
moral sensitivity: highly empathic clusters tend to exhibit steep prosocial
gradients that are powerful but fragile; analytically structured clusters show
flatter but more stable curvature; emotionally reactive clusters display shallow,
noise-dominated dynamics. This pattern is consistent with work linking empathy,
agreeableness, and habituated moral perception to prosocial responsiveness
\cite{Haidt2012,Dancy2004}.

\subsection{Action-Guidance: Habituation and Stability}

\noindent
Virtue ethics accounts for action without appealing to explicit rules or welfare
calculations. Behaviour reflects habituated patterns of attention, affect, and
response that develop over time. This is compatible with the cognitive
architecture analysed earlier: intuitive appraisal is guided by learned
sensitivities, while reflective processes contribute to the stabilisation of
these patterns through self-regulation and identity formation.

\medskip

\noindent
Topologically, virtues correspond to \emph{deep and coherent attractors} that tend
to resist minor contextual noise; vices correspond to shallow or fragmented
structures. This perspective aligns with empirical models of habit formation and
moral perception as acquired sensitivity \cite{Reynolds2006,Wood2016}.

\subsection{Virtue Ethics and Synthetic Perturbation}

\noindent
The virtue-theoretic lens clarifies why the experimental perturbation is
trait-contingent. If synthetic presence does not introduce new norms, than arguably it perturbs the evaluative field in which moral perception operates. Where moral sensitivity is steep and affectively rich, perturbation tends to exert greater influence; where curvature is flatter or more stable, the same perturbation is more readilyabsorbed with minimal behavioural change.

\noindent
This framing renders the central empirical pattern intelligible: attenuation seemed to have been strongest in prosocial--empathic ecologies, weaker in analytical--structured ecologies, and negligible in emotionally reactive ones. From a virtue-theoretic perspective, moral sensitivity shapes not only typical patterns of action, but also susceptibility to environmental modulation.

\noindent
In sum, virtue ethics supplies the final structural component of the normative
framework developed in this chapter. Deontology contributes boundary structure,
consequentialism contributes gradient structure, and virtue ethics contributes
\emph{curvature}: the dispositional geometry that conditions how agents traverse
the evaluative field. Together, these dimensions complete the normative
reconstruction required to interpret the experimental findings before turning to
the remaining traditions and the broader synthesis that follows.

%---------------------------------------------------------
\section{Integrated Ethical Interpretation of the Experimental Results}
\label{sec:integrated_interpretation}

\noindent
With the major normative traditions reconstructed under a shared discipline of
Levels of Abstraction and embedded within the evaluative--topological framework
developed across this thesis, we are now in a position to articulate their joint
relevance for the experimental findings. The purpose of this section is not to
privilege one ethical theory over another, but to show why the behavioural
perturbation associated with synthetic presence becomes ethically intelligible
only when these frameworks are read together. Each theory isolates a distinct
structural aspect of the same phenomenon: how moral salience is transformed into
action within a perturbed evaluative field.

\medskip

\noindent
Read in isolation, the experimental result can be described narrowly as a
reduction in charitable donation under robotic co-presence. Read through the
normative architectures reconstructed in this chapter, the same result acquires
ethical depth: it can be interpreted as revealing how synthetic presence
reconfigures different dimensions of moral orientation—constraint, valuation,
and sensitivity—without introducing new norms or explicit reasons.

\subsection*{Deontological Perspective: Invariant Structure and Moral Uptake}

\noindent
On the deontological reconstruction, duties function as invariant structural
constraints within the evaluative field~\cite{Kant1785,Rawls1971,Scanlon1998,Korsgaard2009}.
The Watching--Eye stimulus, detailed in Chapter~\ref{chap:exp_methods}, implicitly
engages such constraints by cueing accountability, reciprocity, and respect
\cite{Haley2005,Bateson2006}.
 These are not motivational
forces but standards of permissibility that agents typically track through
perceptual and affective sensitivity.

\medskip

\noindent
From this standpoint, the attenuation observed in the Robot condition would not
indicate that participants violated duties or rejected moral norms. Rather, it is
consistent with a weakening in the \emph{uptake} of deontic salience. Because all
explicit cues were held constant across conditions, any systematic reduction in
prosocial response isolates synthetic presence as a factor that interferes with
how deontic invariants are registered within the evaluative field. Deontology
thus provides the conceptual resources to distinguish a benign preference shift
from a deformation in sensitivity to obligation.

\subsection*{Consequentialist Perspective: Gradient Deformation and Outcome Salience}

\noindent
From a consequentialist perspective, moral orientation depends on the perceived
structure of outcomes. Watching-Eye cues are known to steepen the value gradient
favouring prosocial action by increasing anticipated social or reputational
benefits. According to the literature on the topic (see Chapter~\ref{chap:tools_new}) donation thereby becomes locally attractive insofar as it appears to move the agent “uphill” in expected value.

\medskip

\noindent
It might be safe to say that synthetic presence alters this configuration without supplying new outcome information. By introducing an ambiguous social entity, it can flatten or redirect the perceived gradient: attention may be divided, the reputational
signal diluted, and the social meaning of helping rendered less determinate. In
topological terms, the welfare-relevant slope surrounding donation is locally
deformed. The resulting attenuation is therefore compatible with a
consequentialist interpretation: prosocial action appears less strongly favoured
because the perceived payoff landscape has been subtly reshaped.

\subsection*{Virtue-Theoretic Perspective: Dispositional Curvature and Differential Susceptibility}

\noindent
Virtue ethics provides the most direct bridge between normative theory and the
empirical structure of the data. On the virtue-theoretic reconstruction, moral
responsiveness depends on dispositional curvature: the depth, stability, and
integration of the evaluative attractors shaped by character and habituation.

\medskip

\noindent
Cluster analyses in Chapter~\ref{chap:exp_methods} revealed distinct
dispositional ecologies. These differences are not artefactual; they can be read
as the normative signature of virtue-theoretic topology. Highly
prosocial--empathic profiles tend to exhibit steep, affectively rich attractors
that strongly support helping but are also more susceptible to perturbation.
Analytically structured profiles exhibit greater stability but weaker affective
pull. Emotionally reactive profiles display shallow, volatile structure with
limited directional guidance.

\medskip

\noindent
Within the formalism already introduced, this interaction can be represented as
\[
\dot{x}' = f(x;\beta_C) + \delta f(x;\mathscr{R}),
\]
where $f(x;\beta_C)$ captures the dispositional dynamics of each ecology and
$\delta f(x;\mathscr{R})$ the perturbation associated with synthetic presence.
The crucial point is that $\delta f$ is not uniform: it might be safe to say that its effect depends on the
curvature encoded by $\beta_C$. Virtue ethics therefore renders intelligible why
attenuation is cluster-dependent rather than universal. Synthetic presence
interacts with character structure, not with explicit rule-following or outcome
calculation.

\medskip

\noindent
Taken together, these perspectives converge on a single ethical insight. The
experiment does not show that robots impose norms, reason morally, or replace
human judgement. It shows that synthetic presence can reorganise the evaluative
field within which moral perception, valuation, and sensitivity operate. Each
normative framework isolates a different structural facet of this
reorganisation: deontology highlights disrupted uptake of invariant constraints,
consequentialism highlights local deformation of value gradients, and virtue
ethics highlights differential susceptibility rooted in dispositional topology.

\medskip

\noindent
This integrated reading constitutes the core contribution of the chapter. It
shows that the ethical significance of the experimental perturbation cannot be
captured by any single normative lens, nor reduced to descriptive psychology
alone. Only by combining Levels of Abstraction discipline with evaluative
topology does the attenuation associated with synthetic presence become both
empirically grounded and normatively interpretable.

\section{A Machine Ethics Account}

\noindent
The integrated reconstruction developed in this chaptermight suggest structural limitations in the classical Machine Ethics. Across its dominant approaches normative abstractions are treated as if they were behaviour-generating mechanisms operating at the same Level of Abstraction as moral cognition.

\noindent
This error seems to manifest in three systematic ways:
\begin{enumerate}
	\item \textbf{Deontic reduction}: deontological principles are operationalised as executable rules, thereby mislocating duties at the cognitive LoA and rendering them insensitive to perturbations in deontic uptake.
	\item \textbf{Utility reification}: consequentialist utilities are treated as generative drivers of action, obscuring the role of perceptual salience, affective valuation, and social meaning in shaping perceived value gradients.
	\item \textbf{Dispositional omission}: the absence of character- or trait-level topology leaves no resources for explaining cluster-dependent deformation, nor for accounting for why attenuation is strongest precisely where empathic gradients are steepest.
\end{enumerate}

\noindent
What the experiment might suggest, instead, is that moral behaviour might be emerging from \emph{field-level dynamics}: interactions among salience, affect, dispositional structure, and social context that no monolithic normative framework can generate or predict in isolation. If this is the case, the mistake in the research agenda of classical Machine Ethics could be both empirical and architectural—it conflates justificatory structure with causal machinery.

\subsection{Concluding Perspective: Why a Multi-Framework Interpretation Is Necessary}

\noindent
Taken together, the three reconstructed traditions converge on a single claim: \emph{synthetic presence might reshape the evaluative field through which moral salience becomes action}. Each framework captures a distinct dimension of this deformation without reducing it to a single mechanism.

\begin{itemize}
	\item \textbf{Deontology} diagnoses disruptions in sensitivity to invariant normative expectations.
	\item \textbf{Consequentialism} identifies local flattening or redirection of perceived outcome gradients.
	\item \textbf{Virtue ethics} explains why susceptibility to perturbation is mediated by dispositional curvature and latent trait ecology.
\end{itemize}

\noindent
The ethical significance of the experimental attenuation therefore cannot be recovered from any one framework alone. It arises from their joint application to a shared topological substrate: the evaluative field that links perception, appraisal, and action.

\noindent
Under this integrated reading, the experiment might be indicating not merely that robotic co-presence might be altering behaviour, but \emph{how} it could do so—by deforming the structural conditions under which moral cues acquire behavioural force. This conclusion sets the stage for the sentimentalist analysis that follows, where affective vector fields and pre-reflective appraisal dynamics become central to explaining the immediacy and directionality of the perturbation.


\section{Sentimentalism and Emotion-Based Normativity: Affective Vector Fields}
\label{sec:sentimentalism}

\noindent
If deontology fixes the boundaries of moral space and consequentialism tilts its
slopes, sentimentalism attends to the \emph{forces that actually move us within
	it}. In the sentimentalist tradition—classically articulated by Hume and Smith
and developed in contemporary affect-based accounts—moral evaluation is
understood to originate in structured patterns of affective responsiveness to
others \cite{HumeTreatise,Smith1759,Slote2010,Nichols2004}. Moral salience is not
first encountered as a rule or a calculation, but as a pull, a resistance, or a
felt demand that arises within social perception.

\noindent
This perspective connects closely with the empirical pattern isolated in the
experiment. The observed attenuation does not require a change in duties or
outcome rankings; it is plausibly realised through a modulation of affective
uptake. Sentimentalism therefore offers the most proximal normative lens for
interpreting how synthetic presence reshapes moral behaviour.

\subsection{Source of Normativity: Affective Responsiveness and Moral Salience}

\noindent
On a sentimentalist view, normativity is grounded in affective responses such as
empathy, warmth, aversion, and indignation. These responses do not replace
judgement; rather, they furnish the medium through which moral significance is
first registered. In the terms developed earlier in the thesis, they contribute
to the initial curvature of the evaluative field at the cognitive LoA
(see~\ref{chap:moral_primer}).

\noindent
Where deontological constraints remain silent and consequentialist gradients
remain abstract, affective resonance determines what \emph{stands out} as morally
pressing. Sentimentalism thus specifies the affective geometry of moral space:
which features draw agents closer, which repel, and which leave them relatively
unmoved.

\subsection{Mode of Evaluation: Affective Vector Fields}

\noindent
Within the evaluative--topological framework already introduced, sentimentalist
evaluation can be represented as an \emph{affective vector field}. Formally, this
amounts to treating affective responses as directional influences over evaluative
states:
\[
\mathbf{A}(x),
\]
where $x$ denotes the agent’s current evaluative configuration and
$\mathbf{A}(x)$ captures the direction and relative strength of affective pull or
push (e.g.\ empathic attraction, aversive withdrawal).\footnote{This notation
	follows the dynamical formalism introduced earlier in the thesis; no new
	mathematical machinery is introduced here.}

\noindent
This representation is not intended as a claim about explicit computation. It is
a modelling device that renders visible how affective forces bias trajectories
through evaluative space. Moral action, on this view, reflects the integration of
these forces rather than the execution of explicit rules.

\subsection{Action Guidance: Affective Modulation and Perturbation}

\noindent
It is safe to say that--on a philosophical reading--synthetic presence enters this picture as a perturbation to affective flow. Its influence is not propositional but atmospheric: it alters attention, perceived social meaning, and empathic focus. Within the vector-field model, this corresponds to a local attenuation or redirection of affective influence:
\[
\mathbf{A}(x) \;\rightarrow\; \mathbf{A}(x) + \delta \mathbf{A}(x;\mathscr{R}),
\]
where $\mathscr{R}$ denotes robotic co-presence and $\delta \mathbf{A}$ captures
its modulating contribution.

\noindent
The experimental pattern is consistent with this interpretation. In affectively
rich ecologies, where empathic vectors are ordinarily steep, even modest
attenuation is associated with measurable behavioural displacement. In more
analytically structured profiles, where affect plays a weaker guiding role, the
same perturbation is associated with limited impact. The robot does not reverse
moral orientation; it dampens the affective momentum that would otherwise carry
action toward donation.

\subsection{Sentimentalism in Machine Ethics}

\noindent
This analysis might clarifies why we advocate for a persistent blind spot in classical Machine Ethics. Rule-based and utility-based systems lack representational resources for affective modulation. They can encode constraints and optimise values, but they do not model the affective forces through which moral cues become behaviourally salient in the first place.

\noindent
The experiment renders this omission visible. A silent, behaviourally inert robot
is associated with altered moral behaviour not by changing rules or utilities,
but by reshaping the affective field in which moral perception unfolds. A
sentimentalist topology can represent this kind of effect; monolithic Machine
Ethics architectures cannot.

\subsection{Why Sentimentalism Matters for the Experimental Interpretation}

\noindent
Sentimentalism completes the normative reconstruction by accounting for the
\emph{mechanism of immediacy}. Deontology clarifies what remains binding,
consequentialism clarifies how outcomes are ordered, virtue ethics clarifies why
susceptibility varies by character, and sentimentalism clarifies how moral force
is felt in the moment.

\noindent
Taken together, the findings could suggest that synthetic presence perturbs moral
behaviour by deforming the affective topology through which moral salience becomes
action. This insight prepares the ground for the final synthesis of the chapter,
where the interaction between affective vectors, dispositional curvature, and
normative structure is integrated within the broader Levels-of-Abstraction
framework.

\section{Contractualism, Particularism, and Pluralist Normativity}
\label{sec:contractualism_particularism_hybrid}

\noindent
The preceding sections reconstructed deontology, consequentialism, virtue ethics,
and sentimentalism as distinct structural dimensions of the evaluative field. To
complete the normative architecture required for interpreting the experimental
findings, three further frameworks must be considered: \emph{contractualism},
\emph{particularism}, and \emph{pluralist or hybrid models}. Their inclusion is not
encyclopaedic. It is methodological.

\noindent
Each addresses a dimension of moral evaluation that becomes salient precisely in
cases of subtle, context-dependent perturbation—cases in which behaviour shifts
without any explicit change in rules, outcomes, or character. Minimal social
cues, ambiguous agency, and synthetic presence are associated with changes in
interpersonal justification, situational salience, and cross-cutting normative
pressure \cite{Francey2012,Kawamura2017,Malle2016,Bremner2022}. These dimensions
are not fully captured by invariants, gradients, or dispositional curvature
alone.

\subsection{Contractualism: Interpersonal Justification and Evaluative Equilibria}

\noindent
Contractualism locates moral rightness in the requirement that one’s actions be
justifiable to others on principles no one could reasonably reject
\cite{Scanlon1998}. Normativity here is relational: it arises from standing in
relations of mutual accountability rather than from rules, outcomes, or traits.

\noindent
Within the Levels of Abstraction framework, contractualism operates at a
reflective normative LoA. Its practical relevance nevertheless presupposes
cognitive and affective capacities—sensitivity to being answerable, recognition
of others as evaluative agents, and uptake of reactive attitudes
\cite{Strawson1962}. Topologically, contractualism can be represented as defining
\emph{justificatory equilibria} within the evaluative field: regions in which
actions remain stable under interpersonal scrutiny.

\noindent
This structure is implicated in the experimental paradigm. Watching-Eye cues
are known to intensify perceived accountability, strengthening justificatory
pressure toward prosocial action \cite{Francey2012,Kawamura2017}. A humanoid
robot, however, is perceptually social yet normatively indeterminate. Empirical
work suggests that such agents engage social cognition without reliably anchoring
interpersonal roles \cite{Malle2016,Carpenter2016,Bremner2022}. The resulting
configuration is consistent with a deformation of the justificatory field: the
sense of shared evaluative regard becomes less stable.

\noindent
From a contractualist perspective, attenuation under robotic presence can
therefore be interpreted as a weakening of justificatory equilibrium rather than
as a rejection of duty or a reassessment of outcomes. Moral reasons lose practical
force insofar as their interpersonal grounding becomes less determinate.

\subsection{Particularism: Contextual Salience and Local Moral Topologies}

\noindent
Moral particularism rejects fixed principles and invariant reason-valences. What
counts morally depends on context: a consideration may favour an action in one
situation and count against it in another \cite{Dancy2004}. McDowell’s perceptual
account frames this as sensitivity to morally relevant particulars rather than
rule application \cite{McDowell1979}. Contemporary Moral Psychology aligns with
this picture, emphasising the role of attention, affect, and situational framing
in moral uptake \cite{Haidt2001,Greene2001,Narvaez2005}.

\noindent
Topologically, particularism corresponds to an evaluative field composed of
\emph{local salience contours} rather than global invariants or gradients. Moral
appraisal unfolds through the ordering and persistence of what becomes salient
within a given episode.

\noindent
In this setting, synthetic presence functions as a local perturbator. Watching-Eye
cues ordinarily elevate accountability salience rapidly \cite{Francey2012}. The
robot introduces a competing source of social salience—ambiguous in agency and
evaluative status—that reorganises attention and affect
\cite{Bremner2022,Malle2016,Mutlu2009}. The resulting attenuation is not best
understood as a shift in principle or value, but as a reconfiguration of which
cues dominate the evaluative moment.

\noindent
This framing also helps clarify cluster differences. For agents whose evaluative
styles are already highly context-sensitive, additional perturbation introduces
noise rather than systematic displacement. Where moral sensitivity is fluid,
further salience competition has limited marginal effect—a pattern aligned with
the empirical findings.

\subsection{Pluralist Models: Multidimensional Evaluative Manifolds}

\noindent
Pluralist and hybrid theories hold that normativity arises from multiple
irreducible sources: duties, outcomes, character, relationships, and contextual
considerations \cite{Ross2002,Chang2013,Griffin1986}. No single dimension governs
moral judgment. Moral evaluation instead involves navigating a space shaped by
intersecting normative forces.

\noindent
Topologically, pluralism corresponds to a \emph{multi-dimensional evaluative
	manifold}. Constraints, gradients, attractors, affective vectors, and salience
structures coexist and interact. Empirical research supports this architecture:
affective, rule-based, and outcome-sensitive processes operate semi-independently
and dynamically constrain one another
\cite{Haidt2001,Greene2001,Churchland2011}.

\noindent
The experimental results are most naturally interpreted at this manifold level.
The Watching-Eye cue activates accountability expectations; donation engages
welfare considerations; cluster differences reflect dispositional curvature;
robotic presence refracts interpersonal meaning and competes for salience
\cite{Malle2016,Bremner2022,Krach2008}. The observed attenuation could be emerging from their joint reconfiguration rather than from any single normative axis.

\noindent
Pluralism also clarifies why attenuation appears across clusters. Dispositional
differences shape baseline trajectories, but synthetic presence perturbs features
of the shared evaluative topology itself. The robot does not target one normative
dimension; it subtly reshapes several at once. Moral behaviour shifts
accordingly.

\medskip

\noindent
Taken together, contractualism, particularism, and pluralist models complete the
normative picture. They indicate that the experimental effect is not adequately
characterised as a failure of duty, a miscalculation of value, or a defect of
character alone. It is more plausibly interpreted as a field-level deformation
operating through interpersonal justification, contextual salience, and the
interaction of multiple normative forces. With this architecture in place, the
chapter is now positioned to return explicitly to the Levels-of-Abstraction
framework and to draw out its implications for the design, interpretation, and
governance of synthetic moral environments.

\section{Integrative Ethical Interpretation of the Experimental Findings}
\label{sec:integrative_ethical_interpretation}

\noindent
Read through the lens of normative theory, the experimental findings take on a sharper and more general significance. The attenuation of prosocial donation under robotic co-presence is not plausibly explained as the weakening of a single moral principle, motive, or evaluative dimension. Rather, it could be best understood as a \emph{field-level perturbation}: a reconfiguration of the evaluative environment within which diverse moral considerations are ordinarily weighted, integrated, and allowed to become behaviourally effective.

\noindent
Each of the reconstructed normative frameworks isolates a distinct aspect of this displacement:

\begin{enumerate}
	\item \textbf{Deontological perspective: diminished uptake of accountability.}  
	The Watching-Eye cue ordinarily heightens sensitivity to implicit demands of accountability and respect. The robot does not negate these demands, but appears to weaken their felt presence. The observed attenuation is therefore reasonably and cautiously consistent with a disruption in the conditions under which deontic expectations become salient, rather than with any failure of rule-following or obligation recognition.
	
	\item \textbf{Consequentialist perspective: deformation of perceived outcome gradients.}  
	Prosocial donation normally sits on a locally steep welfare-related gradient shaped by reputational, interpersonal, and affective returns. Synthetic presence appears to flatten or blur this gradient, not by altering outcomes themselves, but by reshaping how those outcomes are perceived and weighted within the evaluative field.
	
	\item \textbf{Virtue-theoretic perspective: dispositional expression under contextual modulation.}  
	The cluster structure indicates that moral dispositions remain operative, yet their expression is mediated by an evaluative environment that is itself modifiable. The consistent directional attenuation across clusters is compatible with the view that character does not operate in isolation, but is enacted within a field whose curvature can be shifted by contextual features.
	
	\item \textbf{Contractualist perspective: destabilised justificatory relations.}  
	Moral motivation grounded in mutual answerability depends on recognising oneself as acting under the evaluative regard of others. The robot introduces ambiguity into this interpersonal space, weakening the sense of shared normative standing that ordinarily sustains justificatory pressure toward prosocial action.
	
	\item \textbf{Particularist perspective: reordered salience at the level of the episode.}  
	The evaluative episode itself may be altered. Although the Watching-Eye cue remains present, its priority within the perceptual--affective sequence is displaced by a new and socially ambiguous source of salience. What becomes morally relevant first—and for how long—is subtly reorganised.
	
	\item \textbf{Pluralist--topological perspective: manifold-level displacement.}  
	Taken together, the findings are consistent with what a pluralist model would anticipate when multiple normative dimensions interact with a global perturbation to social meaning. The effect is observable across clusters because it operates on shared features of the evaluative manifold rather than on any single normative axis.
\end{enumerate}

\noindent
These perspectives converge on a unified ethical interpretation:
\bigskip
\begin{center}
	\begin{tcolorbox}[colback=white, colframe=black!60,
		title={Integrative Ethical Conclusion}]
		Synthetic presence may reshape the multi-dimensional evaluative topology through which moral salience becomes action. The perturbation observed in the experiment operates at the level of the evaluative field itself, modulating accountability sensitivity, outcome salience, dispositional expression, justificatory relations, and contextual priority in parallel. No single ethical framework exhausts this phenomenon. The results instead support a pluralist, topologically structured, and empirically constrained account of moral cognition—one in which artificial agents acquire ethical significance not by acting, reasoning, or commanding, but by altering the conditions under which moral meaning is registered and allowed to matter.
	\end{tcolorbox}
\end{center}
\bigskip
\noindent
Placed alongside the thesis’s formal conclusion, this ethical reconstruction does not revise the empirical claims. It re-describes their significance. Where the Conclusion chapter suggested that synthetic presence modulates moral behaviour upstream of deliberation, the present section clarifies \emph{why this matters normatively}: because the modulation occurs at the level where diverse moral reasons are integrated into action-readiness. Ethical influence, in this sense, precedes agency. It arises not from what artificial systems do, but from how their presence reshapes the moral environments humans already inhabit.
