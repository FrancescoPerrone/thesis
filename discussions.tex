\chapter{Discussions}
\label{chap:dis}

\section{Reframing the Central Question}

From various discussion over the past 6 years, it seems reasonable to conclude that a moment familiar to any researcher studying moral cognition exists, when the theoretical abstractions fall away and what remains is the simple, persistent question that brought us here in the first place:

\begin{quote}
	\textit{Why do small, seemingly insignificant presences in our environment alter what we take to be the ‘right’ thing to do?}
\end{quote}

This thesis began from that question—not as an academic curiosity, but as a recognition that our moral lives are extraordinarily sensitive to the texture of the situations in which we act. A shift in posture; a change in the room’s atmosphere; a sense that someone—or something—is watching. These subtleties often shape our decisions long before we would describe ourselves as ``reasoning about ethics.’’

The experiment in the previous chapter demonstrates that this sensitivity extends even to \textbf{synthetic presences}: entities that do not feel, do not reason, and do not act in any recognisably moral sense, yet nonetheless influence the way moral salience flows through a situation. The robot in our study did not speak. It did not move meaningfully. It issued no signals of intent. And yet, its mere co-presence shifted the way participants transformed an affectively charged cue—the child’s face on the charity poster—into a decision about donating.

That shift is subtle in magnitude but rich in structure. And it is precisely this structure that the Discussion Chapter now aims to make sense of.

\section{What This Experiment Actually Shows}

The findings of the previous chapter can be summarised directly:

\begin{itemize}
	\item There is a measurable attenuation of prosocial donation when a humanoid robot is present.
	\item This attenuation is not universal: it is most pronounced among individuals whose dispositional architecture foregrounds empathy, sociability, and interpersonal attunement.
	\item Other psychological profiles appear comparatively inert, showing minimal or no change.
	\item Bayesian estimation reinforces this pattern, revealing a probabilistic skew toward attenuation but with uncertainty distributed across clusters.
	\item The Watching Eye stimulus loses some of its intuitive force—not because empathy collapses, but because the robot’s ontological ambiguity refracts the salience of the moral cue.
\end{itemize}

In short, the experiment reveals a \textbf{topology}, not a cause.  
A \textbf{reconfiguration}, not a negation.  
A \textbf{refractor}, not a suppressor.

The robot alters how moral meaning is \emph{processed}, not whether moral meaning exists.  
This is the key interpretive hinge that the Discussion Chapter builds on.

\section{The Broader Significance: Moral Cognition as a Topological Process}

If we take seriously the intuitionist view of moral judgment—that our moral responses begin as fast, affectively-driven impressions, only later supplemented by reflective reasoning—then the results fit into a wider theoretical arc:

\begin{itemize}
	\item Moral cognition is \textbf{context-sensitive}.
	\item It emerges from an \textbf{interaction} between perceptual cues, affective resonance, and situational structure.
	\item It is modulated by \textbf{latent dispositional ecologies}—ways of attuning to the world that differ structurally across individuals.
\end{itemize}

In this frame, synthetic presences become morally relevant not because they ``reason’’ or ``intend,’’ but because they \textbf{change the context in which intuitive appraisal unfolds}. This reframing is essential. It shifts the problem from ``robots making moral decisions’’ to:

\begin{quote}
	\textit{How does the presence of artificial bodies alter the intuitive pathways through which humans interpret moral signals?}
\end{quote}

It is this reframing that opens the conceptual space for the remainder of the Discussion Chapter:
the reinterpretation of the cluster structures, the integration of the Levels of Abstraction framework, the implications for intuitionist moral psychology, the critique of current Machine Ethics, and the broader ethical and epistemic consequences for AI design.

\section{Structure of the Discussion Chapter}

To maintain clarity and momentum, the Discussion Chapter proceeds along four axes:

\begin{enumerate}
	\item \textbf{Revisiting the findings through the lens of moral cognition} – intuitive processes, Watching-Eye literature, and salience flow.
	\item \textbf{Theoretical integration with Floridi’s Levels of Abstraction} – how synthetic presence appears at different LoAs, and why this matters.
	\item \textbf{Implications for Machine Ethics and the ethics of AI presence} – the limits of rule-based machine morality; synthetic agents as moral perturbators.
	\item \textbf{Consequences for design, governance, and future research} – implications for HRI, LLM-based systems, interactive environments, and moral ecosystems.
\end{enumerate}

Each of these sections integrates content that was intentionally removed from the Methods chapter but retained for this interpretive synthesis: the hypothesis analysis, formal framework commentary, topological interpretation, and trait-contingency structure.

Throughout, the chapter preserves the conceptual vocabulary developed earlier—\emph{evaluative deformation}, \emph{normative displacement}, \emph{interpretive topology}—and maintains the narrative cadence of the Introduction: precise, philosophically grounded, and attentive to the lived texture of moral experience.

