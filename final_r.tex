\chapter{General Discussion and Theoretical Integration}
\label{chap:general_discussion}

\section{Introduction: Why the Experiment Requires a Structural Interpretation}

\noindent
The preceding chapters developed three interconnected strands: (i) a cognitive–affective account of moral judgment, (ii) a normative–philosophical reconstruction of ethical theory through the lenses of Level-of-Abstraction discipline and evaluative topology, and (iii) an empirical demonstration that robotic co-presence systematically attenuates prosocial donation under morally salient conditions. The task of the present chapter is not to repeat these analyses, but to integrate them. It offers a theoretical synthesis that explains \emph{why} the experimental effect occurs, \emph{what} its ethical significance is, and \emph{how} it reshapes the methodological landscape for research in Human–Robot Interaction, moral psychology, and the emerging field of Computational Morality.

\noindent
In this sense, the experiment is not an isolated behavioural result but a \textit{probe} into the architecture of moral cognition. The observed attenuation of prosocial behaviour is theoretically meaningful only when interpreted through the structures developed earlier: dual-process architectures, the Social Intuitionist Model, evaluative topology, and the reconstructed normative frameworks of deontology, consequentialism, virtue ethics, sentimentalism, contractualism, and particularism. The present chapter therefore provides a synoptic interpretation in which the behavioural signature revealed by the data becomes a lens through which the nature of moral cognition—and its vulnerability to perturbation—is rendered theoretically transparent.

\subsection{From Behaviour to Structure: Why a Higher-Level Interpretation is Required}

\noindent
The experimental paradigm—Watching-Eye moral cue embedded within a silent synthetic presence—does not merely generate a difference in donation behaviour; it reveals a deformation of the evaluative field that links moral salience to action. Classical interpretations of donation differences (e.g., generosity, altruism, compliance) lack the conceptual resources to capture this phenomenon. A purely behavioural description would record that participants donated less in the Robot condition, especially those belonging to the Prosocial–Empathic cluster. But such a description omits the structural logic that makes the result scientifically and philosophically significant.

\noindent
The central claim developed throughout the thesis is that \emph{moral behaviour is not invariant under changes to the perceptual–social environment}. The robot’s presence does not overwrite moral norms nor impose new ones; instead, it modifies the cognitive–affective conditions under which evaluative forces act. It shifts attentional allocation, alters affective resonance, and modifies the perceived sociality of the space. In topological terms, the robot introduces a perturbation $\gamma_R$ that deforms the curvature of the evaluative manifold, thereby weakening the salience gradient induced by the Watching-Eye stimulus.

\noindent
A simple behavioural difference thus reflects a deeper structural transformation. As shown by the stratified regression models and the Bayesian estimation of donation differences, this transformation is not homogeneous. Rather, it is filtered through latent psychological structures—captured empirically through the PCA–$k$-means clustering—corresponding to three distinct cognitive–affective ecologies:
\begin{itemize}
	\item the \textbf{Emotionally Reactive / Low-Structure Profile},
	\item the \textbf{Prosocial–Empathic / Warm–Sociable Profile},
	\item and the \textbf{Analytical–Structured / High-Systemizing Profile}.
\end{itemize}

\noindent
These ecologies instantiate different evaluative topologies: different attractor formations, salience sensitivities, and modulation pathways. The key empirical result is that the Prosocial–Empathic cluster—the regime most dependent on affectively grounded social cues—exhibits the sharpest attenuation. This is not an accident; it is a structural consequence of how moral cognition is organised.

\subsection{Why This Chapter Cannot Be Pure “Discussion” in the Conventional Sense}

\noindent
Traditional discussion chapters in empirical theses focus on limitations, alternative interpretations, and future work. While valuable, such a format is inadequate for the present research. The experiment operates at the intersection of cognitive science, social robotics, computational modelling, and normative ethics. The behavioural effect observed (\emph{reduced prosocial donation under synthetic presence}) is only the surface-level manifestation of a deeper phenomenon: the reconfiguration of the evaluative machinery through which moral meaning becomes action.

\noindent
To articulate this phenomenon requires a conceptual integration that cannot be confined to standard “discussion” categories. Instead, the chapter must synthesise:
\begin{enumerate}
	\item the \textbf{cognitive architecture} (dual-process, SIM, dynamic integration);
	\item the \textbf{evaluative geometry} (topology, curvature, gradient flow);
	\item the \textbf{normative reconstruction} (deontic invariants, consequentialist gradients, dispositional attractors, sentimentalist vector fields, contractualist justificatory structure, and particularist salience responsiveness);
	\item and the \textbf{empirical structure} of the data (cluster-specific susceptibility, Bayesian attenuation, topological deformation of the Watching-Eye effect).
\end{enumerate}

\noindent
The present chapter therefore functions as an \emph{interpretive pivot}: it translates the empirical findings into philosophical insight, and reinterprets philosophical frameworks in light of empirical constraints.

\subsection{A Structural Reading of the Core Experimental Result}

\noindent
The empirical pattern can be summarised as follows:
\begin{itemize}
	\item The humanoid robot NAO is perceptually salient but ontologically ambiguous.
	\item The Watching-Eye cue normally induces an empathic salience gradient that increases donation.
	\item The robot introduces a perturbation $\gamma_R$ that competes with, and partially overrides, the empathic cue.
	\item This attenuation is strongest in the Prosocial–Empathic cluster, weaker in the Analytical–Structured cluster, and negligible in the Emotionally Reactive cluster.
\end{itemize}

\noindent
Interpreted through the cognitive framework, this pattern shows that:
\begin{enumerate}
	\item moral appraisal begins with intuitive and affective resonance;
	\item synthetic presence disrupts this resonance by altering attention and perceived sociality;
	\item different dispositional structures process this disruption differently;
	\item resulting behavioural output reflects the deformation of the evaluative field.
\end{enumerate}

\noindent
Interpreted through the normative framework, we obtain:
\begin{itemize}
	\item a \textbf{deontological reading}: synthetic presence weakens the implicit deontic expectations cued by the Watching-Eye prime;
	\item a \textbf{consequentialist reading}: synthetic presence flattens the perceived payoff gradient of helping behaviour;
	\item a \textbf{virtue-ethical reading}: synthetic presence interacts with dispositional attractors, suppressing the stable orientation toward prosociality in the Prosocial–Empathic profile;
	\item a \textbf{sentimentalist reading}: synthetic presence dampens empathic vector fields that ordinarily energise prosocial action;
	\item a \textbf{contractualist reading}: synthetic presence disrupts the relational justifiability equilibrium normally activated by moral cues;
	\item a \textbf{particularist reading}: synthetic presence alters the salience landscape such that local cues no longer generate the same moral significance.
\end{itemize}

\noindent
Thus, each normative theory yields a structurally distinct but empirically convergent interpretation. The ethical significance of the experiment does not lie in any single framework but in the \emph{coherent intersection} of all of them.

\subsection{Why the Synthetic Presence Effect Matters Beyond the Experiment}

\noindent
The attenuation of moral action under synthetic presence is not just an interesting behavioural anomaly; it is a demonstration of a deeper principle: \emph{moral cognition is structurally permeable}. It is sensitive to perturbations that operate below the level of explicit reasoning. It is vulnerable to changes in perceived social ontology. It is modulated by affectively weighted cues whose role is seldom acknowledged in normative theory and almost never incorporated in classical Machine Ethics.

\noindent
This has far-reaching implications:
\begin{enumerate}
	\item It challenges the assumption that artificial agents can be designed according to purely deliberative ethical frameworks.
	\item It reveals that synthetic presence exerts moral influence even without action, speech, intent, or agency.
	\item It shows that human–robot environments are \emph{ethically loaded} by virtue of perceptual and affective structure alone.
	\item It demands a reconsideration of how artificial systems are interpreted within the moral ecology of human decision-making.
\end{enumerate}

\noindent
In short, the experiment demonstrates a fact of philosophical significance: \textit{synthetic agents are not normatively inert}. Their presence, even in silent passivity, can deform the evaluative pathways through which moral meaning becomes action.

\bigskip

\noindent
The remainder of this chapter builds on this foundation. Subsequent sections provide:
\begin{itemize}
	\item a cluster-by-cluster integrative interpretation,
	\item a cross-framework normative synthesis,
	\item a critique of monolithic Machine Ethics,
	\item a reconstruction of Computational Morality grounded in empirical structure,
	\item and a final consolidation of the thesis’ theoretical contributions.
\end{itemize}

\noindent
The goal is not only to interpret the experiment, but to show how the experiment reconfigures the conceptual terrain on which research in moral psychology, HRI, and machine ethics must proceed.
