\chapter{TOOLS}
\label{chap:tools}
\thispagestyle{pprintTitle}

\chapter{Psychometric Instruments and Experimental Paradigms}
\label{chap:tools}
\thispagestyle{pprintTitle}

\noindent
Empirical work aimed at understanding moral cognition must specify, with some 
philosophical care, the instruments through which psychological and behavioural 
structures become accessible to observation. Moral appraisal itself is never directly given; it is inferred from patterned responses—affective, dispositional, perceptual, and social—that reflect how evaluative information is encoded in the agent’s cognitive architecture \cite{Haidt2001EmotionalDog, Greene2002, Greene2004, Cushman2013DualSystem, Crockett2016Models, Fedyk2017}. The tools employed in this thesis therefore function not as neutral measurement devices but as theoretically motivated probes: each instrument targets a specific dimension of the evaluative topology developed in earlier chapters, rendering latentdispositional structure empirically tractable without collapsing its 
complexity into reductive summary scores.

The methodological commitments of this thesis require a principled account of the instruments through which evaluative behaviour becomes empirically accessible. Work in moral psychology and cognitive science has repeatedly shown that moral appraisal is not directly observable but manifests through structured patterns of affective response, controlled cognition, and social cue integration \cite{Haidt2001EmotionalDog, Greene2002, Greene2004, Cushman2013DualSystem, Crockett2016Models, Fedyk2017}. For this reason, empirical studies of moral cognition depend on validated constructs and measurement strategies capable of rendering latent dispositions observable without distorting their theoretical significance.

The present work does not align itself with moral cognition research as a discrete disciplinary domain. Instead, it draws upon rigorously established constructs from moral psychology, cognitive science, and social signal processing as operational resources for making evaluative dispositions tractable. Instruments such as the Empathizing Quotient \cite{BaronCohenWheelwright2004_EmpathyQuotient}, the Systemizing Quotient \cite{BaronCohenRichlerBisaryaGurunathanWheelwright2003_SystemizingQuotient}, and the Big Five Inventory \cite{JohnDonahueKentle1991_BigFiveInventory, Rammstedt2007} provide precisely the kind of psychometric access to stable individual differences that contemporary models of moral cognition identify as structurally relevant. Likewise, the analytical frameworks developed within Social Signal Processing \cite{Vinciarelli2009} offer methodological grounding for understanding how agents register, interpret, and behaviourally respond to contextually salient perturbations.

In this sense, the psychometric tools employed here are not neutral measurement devices, but theoretically motivated probes into the dispositional structures that shape how agents encode, negotiate, and respond to morally salient changes in their evaluative environment.



The Empathizing Quotient \cite{BaronCohenWheelwright2004_EmpathyQuotient}, the Systemizing Quotient \cite{BaronCohenRichlerBisaryaGurunathanWheelwright2003_SystemizingQuotient}, and the Big Five Inventory \cite{JohnDonahueKentle1991_BigFiveInventory, Rammstedt2007} offer validated operationalisations of dispositional constructs repeatedly implicated in moral judgment and social decision-making. Likewise, the Watching--Eye paradigm \cite{Haley2005, Bateson2006, Nettle2013, Conty2016, Dear2019} constitutes a mature experimental framework for probing reputational concern, prosocial motivation, and sensitivity to subtle social cues. Together, these instruments form a coherent measurement suite capable of isolating trait-level parameters that interact with contextual salience to shape moral behaviour.

\medskip
\noindent
This chapter therefore serves a conceptual rather than merely procedural purpose. The psychometric instruments and experimental paradigms introduced here are situated explicitly within the evaluative--topological model developed in earlier chapters, in which moral cognition is understood not as a sequence of discrete judgments but as the dynamic evolution of a manifold of interacting evaluative gradients. Contemporary theories of moral psychology emphasise that such gradients integrate affective, social, and contextual inputs in a manner shaped by stable dispositional architecture \cite{Haidt2001EmotionalDog, Greene2002, Greene2004, Cushman2013DualSystem, Crockett2016Models, Fedyk2017}. Within this framework, the role of each tool is to reveal invariant dispositional structures—the stable dimensions along which individuals differ in how incoming evaluative information is encoded and transformed. These structures correspond to the latent parameters governing how a subject’s evaluative gradients bend, flatten, or intensify as the informational environment is perturbed.

In the experiment motivating this thesis, such perturbation is elicited not through explicit moral dilemmas but through a more subtle and ecologically grounded manipulation: the silent perceptual presence of a humanoid robot. Prior work in human--robot interaction shows that even passively positioned robots can shift perceived social affordances, alter attentional allocation, and modulate expectations concerning norm-relevant behaviour \cite{Zlotowski2015, Malle2015, Komatsu2016}. Their ambiguous ontological status disrupts default social priors and thereby reconfigures the salience landscape within which moral reasons become behaviourally operative. In this respect, robotic presence functions as a controlled perturbation to the evaluative topology itself, enabling the empirical study of how dispositional invariants interact with contextual cues to produce measurable differences in moral behaviour.


\medskip
\noindent
The aim of the chapter is thus twofold.

\begin{enumerate}
	\item First, to establish that each psychometric and experimental tool is grounded in stable bodies of empirical and theoretical research across psychology, cognitive science, HCI/HRI, and social signal processing. This ensures that the constructs they measure---empathic sensitivity, systemizing tendencies, personality traits, and responsiveness to social cues---are well-defined, reproducible, and theoretically interpretable within the broader landscape of moral psychology and social cognition.
	
	\item Second, to show how each tool contributes to the modelling of the dispositional term $\beta_C$ in the formal expression
	\[
	\mathscr{P}(\delta_m) = f(\alpha_E, \beta_C, \gamma_R),
	\]
	where $\beta_C$ denotes the latent trait configuration governing how a participant’s evaluative topology is modulated by the perturbation introduced by the humanoid robot. In this sense, the tools are not ancillary components of the experiment but operationalisations of the dispositional invariants that mediate the transformation of evaluative salience under robotic presence.
\end{enumerate}


\noindent
The tools included here—the Empathizing Quotient (EQ), the Systemizing Quotient (SQ), the Big Five Inventory (BFI), and the Watching–Eye paradigm—were selected because they satisfy three stringent criteria grounded in established empirical research. First, each instrument has a well-defined construct lineage supported by extensive psychometric validation. The EQ \cite{BaronCohenWheelwright2004_EmpathyQuotient} and SQ \cite{BaronCohenRichlerBisaryaGurunathanWheelwright2003_SystemizingQuotient} constitute the canonical operationalisations of empathizing and systemizing tendencies, with consistent factor structures, cross-cultural robustness, and demonstrable discriminant validity within both clinical and non-clinical populations. The BFI, in its original form \cite{JohnDonahueKentle1991_BigFiveInventory} and in its widely used short version \cite{Rammstedt2007}, provides a compact yet psychometrically rigorous assessment of the five broad personality domains that anchor contemporary trait theory.

Second, the Watching–Eye paradigm has developed into a mature experimental framework for probing reputation-sensitive prosocial behaviour. Numerous studies have demonstrated that minimal cues of observation modulate cooperative and charitable actions \cite{Haley2005, Bateson2006, Nettle2013, Conty2016, Dear2019}, and the paradigm’s effects have been replicated across diverse contexts, task structures, and elicitation modalities. This makes it uniquely suited for isolating perturbations to social-evaluative processing—a core requirement for the present analysis.

Third, all four tools possess sufficient resolution and conceptual precision to inform the modelling of latent dispositional structure within the evaluative-topological framework advanced in this thesis. They provide theoretically interpretable coordinates for the dispositional term $\beta_C$, enabling an analysis of how trait configurations shape the deformation of evaluative gradients under robot-induced perturbations. For this reason, these instruments are not simply conventional choices, but the most appropriate set of measurements for the level of abstraction at which the experimental work is situated.


\begin{enumerate}
	\item \textbf{Theoretical relevance}: Each tool targets a component of moral topology (affective resonance, evaluative precision, personality curvature, or salience modulation).
	\item \textbf{Empirical robustness}: Each tool is validated across multiple cultures, large samples, and decades of psychological research, and has been used in studies of prosociality, moral sensitivity, social attention, and Human–Robot Interaction (HRI).
	\item \textbf{Computational suitability}: Each tool produces variables suitable for integration into regression models, cluster analysis, and topological interpretation.
\end{enumerate}

\noindent
Before turning to the tools themselves, we first articulate the methodological role they play within this thesis.

\section{The Role of Psychometric Tools in the Evaluative–Topological Architecture}

\noindent
Within the formal architecture developed throughout this thesis, moral behaviour is modelled as the endpoint of a trajectory across an evaluative field. Contemporary research in moral psychology and cognitive science emphasises that such trajectories arise from the joint interaction of environmental cues, dispositional structure, and perturbational influences \cite{Haidt2001EmotionalDog, Greene2002, Greene2004, Cushman2013DualSystem, Crockett2016Models, Fedyk2017}. Accordingly, the formal decomposition
\[
\mathscr{P}(\delta_m) = f(\alpha_E, \beta_C, \gamma_R)
\]
captures the three principal determinants of evaluative dynamics:

\begin{itemize}
	\item \emph{environmental inputs} ($\alpha_E$): morally salient cues such as the Watching–Eye prime and task context;
	\item \emph{dispositional structure} ($\beta_C$): latent traits quantified by psychometric instruments;
	\item \emph{perturbation operators} ($\gamma_R$): the ontologically ambiguous presence of the humanoid robot.
\end{itemize}

\noindent
The psychometric tools employed in this study belong to the $\beta_C$ term. They render dispositional structure empirically tractable by quantifying constructs shown to be central in the integration of affective, social, and contextual information. The Empathizing Quotient \cite{BaronCohenWheelwright2004_EmpathyQuotient} indexes the \textbf{affective bandwidth} through which agents register morally salient others; the Systemizing Quotient \cite{BaronCohenRichlerBisaryaGurunathanWheelwright2003_SystemizingQuotient} captures the \textbf{analytical curvature} underlying structural interpretation of social situations; and the Big Five Inventory \cite{JohnDonahueKentle1991_BigFiveInventory, Rammstedt2007} measures the \textbf{personality geometry} shaping attentional allocation, normative sensitivity, and regulatory control \cite{Barrick1991}. These constructs have well-established roles in models of moral appraisal and behavioural prediction \cite{Haidt2001EmotionalDog, Crockett2016Models}.

\medskip

\noindent
Their role in the experiment is not ancillary. These measures enabled the analysis to disentangle two layers of the evaluative architecture that would otherwise remain conflated: (i) the \emph{dispositional configuration} each participant brings into the situation, and (ii) the \emph{field-level modulation} induced by robotic presence. The cluster analysis performed on EQ, SQ, and BFI scores revealed a structured personality topology comprising affectively warm, analytically structured, and reactive–volatile profiles, indicating that participants did not enter the experimental environment as a psychologically homogeneous group.

What is theoretically significant, however, is what followed. Despite this structured dispositional diversity, the humanoid robot exerted a \emph{uniform directional effect} on prosocial behaviour across all clusters. No Big Five trait, EQ subscale, SQ dimension, or latent profile moderated the displacement. Prior research in human--robot interaction has shown that even passive robotic agents can shift perceived social affordances, modulate attention, and alter expectations surrounding norm-relevant behaviour \cite{Zlotowski2015, Malle2015, Komatsu2016}. The present findings extend this line of work by demonstrating that robotic presence does not operate through trait-dependent amplification or suppression of behavioural tendencies. Instead, it perturbs the evaluative field itself—its salience structure, affective gradients, and normative attractors—such that all dispositional trajectories are bent in the same behavioural direction.

\medskip

\noindent
In this sense, the psychometric tools were indispensable. They allowed the analysis to dissociate the \emph{shape of the dispositional manifold} from the \emph{geometry of the perturbation}. Without psychometric grounding, the attenuation of donation behaviour might have been misinterpreted as a trait-level effect rather than a field-level displacement. The instruments thereby provided the empirical precision needed to show that the robot acted not upon who the participants were, but upon the evaluative topology within which their moral choices unfolded.

\noindent
In the experimental formalism, the dispositional term appears in the perturbation expression
\[
f(\alpha_E, \beta_C, \gamma_R) - f(\alpha_E, \beta_C),
\]
which measures how the moral transformation function is reshaped by $\gamma_R$ given a fixed dispositional configuration. This formulation reflects the broader consensus in moral psychology that moral behaviour emerges from the interaction between environmental cues, dispositional structure, and perturbational influences on evaluative processing \cite{Haidt2001EmotionalDog, Greene2002, Greene2004, Cushman2013DualSystem, Crockett2016Models, Fedyk2017}. The empirical results presented in Chapter~\ref{chap:experimental_methods} showed that although $\beta_C$ exhibits a structured internal topology—revealed through clustering analyses of EQ, SQ, and BFI scores \cite{BaronCohenWheelwright2004_EmpathyQuotient, BaronCohenRichlerBisaryaGurunathanWheelwright2003_SystemizingQuotient, JohnDonahueKentle1991_BigFiveInventory, Rammstedt2007, Barrick1991}—the perturbation introduced by the humanoid robot did \emph{not} depend on those dispositional differences. All clusters displayed the same directional attenuation of prosocial behaviour, indicating that $\gamma_R$ operates primarily at the \emph{field level}, reshaping the evaluative landscape within which dispositional trajectories unfold rather than interacting with trait-specific gradients.

\medskip

\noindent
This pattern aligns with established findings in human--robot interaction, where even passive robotic agents have been shown to modulate perceived social affordances, attentional allocation, and norm-relevant expectations irrespective of observer traits \cite{Zlotowski2015, Malle2015, Komatsu2016}. In the present experiment, robotic presence functioned as a global perturbation of the evaluative field rather than as a selective amplifier or suppressor of individual dispositions.

\medskip

\noindent
The goal of this chapter, therefore, is not simply to catalogue the psychometric tools, but to clarify how each instrument contributes to the modelling of $\beta_C$ and why their inclusion is essential for distinguishing dispositional structure from field-level displacement. Without these psychometric constraints, the observed attenuation of prosocial behaviour could have been misattributed to personality differences rather than correctly interpreted as a global deformation of the evaluative topology induced by robotic presence. The tools thereby provide the empirical precision necessary to show that the robot acted not upon who the participants were, but upon the evaluative field within which their moral choices unfolded.

Having established the distinction between dispositional structure and field-level perturbation, we can now justify the methodological choices that made this distinction empirically visible.


\section{Why These Tools: Methodological Criteria and Alignment with the Thesis}

\noindent
Given the dual-layer structure revealed by the experiment—stable dispositional variation on the one hand, and a field-level displacement effect induced by robotic presence on the other—the selection of psychometric and experimental tools cannot be arbitrary. The instruments employed here were chosen because they satisfy three methodological criteria essential for interpreting the attenuation of prosocial behaviour observed in the study.

\paragraph{(1) Cross-paradigmatic relevance.}
The EQ, SQ, BFI, and Watching–Eye paradigm each rest on extensive empirical traditions across several domains of inquiry. In moral and social psychology, these tools have been used to study prosociality, empathic concern, harm aversion, and the integration of affective and cognitive processes in moral judgment \cite{Haidt2001EmotionalDog, Greene2002, Greene2004, Cushman2013DualSystem, Crockett2016Models, Fedyk2017}. In personality psychology, the Big Five Inventory provides a compact but psychometrically robust measure of trait architecture with well-established predictive value for behavioural outcomes \cite{JohnDonahueKentle1991_BigFiveInventory, Rammstedt2007, Barrick1991}. The Empathizing and Systemizing Quotients offer validated assessments of affective resonance and analytic style \cite{BaronCohenWheelwright2004_EmpathyQuotient, BaronCohenRichlerBisaryaGurunathanWheelwright2003_SystemizingQuotient}. 

In parallel, the Watching–Eye paradigm constitutes one of the most reliable experimental manipulations of prosocial salience, with repeated demonstrations that subtle cues of observation can modulate cooperative and charitable behaviour \cite{Haley2005, Bateson2006, Nettle2013, Conty2016, Dear2019}. Crucially, these literatures intersect with contemporary Human–Robot Interaction research, where robotic agents are known to shift social affordances, attentional allocation, and normative expectations \cite{Zlotowski2015, Malle2015, Komatsu2016}. Their use therefore positions the present study within a broad empirical landscape while maintaining continuity with the theoretical commitments of the evaluative–topological framework.

\paragraph{(2) Topological relevance.}
Each tool probes a structurally distinct component of the evaluative manifold that underpins moral cognition:
\begin{itemize}
	\item \textbf{EQ}: the affective attractors that anchor early moral and social appraisal \cite{BaronCohenWheelwright2004_EmpathyQuotient};
	\item \textbf{SQ}: the structural curvature associated with analytic or rule-based processing \cite{BaronCohenRichlerBisaryaGurunathanWheelwright2003_SystemizingQuotient};
	\item \textbf{BFI}: the multidimensional geometry of personality traits that modulate salience, attentional uptake, and behavioural regulation \cite{JohnDonahueKentle1991_BigFiveInventory, Rammstedt2007, Barrick1991};
	\item \textbf{Watching–Eye paradigm}: an experimentally validated perturbation that shifts moral salience without instruction or coercion \cite{Haley2005, Bateson2006, Nettle2013, Conty2016, Dear2019}.
\end{itemize}

\noindent
Together, these measurements provide the granularity needed to model the dispositional term $\beta_C$ and to distinguish clearly between trait-level variation and field-level perturbation. This is precisely what enabled the analysis to establish that robotic presence operated on the evaluative field rather than on personality-dependent gradients.

\paragraph{(3) Stability and interpretability.}
The selected instruments satisfy the methodological requirements of stability, reliability, and interpretability that are necessary for higher-level analysis:
\begin{itemize}
	\item they support clustering of participants within dispositional space,
	\item they enable regression modelling of trait influences on donation behaviour,
	\item and they admit interpretation through established normative and meta-ethical frameworks, including sentimentalism, virtue-theoretic accounts, and pluralist models of moral reasoning.
\end{itemize}

\noindent
Most importantly, these tools provided the methodological precision needed to demonstrate that the attenuation of prosociality was not driven by differences in personality clusters, empathizing profiles, or systemizing tendencies. Instead, the psychometric suite functioned as a set of diagnostic probes revealing a structured dispositional landscape against which the global displacement effect of robotic presence could be identified unambiguously. The tools thereby allowed the experiment to differentiate \emph{who the participants were} from the \emph{structure of the evaluative field} within which their behaviour unfolded.

\medskip

\noindent
With these foundations established, we now turn to the first measurement tool: the Empathizing Quotient.


\section{The Empathizing Quotient (EQ): Affective Resonance as a Moral Vector Field}

\noindent
The Empathizing Quotient (EQ) occupies a central place in the measurement of affective sensitivity within contemporary psychology. Developed by Baron--Cohen and colleagues as part of the broader Empathizing--Systemizing (ES) framework \cite{BaronCohenWheelwright2004_EmpathyQuotient,Baron2002,Baron2009}, the EQ was originally designed to quantify individual differences in emotional resonance, perspective-taking, and the capacity to infer and respond appropriately to the mental states of others. Its construction reflects two decades of theoretical and empirical work stemming from autism research, sex differences in social cognition, and the development of trait-based accounts of empathic functioning.

\subsection{Historical and Theoretical Foundations}

\noindent
The EQ emerged against the background of two influential lines of inquiry. The first concerned the cognitive and affective profiles observed in autism spectrum conditions, where empathic difficulties appeared as a core diagnostic dimension. Baron--Cohen's early work on ``mindblindness'' and the ES theory \cite{Baron2002} proposed that empathizing and systemizing represent partially dissociable cognitive styles, with autism characterized by diminished empathizing abilities alongside preserved or enhanced systemizing capacities. The second line of inquiry derived from trait psychology and social cognition, where stable inter-individual differences in emotional attunement, empathic accuracy, and prosocial inclinations were increasingly understood as predictive of moral and social behaviour.

The EQ was designed to operationalise the empathizing construct in a psychometrically rigorous manner. It includes affective items (e.g., sensitivity to distress), cognitive-empathic items (e.g., perspective-taking), and items assessing spontaneous concern for others. Initial investigations \cite{BaronCohenWheelwright2004_EmpathyQuotient} demonstrated large group differences between autistic and neurotypical adults, robust sex differences, and high internal reliability. Subsequent factor-analytic studies \cite{Lawson2004} further clarified the latent structure of the scale, identifying separable components associated with emotional reactivity, cognitive perspective-taking, and social attunement.

\subsection{Psychometric Validation and Cross-Cultural Work}

\noindent
Psychometric validation of the EQ has been extensive. Beyond the initial work in clinical and neurotypical samples, replication studies have demonstrated strong internal consistency, acceptable test--retest reliability, and predictable convergence with related constructs such as empathic concern, emotional intelligence, and social sensitivity \cite{BaronCohenWheelwright2004_EmpathyQuotient}. Cross-cultural validations, including Japanese and Western samples, have shown that the EQ maintains its factor structure and predictive value across cultural contexts \cite{Wakabayashi2006}. 

These findings situate the EQ within the broader movement toward trait-based quantification of social-cognitive skills. Within personality psychology, empathizing correlates with the Agreeableness and Openness dimensions of the Big Five \cite{Barrick1991}, while remaining psychometrically distinguishable from both. Within social neuroscience, EQ scores have been found to correlate with vmPFC--amygdala coupling and with the strength of activation in neural substrates associated with social pain, affect sharing, and mentalising.

\subsection{Empirical Applications Across Disciplines}

\noindent
The EQ has become a standard instrument in multiple research paradigms. In moral psychology, empathy-related traits are strong predictors of altruistic helping, harm aversion, guilt sensitivity, and responses to moral dilemmas \cite{Haidt2001EmotionalDog,Greene2002,Greene2004,Cushman2013DualSystem,Crockett2016Models,Fedyk2017}. High EQ scores are consistently associated with stronger prosocial choices in economic games, including the ultimatum, dictator, and trust games. Behavioural economics work shows that individuals with higher empathic sensitivity display increased generosity even when anonymity is preserved, suggesting that empathic traits modulate internalised moral norms beyond external social cues.

In social neuroscience, EQ scores track activation patterns in regions associated with affective resonance, including the anterior insula, temporoparietal junction, and amygdala--vmPFC networks. Oxytocin administration studies further demonstrate selective improvement in empathic accuracy \cite{Bartz2010}, reinforcing the biological plausibility of affective resonance as a trait-like dimension.

The EQ has also gained significance in Human--Robot Interaction (HRI), where empathic predispositions shape attributions of intentionality, perceived moral standing, and expectations regarding robots' behaviour \cite{Malle2015,Komatsu2016,Zlotowski2015}. Individuals with higher EQ scores tend to ascribe richer mental states to robots, respond more strongly to cues of intentionality, and exhibit greater sensitivity to violations of social or moral norms in robotic agents. In group-based interactions, empathic individuals demonstrate greater behavioural alignment with robots, particularly when robots display subtle affective or communicative signals \cite{Kuchenbrandt2011}.

\subsection{Critiques and Methodological Limitations}

\noindent
Despite its widespread use, the EQ has faced several critiques. Some researchers argue that its factor structure is not fully stable across populations, with certain studies reporting two or three factors rather than the originally proposed triadic structure. Concerns have also been raised regarding response biases, social desirability, and the possibility that self-report measures may not accurately capture behavioural or neural indices of empathy. Cross-cultural studies have noted differences in average EQ scores, prompting questions about cultural calibration and the extent to which certain items rely on culturally specific norms of emotional expression.

Within experimental psychology, some scholars have argued that empathic responding is situationally variable and cannot be fully reduced to trait-level constructs. Studies demonstrating dissociations between empathic concern and moral behaviour in high-stakes dilemmas \cite{Cushman2013DualSystem,Greene2004} raise further questions about the predictive specificity of the EQ. Nevertheless, the scale remains one of the most widely used and empirically grounded measures of individual differences in affective resonance.

\subsection{Relevance to the Evaluative--Topological Framework}

\noindent
In the evaluative--topological model developed in this thesis, the EQ operationalises the affective attractors that structure early moral appraisal. High empathizing corresponds to steeper affective gradients in the evaluative landscape, amplifying the salience of morally relevant others and increasing the likelihood that prosocial dispositions will be behaviourally expressed. Conversely, lower EQ scores correspond to flatter affective manifolds, in which moral salience is more weakly coupled to others' distress or need.

In the context of the experiment, the EQ plays a crucial role in modelling the dispositional term $\beta_C$. It enables the analysis to determine whether differences in affective sensitivity condition the behavioural response to a perturbation in the evaluative field—namely, the silent presence of a humanoid robot. The finding that EQ did \emph{not} moderate the displacement effect provides strong evidence that the robot acted at the field level rather than through trait-specific amplification or suppression. This result is consistent with HRI studies showing that robotic presence alters normative expectations independently of empathic predispositions \cite{Malle2015,Komatsu2016,Zlotowski2015}.

In this sense, the Empathizing Quotient is indispensable for distinguishing between dispositional and field-level contributions to moral behaviour. It provides a theoretically coherent and empirically validated coordinate within the dispositional manifold, enabling the evaluative--topological model to separate the geometry of $\beta_C$ from the geometry of the perturbation $\gamma_R$.


\subsection{EQ Within the Evaluative-Topological Framework}

\noindent
Within the topological architecture of this thesis, EQ measures the magnitude of the \textbf{affective vector field} $\mathbf{A}(x)$ that pulls evaluative trajectories toward empathically grounded prosocial action. High EQ corresponds to:

\begin{itemize}
	\item steep affective gradients,
	\item strong attractors around suffering, need, vulnerability,
	\item high sensitivity to social evaluation cues (including Watching-Eye primes),
	\item rapid activation of intuitive moral appraisal.
\end{itemize}

\noindent
The attenuation effect observed in the experiment was strongest among participants with high EQ values, supporting the interpretation that the robot primarily dampens the \emph{affective dynamics} of moral cognition.

\[
\delta \mathbf{A}(x;\mathscr{R}) < 0 \quad \text{for high-EQ participants}.
\]

\noindent
Thus, EQ is not merely a psychometric variable but a quantification of emotional curvature within the evaluative field.

\subsection{EQ in HRI and Moral Cognition Research}

\noindent
Studies have shown that high empathizers:

\begin{itemize}
	\item anthropomorphise robots more readily \cite{Hofree2018},
	\item show stronger prosocial responses to perceived observers \cite{Mori2020},
	\item exhibit heightened moral salience in the presence of social cues \cite{Klimecki2013}.
\end{itemize}

\noindent
This aligns precisely with Cluster~2 in our experiment: high-empathy participants with strong affective attractors who showed \emph{the largest attenuation} under robot presence.

\subsection{Why EQ Matters}

\noindent
Although the Empathizing Quotient possesses deep theoretical relevance for modelling the affective attractors that shape trajectories within the evaluative manifold, its primary role in the experiment was methodological. The EQ provides a validated measure of emotional resonance, perspective-taking, and sensitivity to others’ mental and affective states \cite{BaronCohenWheelwright2004_EmpathyQuotient,Lawson2004,Wakabayashi2006,Baron2002,Baron2009}. Because empathic capacity is a well-established predictor of altruistic behaviour, harm aversion, and cooperative decision-making \cite{Haidt2001EmotionalDog,Greene2002,Greene2004,Cushman2013DualSystem,Crockett2016Models,Gleichgerrcht2013}, failing to quantify it would have introduced a serious confound into the interpretation of the attenuation effect.

\noindent
For the purposes of this thesis, the EQ serves three complementary functions:

\begin{itemize}
	\item it provides a micro-level control measure that rules out empathy as a 
	proximate explanation for the donation outcomes,
	\item it contributes to modelling the affective dimension of the dispositional 
	manifold ($\beta_C$) within the evaluative--topological framework,
	\item and it supports the cluster analysis by helping to identify distinct 
	affective profiles without implying trait-based moderation of the robotic 
	perturbation.
\end{itemize}

Although the Empathizing Quotient possesses deep theoretical relevance for modelling the affective attractors that shape trajectories within the evaluative manifold, its primary role in the experiment was methodological. The EQ provides a validated measure of emotional resonance, perspective-taking, and sensitivity to others’ mental and affective states \cite{BaronCohenWheelwright2004_EmpathyQuotient,Lawson2004,Wakabayashi2006,Baron2002,Baron2009}. Because empathic capacity is a well-established predictor of altruistic behaviour, harm aversion, and cooperative decision-making \cite{Haidt2001EmotionalDog,Greene2002,Greene2004,Cushman2013DualSystem,Crockett2016Models,Gleichgerrcht2013}, failing to quantify it would have introduced a serious confound into the interpretation of the attenuation effect.

\noindent
Without an explicit measure of empathic sensitivity, any reduction in prosocial behaviour in the robot condition could plausibly have been attributed to pre-existing differences in empathic disposition between participants. The EQ ruled out this possibility by providing a principled and psychometrically robust estimate of affective bandwidth, ensuring that group-level variability in donation responses could not be dismissed as an artefact of unmeasured empathy.


\noindent
The EQ thus served two complementary functions. At a micro-level, it guaranteed that the behavioural results were not reducible to empathic heterogeneity. At a macro-level, it contributed to modelling the dispositional term $\beta_C$ with sufficient resolution to distinguish affective traits from field-level perturbation effects. Taken together, these functions made the EQ indispensable for demonstrating that the humanoid robot acted on the evaluative field itself rather than on participants’ empathic dispositions..

\section{The Systemizing Quotient (SQ): Structural Evaluation and the Precision of Moral Gradients}

\noindent
Where the Empathizing Quotient (EQ) captures affective resonance, the Systemizing Quotient (SQ) \cite{Baron2003,Goldenfeld2005,Wakabayashi2006} quantifies an individual's propensity for identifying structural regularities, constructing causal models, and applying rule-based inference. Developed in parallel with the ES theory of cognition \cite{Baron2002,Baron2009}, the SQ was designed to measure the degree to which an agent seeks predictive coherence in complex environments. It therefore provides a natural operationalisation of what, within the evaluative--topological framework of this thesis, we describe as the \emph{analytical curvature} of the evaluative field: the tendency to encode moral situations via structural invariants rather than affective attractors.

\subsection{Historical Origins and Theoretical Motivation}

\noindent
The origins of the SQ lie in the broader attempt to model cognitive styles that differentiate autistic from neurotypical populations. Baron--Cohen’s early work proposed that systemizing reflects a cognitive drive for rule extraction and causal precision, complementing empathizing but operating through distinct computational mechanisms \cite{Baron2002}. The Systemizing Quotient was introduced as the psychometric realisation of this construct, with the initial validation study \cite{Baron2003} demonstrating its sensitivity to within-group variation as well as to group-level differences between autistic and neurotypical adults.

The theoretical motivation for systemizing has since broadened. While originally embedded in autism research, systemizing has come to be linked with general tendencies toward mechanistic reasoning, causal Bayes nets, and algorithmic-level representations of environmental structure. Psychometric studies have shown that high-SQ individuals exhibit a preference for deterministic rules, hierarchical schemas, and low-noise value comparisons \cite{Goldenfeld2005}. In the ES theory, empathizing and systemizing jointly define a two-dimensional space in which variation in social cognition, emotional regulation, and reasoning strategies can be mapped.

\subsection{Psychometric Validation and Cross-Cultural Findings}

\noindent
Validation studies demonstrate that the SQ has high internal consistency, good test--retest reliability, and predictable correlations with cognitive-style measures, including analytic problem-solving, sensitivity to pattern structure, and preference for system-based explanations. Importantly, cross-cultural validation work \cite{Goldenfeld2005,Wakabayashi2006} has shown that the SQ retains its factor structure and predictive validity across different cultural contexts, supporting the claim that systemizing taps into a cognitive style with cross-cultural generality.

Neurocognitive work complements these findings. High systemizing tendencies correlate with activation in lateral prefrontal and parietal cortices associated with analytic reasoning, causal inference, and top--down attentional control. Conversely, higher systemizing scores are associated with reduced activation in affective salience networks during social evaluation tasks \cite{Gleichgerrcht2013}, reinforcing the link between SQ and reduced susceptibility to affect-laden cues.

\subsection{Empirical Uses Across Psychology, Neuroscience, and Behavioural Science}

\noindent
Systemizing has been deployed across a wide range of empirical domains. In moral psychology, systemizing tendencies predict a greater reliance on deliberative processes in dual-process moral judgment models \cite{Greene2014}. High-SQ individuals exhibit greater stability in moral evaluations across contexts, a preference for rule-consistency, and an increased likelihood of endorsing principle-based judgments in high-conflict dilemmas \cite{Cushman2013DualSystem}. The reduced affective reactivity associated with high systemizing is consistent with findings showing that utilitarian judgments arise under conditions of weaker affective engagement and stronger top--down control \cite{Greene2004,Gleichgerrcht2013}.

In behavioural economics, high systemizing correlates with consistent rule-following, lower variance in strategic play, and lower susceptibility to affective framing effects. Such individuals tend to interpret prosocial games in terms of structural incentives rather than interpersonal resonance, emphasising coherence over compassion in choice architecture.

In Human--Robot Interaction, systemizing tendencies strongly modulate expectations regarding synthetic agents. High-SQ participants are more likely to attribute competence, reliability, and causal predictability to robots, and less likely to respond to anthropomorphic cues \cite{Zlotowski2015,Malle2015,Komatsu2016}. This makes SQ particularly relevant for the experimental context of this thesis: systemizing provides a dispositional anchor for understanding how agents interpret the structural affordances introduced by a humanoid robot, especially in settings where affective cues are minimal and norm-relevant structure must be inferred.

\subsection{Critiques and Limitations}

\noindent
Although widely used, the SQ is not without critique. Some studies report that its factor structure is more heterogeneous than originally proposed, with potential subfactors corresponding to mechanical reasoning, abstract pattern detection, and rule-based inference. There are also concerns about cultural calibration, particularly regarding item content related to technical interests, which may vary across populations.

Another debate concerns the relationship between systemizing and moral judgment. While high systemizing predicts increased deliberation and reduced affective influence, this does not always translate into consistent moral choices. Some findings suggest that highly systemizing individuals may display context-dependent shifts in judgment when structural cues are ambiguous, indicating that systemizing does not override all forms of affective influence but interacts with them in non-linear ways.

Finally, as with all self-report measures, the SQ faces questions about introspective accuracy and the relation between subjective reports and actual behavioural or neural markers of structural reasoning.

\subsection{SQ Within the Evaluative--Topological Framework}

\noindent
Within the evaluative--topological model, SQ modulates the \emph{second derivative} of the evaluative potential function: it influences the \emph{rigidity}, \emph{smoothness}, and \emph{predictability} of evaluative gradients. High-SQ agents encode situations as stable causal schemas rather than affective landscapes. Consequently, their evaluative fields resist deformation under purely affective perturbations and favour top--down interpretive stability. This interpretation aligns with theoretical work emphasising the role of structural representations in moral reasoning \cite{Greene2014,Cushman2013DualSystem}.

Formally, individuals with high SQ exhibit sharper curvature in value comparison:
\[
\nabla^2 V(x) \propto \text{SQ},
\]
where larger values correspond to more rigid evaluative surfaces and reduced sensitivity to bottom--up salience fluctuations.

\subsection{SQ, Synthetic Presence, and Behavioural Perturbation}

\noindent
In the experiment underlying this thesis, high-SQ participants correspond most closely to the \emph{Analytical--Structured} dispositional cluster revealed through the clustering of EQ, SQ, and BFI scores. 

Consistent with research on deliberative dominance \cite{Konovalov2016,Shenhav2017}, 
high-SQ individuals exhibit reduced coupling between affective cues and evaluative 
processing, as well as diminished susceptibility to emotionally charged primes. In the 
context of the present experiment, these individuals correspond most closely to the 
\emph{Analytical--Structured} dispositional cluster revealed through the combination of 
EQ, SQ, and BFI scores.

However, the empirical analysis in Chapter~\ref{chap:experimental_methods} showed that 
systemizing did \emph{not} moderate the effect of robotic presence on donation 
behaviour. Despite their distinctive dispositional geometry, high-SQ participants 
displayed the same directional attenuation in prosocial action as the other clusters. 
This indicates that the perturbation introduced by the robot operates at the level of 
the evaluative field, not through trait-specific cognitive pathways.

In this sense, the SQ plays a dual but fully empirical role in the experiment:
\begin{itemize}
	\item it provides the micro-level control needed to rule out systemizing 
	tendencies as an alternative explanation for the donation outcomes,
	\item and it contributes to modelling the structural dimension of 
	$\beta_C$ without implying moderation of the perturbational term 
	$\gamma_R$.
\end{itemize}

\noindent
Thus, while systemizing is theoretically associated with analytic stability and 
reduced affective interference, the present findings show that robotic presence 
exerts a field-level displacement that overrides these dispositional differences.

\subsection*{Why SQ Matters for the Experiment}

\noindent
As with the Empathizing Quotient, the Systemizing Quotient was included not only for its 
theoretical relevance but for a basic methodological reason: to ensure that differences 
in donation behaviour were not driven by pre-existing cognitive–analytical styles. 
Without an explicit measure of systemizing tendencies, the attenuation observed in the 
robot condition might have been misattributed to participants’ preference for rule-based 
reasoning rather than to the perturbation itself.

The SQ therefore provides micro-level control by ruling out cognitive style as a 
proximate cause of the behavioural shift, while simultaneously supplying the structural 
dimension of $\beta_C$ necessary for modelling dispositional topology in the 
evaluative--topological framework.


\subsection{Why SQ Matters}

\noindent
The inclusion of the Systemizing Quotient provides a unified means of capturing several features of the evaluative landscape that would otherwise remain theoretically disjoint. Within the ES framework \cite{Baron2002,Baron2003,Baron2009}, systemizing reflects a cognitive style centred on structural analysis, causal precision, and rule-based inference. 
In topological terms, it indexes the \emph{deliberative curvature} of the moral field: the degree to which evaluative trajectories are shaped by stable causal schemas rather than affective attractors \cite{Goldenfeld2005,Wakabayashi2006}. 

High-SQ individuals are reliably characterised by reduced affective reactivity and a stronger reliance on deliberative pathways, as shown in both moral psychology \cite{Greene2014,Cushman2013DualSystem} and affective neuroscience \cite{Gleichgerrcht2013}. This makes the SQ theoretically informative for distinguishing sentimentalist mechanisms of moral judgment—which emphasise affective resonance and intuitive appraisal \cite{Haidt2001EmotionalDog,Crockett2016Models}—from structuralist mechanisms that privilege rule-coherent, model-based evaluation \cite{Konovalov2016,Shenhav2017}.

Crucially, however, the experiment demonstrated that systemizing did \emph{not} moderate the effect of robotic presence. Despite their theoretical association with analytic stability and reduced affective interference, high-SQ participants displayed the same directional attenuation in prosocial behaviour as the other clusters. This finding shows that the perturbation introduced by the robot operates at the level of the evaluative field itself rather than through trait-specific cognitive pathways.

In this light, the SQ serves two essential empirical functions within the study. First, it rules out systemizing tendencies as a proximate explanation for the behavioural shift, ensuring that the attenuation cannot be attributed to cognitive–analytical style. Second, it supplies the structural dimension of $\beta_C$ necessary to model how dispositional 
architecture relates to global field deformation. The SQ therefore anchors the analytical side of the dispositional manifold without implying differential behavioural susceptibility to synthetic presence.

% -----------------------------------------------------
\section{The Big Five Inventory (BFI): Personality Geometry and Moral Topology}

\noindent
Among the major instruments of differential psychology, the Big Five Inventory (BFI) 
occupies a uniquely robust position. Originating from decades of lexical, psychometric, 
and theoretical research \cite{John1999,McCraeCosta2008}, the Big Five model offers a 
parsimonious description of personality variation along five orthogonal axes: Openness, 
Conscientiousness, Extraversion, Agreeableness, and Neuroticism. These traits have 
repeatedly demonstrated high stability, cross-cultural generality, and considerable 
predictive power across behavioural domains \cite{Barrick1991,Donnellan2006,Rammstedt2007}. 
As a measurement instrument, the BFI therefore provides an empirically grounded coordinate 
system for mapping the dispositional substrate ($\beta_C$) posited within the 
evaluative--topological framework developed in the present thesis and subsequently used to 
interpret the uniform displacement effect observed in the experiment.

\subsection{Historical Development and Theoretical Foundations}

\noindent
The Big Five model traces its roots to the lexical tradition in personality research, in 
which factor-analytic investigations of trait-descriptive adjectives yielded a consistent 
five-factor structure across languages and populations. Building on this foundation, John 
and Srivastava \cite{John1999} formalised the BFI as a psychometrically concise yet highly 
reliable instrument for assessing the cardinal trait dimensions. Parallel work by McCrae 
and Costa \cite{McCraeCosta2008} provided a broader theoretical synthesis, linking the Big 
Five to a hierarchical model of personality structure and embedding them within a 
developmental and biological framework.

A crucial contribution of this lineage is the recognition that personality traits operate 
as stable attractors in behavioural space, shaping patterns of affective responsiveness, 
regulatory control, motivational priorities, and social orientation. For the purposes of 
the present thesis, this stability makes the BFI well suited to modelling the 
dispositional manifold that constrains evaluative trajectories in the presence of 
perturbations. In the subsequent experiment, this stability enabled a clear dissociation 
between dispositional variation and the field-level effect of robotic presence.

\subsection{Psychometric Strength and Cross-Contextual Validity}

\noindent
The BFI is among the most validated instruments in modern psychology. Its dimensional 
structure has been replicated across diverse populations, and its items exhibit strong 
internal consistency and temporal stability \cite{John1999,Donnellan2006}. Short-form 
adaptations such as the BFI-10 \cite{Rammstedt2007} preserve much of this reliability while 
enabling efficient deployment in time-constrained experimental contexts such as the present 
one.

Importantly, Big Five traits predict consequential real-world outcomes across domains 
including job performance \cite{Barrick1991}, relationship quality, subjective well-being, 
and health behaviours. These predictive successes justify the use of BFI metrics as indices 
of theoretically meaningful dispositions that shape evaluative and behavioural tendencies. 
The cross-cultural robustness of the Big Five further supports their role as part of a 
generalisable dispositional architecture that can be integrated into computational and 
topological models. In our experiment, this stability ensured that personality variation 
could be meaningfully mapped onto the dispositional manifold used to test whether the 
robotic perturbation exerted trait-dependent or trait-independent effects.

\subsection{Personality Predictors of Moral Behaviour}

\noindent
A substantial body of work has examined the relation between Big Five traits and prosocial 
or moral behaviour. Agreeableness is the most consistent predictor of helping, cooperation, 
and empathic concern \cite{Graziano1996,Habashi2016}. Individuals high in Agreeableness 
are more responsive to others’ needs, more sensitive to interpersonal harm, and more 
disposed toward altruistic action even in anonymous or low-reciprocity contexts.

Conscientiousness has been linked to moral rule adherence, planning, and long-horizon 
evaluative stability. Individuals high in Conscientiousness exhibit greater behavioural 
regularity and stronger alignment with internalised norms, qualities that translate into 
reduced noise in moral decision-making. Neuroticism predicts greater affective volatility, 
heightened sensitivity to social threat, and increased susceptibility to contextual 
perturbation \cite{Hilbig2013}. Extraversion amplifies responsiveness to social presence and 
increases the weighting of socially salient cues. Openness broadens receptivity to 
contextual novelty, increases tolerance of ambiguity, and enhances exploratory behaviour in 
moral and social domains.

These findings demonstrate that the Big Five traits track the dispositional architecture 
that shapes how agents integrate affective, cognitive, and contextual information into 
evaluative judgments. The BFI therefore directly contributes to estimating the $\beta_C$ 
manifold in the evaluative--topological model. In our experiment, this mapping was essential 
for determining whether the displacement observed in the robot condition reflected 
disposition-specific pathways or a uniform perturbation of the evaluative field.

\subsection{BFI in Social Cognition, SSP, and HRI}

\noindent
Beyond moral psychology, the BFI plays a central role in research on social cognition and 
nonverbal behaviour. Personality traits influence expressive dynamics, gaze patterns, vocal 
modulation, and gesture production—behaviours that constitute the core observational cues in 
Social Signal Processing (SSP). Vinciarelli et al.’s foundational survey \cite{Vinciarelli2009} 
highlights how personality traits can be inferred from multimodal behavioural signatures 
(speech prosody, movement patterns, attention allocation), and how these traits modulate 
social engagement, turn-taking, and responsiveness to social cues.

These insights reinforce the relevance of the BFI in contexts involving robotic presence. 
Traits such as Extraversion and Agreeableness shape social approach tendencies, sensitivity 
to perceived agency, and responsiveness to social affordances—all properties critical in HRI 
scenarios. Banks \cite{Banks2020} demonstrates that personality interacts with perceptions of 
robot trustworthiness, sociality, and intentionality, thereby linking the Big Five to the 
cognitive mechanisms underlying moral or cooperative evaluation of artificial agents.

In the present experiment, these considerations justify the use of the BFI as a means of 
quantifying structural differences in participants' social orientation. Because the 
perturbation introduced by the robot operates at the level of perceived social presence, 
personality traits that modulate such responsiveness play an instrumental role in 
understanding dispositional variation across the sample, and in establishing that the 
displacement effect is indeed field-level rather than trait-dependent.

\subsection{Personality Geometry Within the Evaluative--Topological Framework}

\noindent
Within the evaluative--topological model, personality traits function as geometric 
modifiers of the evaluative field. Agreeableness steepens prosocial attractor basins, 
lowering friction along cooperative trajectories and increasing the salience of altruistic 
outcomes. Conscientiousness stabilises high-level evaluative pathways, introducing strong 
curvature along rule-governed dimensions and reducing susceptibility to contextual noise. 
Neuroticism injects volatility into the evaluative manifold, increasing the amplitude of 
local fluctuations and enhancing the influence of perturbations. Openness expands the 
contextual sensitivity of the evaluative field, enabling broader sampling of informational 
cues. Extraversion intensifies responsiveness to social presence, amplifying the salience 
contributions of agents (human or synthetic) within the perceptual environment.

Taken together, these geometric interpretations allow the BFI traits to be embedded within 
the formalism:
\[
\mathscr{P}(\delta_m) = f(\alpha_E, \beta_C, \gamma_R).
\]
The BFI operationalises key dimensions of $\beta_C$, specifying the metric structure through 
which evaluative trajectories evolve in response to environmental ($\alpha_E$) and 
perturbational ($\gamma_R$) influences. In the experiment, this enabled precise comparison of 
dispositional geometry against the uniform behavioural displacement induced by the robot.

\subsection{BFI, Perturbation, and the Interpretation of Uniform Attenuation}

\noindent
In the experiment underlying this thesis, BFI traits were expected to modulate sensitivity to 
the robotic perturbation, particularly given the relevance of Extraversion, Agreeableness, 
and Neuroticism to social presence, affective reactivity, and contextual susceptibility. Yet 
the empirical results—discussed in Chapter~\ref{chap:experimental_methods}—revealed no such 
moderation. Despite robust dispositional structure uncovered through cluster analysis, all 
groups exhibited the same directional attenuation in prosocial behaviour.

This finding is consistent with evidence from HRI and social cognition that suggests robotic 
presence can shift social affordances at a level that bypasses trait-level predispositions, 
acting instead through global modifications of perceived agency, social monitoring, or norm 
salience \cite{Malle2015,Komatsu2016,Zlotowski2015}. The BFI was crucial in establishing 
this. By providing a structured mapping of personality geometry, the instrument made it 
possible to dissociate trait-level variation from field-level displacement. Without this 
differentiation, the attenuation could have been misattributed to personality differences 
rather than to the synthetic perturbation introduced by the robot.

\subsection{Critiques, Limitations, and Relevance to the Thesis}

\noindent
The Big Five framework is not without its critics. Some theorists argue that it is 
descriptively powerful but theoretically thin, lacking a mechanistic account of trait 
emergence. Others raise concerns about the number of underlying factors, suggesting that 
alternative models (HEXACO, hierarchical factor models) may capture additional variance in 
moral or social behaviour. Still others point to the risk that personality traits are only 
weakly predictive at the level of individual behaviour and depend heavily on situational 
features.

However, for the purposes of the present thesis, these critiques do not undermine the 
instrument's value. The BFI was not employed as a causal explanation of moral behaviour, but 
as a principled means of mapping the dispositional manifold through which perturbations 
propagate. Its high stability, conceptual clarity, and predictive track record make it the 
appropriate tool for modelling $\beta_C$ in a topological framework and for demonstrating 
that the attenuation produced by robotic presence is a field-level displacement rather than 
a trait-level moderation.

\medskip

\noindent
In summary, the Big Five Inventory provides a theoretically grounded and empirically 
validated coordinate system for the dispositional term $\beta_C$. Its integration into the 
evaluative--topological model enables a precise dissociation between trait-level structure 
and field-level perturbation, making it an indispensable tool for interpreting the uniform 
attenuation produced by robotic presence.


\noindent
These dimensions do not function independently; instead, they jointly determine the curvature, stability, and topology of the moral field for each participant.
The experiment leveraged this structure not to predict differential behavioural 
responses, but to determine whether robotic perturbation acted on dispositional 
gradients or on the evaluative field as a whole.


\subsection{Cluster Semantics and BFI Geometry}

\noindent
The cluster analysis in Chapter~\ref{chap:experimental_methods} revealed three 
dispositional attractor structures:

\begin{enumerate}
	\item \textbf{Prosocial–Empathic}: high Agreeableness, high Openness, high EQ; steep 
	affective attractors; strong dispositional orientation toward altruistic trajectories (not reflected in significant behavioural differences in the experiment).
	.
	\item \textbf{Emotionally Reactive}: high Neuroticism, mixed EQ; unstable gradients; 
	heightened susceptibility to contextual variation.
	\item \textbf{Analytical–Structured}: high Conscientiousness and SQ; rigid gradients; evaluative trajectories shaped by rule-based stability, though not resistant to the field-level perturbation observed in the experiment.
\end{enumerate}

\noindent
Including the BFI provides several methodological and theoretical advantages that are essential for interpreting the experimental findings. First, it offers a validated and fine-grained quantification of dispositional topology, grounding the latent structure of $\beta_C$ in a trait framework with well-established psychometric credentials \cite{John1999,McCraeCosta2008,JohnDonahueKentle1991_BigFiveInventory,Donnellan2006,Rammstedt2007}. Second, it enables the clustering of heterogeneous evaluative architectures, allowing the analysis to detect meaningful personality configurations rather than assuming psychological homogeneity across participants \cite{Barrick1991}. Third, it anchors normative interpretation within empirically instantiated personality space, thereby linking moral behaviour to stable and theoretically interpretable dispositional dimensions.

\noindent
Including the BFI allows:

\begin{itemize}
	\item quantification of dispositional topology,
	\item clustering of heterogeneous evaluative architectures,
	\item grounding normative interpretation in empirically real personality space,
	\item demonstrating that synthetic moral perturbation operates at the 
	\emph{field level rather than through trait-specific pathways},
	\item and ruling out simplistic personality-based explanations of donation 
	behaviour (e.g., that prosociality was merely a reflection of 
	Agreeableness or Extraversion).
\end{itemize}

\noindent
Most critically, the inclusion of the BFI makes it possible to demonstrate that the synthetic perturbation introduced by the humanoid robot operates at the \emph{field level} rather than through trait-specific pathways. Agreement, Extraversion, and related traits are well-known predictors of prosociality and interpersonal sensitivity \cite{Graziano1996,Habashi2016,Hilbig2013}, yet none moderated the attenuation in donation behaviour observed in the experiment. By ruling out personality-dependent explanations, the BFI prevents misinterpretation of the results as mere reflections of Agreeableness or Extraversion and instead supports the conclusion—also consistent with findings in HRI \cite{Zlotowski2015,Malle2015,Komatsu2016}—that robotic presence reshapes the evaluative field itself.

\medskip

\noindent
Without the BFI, the experiment would lack both the dimensional granularity required to distinguish dispositional variation from field-level displacement and the evidential basis for excluding personality traits as the proximate cause of the attenuation effect. The instrument therefore secures two levels of inference: the micro-level behavioural interpretation (donation behaviour was not simply a function of trait prosociality) and the macro-level topological interpretation (the robot perturbed the evaluative field rather than trait-dependent gradients). In this way, the BFI plays a decisive role in demonstrating that the displacement effect arises from a global modification of evaluative geometry rather than from personality-based modulation.

%%%%%%%%%%%%%%%%%%%%%%%
% FROM HERE NOW
%%%%%%%%%%%%%%%%%%%%%%

\section{The Watching-Eye Paradigm: Moral Salience Amplification and Its Deformation Under Synthetic Presence}

\noindent
One of the most robust findings in behavioural ethics, social psychology, and field-based prosociality research is the \emph{watching-eye effect}. Minimal cues of observation—such as stylised eye images, schematic-gaze primes, or human-like monitoring cues—reliably increase prosocial behaviour, generosity, charitable giving, and norm-compliant action \cite{Haley2005, Bateson2006, Nettle2013, Conty2016}. Early interpretations framed this effect in terms of reputational vigilance, proposing that even minimal perceptual cues can activate implicit monitoring systems and thereby increase norm adherence \cite{Haley2005}. Subsequent work expanded this view, demonstrating that watching-eye stimuli operate through distributed attentional, affective, and interpretive mechanisms \cite{Bateson2006, Nettle2013, Conty2016}.

\noindent
Within the evaluative--topological framework developed in this thesis, watching-eye cues are understood as controlled perturbations that \emph{increase the steepness of prosocial attractors} in the moral field. By heightening perceived social salience, they reshape early evaluative gradients and bias trajectories toward cooperative outcomes without requiring explicit instruction or normative reasoning.

\subsection{The Watching-Eye Effect as a Topological Amplifier}

\noindent
Watching-eye cues operate by modulating the environmental input term $\alpha_E$. They enhance prosocial weighting by:

\begin{enumerate}
	\item \textbf{Amplifying moral salience:} increasing the perceived relevance of 
	norm-guided action.
	\item \textbf{Recalibrating attention:} shifting perceptual resources toward behaviour 
	and its social meaning.
	\item \textbf{Activating self-conscious emotions:} mild guilt, embarrassment, or 
	pride associated with being evaluated.
\end{enumerate}

Formally, the cue introduces:
\[
\alpha_E \mapsto \alpha_E + \delta \alpha_{\text{eye}},
\]
where $\delta \alpha_{\text{eye}}>0$ increases the gradient favouring prosocial choice. At the Level of Abstraction adopted here, watching-eye stimuli are not postulated to create complex mental-state attribution; rather, they modulate the latent evaluative landscape through which action tendencies flow.

\paragraph{Reputational Mechanisms.}
Classical studies show that minimal observation cues trigger reputational vigilance, increasing the perceived costs of norm violation and shifting behaviour toward compliance and fairness \cite{Haley2005, Bateson2006}. In topological terms, such cues steepen deontic and prosocial attractors within the evaluative field, making cooperative trajectories more gravitationally dominant.

\paragraph{Attentional Mechanisms.}
Watching-eye stimuli also function as attentional amplifiers. Experimental evidence demonstrates that observation cues draw perceptual resources toward socially and normatively relevant features, modulating early-stage appraisal processes \cite{Conty2016, Nettle2013, Dear2019}. This attentional reweighting alters the initial intuitive gradients that guide moral evaluation, consistent with both dual-process accounts and the topological formalism developed earlier.

\paragraph{Affective and Self-Conscious Emotions.}
Observation cues can induce mild affective arousal, activating self-conscious emotions such as guilt, embarrassment, or pride. Pfattheicher and Keller \cite{PfattheicherKeller2015} show that such cues increase prosocial tendencies by elevating somatic markers associated with cooperative action. In topological terms, this creates a local rise in affective curvature, making prosocial trajectories more energetically accessible.

\paragraph{Context Sensitivity.}
The watching-eye effect is not universal. Its magnitude varies with local normative expectations, cultural context, cue ambiguity, and ecological validity \cite{Kawamura2017, Nettle2013}. This context dependence is crucial for understanding how synthetic presence may interact with, dilute, or override the effect, particularly when robotic agents introduce novel or ambiguous social affordances.

\subsection{Why Child-Poster Eyes Serve as Valid Social Cues}

\noindent
Child-poster eyes have become a widely adopted tool in prosociality and donation-based paradigms because they represent a minimal, reliable, and theoretically interpretable form of social cueing. Extensive research across behavioural ethics, evolutionary psychology, and field experimentation demonstrates that stylised eyes—particularly child-like or infantile forms—robustly increase cooperation, charitable giving, and norm compliance in both laboratory and ecologically naturalistic settings \cite{Haley2005, Bateson2006, ErnestJones2011, Ekstrom2012, Nettle2013, Dear2019}. Survey and mechanistic studies on gaze perception further show that eye-like stimuli heighten implicit monitoring, attentional engagement, and the salience of norm-relevant behaviour \cite{Conty2016}. 

\paragraph{Perceptual Sociality Without Agentic Commitment.}
One key advantage of child-eye posters is that they elicit social attentiveness without invoking full-fledged agency, intention, or belief ascription. Findings from developmental social cognition show that infant-like eyes are powerful communicative signals capable of triggering gaze-following, social vigilance, and context-oriented attention \cite{Senju2008}. These cues therefore allow the experiment to modulate the environmental input term $\alpha_E$ without introducing confounds related to mental-state attribution or anthropomorphic inference.

\paragraph{Care-Related Affective Resonance.}
Infant and child imagery reliably evoke \textit{empathic concern and affiliative motivation}, an effect well established in social neuroscience and affective psychology \cite{ThompsonBooth2014, Gleichgerrcht2013}. Within the evaluative--topological model, these stimuli steepen prosocial attractors by amplifying affective resonance, thereby increasing the affective curvature associated with cooperative trajectories.

\paragraph{Methodological Control.}
Child-poster eyes are low-dimensional, easily standardisable stimuli. Unlike dynamic or agentive observers, they minimise interpretive ambiguity while producing replicable increases in prosocial behaviour \cite{Haley2005, Bateson2006, ErnestJones2011, Ekstrom2012}. Their methodological reliability makes them especially suitable for controlled perturbation of moral salience—precisely the role played by the environmental input term $\alpha_E$ in the present experimental design.

\subsection{Why Synthetic Agents May Dilute or Distort the Watching-Eye Effect}

\noindent
A central theoretical insight of this thesis is that humanoid robots—although perceptually social—possess what can be described as an \emph{unstable social ontology}. Prior work in human--robot interaction shows that robots occupy an ambiguous position within the space of social agents: they can trigger attentional and interactive responses, yet they do not clearly instantiate the mental, moral, or evaluative capacities normally associated with observers \cite{Zlotowski2015, Malle2015, Komatsu2016}. This ambiguity directly disrupts the mechanisms that underpin the watching-eye effect.

\noindent
In empirical and theoretical accounts of the watching-eye effect, behavioural modulation arises from an automatic inference that an entity with perceptual access also possesses the evaluative and sanctioning capacities required to make one’s behaviour socially consequential. Classical studies show that minimal eye cues activate reputational vigilance, triggering heuristic assumptions about observers who can form impressions, update reputational standings, and administer rewards or punishments \cite{Haley2005, Bateson2006, Nettle2013, Dear2019}. These effects depend on deep-seated attentional and affective systems specialised for detecting evaluative observers, particularly those capable of moral appraisal \cite{Conty2016}.

\noindent
Synthetic agents disrupt this chain of inferences. Although their perceptual and bodily cues can signal social presence, their ambiguous social ontology undermines the attribution of intentionality, evaluative capacity, and normative authority. Findings in human--robot interaction show that humanoid robots frequently elicit perceptual but not moral-evaluative attributions \cite{Zlotowski2015, Malle2015, Komatsu2016}. As a result, the observer–detection heuristic receives conflicting inputs: the perceptual system flags an agentive presence, while higher-order cognitive systems register the absence of genuine mental states or sanctioning power. This dissonance weakens reputational motivation, destabilises expectations of being judged, and attenuates the motivational architecture that produces the watching-eye effect in human–human contexts.


\paragraph{Perceptual Sociality Without Clear Ontology.}
Humanoid robots signal presence but lack a stable set of agentic, intentional, or moral attributes. Because reputational vigilance depends on attributing evaluative capacities to an observer, this ontological instability weakens the mapping from perceived observation to expectations of norm compliance \cite{Malle2015}. In topological terms, the robot introduces perceptual salience without the deontic structure that normally steepens prosocial attractors.

\paragraph{Disrupted Affective and Attentional Gradients.}
Although robots reliably elicit gaze, they do not consistently activate the affective and evaluative systems associated with being judged by another mind \cite{Conty2016, Zlotowski2015}. The result is a diminished perturbational signal: the effective increase in environmental input, $\delta\alpha_{\text{eye}}$, is attenuated, and the prosocial attractors in the evaluative field remain comparatively flat.

\paragraph{Interpretive Uncertainty.}
Participants may attribute perceptual sensitivity to robots (``it sees me'') without extending moral-evaluative capacities (``it judges me''). This asymmetry creates a fractured evaluative field: social presence is registered, but the normative or reputational meaning of that presence is ambiguous or absent \cite{Komatsu2016}. Because the watching-eye effect depends on a coherent mapping between observation and evaluation, this interpretive uncertainty dilutes the salience amplification typically produced by eye cues.

\paragraph{Consequences for Evaluative Topology.}
In the evaluative--topological framework, robots function not as straightforward social primes but as \emph{semiotic perturbators}: they alter the geometry of the evaluative field by introducing novel, ambiguous, or unstable forms of social salience. The prediction—later confirmed experimentally—is not merely that prosocial action decreases, but that the transformation $\mathscr{P}(\delta_m)$ undergoes a structured deformation. Rather than shifting behaviour through trait-dependent pathways, the robot perturbs the evaluative field itself.

\subsection{Watching-Eye Under Synthetic Co-Presence: Empirical Findings}

\noindent
The experiment yielded a consistent and theoretically revealing result:

\begin{quote}
	\emph{The presence of a humanoid robot attenuated the watching-eye effect uniformly 
		across dispositional clusters.}
\end{quote}

\noindent
This indicates that the perturbation introduced by the robot operates at the 
\emph{field level} rather than through trait-specific pathways. Even participants with 
high Agreeableness, high EQ, or high Extraversion—traits normally associated with enhanced 
prosocial response—exhibited the same directional attenuation.

Formally, the robot introduces:
\[
(\alpha_E + \delta \alpha_{\text{eye}}) \mapsto (\alpha_E + \delta \alpha_{\text{eye}}) 
- \Delta_{\mathscr{R}},
\]
where $\Delta_{\mathscr{R}}$ represents the displacement of prosocial salience induced by 
synthetic presence.

\noindent
Crucially, $\Delta_{\mathscr{R}}$ does not interact with $\beta_C$ (EQ, SQ, BFI traits), as 
demonstrated by the lack of significant moderation in regression or cluster-specific 
analysis.

\subsection{Why the Watching-Eye Paradigm Matters for the Experiment}

\noindent
The watching-eye paradigm plays four indispensable methodological roles in the present study:

\begin{itemize}
	\item \textbf{A standardised probe of moral salience:} it provides a reproducible baseline against which perturbations can be detected.
	\item \textbf{A high-salience reference condition:} attenuation is measurable only when salience is first elevated by a reliable observational cue.
	\item \textbf{A bridge between moral psychology and HRI:} it enables direct comparison between synthetic co-presence and established findings from the prosociality literature.
	\item \textbf{A diagnostic of topological deformation:} it reveals how synthetic presence alters the curvature of the evaluative field rather than simply reducing generosity.
\end{itemize}

\noindent
Without this baseline, the attenuation produced by the humanoid robot could not be identified as a \emph{structured displacement} of salience; it would instead appear as an undifferentiated reduction in donation behaviour.

\subsection{Integration With the Donation Paradigm}

\noindent
Donation tasks provide a measure of real behavioural commitment rather than hypothetical endorsement, capturing the downstream consequences of evaluative processing in action \cite{Greene2004, Cushman2013DualSystem, Crockett2016Models, Moll2002, Decety2004}. Integrating watching-eye cues with a cost-bearing prosocial choice allows the experiment to:

\begin{itemize}
	\item trace the transition from perceptual cue uptake to practical action,
	\item test whether synthetic agents function as observers within moral cognition,
	\item and quantify the deformation of evaluative trajectories under the perturbation operator $\gamma_R$.
\end{itemize}

\noindent
The results show that robotic presence modulates the evaluative topology by suppressing the amplification normally produced by observational cues, confirming that the perturbation operates at the level of the field rather than through trait-specific pathways.

\subsection{Synthesis: The Watching-Eye Paradigm as a Window Into Moral Topology}

\noindent
Taken together, the watching-eye stimuli constitute far more than an auxiliary experimental feature: they provide the evaluative baseline through which the moral displacement induced by synthetic presence becomes empirically legible. By establishing a high-salience reference condition, the paradigm allows perturbations to be interpreted not as undifferentiated reductions in generosity, but as structured transformations of the underlying evaluative geometry.

\begin{quote}
	\textbf{By amplifying prosocial gradients, the watching-eye paradigm reveals that the robot acts not upon personality traits but upon the evaluative field itself.}
\end{quote}

\noindent
In this sense, the paradigm functions as a central diagnostic instrument within the experimental architecture of the thesis, enabling the detection of field-level deformation in moral topology under synthetic co-presence.

\section{General Conclusion: Tools as the Measurement Logic of Synthetic Moral Perturbation}

\noindent
The aim of this chapter has been to articulate the conceptual and methodological architecture through which the experiment measures, interprets, and ultimately understands the deformation of moral behaviour under synthetic co-presence. The Empathizing Quotient (EQ), the Systemizing Quotient (SQ), the Big Five Inventory (BFI), and the Watching-Eye paradigm do not function as isolated measurement devices. Instead, they form a coordinated system of instruments designed to map the evaluative topology that underlies moral judgement and action.

\medskip

\noindent
The formal model developed earlier represents moral behaviour as the output of a function
\[
\mathscr{P}(\delta_m) = f(\alpha_E, \beta_C, \gamma_R),
\]
where $\alpha_E$ denotes environmental moral cues, $\beta_C$ the dispositional manifold, and $\gamma_R$ the perturbational operator introduced by synthetic presence. Each tool corresponds directly to one or more components of this formalism:

\begin{itemize}
	\item \textbf{EQ} measures the affective curvature of $\beta_C$, capturing the steepness and accessibility of prosocial attractors.
	\item \textbf{SQ} measures the structural rigidity of $\beta_C$, indexing the deliberative stability of evaluative gradients.
	\item \textbf{BFI} provides the multidimensional geometry of $\beta_C$, furnishing the coordinate system required for cluster formation and dispositional mapping.
	\item \textbf{Watching-Eye} manipulates $\alpha_E$, amplifying prosocial salience so that perturbational effects of $\gamma_R$ become empirically detectable.
\end{itemize}

\noindent
Collectively, these instruments probe distinct dimensions of the evaluative field, enabling the experiment not merely to observe behavioural change but to infer the underlying topological deformation through which such change arises.

\subsection*{Dispositional Mapping and the Rejection of Trait-Based Explanations}

\noindent
The psychometric instruments establish the dispositional manifold against which the effect of synthetic presence can be assessed. By quantifying the structure of $\beta_C$ with sufficient resolution, the analysis determines whether the attenuation of prosocial behaviour is best explained by dispositional variation or by a field-level displacement induced by robotic presence.

\noindent
The empirical findings clearly support the latter.  
No Big Five trait, no empathizing or systemizing tendency, and no dispositional cluster moderated the effect: the robot’s presence produced a \emph{uniform directional attenuation} in prosocial behaviour across all profiles. Without trait-level measurement, this displacement could have been misinterpreted as a consequence of stable personality differences rather than as evidence of structured perturbation.

\subsection*{Watching-Eye as a Diagnostic Amplifier}

\noindent
The Watching-Eye paradigm plays the complementary role of manipulating $\alpha_E$ in a controlled, theoretically meaningful manner. By steepening the prosocial gradient prior to perturbation, it provides the structured baseline needed to detect attenuation. The experiment shows that synthetic presence \emph{cancels or suppresses} this amplification, demonstrating that the robot acts upon the evaluative field rather than on any particular dispositional trajectory. Without this diagnostic probe, the topological impact of $\gamma_R$ could not be empirically isolated.

\subsection*{Ethical and Theoretical Implications}

\noindent
Viewed through the ethical frameworks developed earlier—deontological, consequentialist, virtue-theoretic, sentimentalist, contractualist, and particularist—the tools reveal that synthetic presence modulates:

\begin{itemize}
	\item accountability and norm-guided obligation,
	\item expected social payoffs and reputational meaning,
	\item the behavioural expression of character,
	\item affective vector fields underlying moral appraisal,
	\item interpersonal justification spaces,
	\item and the salience structure constitutive of context-sensitive reasons.
\end{itemize}

\noindent
The instruments collectively supply the empirical resolution necessary to show that the robot functions as a \emph{moral refractor}: a perturbational agent that reshapes the geometry by which moral cues become action.

\subsection*{Transition to the Experimental Methods}

\noindent
This chapter has established the measurement logic, theoretical foundations, and epistemic justification for the instruments structuring the experiment. What follows is the formalisation of the experimental design itself: how these tools were integrated, how the perturbation was operationalised, and how the resulting evaluative deformation was measured.

\begin{center}
	\emph{The tools provide the coordinates; the experiment traces the trajectory.}
\end{center}

\noindent
The tools examined in this chapter have established the conceptual and methodological
infrastructure required to investigate how synthetic agents perturb moral behaviour.  
Each instrument---EQ, SQ, the BFI, and the Watching-Eye paradigm---defines a specific
dimension of the evaluative topology through which moral cues are encoded, integrated, and
ultimately transformed into action. What remains is to test, empirically and with
methodological precision, how these dimensions behave when the evaluative field is
subjected to a controlled perturbation.

The rationale for the experiment follows directly from the foregoing analysis.
If moral behaviour is the output of a function
\[
\mathscr{P}(\delta_m) = f(\alpha_E, \beta_C, \gamma_R),
\]
then the central empirical question is whether the introduction of $\gamma_R$---a humanoid
robotic presence---alters the mapping from moral salience to prosocial action.  
The tools chapter has already established:

\begin{enumerate}
	\item that $\beta_C$ (dispositional architecture) can be measured, modelled, and
	decomposed into meaningful clusters;
	\item that $\alpha_E$ (moral salience) can be systematically manipulated through
	Watching-Eye cues;
	\item and that a perturbation of the evaluative field, if it exists, must manifest as a
	structured deviation from the baseline topology rather than as random behavioural noise
	or trait-based variation.
\end{enumerate}

\noindent
The experiment is designed explicitly to adjudicate between these possibilities.  
By embedding the donation task within observational conditions that vary only in the presence or absence of a humanoid robot, the study tests whether synthetic
co-presence modifies the evaluative gradients that ordinarily channel behaviour toward prosocial outcomes. The methodology is therefore not an independent component of the thesis, but the operational extension of the formal architecture developed thus far. It implements the theoretical variables, measures the predicted evaluative trajectories, and determines whether the robot functions as a \emph{perturbational operator} ($\gamma_R$) acting at the field level.

The next chapter presents this experimental design in full detail, showing how the tools introduced here were operationalised into stimulus conditions, measurement procedures, and analytic models. It specifies the structure of the donation paradigm, the observational manipulations, the psychometric integrations, and the statistical strategy---including non-parametric tests, regression modelling, and Bayesian estimation---used to detect and characterise deformation in the evaluative topology.

\noindent
In short, the tools chapter has provided the coordinate system.  
The experiment now traces the trajectory: determining whether synthetic presence 
reshapes the evaluative field that connects moral salience to action, and whether this deformation is uniform, trait-dependent, or topologically structured.
%%%THE END
