\chapter{From Theory to Measurement: Operationalising Dispositions and Moral Perturbation}
\label{chap:tools_new}
\thispagestyle{pprintTitle}

To measure moral cognition is to confront a familiar difficulty: the processes that matter most for guiding action—those early shifts in attention, affect, and evaluative weight—rarely announce themselves. They move quietly, shaping what feels salient long before any explicit judgment is formed. An experiment that aims to detect perturbations in this machinery must therefore proceed with more
than methodological care; it must choose instruments that open the right conceptual windows onto a process that is not directly observable.

The task of this chapter is to make that access possible. The measures introduced below are not selected for convenience, nor for their prevalence in psychologicalresearch, but because each provides a principled entry-point into the evaluative architecture described earlier. They allow upstream processes to leave traces that can be analysed without distorting the phenomenon itself. What follows,
then, is not a catalogue of questionnaires, but the construction of the measurement interface through which evaluative perturbation becomes empirically visible.


Moral appraisal does not enter empirical analysis directly%
~\cite{Greene2001,Greene2002,Cushman2006MoralLearning,Young2010,Crockett2010SerotoninHarm}. What becomes observable are the structured traces it leaves---dispositions, affective responses, attentional shifts, behavioural choices---from which the underlying evaluative process can be inferred~\cite{Decety2004EmpathyNeural,TracyRobins2004Appraisal,Greene2004TMS,Haley2005,Bateson2006,Whitmarsh2021AffectiveAttention}.
Within the evaluative--topological framework developed earlier, measurement is therefore not passive registration but a \emph{mode of access} to structure. The evaluative topology designates the organisation of the evaluative field: the structured set of task-relevant moral appraisals, affective orientations, social expectations, and contextual cues that jointly determine how an agent interprets a situation as calling for one course of action rather than another. It is a “field” in the technical sense: a dynamically organised configuration of influences—cognitive, affective, and social—that exerts graded pressures on judgement and behaviour, such that changes to any one component (e.g., the presence of a humanoid robot) can shift the overall pattern of moral response. To measure moral perturbation is to measure deformation within this topology.\label{def:eval_field} 

The task of this chapter is therefore not to catalogue instruments, but to justify how specific psychometric measures and perturbational manipulations---most centrally, the silent co-presence of a humanoid robot---serve as theoretically motivated probes of that topology. Each tool samples a different dimension of the evaluative field; taken together, they provide an empirical framework capable of detecting whether NAO's presence modulates the pathways through which
moral salience is translated into moral action.

These instruments function not as neutral devices but as \emph{conceptually disciplined interventions}. Each targets a specific dimension of the evaluative
structure represented by the mapping introduced in
Section~\ref{sub:struct}, which formalises how empathic--affective variation,
trait-level dispositions, and synthetic presence contribute to the generation of
moral behaviour. The formalism does not claim mathematical precision beyond this
interpretive role; it provides a controlled vocabulary for expressing how
different components of the evaluative field become experimentally tractable
without collapsing their complexity into reductive summary scores. Their
epistemic role is analogous to the instruments of physics or the conceptual
scaffolds of analytic philosophy: they do not merely record a value but
\emph{constitute the mode of access} through which the phenomenon becomes
empirically available. The analogy is exact. In physics, a gravitational wave or
an electron is not simply ``detected''; it is rendered detectable by an
apparatus that fixes the relevant Level of Abstraction and the dimension of
variation. Philosophical analysis performs an analogous function when specifying
the conceptual lens through which inference, normativity, or agency become
visible. Measurement is thus an act of disciplined construction, not passive
reception.

The same principle governs the tools used here. Briefly, measures of empathising and systematising dispositions (EQ/SQ) do not purport to exhaust personality; they isolate axes of cognitive--affective variation known to shape sensitivity to salience and intuitive processing in moral cognition%
~\cite{BaronCohen2003,BaronCohenWheelwright2004,BaronCohen2009}. The Big Five Inventory (BFI) captures dispositional gradients that interact with evaluative weighting, vigilance, and behavioural inhibition%
~\cite{John1991,JohnSrivastava1999,Rammstedt2007}. These instruments are therefore not psychological ``thermometers'' but theoretically justified interventions into the \emph{evaluative field}, each selecting the dimensions along which individual differences become experimentally meaningful.

This methodological stance, in turn, informs the design of the experimental setting following the Watching--Eye paradigm as an experinental template. The stimulus is not treated as a behavioural curiosity but as a calibrated
perturbation:
\begin{center}
	\begin{leftbar}
		\textit{a contextual operator on accountability salience and affective
			vigilance at the cognitive LoA identified earlier.}
	\end{leftbar}
\end{center}

The Watching--Eye cue functions as a controlled modification of the evaluative
landscape, providing a channel through which implicit social meaning acquires
behavioural force. As in physics, the measurement depends not only on the
quantity under investigation but on the entire apparatus that renders the
phenomenon measurable.

What follows in this chapter therefore treats the instruments not as
psychological artefacts but as components of an \emph{epistemic strategy}. Each measure is examined in light of three questions:  (i) which aspect of the evaluative architecture it operationalises; (ii) what assumptions it encodes about cognitive–affective processing; and (iii) how it constrains the interpretation of the behavioural perturbations that constitute the empirical core of the thesis.

Having established the epistemic role of these instruments, we can now turn to their empirical deployment. Every measure introduced here is a way of gaining access to something that never appears on its own—the evaluative machinery by which agents register what matters. If the previous chapter showed how moral
appraisal is shaped by attention, affect, and social meaning, the task now is to make those structures experimentally visible. The next subsection therefore
describes how dispositional architecture and controlled perturbation are operationalised in practice, and how the measurement suite becomes the means by which subtle deformations of the evaluative field reveal themselves in behaviour.


\section{Perturbation as Measurement: The Experimental Context}

The experiment developed in this thesis does not measure moral cognition by presenting participants with explicit dilemmas or by eliciting articulated judgments. Instead, it probes the evaluative architecture indirectly, through a precisely calibrated perturbation of the perceptual environment. The silent presence of a humanoid robot serves as this perturbation. Prior work in Human--Robot Interaction shows that even non-interactive robots can reshape 
perceived social affordances, redirect attentional gradients, and alter expectations about norm-relevant behaviour%
~\cite{Zlotowski2015,Malle2015,Komatsu2016}. Their ambiguous social ontology—perceptually agentic yet behaviourally indeterminate—disrupts default social priors and reorganises the salience structure within which moral reasons become 
behaviourally operative.

Within this framework, robotic presence is not a background feature of the 
experimental setting; it is the experimental instrument. It functions as a 
controlled perturbation of the evaluative field, allowing the study to examine 
how dispositional invariants interact with contextual cues to produce measurable 
differences in prosocial behaviour. The Watching--Eye cue provides a second, 
orthogonal perturbation, modulating accountability salience and intuitive social 
monitoring. Together, these manipulations define the salience structure within 
which participants navigate moral action.


\subsection{Purpose and Structure of this Chapter}
\label{sub:struct}

The work undertaken in this chapter is not organisational housekeeping; it is an
attempt to expose the structure that makes the experiment scientifically
meaningful. If moral behaviour is the surface expression of deeper
cognitive--affective dynamics, then any empirical study must first specify
\emph{which dimensions of that structure are being accessed, and by what kind of
	instrument}. The chapter therefore has a dual purpose, each corresponding to a
different layer of the evaluative architecture.

\label{def:formalism}
\begin{enumerate}
	\item First, it establishes the empirical and theoretical foundations of the
	psychometric and contextual tools employed in the study. These instruments
	are not selected for convenience: each targets a specific component of the
	evaluative substrate—affective resonance, systemising precision, personality 
	curvature, or salience modulation—whose interaction determines how moral 
	salience becomes behaviourally operative.
	
	\item Second, to show how each tool contributes to the modelling of the dispositional term $\beta_C$ in the formal expression
	\[
	\mathscr{P}(\delta_m) = f(\alpha_E, \beta_C, \gamma_R).
	\]
	Here $\mathscr{P}(\delta_m)$ denotes the probability of a morally relevant behavioural outcome (e.g.\ donation) in a given trial; $\alpha_E$ collects the task and environmental parameters (Watching--Eye cue, payoff structure, contextual framing); $\beta_C$ encodes the participant's latent cognitive--affective disposition; and $\gamma_R$ indexes the presence or absence (and configuration) of the humanoid robot. The mapping $f$ is not introduced as a fully specified quantitative model but as a structured vocabulary for talking about how environmental conditions, dispositional architecture, and synthetic presence jointly shape the evaluative field. In this sense, the tools are not ancillary components of the experiment but operationalisations of the dispositional invariants that mediate the transformation of evaluative salience under robotic presence.
\end{enumerate}

The instruments employed in this thesis---the Empathizing Quotient (EQ), the
Systemizing Quotient (SQ), the Big Five Inventory (BFI), and the Watching--Eye
paradigm---were selected because they satisfy three stringent criteria grounded
in established empirical research%
~\cite{BaronCohenWheelwright2004,BaronCohen2003,BaronCohen2009,John1991,
	JohnSrivastava1999,Rammstedt2007,Haley2005,Bateson2006,Dear2019}:

\begin{enumerate}
	\item \textbf{Theoretical relevance.} Each tool targets a component of the
	evaluative architecture identified earlier: affective resonance, evaluative
	precision, personality curvature, or salience modulation%
	\cite{Decety2004,Crockett2016,Conty2016,Vinciarelli2009}.
	
	\item \textbf{Empirical robustness.} Each tool has been validated across
	multiple cultures, large samples, and decades of psychological research, and
	has been employed in studies of prosociality, moral sensitivity, social
	attention, and HRI~\cite{Wakabayashi2006,Goldenfeld2005,Lawson2004,Habashi2016,Haley2005,Bateson2006,ErnestJones2011,Kuchenbrandt2011,Malle2016,Bremner2022}.
	
	\item \textbf{Computational suitability.} Each tool yields variables suitable
	for integration into regression models, cluster analysis, and topological
	interpretation, allowing dispositional and contextual parameters to be related
	systematically to behavioural outcomes%
	\cite{Cushman2013,Konovalov2016,Shenhav2017}.
\end{enumerate}

Given the relatively modest sample size ($N=71$), it is natural to ask how this
measurement framework remains empirically credible:

\begin{center}
	\begin{leftbar}
		\textit{How does the theoretical weight of the measurement architecture 
			reconcile with a study based on seventy-one participants, and have these 
			tools been employed in comparable contexts?}
	\end{leftbar}
\end{center}

\noindent
The answer lies in the nature of the constructs under investigation. The tools
used in this thesis were not chosen because they require large samples for
exploratory factor recovery, but because they index \emph{structurally stable}
psychological dimensions. Their psychometric properties---factor structure,
reliability, discriminant validity---have been established in large-scale
studies involving thousands of participants across diverse populations%
~\cite{BaronCohenWheelwright2004,BaronCohen2003,BaronCohen2009,John1991,
	JohnSrivastava1999,Rammstedt2007,Wakabayashi2006,Goldenfeld2005,Lawson2004}.
In that sense, the present study does not re-validate the instruments; it
leverages constructs whose statistical scaffolding is already in place.

Crucially, the goal of the experiment is not to discover new personality
factors or infer latent structure from the data. It is to examine how \emph{known
	dispositional invariants} interact with a controlled perturbation of the
evaluative field (see~\ref{def:eval_field}, page~\pageref{def:eval_field}). Small-to-moderate sample sizes are standard in this domain:
replications of the Watching--Eye effect typically employ samples between 40 and
120 participants%
~\cite{Haley2005,Bateson2006,Nettle2013,Dear2019}, and HRI studies on the
behavioural impact of robotic presence operate in similar ranges%
~\cite{Zlotowski2015,Malle2016,Bremner2022}. Studies integrating personality
measures into moral or prosocial decision paradigms (e.g.\ BFI or EQ/SQ as
predictors of prosociality, empathic concern, or social attention) likewise
often rely on samples of comparable scale%
~\cite{Pfattheicher2015,Habashi2016,Komatsu2016,Yamagishi2014}.

The inferential strategy adopted here reflects this precedent. Dispositional
measures are treated as low-dimensional, theoretically structured parameters
that influence the deformation of evaluative gradients; the analysis does not
attempt to reconstruct the topology of moral cognition from the present dataset
alone. The statistical burden falls on detecting systematic, directional shifts
in behaviour induced by the experimental manipulations, not on extracting
high-dimensional latent structure from sparse data. For that purpose, a
well-powered design requires a clean manipulation, validated constructs, and an
analysis aligned with the theoretical architecture%
~\cite{Cushman2013,Konovalov2016,Shenhav2017} rather than sheer sample size.

In short, the experiment does not aim to estimate the entire evaluative topology
\emph{de novo}. It examines how a synthetic agent perturbs an already
well-characterised structure. The sample size is calibrated not to psychometric
exploration but to experimental contrast:

\noindent
\begin{center}
	\begin{leftbar}
		\textit{The detection of whether robotic presence produces a measurable 
			attenuation of prosocial action across dispositional profiles.}
	\end{leftbar}
\end{center}


\noindent
Evidence from HRI, SSP, and moral psychology indicates that perturbations of
this magnitude are robustly detectable with samples of the size employed here%
~\cite{Pentland2007,Vinciarelli2009,Conty2016,Bremner2022}.

With this clarification in place, we can now turn to the instruments
themselves. Their role in the thesis is not a matter of psychometric
convenience, but of \emph{theoretical access}: they expose the latent
structures through which evaluative salience is processed and transformed---structures
that, as the experiment will show, can be subtly but measurably deformed by the
presence of a synthetic agent.


\section{The Role of Psychometric Tools in the Evaluative--Topological Architecture}

\noindent
In the framework developed thus far, moral behaviour is modelled as the endpoint
of a trajectory across an evaluative field. Contemporary work in moral psychology
and cognitive science converges on the view that such trajectories are shaped by
the coordinated influence of environmental cues, dispositional structure, and
perturbational forces%
\cite{Haidt2001EmotionalDog,Greene2002,Greene2004,Cushman2013DualSystem,
	Crockett2016Models,Fedyk2017}. This relationship was formalised in the previous
section through the schematic decomposition
\[
\mathscr{P}(\delta_m) = f(\alpha_E, \beta_C, \gamma_R),
\]
whose components were introduced and motivated earlier. What matters here is how
psychometric tools allow the dispositional term $\beta_C$ to become empirically
visible.

\medskip

\noindent
Within this architecture, the role of psychometrics is not descriptive but
structural. The Empathizing Quotient (EQ), Systemizing Quotient (SQ), and Big
Five Inventory (BFI) operationalise latent properties of the evaluative field:
affective bandwidth, analytic structuring tendencies, and broad personality
curvature. Each instrument isolates a dimension along which individuals differ in how they register, weight, and transform morally salient information. These constructs do not exhaust the space of possible dispositions, but they provide \emph{theoretically grounded coordinates} for the manifold through which evaluative trajectories flow~\cite{BaronCohenWheelwright2004_EmpathyQuotient,
	BaronCohenRichlerBisaryaGurunathanWheelwright2003_SystemizingQuotient,JohnDonahueKentle1991_BigFiveInventory,Rammstedt2007}.


\noindent
The methodological necessity of these tools becomes clear once we consider the counterfactual. Without a dispositional coordinate system, heterogeneity among participants would remain unmodelled, and any perturbation effect could be misattributed to uncontrolled trait variance. As it will become clearer in I Chapter~\ref{chap:exp_methods}, the cluster structure recovered
from EQ, SQ, and BFI scores demonstrates that this heterogeneity is not noise but a \emph{non-trivial topology}: profiles centred on empathic warmth, analytical structuring, and emotional reactivity emerge as distinct basins in the dispositional landscape. Participants did not enter the experiment as psychologically interchangeable agents.

\noindent
What follows is the critical empirical observation. Despite this internal
structure, the humanoid robot produced a \emph{uniform directional attenuation}
of prosocial behaviour across all dispositional profiles. No component score, no
trait factor, and no latent cluster moderated the effect. This aligns with
findings in Human--Robot Interaction showing that passive robots can globally
reshape social affordances and attentional dynamics%
\cite{Zlotowski2015,Malle2015,Komatsu2016}. The present study \textbf{extends that literature} by demonstrating that the robot acts not on trait-linked pathways, but on the evaluative field itself: shifting salience gradients, dampening affective trajectories, and deforming the topology through which all dispositions acquire behavioural expression.

\noindent
The psychometric instruments are indispensable for isolating this phenomenon.
They allow the analysis to separate the \emph{shape of the dispositional
	manifold} from the \emph{geometry of the perturbation}. In the formalism
introduced and explained earlier (page~\pageref{def:formalism}), the contrast of interest can be expressed as
\[
f(\alpha_E, \beta_C, \gamma_R) \;-\; f(\alpha_E, \beta_C),
\]
a comparison that holds the dispositional coordinates $\beta_C$ fixed while
introducing the perturbation operator $\gamma_R$. 

The function $f$ represents a \emph{structural mapping}
introduced earlier to express how observable moral behaviour depends on three classes of determinants. Given a fixed environmental configuration $\alpha_E$ (e.g., the Watching--Eye cue and task context) and a stable dispositional profile~$\beta_C$, the expression $f(\alpha_E, \beta_C)$ denotes the pattern of
moral behaviour expected \emph{in the absence} of synthetic presence. The term $f(\alpha_E, \beta_C, \gamma_R)$ represents the corresponding pattern \emph{when the robotic perturbation is active}. The difference
\[
f(\alpha_E, \beta_C, \gamma_R) - f(\alpha_E, \beta_C)
\]
therefore isolates the contribution of the perturbation operator $\gamma_R$ while holding both environmental cues and dispositional structure fixed. It is a formal way of expressing a counterfactual comparison: how the same evaluative architecture behaves with and without the synthetic agent.


The empirical result in Chapter~\ref{chap:exp_methods} is unambiguous: although $\beta_C$ exhibits a structured internal topology, the robot’s presence induces a deformation that is effectively global. This is the behavioural signature of a field-level operator, not a trait-contingent stimulus.

\noindent
The purpose of this section is to clarify the epistemic function of the tools we use. Psychometric tools make visible the latent
evaluative substrate; the experiment reveals how synthetic presence acts upon that substrate. Only by mapping dispositional structure can we show that the robot’s influence bypasses trait-specific channels and instead targets the conditions under which moral trajectories unfold.

\noindent
With this distinction in place---between the manifold of dispositions and the global perturbation imposed upon it---we can now examine the individual tools in their own right. Each measure is introduced in light of three questions:

(i)~which aspect of the evaluative architecture it operationalises;
(ii)~what assumptions it encodes about cognitive--affective processing; and (iii)~how it constrains the interpretation of the behavioural perturbations that form the empirical core of the thesis.

\section{Why These Tools: Methodological Criteria and Alignment with the Thesis}

\noindent
The measurement strategy of this thesis cannot be grounded in convenience or disciplinary habit. Once moral behaviour is modelled as the endpoint of a trajectory shaped by environmental cues, dispositional architecture, and perturbational forces, the empirical task becomes clear: the instruments must allow these components to be \emph{distinguished in practice}. In particular, the methods must permit a clean separation between variation arising from stable cognitive--affective dispositions and variation induced by a deformation of the evaluative field.

The psychometric and experimental tools selected here were chosen precisely because they satisfy this requirement. Each instrument targets a theoretically motivated dimension of the evaluative topology and has a well-established empirical profile that makes its role interpretable within the framework developed in Chapters~\ref{chap:lit_rev} and~\ref{chap:moral_primer}. The criteria below articulate the methodological reasons for their inclusion \emph{prior} to any empirical result.

\paragraph{(1) Cross-paradigmatic relevance.}
The EQ, SQ, BFI, and Watching--Eye paradigm each derive from empirical traditions spanning moral psychology, social cognition, personality research, and Human--Robot Interaction. Across these literatures, they have been used to study prosociality, empathic concern, harm aversion, cognitive style, and the integration of affective and deliberative processes in moral appraisal%
~\cite{Haidt2001EmotionalDog,Greene2002,Greene2004,Cushman2013DualSystem,
	Crockett2016Models,Fedyk2017}.  
The Big Five Inventory operationalises broad personality architecture with robust behavioural predictive validity%
~\cite{JohnDonahueKentle1991_BigFiveInventory,Rammstedt2007,Barrick1991}.  
The Empathizing and Systemizing Quotients provide validated assessments of affective resonance and analytic curvature%
~\cite{BaronCohenWheelwright2004_EmpathyQuotient,
	BaronCohenRichlerBisaryaGurunathanWheelwright2003_SystemizingQuotient}.  
The Watching--Eye paradigm is among the most reliable manipulations of prosocial salience, repeatedly demonstrating that minimal cues of observation modulate cooperative behaviour%
~\cite{Haley2005,Bateson2006,Nettle2013,Conty2016,Dear2019}.  
These convergences align the present study with a mature empirical landscape while remaining faithful to the evaluative--topological framework.

\paragraph{(2) Topological relevance.}
Each tool probes a distinct structural component of the evaluative manifold:
\begin{itemize}
	\item \textbf{EQ}: affective attractors shaping early intuitive appraisal;
	\item \textbf{SQ}: analytic curvature influencing interpretive structure;
	\item \textbf{BFI}: personality geometry modulating salience, attention, and regulatory control;
	\item \textbf{Watching--Eye}: a validated perturbation of accountability salience operating at the cognitive LoA.
\end{itemize}
Together, these measures provide the granularity required to instantiate the dispositional term~$\beta_C$ (see page~\pageref{def:formalism} for an introduction to the terms) in the expression
\[
\mathscr{P}(\delta_m) = f(\alpha_E, \beta_C, \gamma_R),
\]
and thereby allow dispositional variation to be distinguished from field-level perturbation. This is central to the empirical question: whether robotic presence acts \emph{on traits} or \emph{on the evaluative field} within which traits express themselves.

\paragraph{(3) Stability and interpretability.}
The selected instruments satisfy three further requirements:
\begin{itemize}
	\item \textbf{Stability}: each has extensive psychometric validation across cultures and populations;
	\item \textbf{Analytical tractability}: each yields variables suitable for clustering, regression, and topological comparison;
	\item \textbf{Interpretability}: each maps onto established accounts in moral psychology, enabling behavioural effects to be related back to theoretical structure.
\end{itemize}

\noindent
Most importantly, these tools provide the precision required to dissociate \emph{who participants are} from the \emph{evaluative conditions} under which they act. As the analysis will show, the psychometric suite revealed a structured dispositional landscape, yet robotic presence attenuated prosocial behaviour \emph{irrespective} of that structure. The instruments therefore make visible the distinction between dispositional architecture and field-level deformation—an interpretive separation that would be impossible without them.

\medskip

\noindent
With these criteria established, we now turn to the first measurement instrument: the Empathizing Quotient.


\section{The Empathizing Quotient (EQ): Affective Resonance as Evaluative Curvature}

\noindent
The Empathizing Quotient (EQ) provides a validated measure of affective resonance—an individual’s capacity to detect, register, and respond to the emotional and psychological states of others \cite{BaronCohenWheelwright2004_EmpathyQuotient}. Originally developed within the Empathizing--Systemizing framework \cite{Baron2002,Baron2009}, the EQ captures both emotional reactivity and cognitive perspective-taking, two mechanisms repeatedly shown to influence prosocial behaviour, harm aversion, and sensitivity to moral salience \cite{Haidt2001EmotionalDog,Greene2002,Cushman2013DualSystem,Crockett2016Models}.

\subsection*{Why EQ Matters Conceptually}
Within the evaluative--topological model developed in this thesis, empathizing corresponds to the \emph{affective curvature} of the evaluative field. High EQ scores indicate steep affective gradients: morally relevant others appear more salient, distress is more motivationally weighted, and the transition from appraisal to prosocial action becomes more strongly guided by affective dynamics. Low EQ profiles, by contrast, reflect flatter affective manifolds in which moral cues exert weaker pull.

\noindent
In this sense, the EQ is not merely a trait measure; it provides a quantitative coordinate for the dispositional term $\beta_C$ in the mapping
\[
\mathscr{P}(\delta_m) = f(\alpha_E, \beta_C, \gamma_R),
\]
where $\beta_C$ denotes the stable parameters shaping how evaluative information is transformed into behaviour.

\subsection*{Historical and Psychometric Grounding}
Empirically, the EQ has a robust record: strong internal reliability, stable factor structure across cultures \cite{Wakabayashi2006}, convergence with related constructs (empathic concern, emotional intelligence), and predictive validity for prosociality in behavioural economic tasks. Neurocognitive studies further show correlations between EQ scores and activation in vmPFC, anterior insula, and TPJ—regions central to affective resonance and mental-state attribution \cite{Moll2002,Decety2004} (refer to section~\ref{sub:neuro}, page~\pageref{sub:neuro} for a discussion on brain regions).

These features make the EQ particularly suitable for this thesis: it is theoretically interpretable, computationally tractable, and empirically grounded.

\subsection{EQ and Synthetic Presence}
The central scientific function of EQ in this experiment was to determine whether empathic sensitivity moderated the attenuation effect introduced by the humanoid robot. One plausible hypothesis, grounded in moral psychology and HRI, is that high-empathy individuals would exhibit stronger prosociality and possibly stronger perturbation under synthetic social cues \cite{Kuchenbrandt2011,Zlotowski2015}.

\noindent
The data did not support this possibility. EQ did \emph{not} moderate the
displacement effect: both higher- and lower-empathy participants tended to show
reduced prosocial donation in the robot condition. This pattern suggests the
following interpretive reading:

\begin{center}
	\emph{"The robot may have influenced the evaluative field itself rather than
		trait-dependent gradients within it."}
\end{center}

\noindent
This interpretation is consistent with findings in HRI indicating that robotic
presence can modulate attentional and social–evaluative processing in ways that
do not strongly depend on empathic predisposition~\cite{Malle2015,Komatsu2016}.


\subsection{Methodological Role in the Thesis}
The EQ served two indispensable methodological purposes:

\begin{enumerate}
	\item \textbf{Controlling for affective heterogeneity}.  
	Without a measure of empathic sensitivity, reductions in donation could have been attributed to unmeasured differences in participants’ empathy levels. The EQ rules out this confound.
	
	\item \textbf{Modelling the affective dimension of $\beta_C$}.  
	EQ provides the affective coordinate of the dispositional manifold, enabling cluster analysis and regression models to distinguish dispositional shape from field-level perturbation.
\end{enumerate}

\noindent
Thus, even though affective resonance plays an important role in moral cognition, the experimental pattern suggests that the perturbation introduced by the humanoid robot operated \emph{upstream} of empathy—modulating aspects of the evaluative topology rather than amplifying or dampening empathic traits.

\noindent
Having considered the affective dimension of $\beta_C$, we now turn to its analytical counterpart: the Systemizing Quotient.



\section{The Systemizing Quotient (SQ): Structural Precision in the Evaluative Field}

\noindent
Where the Empathizing Quotient (EQ) indexes affective resonance, the Systemizing Quotient (SQ) \cite{Baron2003,Goldenfeld2005,Wakabayashi2006} quantifies a cognitive style characterised by rule extraction, structural analysis, and the search for causal regularities. Within the evaluative--topological model developed in this thesis, the SQ corresponds to the \emph{analytical curvature} of the evaluative field: the extent to which agents encode situations via stable structural relations rather than affective gradients.

\subsection{Theoretical Background and Psychometric Foundations}
The SQ emerged from the Empathizing--Systemizing framework \cite{Baron2002,Baron2009}, originally designed to capture the dissociability of affective versus rule-based processing in autism research. Subsequent work broadened this motivation: systemizing is now associated with mechanistic reasoning, pattern extraction, predictive modelling, and a preference for low-noise, high-coherence causal schemas \cite{Goldenfeld2005}. Psychometric studies demonstrate high internal reliability, cross-cultural robustness, and predictable correlations with analytic problem-solving and rule-consistent behaviour.

Neurocognitively, higher SQ scores have been associated with increased lateral prefrontal and parietal activation during analytic reasoning~\cite{Fritz2016,Wheelwright2006,Ecker2016,BaronCohen2005NeuralBasis}, and with reduced activation in affective salience networks during certain social tasks~\cite{Gleichgerrcht2013}. Taken together, these findings are consistent with interpreting SQ as indexing a cognitive style that places greater weight on structural stability than on affective modulation.


\subsection{SQ Across Moral Psychology and HRI}
In moral psychology, systemizing predicts greater reliance on deliberative processing, reduced affective interference, and increased endorsement of principle-based judgments 
in high-conflict dilemmas~\cite{Greene2014,Cushman2013DualSystem,BaronCohen2003,Chakroff2016,PaxtonGreene2010}.  In behavioural economics, high-SQ individuals show more consistent strategic patterns 
and reduced susceptibility to framing effects~\cite{DeMartino2006,Krajbich2015,Grice2017}.

In Human--Robot Interaction, systemizing tendencies shape expectations about synthetic 
agents: high-SQ participants tend to interpret robots through structural and functional 
cues rather than anthropomorphic ones and attribute competence and reliability more 
readily than emotional or social qualities~\cite{Zlotowski2015,Malle2015,Komatsu2016,Thellman2017,Spatola2021}. 
This makes the SQ especially relevant in the present experiment, where the perturbation 
introduced by the robot is primarily structural rather than affective.


\subsection{Systemizing Quotient (SQ): Structural Bias and Evaluative Rigidity}

Within the evaluative–topological framework developed in Chapter~\ref{chap:moral_primer}, 
each dispositional measure contributes a distinct dimension to the latent 
configuration $\beta_C$. Whereas the Empathizing Quotient (EQ) captures the 
affective attractors that steepen or flatten moral salience, the Systemizing 
Quotient (SQ) indexes a different property of the evaluative field: the degree to 
which agents privilege structural invariants, analytic coherence, and rule-based 
regularities when interpreting a situation.

In the functional decomposition
\[
\mathscr{P}(\delta_m) = f(\alpha_E, \beta_C, \gamma_R),
\]
SQ enters as a shaping parameter of $\beta_C$: it modulates the \emph{rigidity}, 
\emph{smoothness}, and \emph{resistance to perturbation} of evaluative 
trajectories. High-SQ agents tend to encode situations through stable relational 
patterns; their evaluative landscape is less susceptible to low-level shifts in 
affect or fleeting cues of social meaning. Conversely, lower SQ reflects a more 
flexible, affect-sensitive terrain in which subtle perturbations may redirect the 
trajectory from appraisal to action.

This topological intuition can be expressed heuristically by treating the 
evaluative potential $V(x)$ as a surface whose curvature reflects the 
systemizing tendency:
\[
\nabla^2 V(x) \;\propto\; \mathrm{SQ}.
\]
The Laplacian here is not introduced as a mechanistic claim about neural 
computation but as a conceptual device: higher curvature corresponds to a 
stiffer evaluative surface—one whose gradients change slowly and whose 
trajectories are less easily diverted by perturbational forces. Lower curvature 
marks a softer topology, where local salience cues exert proportionally greater 
influence on evaluative flow.

In this way, SQ provides one of the coordinate axes through which the 
dispositional manifold becomes empirically visible. Its contribution to $\beta_C$ 
is not descriptive ornamentation but a structural constraint on how agents weigh 
context, interpret relational structure, and integrate perturbational input. As 
the later cluster analysis shows, variation in SQ participates in shaping the 
dispositional geometry; but the perturbation induced by robotic presence operates 
at a field level that exceeds these trait-contingent contours—a dissociation that 
the formalism above is designed to make intelligible.


\subsection{SQ, Synthetic Presence, and Field-Level Perturbation}
The experiment in Chapter~\ref{chap:exp_methods} indicated that SQ did
\emph{not} moderate the behavioural attenuation observed in the presence of the
humanoid robot. Participants with higher SQ—who might be expected to rely more
heavily on rule-based evaluative strategies—showed a similar directional
reduction in prosocial behaviour to both higher- and lower-empathy participants.

This pattern is conceptually informative. It suggests that the influence of
robotic presence operated not on participants’ cognitive styles but on aspects
of the \emph{evaluative field itself}. Systemizing tendencies did not appear to
buffer, amplify, or redirect the behavioural effect.

\begin{center}
	\emph{The perturbation introduced by the robot appears to have been global
		rather than trait-specific.}
\end{center}


This aligns with existing HRI work showing that ambiguous synthetic agents alter social affordances and attentional dynamics independently of analytic or empathic predispositions \cite{Zlotowski2015,Malle2015}.

\subsection{Methodological Significance}
SQ served two methodological functions within the experiment:

\begin{enumerate}
	\item \textbf{Controlling for cognitive style}.  
	Without an explicit measure of systemizing tendencies, attenuation could have been misattributed to analytic disposition rather than environmental perturbation.
	
	\item \textbf{Modelling the structural dimension of $\beta_C$}.  
	SQ provides the analytical coordinate within the dispositional manifold, enabling the analysis to distinguish dispositional geometry from field-level displacement.
\end{enumerate}

\noindent
Together with the EQ, the Systemizing Quotient ensures that dispositional structure is properly characterised before interpreting the behavioural impact of robotic presence. The next tool completes this picture: the Big Five Inventory, which captures broad personality geometry beyond empathy and systemizing.

% -----------------------------------------------------
\section{The Big Five Inventory (BFI): Personality Geometry Within the Evaluative Topology}

The Big Five Inventory (BFI) is one of the most extensively validated instruments
in differential psychology~\cite{John1991,JohnSrivastava1999,Rammstedt2007}.  
Decades of lexical, psychometric, and cross-cultural research have shown that  
Openness, Conscientiousness, Extraversion, Agreeableness, and Neuroticism 
constitute a robust low-dimensional structure for describing stable 
individual differences~\cite{John1999,McCraeCosta2008}.  
Within the evaluative--topological framework developed in 
Chapter~\ref{chap:moral_primer}, these traits supply a principled coordinate 
system for locating each participant’s contribution to the dispositional term 
$\beta_C$ in the functional decomposition
\[
\mathscr{P}(\delta_m) = f(\alpha_E, \beta_C, \gamma_R),
\]
where $\alpha_E$ captures environmental salience, $\beta_C$ captures stable 
evaluative tendencies, and $\gamma_R$ denotes the perturbation introduced by the 
humanoid robot.

\subsection{Why Personality Matters for a Topological Account of Moral Cognition}

Personality traits shape how situations are perceived, which cues are weighted,
and how affective signals enter downstream judgement~\cite{Funder2001,Roberts2006,DeYoung2010,Slovic2007}.
 They influence the 
steepness of prosocial attractors, the volatility of evaluative trajectories, 
and the degree to which social presence modulates behaviour.  
For a thesis concerned with \emph{perturbations of the evaluative field} (see page~\pageref{def:eval_field}), the BFI 
provides the structural background against which any deformation must be 
interpreted. Without a model of trait geometry, one could not determine whether 
the attenuation induced by robotic presence reflected genuine field-level
displacement or merely the aggregation of heterogeneous personality effects.

\subsection{Psychometric Stability and Cross-Domain Predictive Value}

The BFI exhibits strong internal reliability, stable factor structure across 
cultures, and predictive value across a wide range of behavioural domains: 
cooperation, social engagement, norm adherence, and affective regulation 
\cite{Barrick1991,Graziano1996,Habashi2016}.  
Short forms such as the BFI-10 preserve this structure while remaining suitable 
for time-constrained experimental settings~\cite{Rammstedt2007}.  
These qualities make the BFI particularly well suited to modelling $\beta_C$ 
within an evaluative–topological framework that requires dispositions to act as 
stable constraints on the flow of salience and affect.

\subsection{Personality and Moral Behaviour Under Social Presence}

Each Big Five trait has a theoretically grounded role in shaping evaluative 
processing:
\begin{itemize}
	\item \textbf{Agreeableness} steepens prosocial basins and increases sensitivity to interpersonal harm.
	\item \textbf{Conscientiousness} stabilises evaluative gradients and supports rule-consistent trajectories.
	\item \textbf{Neuroticism} introduces volatility and amplifies contextual reactivity.
	\item \textbf{Extraversion} enhances responsiveness to social presence and implicit monitoring cues.
	\item \textbf{Openness} broadens contextual sampling and moderates tolerance for ambiguity.
\end{itemize}

HRI studies show that these dimensions influence how artificial agents are 
perceived---as social partners, observers, or normatively relevant entities 
\cite{Zlotowski2015,Banks2020}.  
The BFI therefore enables the experiment to test whether personality-dependent 
interpretations of robotic presence contribute to prosocial modulation or whether 
the effect arises at a level independent of personality variation.

\subsection{Personality Geometry in the Evaluative--Topological Model}

Within the topological formalism, the BFI quantifies the geometry of $\beta_C$:  
the curvature, stability, and directionality of each participant’s evaluative 
field.  Traits can be interpreted as shaping particular geometric properties:

\begin{itemize}
	\item Agreeableness deepens altruistic attractors.
	\item Conscientiousness smooths and stabilises the evaluative surface.
	\item Neuroticism introduces local turbulence and heightened gradient sensitivity.
	\item Extraversion strengthens channels responsive to social cues.
	\item Openness expands the contextual manifold explored during appraisal.
\end{itemize}

These interpretations allow personality structure to be integrated without
reducing behaviour to trait levels: traits constrain trajectories but do not 
determine them.

\subsection{Cluster Analysis: Mapping the Dispositional Manifold}

The cluster analysis reported in Chapter~\ref{chap:exp_methods} revealed three recurrent dispositional configurations:

\begin{enumerate}
	\item \textbf{Prosocial--Empathic}: high Agreeableness combined with high EQ, 
	producing steep affective attractors for interpersonal concern.
	
	\item \textbf{Emotionally Reactive}: high Neuroticism and unstable evaluative curvature, 
	yielding heightened sensitivity to contextual fluctuation.
	
	\item \textbf{Analytical--Structured}: high Conscientiousness together with high SQ, 
	generating rigid, rule-oriented evaluative pathways.
\end{enumerate}

These profiles show that participants did not enter the experiment as 
psychologically interchangeable agents.  
The dispositional manifold exhibited \emph{structured heterogeneity}: 
distinct attractor regions with characteristic evaluative tendencies.  
In the evaluative--topological framework, the BFI and related measures provide the 
coordinate system through which this manifold becomes empirically visible.  
Without such structure, the subsequent perturbation analysis would lack an anchor 
for distinguishing individual variation from field-level dynamics.

\subsection{The Key Empirical Result: A Uniform Field-Level Displacement}

Against this backdrop of meaningful dispositional diversity, the experiment 
produced a striking and theoretically informative result:

\begin{center}
	\begin{leftbar}
		\textit{The humanoid robot induced a uniform attenuation of prosocial behaviour 
			across all clusters.}
	\end{leftbar}
\end{center}

No Big Five dimension—and no latent cluster—moderated the effect.

This pattern is not merely a negative finding; it is the empirical signature of a 
\emph{field-level perturbation}.  
Had the effect been trait-contingent, clusters would have separated in their 
behavioural response. Instead, the attenuation aligned across the entire 
dispositional manifold.  
The perturbation introduced by the robot acted not on personality-specific 
pathways, but on the evaluative topology itself:  
reshaping salience gradients, dampening affective pull, and shifting the 
conditions under which moral appraisal becomes behaviourally operative.

In this sense, the psychometric tools were indispensable.  
They enabled a clean dissociation between:

\begin{itemize}
	\item the \emph{shape of the dispositional manifold} ($\beta_C$), and  
	\item the \emph{geometry of the perturbation} ($\gamma_R$).
\end{itemize}

Only with a well-characterised $\beta_C$ could the experiment demonstrate that 
the robot’s influence was globally oriented—an operator on the evaluative field, 
not a stimulus whose meaning depended on personality variation.

With the role of personality geometry clarified, we now turn to the final 
psychometric component of $\beta_C$: the Systemizing Quotient.


\subsection{Methodological Significance}
The BFI provides the evidential basis for distinguishing between:

\begin{itemize}
	\item \textbf{dispositional geometry} (the shape of $\beta_C$), and
	\item \textbf{field-level deformation} induced by the robotic perturbation ($\gamma_R$).
\end{itemize}

Without the BFI, the attenuation could easily have been misinterpreted as a by-product of personality—differences in Agreeableness, Extraversion, or Neuroticism—rather than as a global shift in the evaluative field.

The analysis of BFI scores brings the dispositional component of the model into clear view:

\begin{center}
	\begin{leftbar}
		\textit{The personality manifold is structured, yet the attenuation induced by 
			robotic presence does not track personality. It deforms the evaluative field 
			globally.}
	\end{leftbar}
\end{center}

The BFI delivers three results essential for the evaluative--topological framework:

\begin{enumerate}
	\item It confirms that participants exhibit theoretically meaningful 
	dispositional variation rather than psychological homogeneity.
	
	\item It provides the coordinate basis for the cluster structure of $\beta_C$, 
	making the dispositional manifold empirically visible.
	
	\item It shows that the robot’s influence is not trait-contingent: the observed 
	attenuation arises from a field-level shift, not from personality-driven pathways.
\end{enumerate}

These results complete the characterisation of $\beta_C$. With the geometry of  dispositional structure established—and with trait-based explanations ruled out—we  can turn to the design of the perturbation itself: the Watching--Eye paradigm and  the silent humanoid robot that reconfigures the evaluative field.


\section{The Watching--Eye Paradigm: Amplifying Moral Salience and Revealing Field-Level Deformation}

\noindent
Across behavioural ethics, social psychology, and field studies of cooperation, a 
remarkably stable finding recurs: minimal cues of being observed---stylised eyes, 
schematic pupils, or even simple dot configurations---increase charitable giving, 
norm compliance, and prosocial behaviour%
\cite{Haley2005,Bateson2006,Nettle2013,Conty2016}.  
No instruction, coercion, or explicit social information is required. The effect 
arises because these cues selectively heighten the salience of norm-relevant 
action.

Within the evaluative--topological framework developed in earlier chapters, 
watching--eye stimuli are best understood as \emph{topological amplifiers}: they 
increase the weight of prosocial directions in the evaluative field by signalling 
that one’s behaviour carries implicit social meaning. This makes them the ideal 
baseline perturbation against which to test whether synthetic presence deforms 
evaluative trajectories in a fundamentally different way.

\subsection{Watching--Eye Cues as Topological Amplifiers}

\noindent
Early explanations cast the watching--eye effect as an implicit reputational 
computation---a sense that one’s behaviour is observable and therefore subject to 
social judgement%
\cite{Haley2005,Bateson2006}.  
Contemporary models provide a more mechanistic account: the effect reflects a 
coordinated modulation of

\begin{itemize}
	\item \textbf{attentional uptake}, increasing sensitivity to norm-relevant cues;
	\item \textbf{affective arousal}, particularly self-conscious emotions linked to evaluation;
	\item \textbf{interpretive expectation}, via implicit social-monitoring systems.
\end{itemize}

\noindent
In the formalism already introduced, the environmental input 
$\alpha_E$ encodes contextual features that influence early evaluative appraisal.  
A watching--eye cue increases the prosocial component of this input. Formally, we 
express this as:

\[
\alpha_E \;\mapsto\; \alpha_E + \delta\alpha_{\text{eye}}, 
\qquad \delta\alpha_{\text{eye}} > 0.
\]

\noindent
This notation does not imply a specific quantitative metric. It indicates that 
the cue adds a \emph{directional} contribution to the evaluative field: prosocial 
gradients steepen, making cooperative trajectories more accessible in the 
resulting cognitive--affective dynamics.

\subsection{Why Child-Pair Eyes Provide a Clean Baseline}

\noindent
Child-eye posters are widely used in prosociality research because they combine 
high perceptual salience with minimal interpretive content. Decades of evidence 
show that stylised child eyes:

\begin{itemize}
	\item robustly increase prosocial behaviour across cultures and modalities%
	\cite{Haley2005,Bateson2006,ErnestJones2011,Ekstrom2012,Nettle2013};
	\item evoke empathic and care-oriented affective responses%
	\cite{ThompsonBooth2014};
	\item heighten vigilance without implying the presence of a moral agent%
	\cite{Conty2016}.
\end{itemize}

\noindent
They therefore provide the ideal experimental baseline: a 
\emph{high-salience, low-interpretation} perturbation.  
The cue is strong enough to elevate prosocial gradients, yet conceptually simple 
enough to avoid confounds involving mind attribution, intentionality, or 
interpersonal inference.

In the context of this thesis, the watching--eye paradigm serves a precise 
theoretical role. It establishes how the evaluative field responds to a 
well-understood salience amplifier. Against this backdrop, the perturbation 
introduced by the humanoid robot can be evaluated not simply in terms of outcome, 
but as a deformation of the underlying evaluative topology.


\subsection{Why Synthetic Presence Dilutes or Distorts the Effect}

\noindent
The central theoretical claim of the thesis—that humanoid robots act as \emph{perturbation operators} on the evaluative field—becomes particularly clear when considering their interaction with watching-eye cues.

Humanoid robots are perceptually social but ontologically indeterminate. They are seen, but not reliably understood, as bearers of evaluative or moral capacities \cite{Zlotowski2015,Malle2015,Komatsu2016}. This ambiguity weakens all three mechanisms that normally support the watching-eye effect:

\begin{enumerate}
	\item \textbf{Reputational inference is unstable.} Robots rarely trigger the implicit assumption that one is being morally evaluated.
	\item \textbf{Affective resonance is dampened.} Observation by a non-agentive entity does not engage self-conscious emotions strongly.
	\item \textbf{Attentional cues conflict.} The perceptual system registers social presence; higher-order systems deny full agency.
\end{enumerate}

The result is a fractured evaluative landscape: the cue ``someone is watching'' is present at the perceptual level, but stripped of the evaluative force that normally steepens prosocial attractors.

\subsection{Empirical Finding: Uniform Attenuation of the Watching--Eye Effect}

\noindent
The experiment yields a clear result:

\begin{quote}
	\emph{The presence of a humanoid robot uniformly attenuated the watching--eye effect across all dispositional clusters.}
\end{quote}

\noindent
Participants normally responsive to cues of observation---including those high in 
Agreeableness, Extraversion, or Empathizing Quotient (EQ)---showed the same 
directional reduction in prosocial action when the robot was present.  

To express this within the evaluative--topological formalism, recall that:

\begin{itemize}
	\item $\alpha_E$ denotes the environmental input that shapes early evaluative appraisal;
	\item $\delta\alpha_{\text{eye}}$ represents the salience increment produced by the watching--eye cue (a positive shift in prosocial weighting);
	\item $\Delta_{\mathscr{R}}$ denotes the deformation of the evaluative field induced by robotic presence.
\end{itemize}

\noindent
The behavioural pattern can then be summarised as:

\[
(\alpha_E + \delta\alpha_{\text{eye}})
\;\mapsto\;
(\alpha_E + \delta\alpha_{\text{eye}}) - \Delta_{\mathscr{R}}.
\]

\noindent
This notation does not specify a numerical metric. It encodes the structural fact 
that the robot imposes a \emph{countervailing field-level displacement} 
($\Delta_{\mathscr{R}}$) that reduces the effective prosocial gradient introduced 
by the eye cue. Crucially, the displacement occurs \emph{regardless of the 
	participant’s dispositional configuration} $\beta_C$.  
No Big Five trait, no EQ or SQ dimension, and no latent cluster moderated the 
attenuation. This is the signature of a perturbation that acts on the 
\emph{evaluative field itself}, not on specific trait-dependent pathways.

\subsection{Why the Watching--Eye Paradigm Is Indispensable}

\noindent
Within this thesis, the watching--eye paradigm is not an auxiliary manipulation; 
it is a methodological anchor. It serves four essential functions:

\begin{itemize}
	\item \textbf{A reliable high-salience baseline:} the eye cue consistently 
	steepens prosocial gradients, allowing attenuation to be detected with 
	precision.
	
	\item \textbf{Continuity with moral psychology:} the paradigm embeds the 
	experiment within a long empirical tradition, enabling direct interpretive 
	comparison with decades of prosociality research.
	
	\item \textbf{Isolation of perturbation effects:} attenuation is meaningful 
	only when there is something to attenuate. The eye cue provides this 
	necessary initial elevation of salience.
	
	\item \textbf{Revelation of topological structure:} the contrast between 
	amplification (eye cue) and deformation (robot) allows the evaluative 
	topology to be interrogated rather than merely described.
\end{itemize}

\noindent
Without this paradigm, the behavioural shift could not be interpreted as a 
\emph{field-level deformation} of evaluative structure.

\subsection{Integration With Costly Prosocial Action}

\noindent
Donation tasks provide observable moral behaviour rather than abstract moral 
judgement. Their integration with watching--eye cues allows the experiment to 
trace the evaluative sequence from:

\begin{enumerate}
	\item \textbf{cue uptake} (the perceptual registration of observation),
	\item \textbf{salience amplification} (increased weighting of prosocial 
	trajectories), to
	\item \textbf{action selection} (allocation of real resources).
\end{enumerate}

\noindent
The robot’s attenuation of this sequence demonstrates that synthetic presence 
modifies the mapping from perceptual cue to moral action. In topological terms, 
the robot does not erase the watching--eye effect; it \emph{counteracts} it by 
introducing a deformation of the evaluative field through which all trajectories 
must pass.  


\subsection{Synthesis: A Window Into Moral Topology}

\noindent
The watching--eye paradigm functions as the critical hinge of the measurement 
architecture. By steepening prosocial gradients, it renders the structure of the 
evaluative field empirically visible. By attenuating these gradients, the humanoid 
robot exposes the central empirical insight of the thesis:

\begin{quote}
	\textbf{Synthetic presence acts on the evaluative field itself, not on 
		personality-dependent pathways.}
\end{quote}

\noindent
The watching--eye effect thus provides the diagnostic contrast through which the 
robot’s field-level deformation becomes observable.

\section{General Conclusion: Measurement as the Logic of Synthetic Moral Perturbation}

\noindent
This chapter has not merely catalogued instruments. It has constructed the 
\emph{measurement logic} that makes synthetic moral perturbation empirically legible.  
The Empathizing Quotient (EQ), Systemizing Quotient (SQ), Big Five Inventory (BFI), and 
the Watching--Eye paradigm form an integrated system of epistemic probes: each is 
theoretically grounded, psychometrically validated, and methodologically necessary for 
making the evaluative topology of moral cognition accessible without flattening it.

\medskip

\noindent
The formal architecture developed earlier models moral behaviour as the output of a 
mapping
\[
\mathscr{P}(\delta_m) = f(\alpha_E, \beta_C, \gamma_R),
\]
where $\alpha_E$ encodes environmental salience, $\beta_C$ the dispositional manifold, 
and $\gamma_R$ the perturbation induced by synthetic presence.  
Each measurement tool targets one of these components:

\begin{itemize}
	\item \textbf{EQ} probes affective attractors within $\beta_C$.
	\item \textbf{SQ} probes structural curvature and analytic rigidity within $\beta_C$.
	\item \textbf{BFI} provides the coordinate geometry through which the topology of 
	$\beta_C$ becomes measurable.
	\item \textbf{Watching--Eye cues} modulate $\alpha_E$ by steepening prosocial 
	gradients, enabling displacement by $\gamma_R$ to be detected.
\end{itemize}

\noindent
Individually, these instruments measure meaningful constructs.  
Collectively, they make the evaluative topology \emph{empirically visible}.

\subsection{Dispositional Mapping: A Structured Manifold, Not a Confound}

\noindent
The EQ, SQ, and BFI jointly revealed a structured dispositional manifold: 
Affective--Prosocial, Emotionally Reactive, and Analytical--Structured profiles.  
This establishes that participants entered the experiment with real, theoretically 
interpretable heterogeneity.

\medskip

\noindent
Yet the experiment demonstrated a decisive dissociation:

\begin{center}
	\emph{The humanoid robot attenuated prosocial action uniformly across all clusters.}
\end{center}

\noindent
The dispositional manifold $\beta_C$ was not the locus of modulation.  
The perturbation acted on the evaluative field itself.  
Without psychometric resolution, this inference would have been impossible: the uniform 
attenuation could have been mistaken for trait-driven variance rather than field-level 
deformation.

\subsection{Watching--Eye Cues as Diagnostic Amplifiers}

\noindent
The watching--eye manipulation provided the complementary half of the measurement logic.  
By steepening prosocial gradients in $\alpha_E$, it yielded the high-salience baseline 
necessary to detect synthetic attenuation. The robot did not merely reduce generosity; it 
\emph{neutralised a well-established amplifier of moral salience}.  
This interaction is the clearest behavioural signature of a perturbation operating at the 
level of evaluative topology.

\noindent
In theoretical terms: the eyes amplify the gradient; the robot deforms the landscape.

\subsection{Philosophical and Ethical Meaning}

\noindent
Placed against the philosophical frameworks introduced earlier, the findings reveal:

\begin{itemize}
	\item \textbf{Against rationalist models:} the perturbation bypasses deliberation.
	\item \textbf{Against virtue-ethical accounts:} stable dispositions do not shield 
	agents from synthetic deformation.
	\item \textbf{Against sentimentalist explanations alone:} high-empathy profiles are 
	attenuated equally.
	\item \textbf{Against classical Machine Ethics:} the moral significance of AI lies 
	not in artificial agency but in field-level modulation.
\end{itemize}

\noindent
Synthetic systems do not enter moral space as agents but as \emph{operators}: they bend the 
evaluative field through which moral meaning becomes action.

\subsection{Methodological Synthesis: The Tools as Epistemic Infrastructure}

\noindent
This chapter has built the epistemic infrastructure required for the experiment. It has 
shown that:

\begin{enumerate}
	\item moral behaviour must be analysed through distinct layers of environmental input, 
	dispositional structure, and perturbational force;
	\item psychometrics provides the resolution needed to map $\beta_C$ and rule out 
	trait-based explanations;
	\item observational cues provide the leverage needed to modulate $\alpha_E$ in a 
	controlled manner;
	\item synthetic presence must be interpreted as a deformation of \emph{evaluative 
		topology}, not as a trait-contingent stimulus.
\end{enumerate}

\noindent
The tools are therefore not auxiliary components of the study—they are the 
\emph{conditions of intelligibility} for its central empirical claim.


\section{Transition to Experimental Methods}

\noindent
The work of this chapter has been to give the experiment something it cannot supply for itself: an architecture within which its measurements become intelligible. We now have the conceptual scaffolding needed to understand what a perturbation of moral cognition would look like, and—equally important—what it would not. The evaluative field, the dispositions that curve it, the cues that tilt its gradients: these are no longer abstractions but the coordinates from which empirical observation becomes possible.

\noindent
The next chapter turns from architecture to instrumentation. It shows how these evaluative structures were translated into stimuli, procedures, and statistical models; how deformation is rendered measurable; and how dispositional parameters can be tracked as they respond to contextual change.

\begin{center}
	\begin{leftbar}
		\textit{The tools provide the coordinates; the experiment traces the trajectory.}
	\end{leftbar}
\end{center}

The guiding question for the remainder of the thesis can now be stated with precision:

\begin{center}
	\begin{leftbar}
		\textit{Does synthetic presence reshape the evaluative field through which moral salience becomes action?}
	\end{leftbar}
\end{center}

What follows is the methodological counterpart to the conceptual work developed here. The instruments introduced in the next chapter are not merely data-gathering devices; they are the means by which the evaluative landscape becomes empirically visible. With these foundations in place, the thesis moves—\emph{from structure to measurement, from possibility to test}—into the experimental core of the argument.
