\chapter{Tools of Measurement, Framework and Experimental Design}
\label{chap:tools_new}
\thispagestyle{pprintTitle}

\section{Tools of Measurement}

Empirical work aimed at understanding moral cognition must specify, with some philosophical care, the instruments through which psychological and behavioural  structures become accessible to observation. Moral appraisal itself is never directly given; it does not present as a datum in the way a magnetic field or a 
spectral line may. \textit{It is inferred from patterned responses}: affective, dispositional, perceptual, and social that reveal how evaluative information is encoded and transformed within the agent’s cognitive architecture~\cite{Haidt2001EmotionalDog, Greene2002, Greene2004, Cushman2013DualSystem, Crockett2016Models, Fedyk2017}. 

The instruments employed in this thesis therefore function not as neutral measurement devices but as \emph{theoretically motivated probes}. Each tool targets a specific dimension of the evaluative topology developed in earlier chapters, allowing latent dispositional structure to be rendered empirically tractable without collapsing its complexity into reductive summary 
scores. Their significance is thus analogous to that of the instruments of physics and the conceptual scaffolds of philosophy: they do not merely “record” a value but \emph{constitute the mode of access} through which the phenomenon becomes observable at all.

In this respect, psychological instruments resemble the measuring practices of both laboratory physics and analytic philosophy. In physics, one does not detect an electron or a gravitational wave without the apparatus that makes such a 
phenomenon measurable; the device does not simply capture reality but partially \emph{defines} the phenomenon by specifying the \textit{level of abstraction} and the dimension of variation it reveals. Similarly, philosophical analysis specifies 
the conceptual lens through which reasoning, inference, or normativity become discernible. Measurement is not passive reception but disciplined \textit{construction of access}. 

The same principle governs the tools used here. 
Measures of systematising and empathising dispositions (EQ/SQ) do not claim to exhaust personality, but they isolate axes of cognitive--affective variation known to shape intuitive and deliberative routes in moral cognition~\cite{BaronCohen2003,BaronCohenWheelwright2004,BaronCohen2009}. The Big Five Inventory (BFI) captures broad dispositional gradients that interact with evaluative salience and behavioural inhibition \cite{John1991,JohnSrivastava1999,Rammstedt2007}. 
These instruments are therefore not “psychological thermometers” but structured interventions into the \textit{evaluative field}: each selects, with theoretical justification, the dimensions along which individual differences become experimentally meaningful.

This methodological stance also informs the design of the Watching-Eye paradigm used in the experimental study. The effect is not introduced as a folkloric behavioural curiosity but as a calibrated environmental perturbation: 
\bigskip
\noindent
\begin{center}
	\begin{leftbar}
		\textit{a means of modulating accountability salience and affective vigilance at the cognitive level of abstraction identified earlier. }
	\end{leftbar}
\end{center}

\bigskip
\noindent
The Watching-Eye cue is thus treated as a \emph{contextual operator} on the evaluative topology, providing a controlled channel through which implicit social meaning acquires behavioural 
force. Its careful specification is essential, for---as in physics---the measurement depends not only on the quantity being observed but on the entire apparatus through which observation is made possible.

Throughout this chapter, the emphasis will therefore not be on the instruments as psychological artefacts, but on their \textit{epistemic role} in the explanatory framework of the thesis: how each measure maps onto the evaluative architecture, what assumptions it encodes, and how it constrains the interpretation of the behavioural data that follows.


\section{Measurement as Theoretical Access}

The methodological commitments of this thesis require a principled account of the 
instruments through which evaluative behaviour becomes empirically accessible. 
Work in moral psychology and cognitive science has repeatedly shown that moral appraisal is not directly observable but manifests through structured patterns of 
affective response, controlled cognition, and social cue integration~\cite{Haidt2001EmotionalDog, Greene2002, Greene2004, Cushman2013DualSystem, Crockett2016Models, Fedyk2017}. For this reason, empirical studies of moral cognition depend on validated constructs and measurement strategies capable of rendering latent dispositions observable without distorting their theoretical 
significance.

The present work does not align itself with moral cognition research as a discrete disciplinary domain. Instead, it draws upon rigorously established constructs from moral psychology, cognitive science, and social signal processing as operational resources for making evaluative dispositions tractable. Instruments such as the Empathizing Quotient~\cite{BaronCohenWheelwright2004_EmpathyQuotient}, the Systemizing Quotient~\cite{BaronCohenRichlerBisaryaGurunathanWheelwright2003_SystemizingQuotient}, and the Big Five Inventory~\cite{JohnDonahueKentle1991_BigFiveInventory, Rammstedt2007} provide precisely the kind of psychometric access to stable individual differences that contemporary models of moral cognition identify as structurally relevant. Likewise, the analytical frameworks developed within Social Signal Processing 
\cite{Vinciarelli2009} offer methodological grounding for understanding how agents 
register, interpret, and respond to contextually salient perturbations.

In this sense, the psychometric tools employed here are not neutral measurement devices, but \emph{theoretically motivated probes} into the dispositional structures that shape how agents encode,negotiate, and respond to morally salient changes in their evaluative environment.

\subsection{A Coherent Measurement Suite}

The Empathizing Quotient (EQ), the Systemizing Quotient (SQ), and the Big Five Inventory (BFI) offer validated operationalisations of dispositional constructs repeatedly implicated in moral judgment and social decision-making. Likewise, the Watching--Eye paradigm \cite{Haley2005, Bateson2006, Nettle2013, Conty2016, Dear2019} 
constitutes a mature experimental framework for probing reputational concern, prosocial motivation, and sensitivity to subtle social cues. Together, these instruments form a coherent measurement suite capable of isolating trait-level parameters that interact with contextual salience to shape moral behaviour.

If the theoretical chapters have argued that moral cognition is best understood as an evaluative topology shaped by attention, affect, social meaning, and trait-level curvature, then the tools introduced here function as the \emph{coordinate system} through which that topology becomes visible. Their role 
is not to reduce personality to numbers, nor to treat empathy or systemizing as atomic psychological entities, but to expose the \textit{invariant structures along which agents differ in how they process moral salience}.


\subsection{Measurement in Experimental Context}

In the experiment motivating this thesis, evaluative perturbation is elicited not 
through explicit moral dilemmas but through a more subtle and ecologically 
grounded manipulation: the silent perceptual presence of a humanoid robot. Prior 
work in human--robot interaction shows that even passively positioned robots can 
shift perceived social affordances, alter attentional allocation, and modulate 
expectations concerning norm-relevant behaviour \cite{Zlotowski2015, Malle2015, 
	Komatsu2016}. Their ambiguous ontological status disrupts default social priors 
and thereby reconfigures the salience landscape within which moral reasons become 
behaviourally operative.

In this respect, robotic presence functions as a controlled perturbation of the 
evaluative topology itself, enabling the empirical study of how dispositional 
invariants interact with contextual cues to produce measurable differences in 
moral behaviour. The Watching--Eye cue provides a second calibrated perturbation; 
together, they specify the salience structure of the environment in which 
participants navigate moral action.

\subsection{Purpose and Structure of this Chapter}

The aim of the chapter is thus twofold.

\begin{enumerate}
	\item First, to establish that each psychometric and experimental tool is 
	grounded in stable bodies of empirical and theoretical research across 
	psychology, cognitive science, HCI/HRI, and social signal processing. This 
	ensures that the constructs they measure---empathic sensitivity, systemizing 
	tendencies, personality traits, and responsiveness to social cues---are 
	well-defined, reproducible, and theoretically interpretable within the 
	broader landscape of moral psychology and social cognition.
	
	\item Second, to show how each tool contributes to the modelling of the 
	dispositional term $\beta_C$ in the formal expression
	\[
	\mathscr{P}(\delta_m) = f(\alpha_E, \beta_C, \gamma_R),
	\]
	where $\beta_C$ denotes the latent trait configuration governing how a 
	participant’s evaluative topology is modulated by the perturbation introduced 
	by the humanoid robot. In this sense, the tools are not ancillary components 
	of the experiment but operationalisations of the dispositional invariants 
	that mediate the transformation of evaluative salience under robotic 
	presence.
\end{enumerate}

The instruments included here---the Empathizing Quotient (EQ), the Systemizing Quotient (SQ), the Big Five Inventory (BFI), and the Watching--Eye paradigm---were selected because they satisfy three stringent criteria grounded in established empirical research~\cite{BaronCohenWheelwright2004,BaronCohen2003,BaronCohen2009,John1991,JohnSrivastava1999,Rammstedt2007,Haley2005,Bateson2006,Dear2019}:

\begin{enumerate}
	\item \textbf{Theoretical relevance:} Each tool targets a component of moral 
	topology (affective resonance, evaluative precision, personality curvature, 
	or salience modulation) \cite{Decety2004,Crockett2016,Conty2016,Vinciarelli2009}. 
	
	\item \textbf{Empirical robustness:} Each tool is validated across multiple 
	cultures, large samples, and decades of psychological research, and has been 
	used in studies of prosociality, moral sensitivity, social attention, and 
	Human--Robot Interaction (HRI) 
	\cite{Wakabayashi2006,Goldenfeld2005,Lawson2004,Habashi2016,Haley2005,Bateson2006,ErnestJones2011,Kuchenbrandt2011,Malle2016,Bremner2022}. 
	
	\item \textbf{Computational suitability:} Each tool produces variables 
	suitable for integration into regression models, cluster analysis, and 
	topological interpretation \cite{Cushman2013,Konovalov2016,Shenhav2017}. 
\end{enumerate}


Before turning to the tools themselves, we first articulate the methodological role they play within this thesis. Their significance lies not in psychometric convenience but in their ability to expose the latent structures through which evaluative salience is processed and transformed---structures that, as the experiment will show, can be subtly but measurably deformed by the presence of a synthetic agent.

At this point we can raise an important question: 
\bigskip
\noindent
\begin{center}
	\begin{leftbar}
		\textit{How does the theoretical weight of the measurement framework reconcile with the fact that the experiment involves a relatively modest sample of seventy-one participants? And further, have these tools been employed in comparable studies with similar data constraints?}
	\end{leftbar}
\end{center}

\bigskip
\noindent
\noindent
The answer reveals something essential about the architecture of the project. 
The tools used in this thesis---EQ, SQ, the BFI, and the Watching--Eye paradigm---were 
not selected because they demand large samples for interpretability, but because 
they target \emph{structurally stable} psychological constructs. These instruments are grounded in decades of psychometric work involving thousands of participants, 
and their factor structures, reliability profiles, and discriminant properties are 
well established across diverse populations 
\cite{BaronCohenWheelwright2004,BaronCohen2003,BaronCohen2009,John1991,JohnSrivastava1999,Rammstedt2007,Wakabayashi2006,Goldenfeld2005,Lawson2004}. 
In effect, they carry their statistical scaffolding with them. A study does not 
need to reproduce the entire validation literature; it inherits the stability of 
constructs that have already been exhaustively characterised.

\medskip

\noindent
This matters because the goal of the experiment is not to discover new personality factors nor to recover latent dimensions from scratch. It is to examine how \emph{known dispositional invariants} interact with a controlled perturbation of the evaluative field. Small-to-moderate sample studies are the norm in this domain. Replications of the Watching--Eye effect routinely employ 
samples ranging from 40 to 120 participants~\cite{Haley2005,Bateson2006,Nettle2013,Dear2019}, and HRI studies investigating the behavioural impact of robotic presence commonly operate within similar ranges \cite{Zlotowski2015,Malle2016,Bremner2022}. Even studies integrating personality measures into moral-decision paradigms---for example, the use of BFI or EQ/SQ to predict prosociality, empathic concern, or social attention---often rely on samples of comparable scale~\cite{Pfattheicher2015,Habashi2016,Komatsu2016,Yamagishi2014}.

\medskip

\noindent
The statistical strategy deployed in this thesis reflects this precedent. Rather than fitting high-dimensional models or mining for latent structure, the analysis treats dispositional measures as low-dimensional, theoretically structured parameters influencing the deformation of evaluative gradients. The inferential load therefore falls not on discovering complex patterns in sparse data, but on detecting systematic, directional shifts in behaviour induced by the experimental manipulation. For this purpose, a well-powered design does not require a large sample; it requires a clean manipulation, validated constructs, and an analysis 
aligned with the theoretical architecture~\cite{Cushman2013,Konovalov2016,Shenhav2017}.

\medskip

\noindent
In short, the experiment does not attempt to estimate the topology of moral cognition from scratch. It examines how a synthetic agent perturbs an already well-understood structure. The sample size is calibrated not to psychometric exploration but to experimental contrast: 


\bigskip
\noindent
\begin{center}
	\begin{leftbar}
		\textit{detecting whether robotic presence 
			produces a measurable attenuation of prosocial action across dispositional profiles.}
	\end{leftbar}
\end{center}

\bigskip
\noindent
Numerous studies across HRI, SSP, and moral psychology demonstrate that such effects are robustly detectable with sample sizes of the magnitude employed here~\cite{Pentland2007,Vinciarelli2009,Conty2016,Bremner2022}.

\medskip
\noindent
With this clarification in place, we now turn to the tools themselves. Their significance lies not in psychometric convenience, but in their ability to expose the latent structures through which evaluative salience is processed and 
transformed---structures that, as the experiment will show, can be subtly but measurably deformed by the presence of a synthetic agent.


\section{The Role of Psychometric Tools in the Evaluative--Topological Architecture}

\noindent
Within the framework developed thus far, moral behaviour is modelled as the endpoint of a trajectory across an evaluative field. Contemporary work in moral psychology and cognitive science converges on the idea that such trajectories arise from the coordinated influence of three factors: environmental cues, dispositional structure, and perturbational forces \cite{Haidt2001EmotionalDog, Greene2002, Greene2004, Cushman2013DualSystem, Crockett2016Models, Fedyk2017}. This is captured, in schematic form, by the functional decomposition
\[
\mathscr{P}(\delta_m) = f(\alpha_E, \beta_C, \gamma_R),
\]
where:
\begin{itemize}
	\item $\alpha_E$ encodes the \emph{environmental inputs}, such as the Watching--Eye prime and task context;
	\item $\beta_C$ represents the \emph{dispositional configuration} measured by psychometric instruments;
	\item $\gamma_R$ denotes the \emph{perturbation operator} introduced by the humanoid robot.
\end{itemize}

\noindent
The psychometric tools employed in this study populate the $\beta_C$ term. They render stable individual differences empirically visible by quantifying constructs known to influence how agents register and integrate affective, social, and contextual information. The Empathizing Quotient \cite{BaronCohenWheelwright2004_EmpathyQuotient} indexes the \emph{affective bandwidth} through which others become morally salient; the Systemizing Quotient \cite{BaronCohenRichlerBisaryaGurunathanWheelwright2003_SystemizingQuotient} captures the \emph{structural bias} shaping analytic interpretation; and the Big Five Inventory \cite{JohnDonahueKentle1991_BigFiveInventory, Rammstedt2007} maps broad \emph{personality curvature} influencing attention, norm-sensitivity, and regulatory control \cite{Barrick1991}. These variables do not exhaust the dispositional space, but they provide theoretically grounded coordinates on dimensions repeatedly implicated in moral appraisal and prosocial action \cite{Haidt2001EmotionalDog, Crockett2016Models}.

\medskip

\noindent
Their role in the analysis is therefore structural rather than decorative. Without them, dispositional heterogeneity would remain unmodelled, and any perturbation effect could be mistakenly attributed to uncontrolled trait variance. The cluster analysis of EQ, SQ, and BFI scores indeed revealed a non-trivial dispositional topology: affectively warm profiles, analytically structured profiles, and reactive–volatile profiles. Participants did not enter the experimental setting as psychologically interchangeable agents.

\medskip

\noindent
What matters is what came next. Despite this structured diversity, the humanoid robot produced a \emph{uniform directional attenuation} of prosocial behaviour across all dispositional clusters. No Big Five trait, EQ dimension, SQ factor, or latent profile moderated the effect. This finding parallels results in human--robot interaction showing that even passive robots can globally modulate social affordances, attentional allocation, and normative expectations \cite{Zlotowski2015, Malle2015, Komatsu2016}. The present study extends that literature by demonstrating that robotic presence does not selectively amplify or suppress particular traits. Instead, it acts on the \emph{evaluative field} itself: reshaping salience gradients, damping affective trajectories, and shifting the topology within which all trait-based pathways unfold.

\medskip

\noindent
The psychometric tools were indispensable for establishing this point. They allowed the analysis to dissociate the \emph{shape of the dispositional manifold} from the \emph{geometry of the perturbation}. In the experimental formalism, dispositional structure enters the perturbation comparison as
\[
f(\alpha_E, \beta_C, \gamma_R) - f(\alpha_E, \beta_C),
\]
a difference that isolates the contribution of $\gamma_R$ while holding $\beta_C$ fixed. The empirical pattern in Chapter~\ref{chap:exp_methods} showed that while $\beta_C$ exhibits a structured internal topology, the perturbation generated by the robot overwhelms trait-specific differences and applies a global deformation to the evaluative landscape. This is precisely the signature of a field-level operator rather than a trait-contingent stimulus.

\noindent
Thus, the purpose of this section is not simply to catalogue the tools, but to clarify their methodological necessity. They provide the coordinates required to demonstrate that the robot acted not on \emph{who} participants were, but on the evaluative conditions under which their moral trajectories unfolded. Psychometrics makes visible the dispositional substrate; the experiment reveals the topological deformation imposed upon it.

\noindent
With this distinction in place---between dispositional structure and field-level perturbation---we can now turn to a concise examination of the specific constructs measured by each instrument and how they map onto the evaluative--topological architecture of moral cognition.

%%FROM HERE%%%


\section{Why These Tools: Methodological Criteria and Alignment with the Thesis}

\noindent
Given the dual-layer structure revealed by the experiment—stable dispositional variation on one hand, and a field-level displacement induced by robotic presence on the other—the choice of psychometric and experimental instruments cannot be arbitrary. The tools selected here satisfy three methodological criteria that are essential for interpreting the attenuation of prosocial behaviour in a theoretically meaningful way.

\paragraph{(1) Cross-paradigmatic relevance.}
The EQ, SQ, BFI, and Watching--Eye paradigm each derive from long-standing empirical traditions spanning moral psychology, social cognition, personality research, and Human--Robot Interaction. Across these literatures, they have been used to study prosociality, empathic concern, harm aversion, cognitive style, and the integration of affective and deliberative processes in moral evaluation \cite{Haidt2001EmotionalDog, Greene2002, Greene2004, Cushman2013DualSystem, Crockett2016Models, Fedyk2017}.  

The Big Five Inventory remains the canonical operationalisation of broad personality architecture with well-established predictive value for behavioural outcomes \cite{JohnDonahueKentle1991_BigFiveInventory, Rammstedt2007, Barrick1991}. The Empathizing and Systemizing Quotients provide validated assessments of affective resonance and analytic curvature \cite{BaronCohenWheelwright2004_EmpathyQuotient, BaronCohenRichlerBisaryaGurunathanWheelwright2003_SystemizingQuotient}. Meanwhile, the Watching--Eye paradigm constitutes one of the most robust manipulations of prosocial salience, repeatedly demonstrating that minimal cues of observation modulate cooperative and charitable behaviour \cite{Haley2005, Bateson2006, Nettle2013, Conty2016, Dear2019}.  

Taken together, these instruments align the present study with a broad empirical landscape while remaining faithful to the evaluative--topological framework established earlier.

\paragraph{(2) Topological relevance.}
Each tool probes a structurally distinct component of the evaluative manifold:
\begin{itemize}
	\item \textbf{EQ}: the affective attractors anchoring early moral appraisal \cite{BaronCohenWheelwright2004_EmpathyQuotient};
	\item \textbf{SQ}: the curvature associated with analytic or rule-based processing \cite{BaronCohenRichlerBisaryaGurunathanWheelwright2003_SystemizingQuotient};
	\item \textbf{BFI}: the personality geometry modulating salience, attention, and regulatory control \cite{JohnDonahueKentle1991_BigFiveInventory, Rammstedt2007, Barrick1991};
	\item \textbf{Watching--Eye}: a validated perturbation of moral salience without instruction or coercion \cite{Haley2005, Bateson2006, Nettle2013, Conty2016, Dear2019}.
\end{itemize}

\noindent
This heterogeneity of scope provides the granularity needed to model the dispositional term $\beta_C$ in the formal expression
\[
\mathscr{P}(\delta_m) = f(\alpha_E, \beta_C, \gamma_R),
\]
and to cleanly distinguish trait-level variation from field-level perturbation.  
This distinction is the key empirical insight: robotic presence acted on the evaluative field rather than on personality-dependent gradients.

\paragraph{(3) Stability and interpretability.}
The selected instruments satisfy three further requirements:
\begin{itemize}
	\item \textbf{Stability}: each has robust psychometric validation across cultures and samples;
	\item \textbf{Analytical tractability}: each yields variables suitable for clustering, regression, and topological modelling;
	\item \textbf{Interpretability}: each connects to established moral-psychological and philosophical accounts, enabling behavioural findings to be integrated with theoretical models of moral appraisal.
\end{itemize}

\noindent
Most importantly, these tools provided the precision required to demonstrate that the attenuation effect was not driven by personality configurations, empathizing profiles, or systemizing tendencies. The psychometric suite revealed a structured dispositional landscape, but the robot altered behaviour \emph{irrespective} of that structure. The tools therefore allowed the experiment to distinguish \emph{who the participants were} from the \emph{geometry of the evaluative field} within which their choices were made.

\medskip

\noindent
With these criteria established, we now turn to the first measurement instrument: the Empathizing Quotient.

%% FROM HERE %%%

\section{The Empathizing Quotient (EQ): Affective Resonance as Evaluative Curvature}

\noindent
The Empathizing Quotient (EQ) provides a validated measure of affective resonance—an individual’s capacity to detect, register, and respond to the emotional and psychological states of others \cite{BaronCohenWheelwright2004_EmpathyQuotient}. Originally developed within the Empathizing--Systemizing framework \cite{Baron2002,Baron2009}, the EQ captures both emotional reactivity and cognitive perspective-taking, two mechanisms repeatedly shown to influence prosocial behaviour, harm aversion, and sensitivity to moral salience \cite{Haidt2001EmotionalDog,Greene2002,Cushman2013DualSystem,Crockett2016Models}.

\subsection*{Why EQ Matters Conceptually}
Within the evaluative--topological model developed in this thesis, empathizing corresponds to the \emph{affective curvature} of the evaluative field. High EQ scores indicate steep affective gradients: morally relevant others appear more salient, distress is more motivationally weighted, and the transition from appraisal to prosocial action becomes more strongly guided by affective dynamics. Low EQ profiles, by contrast, reflect flatter affective manifolds in which moral cues exert weaker pull.

\noindent
In this sense, the EQ is not merely a trait measure; it provides a quantitative coordinate for the dispositional term $\beta_C$ in the mapping
\[
\mathscr{P}(\delta_m) = f(\alpha_E, \beta_C, \gamma_R),
\]
where $\beta_C$ denotes the stable parameters shaping how evaluative information is transformed into behaviour.

\subsection*{Historical and Psychometric Grounding}
Empirically, the EQ has a robust record: strong internal reliability, stable factor structure across cultures \cite{Wakabayashi2006}, convergence with related constructs (empathic concern, emotional intelligence), and predictive validity for prosociality in behavioural economic tasks. Neurocognitive studies further show correlations between EQ scores and activation in vmPFC, anterior insula, and TPJ—regions central to affective resonance and mental-state attribution \cite{Moll2002,Decety2004}.

These features make the EQ particularly suitable for this thesis: it is theoretically interpretable, computationally tractable, and empirically grounded.

\subsection{EQ and Synthetic Presence}
The central scientific function of EQ in this experiment was to determine whether empathic sensitivity moderated the attenuation effect introduced by the humanoid robot. One plausible hypothesis, grounded in moral psychology and HRI, is that high-empathy individuals would exhibit stronger prosociality and possibly stronger perturbation under synthetic social cues \cite{Kuchenbrandt2011,Zlotowski2015}.

\noindent
The data ruled this out. EQ did \emph{not} moderate the displacement effect. High- and low-empathy participants alike showed reduced prosocial donation in the robot condition. This finding is theoretically decisive:

\begin{center}
	\emph{the robot altered the evaluative field itself, not the trait-dependent gradients within it.}
\end{center}

\noindent
This result aligns with evidence from HRI showing that robotic presence modulates attentional and social-evaluative processing independently of empathic predisposition \cite{Malle2015,Komatsu2016}.

\subsection{Methodological Role in the Thesis}
The EQ served two indispensable methodological purposes:

\begin{enumerate}
	\item \textbf{Controlling for affective heterogeneity}.  
	Without a measure of empathic sensitivity, reductions in donation could have been attributed to unmeasured differences in participants’ empathy levels. The EQ rules out this confound.
	
	\item \textbf{Modelling the affective dimension of $\beta_C$}.  
	EQ provides the affective coordinate of the dispositional manifold, enabling cluster analysis and regression models to distinguish dispositional shape from field-level perturbation.
\end{enumerate}

\noindent
Thus, even though affective resonance is central to moral cognition, the experiment shows that the perturbation introduced by the humanoid robot acted \emph{upstream} of empathy—altering the evaluative topology rather than amplifying or suppressing empathic traits.

\medskip

\noindent
With the affective dimension of $\beta_C$ established, we now turn to the analytical dimension: the Systemizing Quotient.


\section{The Systemizing Quotient (SQ): Structural Precision in the Evaluative Field}

\noindent
Where the Empathizing Quotient (EQ) indexes affective resonance, the Systemizing Quotient (SQ) \cite{Baron2003,Goldenfeld2005,Wakabayashi2006} quantifies a cognitive style characterised by rule extraction, structural analysis, and the search for causal regularities. Within the evaluative--topological model developed in this thesis, the SQ corresponds to the \emph{analytical curvature} of the evaluative field: the extent to which agents encode situations via stable structural relations rather than affective gradients.

\subsection{Theoretical Background and Psychometric Foundations}
The SQ emerged from the Empathizing--Systemizing framework \cite{Baron2002,Baron2009}, originally designed to capture the dissociability of affective versus rule-based processing in autism research. Subsequent work broadened this motivation: systemizing is now associated with mechanistic reasoning, pattern extraction, predictive modelling, and a preference for low-noise, high-coherence causal schemas \cite{Goldenfeld2005}. Psychometric studies demonstrate high internal reliability, cross-cultural robustness, and predictable correlations with analytic problem-solving and rule-consistent behaviour.

Neurocognitively, higher SQ scores correlate with lateral prefrontal and parietal activation during analytic reasoning; they are also associated with reduced activation in affective salience networks during social tasks \cite{Gleichgerrcht2013}. These findings support the interpretation of SQ as measuring a cognitive style that privileges structural stability over affective modulation.

\subsection{SQ Across Moral Psychology and HRI}
In moral psychology, systemizing predicts greater reliance on deliberative processing, reduced affective interference, and increased endorsement of principle-based judgments in high-conflict dilemmas \cite{Greene2014,Cushman2013DualSystem}. In behavioural economics, high-SQ individuals show more consistent strategic patterns and reduced susceptibility to framing effects.

In Human--Robot Interaction, systemizing tendencies shape expectations about synthetic agents: high-SQ participants tend to interpret robots through structural and functional cues rather than anthropomorphic ones and attribute competence and reliability more readily than emotional or social qualities \cite{Zlotowski2015,Malle2015,Komatsu2016}. This makes the SQ especially relevant in the present experiment, where the perturbation introduced by the robot is primarily structural rather than affective.

\subsection{SQ in the Evaluative--Topological Framework}
In the formalism of this thesis, SQ contributes to the dispositional term $\beta_C$ in
\[
\mathscr{P}(\delta_m) = f(\alpha_E, \beta_C, \gamma_R).
\]
Where EQ shapes the \emph{steepness} of affective gradients, SQ shapes the \emph{rigidity} and \emph{smoothness} of evaluative trajectories. High systemizing corresponds to a more stable evaluative surface: situations are encoded through structural invariants, making low-level affective perturbations less influential.

This intuition can be expressed heuristically through curvature:
\[
\nabla^2 V(x) \propto \text{SQ},
\]
where larger values indicate more rigid evaluative structures.

\subsection{SQ, Synthetic Presence, and Field-Level Perturbation}
Despite these theoretical expectations, the experiment showed that SQ did \emph{not} moderate the behavioural attenuation caused by the humanoid robot. High-SQ individuals—those most likely to rely on rule-based evaluation—displayed the same directional reduction in prosocial behaviour as high-empathy and low-empathy participants.

This finding is conceptually important. It demonstrates that robotic presence operated not on the cognitive style of participants but on the \emph{evaluative field itself}. Systemizing tendencies did not buffer, amplify, or redirect the behavioural effect.

\begin{center}
	\emph{The perturbation introduced by the robot was global, not trait-specific.}
\end{center}

This aligns with existing HRI work showing that ambiguous synthetic agents alter social affordances and attentional dynamics independently of analytic or empathic predispositions \cite{Zlotowski2015,Malle2015}.

\subsection{Methodological Significance}
SQ served two methodological functions within the experiment:

\begin{enumerate}
	\item \textbf{Controlling for cognitive style}.  
	Without an explicit measure of systemizing tendencies, attenuation could have been misattributed to analytic disposition rather than environmental perturbation.
	
	\item \textbf{Modelling the structural dimension of $\beta_C$}.  
	SQ provides the analytical coordinate within the dispositional manifold, enabling the analysis to distinguish dispositional geometry from field-level displacement.
\end{enumerate}

\noindent
Together with the EQ, the Systemizing Quotient ensures that dispositional structure is properly characterised before interpreting the behavioural impact of robotic presence. The next tool completes this picture: the Big Five Inventory, which captures broad personality geometry beyond empathy and systemizing.

% -----------------------------------------------------
\section{The Big Five Inventory (BFI): Personality Geometry and Evaluative Topology}

\noindent
The Big Five Inventory (BFI) is one of the most robust instruments in differential psychology. It distils decades of lexical and psychometric research into five broad, cross-culturally stable dimensions—Openness, Conscientiousness, Extraversion, Agreeableness, and Neuroticism \cite{John1999,McCraeCosta2008}. Within the evaluative--topological model introduced earlier, these traits provide a principled coordinate system for mapping the dispositional manifold ($\beta_C$): the stable personality geometry through which evaluative trajectories take shape.

\subsection{Why Personality Matters for This Thesis}
Personality traits function as attractors and modulators in behavioural space. They influence affective responsiveness, attentional allocation, social orientation, and regulatory stability—precisely the mechanisms identified in earlier chapters as constitutive of moral cognition. The BFI therefore allows us to characterise the dispositional background against which the perturbation introduced by the humanoid robot operates.

Crucially, the BFI offers the stability required for distinguishing dispositional structure from the field-level displacement effect observed in the experiment. Without a measure of trait geometry, we would lack the dimensional resolution necessary to determine whether the attenuation of prosocial behaviour reflected personality differences or a global perturbation of the evaluative field.

\subsection{Psychometric Strength and Cross-Domain Predictive Value}
The BFI is among the most widely validated trait measures in psychology. Its factor structure replicates across cultures; its items display strong internal reliability; and short forms such as the BFI-10 preserve psychometric clarity under experimental time constraints \cite{Rammstedt2007}. The Big Five dimensions predict a wide range of behavioural outcomes—social engagement, helping, rule adherence, and responsiveness to interpersonal cues \cite{Barrick1991,Graziano1996,Habashi2016}. These properties make the BFI an ideal tool for modelling $\beta_C$ within a topological framework concerned with how agents integrate contextual and affective information into action.

\subsection{Personality, Moral Behaviour, and Social Presence}
Each Big Five trait has theoretical relevance for moral behaviour:
\begin{itemize}
	\item \textbf{Agreeableness} steepens prosocial attractors and predicts cooperation, altruism, and sensitivity to interpersonal harm.
	\item \textbf{Conscientiousness} stabilises evaluative trajectories and supports rule-consistent behaviour.
	\item \textbf{Neuroticism} introduces volatility and heightens susceptibility to contextual variation.
	\item \textbf{Extraversion} amplifies responsiveness to social presence and perceived observation.
	\item \textbf{Openness} broadens contextual sampling and modulates tolerance for ambiguity.
\end{itemize}

In Human--Robot Interaction, these traits influence how artificial agents are perceived—whether as social entities, competent tools, or norm-relevant observers \cite{Zlotowski2015,Banks2020}. The BFI therefore ensures that personality-driven interpretations of robotic presence can be empirically tested rather than assumed.

\subsection{Personality Geometry in the Evaluative--Topological Model}
Within the formalism
\[
\mathscr{P}(\delta_m) = f(\alpha_E, \beta_C, \gamma_R),
\]
the BFI quantifies the geometry of $\beta_C$. Personality traits define the curvature, stability, and directionality of the evaluative field for each participant:

\begin{itemize}
	\item Agreeableness deepens altruistic basins.
	\item Conscientiousness smooths and stabilises evaluative gradients.
	\item Neuroticism increases local fluctuations.
	\item Extraversion amplifies social input channels.
	\item Openness expands contextual sensitivity.
\end{itemize}

These geometric interpretations allow personality to be formally integrated into the evaluative architecture without reducing behaviour to trait-level dispositions.

\subsection{Cluster Analysis: Making Personality Geometry Visible}
The cluster analysis (Chapter~\ref{chap:experimental_methods}) revealed three dispositional attractors:

\begin{enumerate}
	\item \textbf{Prosocial–Empathic}: high Agreeableness and high EQ; strong affective attractors.
	\item \textbf{Emotionally Reactive}: high Neuroticism; unstable gradients and high volatility.
	\item \textbf{Analytical–Structured}: high Conscientiousness and high SQ; rigid evaluative curvature.
\end{enumerate}

These clusters show that participants entered the experiment with \emph{structured dispositional diversity}. Psychological homogeneity cannot be assumed; it had to be measured.

\subsection{The Key Empirical Result: Uniform Displacement}
Despite these pronounced dispositional differences, the experiment revealed a striking result:

\begin{center}
	\textbf{The humanoid robot produced a uniform attenuation of prosocial behaviour across all clusters.}
\end{center}

No Big Five trait—and no cluster—moderated the effect.

This finding is decisive. It demonstrates that the perturbation introduced by the robot acts at the \emph{field level}. It reshapes the evaluative topology itself, not the trait-specific pathways that populate it. This aligns with work in HRI showing that robotic presence can shift perceived social affordances independently of personality \cite{Malle2015,Komatsu2016}.

\subsection{Methodological Significance}
The BFI provides the evidential basis for distinguishing between:

\begin{itemize}
	\item \textbf{dispositional geometry} (the shape of $\beta_C$), and
	\item \textbf{field-level deformation} induced by the robotic perturbation ($\gamma_R$).
\end{itemize}

Without the BFI, the attenuation could easily have been misinterpreted as a by-product of personality—differences in Agreeableness, Extraversion, or Neuroticism—rather than as a global shift in the evaluative field.

\subsection{Point of the Situation: What the BFI Shows}
At this stage in the book, the tools chapter reaches its central conclusion:

\begin{center}
	\textit{The personality manifold is structured, but robotic presence bends the evaluative field in a direction that does not depend on personality.}
\end{center}

The BFI demonstrates three essential achievements:

\begin{enumerate}
	\item It verifies that participants differ dispositionally in meaningful, theoretically interpretable ways.
	\item It anchors the cluster analysis that reveals the architecture of $\beta_C$.
	\item It proves that the behavioural attenuation is not trait-driven but topology-driven: a global deformation of evaluative structure induced by synthetic presence.
\end{enumerate}

\noindent
This completes the dispositional component of the evaluative–topological model and prepares the ground for the next chapter. Having established the geometry of $\beta_C$ and ruled out trait-based explanations, we can now turn to the design of the experimental perturbation itself: the Watching–Eye paradigm and the silent humanoid robot that reconfigures the evaluative field.

%%%%%%%%%%%%%%%%%%%%%%%
% FROM HERE NOW
%%%%%%%%%%%%%%%%%%%%%%

\section{The Watching--Eye Paradigm: Amplifying Moral Salience and Revealing Field-Level Deformation}

\noindent
Across behavioural ethics, social psychology, and field experiments on prosociality, one finding has proven remarkably robust: minimal cues of being observed—stylised eyes, schematic gaze, or even two black circles resembling pupils—reliably increase cooperation, charitable giving, and norm compliance \cite{Haley2005,Bateson2006,Nettle2013,Conty2016}. This ``watching--eye effect'' operates without instruction or coercion. It is a perturbation of the perceptual environment that increases the salience of norm-relevant behaviour.

Within the evaluative--topological framework developed earlier, watching-eye cues function as controlled amplifiers of moral salience: they steepen prosocial attractors in the evaluative field by increasing the perceived social meaning of one’s actions. This makes them the ideal baseline against which to detect whether synthetic presence deforms evaluative trajectories.

\subsection{Watching--Eye Cues as Topological Amplifiers}

\noindent
Classical interpretations framed the effect in terms of reputational vigilance: an implicit inference that one’s behaviour is observable and potentially judged by others \cite{Haley2005,Bateson2006}. More recent accounts show that the effect emerges from the coordinated modulation of:

\begin{itemize}
	\item \textbf{attentional uptake} of norm-relevant cues,
	\item \textbf{affective arousal} associated with evaluation or self-conscious emotions,
	\item \textbf{interpretive expectations} shaped by implicit social monitoring systems.
\end{itemize}

Formally, watching-eye cues operate on the environmental input term by increasing prosocial weighting:
\[
\alpha_E \mapsto \alpha_E + \delta\alpha_{\text{eye}}, \qquad \delta\alpha_{\text{eye}} > 0.
\]
This steepens the initial gradients through which intuitive appraisals evolve, making cooperative trajectories more accessible in the evaluative field.

\subsection{Why Child-Pair Eyes Provide a Clean Experimental Baseline}

\noindent
Child-eye posters are widely used in prosociality experiments because they combine perceptual sociality with minimal conceptual content. Decades of work demonstrate that stylised child eyes:

\begin{itemize}
	\item robustly increase prosocial behaviour across cultures and settings \cite{Haley2005,Bateson2006,ErnestJones2011,Ekstrom2012,Nettle2013};
	\item evoke empathic and care-based affective responses \cite{ThompsonBooth2014};
	\item amplify attentional vigilance without implying the presence of a moral agent \cite{Conty2016}.
\end{itemize}

This makes them ideal for experimental use. They provide a \emph{high-salience but low-interpretation} cue: strong enough to elevate prosocial gradients, simple enough not to introduce confounds involving mind attribution or intentionality.

\subsection{Why Synthetic Presence Dilutes or Distorts the Effect}

\noindent
The central theoretical claim of the thesis—that humanoid robots act as \emph{perturbation operators} on the evaluative field—becomes particularly clear when considering their interaction with watching-eye cues.

Humanoid robots are perceptually social but ontologically indeterminate. They are seen, but not reliably understood, as bearers of evaluative or moral capacities \cite{Zlotowski2015,Malle2015,Komatsu2016}. This ambiguity weakens all three mechanisms that normally support the watching-eye effect:

\begin{enumerate}
	\item \textbf{Reputational inference is unstable.} Robots rarely trigger the implicit assumption that one is being morally evaluated.
	\item \textbf{Affective resonance is dampened.} Observation by a non-agentive entity does not engage self-conscious emotions strongly.
	\item \textbf{Attentional cues conflict.} The perceptual system registers social presence; higher-order systems deny full agency.
\end{enumerate}

The result is a fractured evaluative landscape: the cue ``someone is watching'' is present at the perceptual level, but stripped of the evaluative force that normally steepens prosocial attractors.

\subsection{Empirical Finding: Uniform Attenuation of the Watching--Eye Effect}

\noindent
The experiment confirms this prediction:

\begin{quote}
	\emph{The presence of a humanoid robot uniformly attenuated the watching--eye effect across all dispositional clusters.}
\end{quote}

Even participants with traits associated with high social sensitivity (Agreeableness, Extraversion, EQ) showed the same directional decrease in prosocial behaviour. Formally:
\[
(\alpha_E + \delta\alpha_{\text{eye}}) \mapsto (\alpha_E + \delta\alpha_{\text{eye}}) - \Delta_{\mathscr{R}},
\]
where $\Delta_{\mathscr{R}}$ is a field-level displacement induced by the robot. The absence of moderation by EQ, SQ, or any BFI trait demonstrates that this displacement operates independently of dispositional geometry.

\subsection{Why the Watching--Eye Paradigm Is Indispensable}

\noindent
For the purposes of this thesis, the watching-eye paradigm serves four methodological functions:

\begin{itemize}
	\item \textbf{It provides a reliable high-salience baseline} against which attenuation can be detected.
	\item \textbf{It links the experiment to established moral psychology}, enabling direct comparison with decades of prosociality research.
	\item \textbf{It isolates genuine perturbation effects}, since attenuation can only occur if salience is first elevated.
	\item \textbf{It reveals the topology of moral cognition}, showing how synthetic presence deforms evaluative gradients rather than simply reducing generosity.
\end{itemize}

Without this paradigm, the behavioural shift could not be interpreted as a deformation of the evaluative field.

\subsection{Integration With Costly Prosocial Action}

\noindent
Donation tasks provide observable moral action rather than abstract moral judgment. Their integration with watching-eye cues allows the experiment to follow the evaluative trajectory from:

\begin{enumerate}
	\item cue uptake, to
	\item salience amplification, to
	\item action selection.
\end{enumerate}

The robot’s attenuation of this sequence demonstrates that synthetic agents alter the mapping from perceptual cues to moral behaviour.

\subsection{Synthesis: A Window Into Moral Topology}

\noindent
The watching-eye paradigm serves as a conceptual and methodological hinge in the experiment. By steepening prosocial gradients, it makes the evaluative field’s structure visible. By attenuating these gradients, the humanoid robot reveals the central result of the thesis:

\begin{quote}
	\textbf{Synthetic presence acts on the evaluative field itself rather than on personality-dependent pathways.}
\end{quote}

\noindent
The watching-eye effect therefore provides the diagnostic contrast needed to show how robotic co-presence deforms the topology of moral cognition.


\section{General Conclusion: Measurement as the Logic of Synthetic Moral Perturbation}

\noindent
This chapter has developed far more than a list of instruments. It has established the 
measurement logic of the entire thesis: the conceptual grammar through which synthetic 
presence becomes empirically legible. The Empathizing Quotient (EQ), Systemizing Quotient 
(SQ), Big Five Inventory (BFI), and the Watching--Eye paradigm form a unified system of 
epistemic probes. Each is theoretically grounded, psychologically validated, and 
methodologically indispensable for making the evaluative topology of moral cognition 
observable without reducing it to caricature.

\medskip

\noindent
The formal architecture introduced earlier models moral behaviour as the output of a 
mapping
\[
\mathscr{P}(\delta_m) = f(\alpha_E, \beta_C, \gamma_R),
\]
where:
\begin{itemize}
	\item $\alpha_E$ captures the structure of environmental moral cues;
	\item $\beta_C$ denotes the dispositional manifold shaping evaluative uptake;
	\item $\gamma_R$ represents the perturbational operator introduced by synthetic presence.
\end{itemize}

\noindent
Each measurement tool corresponds to a distinct component of this model:

\begin{itemize}
	\item \textbf{EQ} probes the affective attractors of $\beta_C$: the steepness, reach, 
	and accessibility of prosocial gradients.
	\item \textbf{SQ} probes the structural curvature of $\beta_C$: the deliberative 
	rigidity, rule-coherence, and model-based stability of evaluative trajectories.
	\item \textbf{BFI} provides the \emph{coordinate system} of $\beta_C$: the multi-dimensional 
	personality geometry needed to identify dispositional clusters and map their evaluative 
	signatures.
	\item \textbf{Watching--Eye cues} perturb $\alpha_E$: they steepen prosocial gradients 
	and thereby create the diagnostic contrast necessary to observe displacement by $\gamma_R$.
\end{itemize}

\noindent
Taken together, these tools do not simply ``measure variables.''  
They give the experiment an \emph{evaluative topology}—a structured moral landscape within 
which deformation can be detected, described, and interpreted.

\subsection{Dispositional Mapping: A Structured Manifold, Not a Confound}

\noindent
A major contribution of this chapter is the demonstration that dispositional diversity is 
structured, measurable, and separable from perturbational effects. The cluster analysis 
derived from EQ, SQ, and BFI revealed three dispositional attractor types—Affective--Prosocial, 
Emotionally Reactive, and Analytical--Structured—each characterised by distinct evaluative 
curvature.

\noindent
Yet the experiment showed a striking and theoretically decisive pattern:

\begin{center}
	\emph{The humanoid robot attenuated prosocial action uniformly across all clusters.}
\end{center}

\noindent
This is a non-trivial finding. It rules out trait-level explanations—agreeableness, 
extraversion, empathy, systemizing, emotional volatility—as proximate drivers of the 
attenuation. The dispositional manifold $\beta_C$ was not the site of modulation.

\emph{Instead, the perturbation operated at the level of the evaluative field itself.}

Without the psychometric tools, this inference would have been impossible: the attenuation 
could have been misread as personality noise rather than as a genuine deformation of moral 
topology.

\subsection{Watching--Eye Cues as Diagnostic Amplifiers}

\noindent
The Watching--Eye paradigm provided the complementary half of the measurement logic.  
By steepening prosocial attractors in $\alpha_E$, it created the high-salience baseline 
against which synthetic attenuation became visible. The robot’s presence did not merely 
reduce generosity—it \emph{neutralised a well-established amplifier of moral salience}.  
This interaction is the clearest empirical signature of field-level perturbation.

\noindent
In theoretical terms, the eyes amplified the gradient; the robot deformed the landscape.

Only the combination of psychometric mapping (of $\beta_C$) and salience amplification (of 
$\alpha_E$) allowed this deformation to be isolated as an operation of $\gamma_R$.

\subsection{Philosophical and Ethical Meaning}

\noindent
Placed in dialogue with the philosophical frameworks introduced earlier, the tools reveal 
the following:

\begin{itemize}
	\item \textbf{Against rationalist models:} the perturbation bypasses deliberation.
	\item \textbf{Against virtue-theoretic accounts:} stable dispositions do not shield 
	agents from synthetic deformation.
	\item \textbf{Against sentimentalist explanations alone:} the effect persists even in 
	high-empathic profiles.
	\item \textbf{Against Machine Ethics assumptions:} moral significance lies not in the 
	agent (robot) but in the environment the agent reshapes.
\end{itemize}

\noindent
The instruments thus do double philosophical work:  
they expose the mechanisms through which moral action is generated, and they reveal the 
conceptual blind spots in contemporary ethical thinking about artificial agents.

Synthetic presence does not simply ``influence'' behaviour; it refracts the geometry through 
which moral meaning becomes action. It is neither a moral agent nor merely a tool. It is a 
\emph{moral perturbator}: an entity capable of bending the evaluative field.

\subsection{Methodological Synthesis: The Tools as Epistemic Infrastructure}

\noindent
This chapter has constructed the epistemic infrastructure required for the experiment.  
It has shown that:

\begin{enumerate}
	\item moral behaviour can only be interpreted through a model that distinguishes 
	environmental cues, dispositional structure, and perturbational operators;
	\item psychometric tools provide the resolution needed to map $\beta_C$ precisely 
	enough to rule out trait-based explanations;
	\item observational cues provide the experimental leverage needed to manipulate 
	$\alpha_E$ in a controlled and theoretically meaningful manner;
	\item synthetic presence must therefore be analysed as a deformation of 
	\emph{evaluative topology}, not as a stimulus acting upon isolated traits.
\end{enumerate}

\noindent
In this sense, the tools are not auxiliary components of the experiment—they are the 
\emph{conditions of intelligibility} for its results.

\subsection{Transition to the Experimental Methods}

\noindent
The next chapter operationalises everything established here.  
It translates the theoretical variables into stimuli, tasks, and statistical models. It 
describes how psychometric instruments were administered, how salience modulation was 
implemented, how synthetic presence was introduced, and how evaluative deformation was 
quantified.

\begin{center}
	\textit{The tools provide the coordinates; the experiment traces the trajectory.}
\end{center}

\noindent
The question that now motivates the remainder of the thesis is precise:

\begin{quote}
	\emph{Does synthetic presence reshape the evaluative field through which moral 
		salience becomes action?}
\end{quote}

\noindent
The methodological architecture developed in this chapter ensures that the experiment can 
answer this question with conceptual clarity, empirical rigor, and philosophical depth.
