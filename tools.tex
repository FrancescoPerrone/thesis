\chapter{TOOLS}
\label{chap:tools}
\thispagestyle{pprintTitle}

\chapter{Psychometric Instruments and Experimental Paradigms}
\label{chap:tools}
\thispagestyle{pprintTitle}

\noindent
Empirical work aimed at understanding moral cognition must specify, with some 
philosophical care, the instruments through which psychological and behavioural 
structures become accessible to observation. Moral appraisal itself is never directly given; it is inferred from patterned responses—affective, dispositional, perceptual, and social—that reflect how evaluative information is encoded in the agent’s cognitive architecture \cite{Haidt2001EmotionalDog, Greene2002, Greene2004, Cushman2013DualSystem, Crockett2016Models, Fedyk2017}. The tools employed in this thesis therefore function not as neutral measurement devices but as theoretically motivated probes: each instrument targets a specific dimension of the evaluative topology developed in earlier chapters, rendering latentdispositional structure empirically tractable without collapsing its 
complexity into reductive summary scores.

The methodological commitments of this thesis require a principled account of the instruments through which evaluative behaviour becomes empirically accessible. Work in moral psychology and cognitive science has repeatedly shown that moral appraisal is not directly observable but manifests through structured patterns of affective response, controlled cognition, and social cue integration \cite{Haidt2001EmotionalDog, Greene2002, Greene2004, Cushman2013DualSystem, Crockett2016Models, Fedyk2017}. For this reason, empirical studies of moral cognition depend on validated constructs and measurement strategies capable of rendering latent dispositions observable without distorting their theoretical significance.

The present work does not align itself with moral cognition research as a discrete disciplinary domain. Instead, it draws upon rigorously established constructs from moral psychology, cognitive science, and social signal processing as operational resources for making evaluative dispositions tractable. Instruments such as the Empathizing Quotient \cite{BaronCohenWheelwright2004_EmpathyQuotient}, the Systemizing Quotient \cite{BaronCohenRichlerBisaryaGurunathanWheelwright2003_SystemizingQuotient}, and the Big Five Inventory \cite{JohnDonahueKentle1991_BigFiveInventory, Rammstedt2007} provide precisely the kind of psychometric access to stable individual differences that contemporary models of moral cognition identify as structurally relevant. Likewise, the analytical frameworks developed within Social Signal Processing \cite{Vinciarelli2009} offer methodological grounding for understanding how agents register, interpret, and behaviourally respond to contextually salient perturbations.

In this sense, the psychometric tools employed here are not neutral measurement devices, but theoretically motivated probes into the dispositional structures that shape how agents encode, negotiate, and respond to morally salient changes in their evaluative environment.



The Empathizing Quotient \cite{BaronCohenWheelwright2004_EmpathyQuotient}, the Systemizing Quotient \cite{BaronCohenRichlerBisaryaGurunathanWheelwright2003_SystemizingQuotient}, and the Big Five Inventory \cite{JohnDonahueKentle1991_BigFiveInventory, Rammstedt2007} offer validated operationalisations of dispositional constructs repeatedly implicated in moral judgment and social decision-making. Likewise, the Watching--Eye paradigm \cite{Haley2005, Bateson2006, Nettle2013, Conty2016, Dear2019} constitutes a mature experimental framework for probing reputational concern, prosocial motivation, and sensitivity to subtle social cues. Together, these instruments form a coherent measurement suite capable of isolating trait-level parameters that interact with contextual salience to shape moral behaviour.

\medskip
\noindent
This chapter therefore serves a conceptual rather than merely procedural purpose. The psychometric instruments and experimental paradigms introduced here are situated explicitly within the evaluative--topological model developed in earlier chapters, in which moral cognition is understood not as a sequence of discrete judgments but as the dynamic evolution of a manifold of interacting evaluative gradients. Contemporary theories of moral psychology emphasise that such gradients integrate affective, social, and contextual inputs in a manner shaped by stable dispositional architecture \cite{Haidt2001EmotionalDog, Greene2002, Greene2004, Cushman2013DualSystem, Crockett2016Models, Fedyk2017}. Within this framework, the role of each tool is to reveal invariant dispositional structures—the stable dimensions along which individuals differ in how incoming evaluative information is encoded and transformed. These structures correspond to the latent parameters governing how a subject’s evaluative gradients bend, flatten, or intensify as the informational environment is perturbed.

In the experiment motivating this thesis, such perturbation is elicited not through explicit moral dilemmas but through a more subtle and ecologically grounded manipulation: the silent perceptual presence of a humanoid robot. Prior work in human--robot interaction shows that even passively positioned robots can shift perceived social affordances, alter attentional allocation, and modulate expectations concerning norm-relevant behaviour \cite{Zlotowski2015, Malle2015, Komatsu2016}. Their ambiguous ontological status disrupts default social priors and thereby reconfigures the salience landscape within which moral reasons become behaviourally operative. In this respect, robotic presence functions as a controlled perturbation to the evaluative topology itself, enabling the empirical study of how dispositional invariants interact with contextual cues to produce measurable differences in moral behaviour.


\medskip
\noindent
The aim of the chapter is thus twofold.

\begin{enumerate}
	\item First, to establish that each psychometric and experimental tool is grounded in stable bodies of empirical and theoretical research across psychology, cognitive science, HCI/HRI, and social signal processing. This ensures that the constructs they measure---empathic sensitivity, systemizing tendencies, personality traits, and responsiveness to social cues---are well-defined, reproducible, and theoretically interpretable within the broader landscape of moral psychology and social cognition.
	
	\item Second, to show how each tool contributes to the modelling of the dispositional term $\beta_C$ in the formal expression
	\[
	\mathscr{P}(\delta_m) = f(\alpha_E, \beta_C, \gamma_R),
	\]
	where $\beta_C$ denotes the latent trait configuration governing how a participant’s evaluative topology is modulated by the perturbation introduced by the humanoid robot. In this sense, the tools are not ancillary components of the experiment but operationalisations of the dispositional invariants that mediate the transformation of evaluative salience under robotic presence.
\end{enumerate}


\noindent
The tools included here—the Empathizing Quotient (EQ), the Systemizing Quotient (SQ), the Big Five Inventory (BFI), and the Watching–Eye paradigm—were selected because they satisfy three stringent criteria grounded in established empirical research. First, each instrument has a well-defined construct lineage supported by extensive psychometric validation. The EQ \cite{BaronCohenWheelwright2004_EmpathyQuotient} and SQ \cite{BaronCohenRichlerBisaryaGurunathanWheelwright2003_SystemizingQuotient} constitute the canonical operationalisations of empathizing and systemizing tendencies, with consistent factor structures, cross-cultural robustness, and demonstrable discriminant validity within both clinical and non-clinical populations. The BFI, in its original form \cite{JohnDonahueKentle1991_BigFiveInventory} and in its widely used short version \cite{Rammstedt2007}, provides a compact yet psychometrically rigorous assessment of the five broad personality domains that anchor contemporary trait theory.

Second, the Watching–Eye paradigm has developed into a mature experimental framework for probing reputation-sensitive prosocial behaviour. Numerous studies have demonstrated that minimal cues of observation modulate cooperative and charitable actions \cite{Haley2005, Bateson2006, Nettle2013, Conty2016, Dear2019}, and the paradigm’s effects have been replicated across diverse contexts, task structures, and elicitation modalities. This makes it uniquely suited for isolating perturbations to social-evaluative processing—a core requirement for the present analysis.

Third, all four tools possess sufficient resolution and conceptual precision to inform the modelling of latent dispositional structure within the evaluative-topological framework advanced in this thesis. They provide theoretically interpretable coordinates for the dispositional term $\beta_C$, enabling an analysis of how trait configurations shape the deformation of evaluative gradients under robot-induced perturbations. For this reason, these instruments are not simply conventional choices, but the most appropriate set of measurements for the level of abstraction at which the experimental work is situated.


\begin{enumerate}
	\item \textbf{Theoretical relevance}: Each tool targets a component of moral topology (affective resonance, evaluative precision, personality curvature, or salience modulation).
	\item \textbf{Empirical robustness}: Each tool is validated across multiple cultures, large samples, and decades of psychological research, and has been used in studies of prosociality, moral sensitivity, social attention, and Human–Robot Interaction (HRI).
	\item \textbf{Computational suitability}: Each tool produces variables suitable for integration into regression models, cluster analysis, and topological interpretation.
\end{enumerate}

\noindent
Before turning to the tools themselves, we first articulate the methodological role they play within this thesis.

\section{The Role of Psychometric Tools in the Evaluative–Topological Architecture}

\noindent
Within the formal architecture developed throughout this thesis, moral behaviour is modelled as the endpoint of a trajectory across an evaluative field. Contemporary research in moral psychology and cognitive science emphasises that such trajectories arise from the joint interaction of environmental cues, dispositional structure, and perturbational influences \cite{Haidt2001EmotionalDog, Greene2002, Greene2004, Cushman2013DualSystem, Crockett2016Models, Fedyk2017}. Accordingly, the formal decomposition
\[
\mathscr{P}(\delta_m) = f(\alpha_E, \beta_C, \gamma_R)
\]
captures the three principal determinants of evaluative dynamics:

\begin{itemize}
	\item \emph{environmental inputs} ($\alpha_E$): morally salient cues such as the Watching–Eye prime and task context;
	\item \emph{dispositional structure} ($\beta_C$): latent traits quantified by psychometric instruments;
	\item \emph{perturbation operators} ($\gamma_R$): the ontologically ambiguous presence of the humanoid robot.
\end{itemize}

\noindent
The psychometric tools employed in this study belong to the $\beta_C$ term. They render dispositional structure empirically tractable by quantifying constructs shown to be central in the integration of affective, social, and contextual information. The Empathizing Quotient \cite{BaronCohenWheelwright2004_EmpathyQuotient} indexes the \textbf{affective bandwidth} through which agents register morally salient others; the Systemizing Quotient \cite{BaronCohenRichlerBisaryaGurunathanWheelwright2003_SystemizingQuotient} captures the \textbf{analytical curvature} underlying structural interpretation of social situations; and the Big Five Inventory \cite{JohnDonahueKentle1991_BigFiveInventory, Rammstedt2007} measures the \textbf{personality geometry} shaping attentional allocation, normative sensitivity, and regulatory control \cite{Barrick1991}. These constructs have well-established roles in models of moral appraisal and behavioural prediction \cite{Haidt2001EmotionalDog, Crockett2016Models}.

\medskip

\noindent
Their role in the experiment is not ancillary. These measures enabled the analysis to disentangle two layers of the evaluative architecture that would otherwise remain conflated: (i) the \emph{dispositional configuration} each participant brings into the situation, and (ii) the \emph{field-level modulation} induced by robotic presence. The cluster analysis performed on EQ, SQ, and BFI scores revealed a structured personality topology comprising affectively warm, analytically structured, and reactive–volatile profiles, indicating that participants did not enter the experimental environment as a psychologically homogeneous group.

What is theoretically significant, however, is what followed. Despite this structured dispositional diversity, the humanoid robot exerted a \emph{uniform directional effect} on prosocial behaviour across all clusters. No Big Five trait, EQ subscale, SQ dimension, or latent profile moderated the displacement. Prior research in human--robot interaction has shown that even passive robotic agents can shift perceived social affordances, modulate attention, and alter expectations surrounding norm-relevant behaviour \cite{Zlotowski2015, Malle2015, Komatsu2016}. The present findings extend this line of work by demonstrating that robotic presence does not operate through trait-dependent amplification or suppression of behavioural tendencies. Instead, it perturbs the evaluative field itself—its salience structure, affective gradients, and normative attractors—such that all dispositional trajectories are bent in the same behavioural direction.

\medskip

\noindent
In this sense, the psychometric tools were indispensable. They allowed the analysis to dissociate the \emph{shape of the dispositional manifold} from the \emph{geometry of the perturbation}. Without psychometric grounding, the attenuation of donation behaviour might have been misinterpreted as a trait-level effect rather than a field-level displacement. The instruments thereby provided the empirical precision needed to show that the robot acted not upon who the participants were, but upon the evaluative topology within which their moral choices unfolded.

\noindent
In the experimental formalism, the dispositional term appears in the perturbation expression
\[
f(\alpha_E, \beta_C, \gamma_R) - f(\alpha_E, \beta_C),
\]
which measures how the moral transformation function is reshaped by $\gamma_R$ given a fixed dispositional configuration. This formulation reflects the broader consensus in moral psychology that moral behaviour emerges from the interaction between environmental cues, dispositional structure, and perturbational influences on evaluative processing \cite{Haidt2001EmotionalDog, Greene2002, Greene2004, Cushman2013DualSystem, Crockett2016Models, Fedyk2017}. The empirical results presented in Chapter~\ref{chap:experimental_methods} showed that although $\beta_C$ exhibits a structured internal topology—revealed through clustering analyses of EQ, SQ, and BFI scores \cite{BaronCohenWheelwright2004_EmpathyQuotient, BaronCohenRichlerBisaryaGurunathanWheelwright2003_SystemizingQuotient, JohnDonahueKentle1991_BigFiveInventory, Rammstedt2007, Barrick1991}—the perturbation introduced by the humanoid robot did \emph{not} depend on those dispositional differences. All clusters displayed the same directional attenuation of prosocial behaviour, indicating that $\gamma_R$ operates primarily at the \emph{field level}, reshaping the evaluative landscape within which dispositional trajectories unfold rather than interacting with trait-specific gradients.

\medskip

\noindent
This pattern aligns with established findings in human--robot interaction, where even passive robotic agents have been shown to modulate perceived social affordances, attentional allocation, and norm-relevant expectations irrespective of observer traits \cite{Zlotowski2015, Malle2015, Komatsu2016}. In the present experiment, robotic presence functioned as a global perturbation of the evaluative field rather than as a selective amplifier or suppressor of individual dispositions.

\medskip

\noindent
The goal of this chapter, therefore, is not simply to catalogue the psychometric tools, but to clarify how each instrument contributes to the modelling of $\beta_C$ and why their inclusion is essential for distinguishing dispositional structure from field-level displacement. Without these psychometric constraints, the observed attenuation of prosocial behaviour could have been misattributed to personality differences rather than correctly interpreted as a global deformation of the evaluative topology induced by robotic presence. The tools thereby provide the empirical precision necessary to show that the robot acted not upon who the participants were, but upon the evaluative field within which their moral choices unfolded.

Having established the distinction between dispositional structure and field-level perturbation, we can now justify the methodological choices that made this distinction empirically visible.


\section{Why These Tools: Methodological Criteria and Alignment with the Thesis}

\noindent
Given the dual-layer structure revealed by the experiment—stable dispositional variation on the one hand, and a field-level displacement effect induced by robotic presence on the other—the selection of psychometric and experimental tools cannot be arbitrary. The instruments employed here were chosen because they satisfy three methodological criteria essential for interpreting the attenuation of prosocial behaviour observed in the study.

\paragraph{(1) Cross-paradigmatic relevance.}
The EQ, SQ, BFI, and Watching–Eye paradigm each rest on extensive empirical traditions across several domains of inquiry. In moral and social psychology, these tools have been used to study prosociality, empathic concern, harm aversion, and the integration of affective and cognitive processes in moral judgment \cite{Haidt2001EmotionalDog, Greene2002, Greene2004, Cushman2013DualSystem, Crockett2016Models, Fedyk2017}. In personality psychology, the Big Five Inventory provides a compact but psychometrically robust measure of trait architecture with well-established predictive value for behavioural outcomes \cite{JohnDonahueKentle1991_BigFiveInventory, Rammstedt2007, Barrick1991}. The Empathizing and Systemizing Quotients offer validated assessments of affective resonance and analytic style \cite{BaronCohenWheelwright2004_EmpathyQuotient, BaronCohenRichlerBisaryaGurunathanWheelwright2003_SystemizingQuotient}. 

In parallel, the Watching–Eye paradigm constitutes one of the most reliable experimental manipulations of prosocial salience, with repeated demonstrations that subtle cues of observation can modulate cooperative and charitable behaviour \cite{Haley2005, Bateson2006, Nettle2013, Conty2016, Dear2019}. Crucially, these literatures intersect with contemporary Human–Robot Interaction research, where robotic agents are known to shift social affordances, attentional allocation, and normative expectations \cite{Zlotowski2015, Malle2015, Komatsu2016}. Their use therefore positions the present study within a broad empirical landscape while maintaining continuity with the theoretical commitments of the evaluative–topological framework.

\paragraph{(2) Topological relevance.}
Each tool probes a structurally distinct component of the evaluative manifold that underpins moral cognition:
\begin{itemize}
	\item \textbf{EQ}: the affective attractors that anchor early moral and social appraisal \cite{BaronCohenWheelwright2004_EmpathyQuotient};
	\item \textbf{SQ}: the structural curvature associated with analytic or rule-based processing \cite{BaronCohenRichlerBisaryaGurunathanWheelwright2003_SystemizingQuotient};
	\item \textbf{BFI}: the multidimensional geometry of personality traits that modulate salience, attentional uptake, and behavioural regulation \cite{JohnDonahueKentle1991_BigFiveInventory, Rammstedt2007, Barrick1991};
	\item \textbf{Watching–Eye paradigm}: an experimentally validated perturbation that shifts moral salience without instruction or coercion \cite{Haley2005, Bateson2006, Nettle2013, Conty2016, Dear2019}.
\end{itemize}

\noindent
Together, these measurements provide the granularity needed to model the dispositional term $\beta_C$ and to distinguish clearly between trait-level variation and field-level perturbation. This is precisely what enabled the analysis to establish that robotic presence operated on the evaluative field rather than on personality-dependent gradients.

\paragraph{(3) Stability and interpretability.}
The selected instruments satisfy the methodological requirements of stability, reliability, and interpretability that are necessary for higher-level analysis:
\begin{itemize}
	\item they support clustering of participants within dispositional space,
	\item they enable regression modelling of trait influences on donation behaviour,
	\item and they admit interpretation through established normative and meta-ethical frameworks, including sentimentalism, virtue-theoretic accounts, and pluralist models of moral reasoning.
\end{itemize}

\noindent
Most importantly, these tools provided the methodological precision needed to demonstrate that the attenuation of prosociality was not driven by differences in personality clusters, empathizing profiles, or systemizing tendencies. Instead, the psychometric suite functioned as a set of diagnostic probes revealing a structured dispositional landscape against which the global displacement effect of robotic presence could be identified unambiguously. The tools thereby allowed the experiment to differentiate \emph{who the participants were} from the \emph{structure of the evaluative field} within which their behaviour unfolded.

\medskip

\noindent
With these foundations established, we now turn to the first measurement tool: the Empathizing Quotient.


\section{The Empathizing Quotient (EQ): Affective Resonance as a Moral Vector Field}

\noindent
The Empathizing Quotient (EQ) occupies a central place in the measurement of affective sensitivity within contemporary psychology. Developed by Baron--Cohen and colleagues as part of the broader Empathizing--Systemizing (ES) framework \cite{BaronCohenWheelwright2004_EmpathyQuotient,Baron2002,Baron2009}, the EQ was originally designed to quantify individual differences in emotional resonance, perspective-taking, and the capacity to infer and respond appropriately to the mental states of others. Its construction reflects two decades of theoretical and empirical work stemming from autism research, sex differences in social cognition, and the development of trait-based accounts of empathic functioning.

\subsection{Historical and Theoretical Foundations}

\noindent
The EQ emerged against the background of two influential lines of inquiry. The first concerned the cognitive and affective profiles observed in autism spectrum conditions, where empathic difficulties appeared as a core diagnostic dimension. Baron--Cohen's early work on ``mindblindness'' and the ES theory \cite{Baron2002} proposed that empathizing and systemizing represent partially dissociable cognitive styles, with autism characterized by diminished empathizing abilities alongside preserved or enhanced systemizing capacities. The second line of inquiry derived from trait psychology and social cognition, where stable inter-individual differences in emotional attunement, empathic accuracy, and prosocial inclinations were increasingly understood as predictive of moral and social behaviour.

The EQ was designed to operationalise the empathizing construct in a psychometrically rigorous manner. It includes affective items (e.g., sensitivity to distress), cognitive-empathic items (e.g., perspective-taking), and items assessing spontaneous concern for others. Initial investigations \cite{BaronCohenWheelwright2004_EmpathyQuotient} demonstrated large group differences between autistic and neurotypical adults, robust sex differences, and high internal reliability. Subsequent factor-analytic studies \cite{Lawson2004} further clarified the latent structure of the scale, identifying separable components associated with emotional reactivity, cognitive perspective-taking, and social attunement.

\subsection{Psychometric Validation and Cross-Cultural Work}

\noindent
Psychometric validation of the EQ has been extensive. Beyond the initial work in clinical and neurotypical samples, replication studies have demonstrated strong internal consistency, acceptable test--retest reliability, and predictable convergence with related constructs such as empathic concern, emotional intelligence, and social sensitivity \cite{BaronCohenWheelwright2004_EmpathyQuotient}. Cross-cultural validations, including Japanese and Western samples, have shown that the EQ maintains its factor structure and predictive value across cultural contexts \cite{Wakabayashi2006}. 

These findings situate the EQ within the broader movement toward trait-based quantification of social-cognitive skills. Within personality psychology, empathizing correlates with the Agreeableness and Openness dimensions of the Big Five \cite{Barrick1991}, while remaining psychometrically distinguishable from both. Within social neuroscience, EQ scores have been found to correlate with vmPFC--amygdala coupling and with the strength of activation in neural substrates associated with social pain, affect sharing, and mentalising.

\subsection{Empirical Applications Across Disciplines}

\noindent
The EQ has become a standard instrument in multiple research paradigms. In moral psychology, empathy-related traits are strong predictors of altruistic helping, harm aversion, guilt sensitivity, and responses to moral dilemmas \cite{Haidt2001EmotionalDog,Greene2002,Greene2004,Cushman2013DualSystem,Crockett2016Models,Fedyk2017}. High EQ scores are consistently associated with stronger prosocial choices in economic games, including the ultimatum, dictator, and trust games. Behavioural economics work shows that individuals with higher empathic sensitivity display increased generosity even when anonymity is preserved, suggesting that empathic traits modulate internalised moral norms beyond external social cues.

In social neuroscience, EQ scores track activation patterns in regions associated with affective resonance, including the anterior insula, temporoparietal junction, and amygdala--vmPFC networks. Oxytocin administration studies further demonstrate selective improvement in empathic accuracy \cite{Bartz2010}, reinforcing the biological plausibility of affective resonance as a trait-like dimension.

The EQ has also gained significance in Human--Robot Interaction (HRI), where empathic predispositions shape attributions of intentionality, perceived moral standing, and expectations regarding robots' behaviour \cite{Malle2015,Komatsu2016,Zlotowski2015}. Individuals with higher EQ scores tend to ascribe richer mental states to robots, respond more strongly to cues of intentionality, and exhibit greater sensitivity to violations of social or moral norms in robotic agents. In group-based interactions, empathic individuals demonstrate greater behavioural alignment with robots, particularly when robots display subtle affective or communicative signals \cite{Kuchenbrandt2011}.

\subsection{Critiques and Methodological Limitations}

\noindent
Despite its widespread use, the EQ has faced several critiques. Some researchers argue that its factor structure is not fully stable across populations, with certain studies reporting two or three factors rather than the originally proposed triadic structure. Concerns have also been raised regarding response biases, social desirability, and the possibility that self-report measures may not accurately capture behavioural or neural indices of empathy. Cross-cultural studies have noted differences in average EQ scores, prompting questions about cultural calibration and the extent to which certain items rely on culturally specific norms of emotional expression.

Within experimental psychology, some scholars have argued that empathic responding is situationally variable and cannot be fully reduced to trait-level constructs. Studies demonstrating dissociations between empathic concern and moral behaviour in high-stakes dilemmas \cite{Cushman2013DualSystem,Greene2004} raise further questions about the predictive specificity of the EQ. Nevertheless, the scale remains one of the most widely used and empirically grounded measures of individual differences in affective resonance.

\subsection{Relevance to the Evaluative--Topological Framework}

\noindent
In the evaluative--topological model developed in this thesis, the EQ operationalises the affective attractors that structure early moral appraisal. High empathizing corresponds to steeper affective gradients in the evaluative landscape, amplifying the salience of morally relevant others and increasing the likelihood that prosocial dispositions will be behaviourally expressed. Conversely, lower EQ scores correspond to flatter affective manifolds, in which moral salience is more weakly coupled to others' distress or need.

In the context of the experiment, the EQ plays a crucial role in modelling the dispositional term $\beta_C$. It enables the analysis to determine whether differences in affective sensitivity condition the behavioural response to a perturbation in the evaluative field—namely, the silent presence of a humanoid robot. The finding that EQ did \emph{not} moderate the displacement effect provides strong evidence that the robot acted at the field level rather than through trait-specific amplification or suppression. This result is consistent with HRI studies showing that robotic presence alters normative expectations independently of empathic predispositions \cite{Malle2015,Komatsu2016,Zlotowski2015}.

In this sense, the Empathizing Quotient is indispensable for distinguishing between dispositional and field-level contributions to moral behaviour. It provides a theoretically coherent and empirically validated coordinate within the dispositional manifold, enabling the evaluative--topological model to separate the geometry of $\beta_C$ from the geometry of the perturbation $\gamma_R$.


\subsection{EQ Within the Evaluative-Topological Framework}

\noindent
Within the topological architecture of this thesis, EQ measures the magnitude of the \textbf{affective vector field} $\mathbf{A}(x)$ that pulls evaluative trajectories toward empathically grounded prosocial action. High EQ corresponds to:

\begin{itemize}
	\item steep affective gradients,
	\item strong attractors around suffering, need, vulnerability,
	\item high sensitivity to social evaluation cues (including Watching-Eye primes),
	\item rapid activation of intuitive moral appraisal.
\end{itemize}

\noindent
The attenuation effect observed in the experiment was strongest among participants with high EQ values, supporting the interpretation that the robot primarily dampens the \emph{affective dynamics} of moral cognition.

\[
\delta \mathbf{A}(x;\mathscr{R}) < 0 \quad \text{for high-EQ participants}.
\]

\noindent
Thus, EQ is not merely a psychometric variable but a quantification of emotional curvature within the evaluative field.

\subsection{EQ in HRI and Moral Cognition Research}

\noindent
Studies have shown that high empathizers:

\begin{itemize}
	\item anthropomorphise robots more readily \cite{Hofree2018},
	\item show stronger prosocial responses to perceived observers \cite{Mori2020},
	\item exhibit heightened moral salience in the presence of social cues \cite{Klimecki2013}.
\end{itemize}

\noindent
This aligns precisely with Cluster~2 in our experiment: high-empathy participants with strong affective attractors who showed \emph{the largest attenuation} under robot presence.

\subsection{Why EQ Matters}

\noindent
For the purposes of this thesis, EQ provides:

\begin{itemize}
	\item a measurable dimension of sentimentalist moral theory (Humean affect),
	\item an empirical anchor for affective topology (vector field curvature),
	\item a predictor for cluster assignment,
	\item and a mechanism through which $\gamma_R$ exerts maximal perturbation.
\end{itemize}

\noindent
EQ therefore captures the essential insight that moral cognition is—both philosophically and empirically—an affectively structured evaluative process.

\section{The Systemizing Quotient (SQ): Structural Evaluation and the Precision of Moral Gradients}

\noindent
Where the Empathizing Quotient (EQ) captures affective resonance, the Systemizing Quotient (SQ) \cite{Baron2003,Goldenfeld2005,Wakabayashi2006} quantifies an individual's propensity for identifying structural regularities, constructing causal models, and applying rule-based inference. Developed in parallel with the ES theory of cognition \cite{Baron2002,Baron2009}, the SQ was designed to measure the degree to which an agent seeks predictive coherence in complex environments. It therefore provides a natural operationalisation of what, within the evaluative--topological framework of this thesis, we describe as the \emph{analytical curvature} of the evaluative field: the tendency to encode moral situations via structural invariants rather than affective attractors.

\subsection{Historical Origins and Theoretical Motivation}

\noindent
The origins of the SQ lie in the broader attempt to model cognitive styles that differentiate autistic from neurotypical populations. Baron--Cohen’s early work proposed that systemizing reflects a cognitive drive for rule extraction and causal precision, complementing empathizing but operating through distinct computational mechanisms \cite{Baron2002}. The Systemizing Quotient was introduced as the psychometric realisation of this construct, with the initial validation study \cite{Baron2003} demonstrating its sensitivity to within-group variation as well as to group-level differences between autistic and neurotypical adults.

The theoretical motivation for systemizing has since broadened. While originally embedded in autism research, systemizing has come to be linked with general tendencies toward mechanistic reasoning, causal Bayes nets, and algorithmic-level representations of environmental structure. Psychometric studies have shown that high-SQ individuals exhibit a preference for deterministic rules, hierarchical schemas, and low-noise value comparisons \cite{Goldenfeld2005}. In the ES theory, empathizing and systemizing jointly define a two-dimensional space in which variation in social cognition, emotional regulation, and reasoning strategies can be mapped.

\subsection{Psychometric Validation and Cross-Cultural Findings}

\noindent
Validation studies demonstrate that the SQ has high internal consistency, good test--retest reliability, and predictable correlations with cognitive-style measures, including analytic problem-solving, sensitivity to pattern structure, and preference for system-based explanations. Importantly, cross-cultural validation work \cite{Goldenfeld2005,Wakabayashi2006} has shown that the SQ retains its factor structure and predictive validity across different cultural contexts, supporting the claim that systemizing taps into a cognitive style with cross-cultural generality.

Neurocognitive work complements these findings. High systemizing tendencies correlate with activation in lateral prefrontal and parietal cortices associated with analytic reasoning, causal inference, and top--down attentional control. Conversely, higher systemizing scores are associated with reduced activation in affective salience networks during social evaluation tasks \cite{Gleichgerrcht2013}, reinforcing the link between SQ and reduced susceptibility to affect-laden cues.

\subsection{Empirical Uses Across Psychology, Neuroscience, and Behavioural Science}

\noindent
Systemizing has been deployed across a wide range of empirical domains. In moral psychology, systemizing tendencies predict a greater reliance on deliberative processes in dual-process moral judgment models \cite{Greene2014}. High-SQ individuals exhibit greater stability in moral evaluations across contexts, a preference for rule-consistency, and an increased likelihood of endorsing principle-based judgments in high-conflict dilemmas \cite{Cushman2013DualSystem}. The reduced affective reactivity associated with high systemizing is consistent with findings showing that utilitarian judgments arise under conditions of weaker affective engagement and stronger top--down control \cite{Greene2004,Gleichgerrcht2013}.

In behavioural economics, high systemizing correlates with consistent rule-following, lower variance in strategic play, and lower susceptibility to affective framing effects. Such individuals tend to interpret prosocial games in terms of structural incentives rather than interpersonal resonance, emphasising coherence over compassion in choice architecture.

In Human--Robot Interaction, systemizing tendencies strongly modulate expectations regarding synthetic agents. High-SQ participants are more likely to attribute competence, reliability, and causal predictability to robots, and less likely to respond to anthropomorphic cues \cite{Zlotowski2015,Malle2015,Komatsu2016}. This makes SQ particularly relevant for the experimental context of this thesis: systemizing provides a dispositional anchor for understanding how agents interpret the structural affordances introduced by a humanoid robot, especially in settings where affective cues are minimal and norm-relevant structure must be inferred.

\subsection{Critiques and Limitations}

\noindent
Although widely used, the SQ is not without critique. Some studies report that its factor structure is more heterogeneous than originally proposed, with potential subfactors corresponding to mechanical reasoning, abstract pattern detection, and rule-based inference. There are also concerns about cultural calibration, particularly regarding item content related to technical interests, which may vary across populations.

Another debate concerns the relationship between systemizing and moral judgment. While high systemizing predicts increased deliberation and reduced affective influence, this does not always translate into consistent moral choices. Some findings suggest that highly systemizing individuals may display context-dependent shifts in judgment when structural cues are ambiguous, indicating that systemizing does not override all forms of affective influence but interacts with them in non-linear ways.

Finally, as with all self-report measures, the SQ faces questions about introspective accuracy and the relation between subjective reports and actual behavioural or neural markers of structural reasoning.

\subsection{SQ Within the Evaluative--Topological Framework}

\noindent
Within the evaluative--topological model, SQ modulates the \emph{second derivative} of the evaluative potential function: it influences the \emph{rigidity}, \emph{smoothness}, and \emph{predictability} of evaluative gradients. High-SQ agents encode situations as stable causal schemas rather than affective landscapes. Consequently, their evaluative fields resist deformation under purely affective perturbations and favour top--down interpretive stability. This interpretation aligns with theoretical work emphasising the role of structural representations in moral reasoning \cite{Greene2014,Cushman2013DualSystem}.

Formally, individuals with high SQ exhibit sharper curvature in value comparison:
\[
\nabla^2 V(x) \propto \text{SQ},
\]
where larger values correspond to more rigid evaluative surfaces and reduced sensitivity to bottom--up salience fluctuations.

\subsection{SQ, Synthetic Presence, and Behavioural Perturbation}

\noindent
In the experiment underlying this thesis, high-SQ participants correspond most closely to the \emph{Analytical--Structured} dispositional cluster revealed through the clustering of EQ, SQ, and BFI scores. Consistent with research on deliberative dominance~\cite{} \cite{Konovalov2016,Shenhav2017}, these individuals displayed:
\begin{itemize}
	\item reduced coupling between moral salience and affective resonance,
	\item diminished susceptibility to affective primes,
	\item and the \emph{least} attenuation in prosocial behaviour under robot-induced perturbation.
\end{itemize}

This pattern aligns with HRI studies showing that systemizing modulates expectations of robotic competence and reduces affective interference in human--robot encounters \cite{Zlotowski2015,Komatsu2016}. In this sense, SQ plays a dual role:
\begin{itemize}
	\item it identifies participants whose evaluative fields are structurally robust under affective perturbation,
	\item and it reveals how synthetic presence interacts asymmetrically with the dispositional manifold.
\end{itemize}

\noindent
Thus, the Systemizing Quotient provides a theoretically indispensable coordinate of $\beta_C$, capturing the structural discipline with which agents parse moral environments and revealing how analytic stability interacts with perturbational effects in the evaluative field.

%%%%%%%%%%
% HERE
%%%%%%%%%%

\subsection{Why SQ Matters}

\noindent
The inclusion of the Systemizing Quotient provides a unified means of capturing several features of the evaluative landscape that would otherwise remain theoretically disjoint. Within the ES framework \cite{Baron2002,Baron2003,Baron2009}, systemizing reflects a cognitive style centred on structural analysis, causal precision, and rule-based inference. In topological terms, it indexes the \emph{deliberative curvature} of the moral field: the degree to which evaluative trajectories are shaped by stable causal schemas rather than affective attractors \cite{Goldenfeld2005,Wakabayashi2006}. 

High-SQ individuals are reliably characterised by reduced affective reactivity and a stronger reliance on deliberative pathways, as shown in both moral psychology \cite{Greene2014,Cushman2013DualSystem} and affective neuroscience \cite{Gleichgerrcht2013}. This makes SQ an effective predictor of resilience to affective suppression: analytically oriented evaluative topologies exhibit greater rigidity and lower susceptibility to perturbations driven by salience cues or affect-laden stimuli. At a theoretical level, SQ thereby anchors a comparison between sentimentalist mechanisms of moral judgment---which emphasise affective resonance and intuitive appraisal \cite{Haidt2001EmotionalDog,Crockett2016Models}---and structuralist mechanisms that privilege rule-coherent, model-based evaluation \cite{Konovalov2016,Shenhav2017}.

Crucially, SQ also clarifies why the robot did not perturb all subjects uniformly. Work in Human--Robot Interaction shows that systemizing modulates expectations of synthetic agents, attenuating the influence of anthropomorphic cues and reducing affective misattribution \cite{Zlotowski2015,Malle2015,Komatsu2016}. In the evaluative--topological framework, analytically stabilised topologies possess more rigid gradient structure and therefore exhibit reduced deformation under synthetic perturbation. High-SQ participants thus display a form of \emph{topological resilience}: their evaluative surfaces resist the salience shift induced by robotic presence, leading to weaker behavioural displacement relative to affectively dominated configurations.



% -----------------------------------------------------
\section{The Big Five Inventory (BFI): Personality Geometry and Moral Topology}

\noindent
Among the major instruments of differential psychology, the Big Five Inventory (BFI) 
occupies a uniquely robust position. Originating from decades of lexical, psychometric, 
and theoretical research \cite{John1999,McCraeCosta2008}, the Big Five model offers a 
parsimonious description of personality variation along five orthogonal axes: Openness, 
Conscientiousness, Extraversion, Agreeableness, and Neuroticism. These traits have 
repeatedly demonstrated high stability, cross-cultural generality, and considerable 
predictive power across behavioural domains \cite{Barrick1991,Donnellan2006,Rammstedt2007}. 
As a measurement instrument, the BFI therefore provides an empirically grounded coordinate 
system for mapping the dispositional substrate ($\beta_C$) posited within the 
evaluative--topological framework developed in the present thesis and subsequently used to 
interpret the uniform displacement effect observed in the experiment.

\subsection{Historical Development and Theoretical Foundations}

\noindent
The Big Five model traces its roots to the lexical tradition in personality research, in 
which factor-analytic investigations of trait-descriptive adjectives yielded a consistent 
five-factor structure across languages and populations. Building on this foundation, John 
and Srivastava \cite{John1999} formalised the BFI as a psychometrically concise yet highly 
reliable instrument for assessing the cardinal trait dimensions. Parallel work by McCrae 
and Costa \cite{McCraeCosta2008} provided a broader theoretical synthesis, linking the Big 
Five to a hierarchical model of personality structure and embedding them within a 
developmental and biological framework.

A crucial contribution of this lineage is the recognition that personality traits operate 
as stable attractors in behavioural space, shaping patterns of affective responsiveness, 
regulatory control, motivational priorities, and social orientation. For the purposes of 
the present thesis, this stability makes the BFI well suited to modelling the 
dispositional manifold that constrains evaluative trajectories in the presence of 
perturbations. In the subsequent experiment, this stability enabled a clear dissociation 
between dispositional variation and the field-level effect of robotic presence.

\subsection{Psychometric Strength and Cross-Contextual Validity}

\noindent
The BFI is among the most validated instruments in modern psychology. Its dimensional 
structure has been replicated across diverse populations, and its items exhibit strong 
internal consistency and temporal stability \cite{John1999,Donnellan2006}. Short-form 
adaptations such as the BFI-10 \cite{Rammstedt2007} preserve much of this reliability while 
enabling efficient deployment in time-constrained experimental contexts such as the present 
one.

Importantly, Big Five traits predict consequential real-world outcomes across domains 
including job performance \cite{Barrick1991}, relationship quality, subjective well-being, 
and health behaviours. These predictive successes justify the use of BFI metrics as indices 
of theoretically meaningful dispositions that shape evaluative and behavioural tendencies. 
The cross-cultural robustness of the Big Five further supports their role as part of a 
generalisable dispositional architecture that can be integrated into computational and 
topological models. In our experiment, this stability ensured that personality variation 
could be meaningfully mapped onto the dispositional manifold used to test whether the 
robotic perturbation exerted trait-dependent or trait-independent effects.

\subsection{Personality Predictors of Moral Behaviour}

\noindent
A substantial body of work has examined the relation between Big Five traits and prosocial 
or moral behaviour. Agreeableness is the most consistent predictor of helping, cooperation, 
and empathic concern \cite{Graziano1996,Habashi2016}. Individuals high in Agreeableness 
are more responsive to others’ needs, more sensitive to interpersonal harm, and more 
disposed toward altruistic action even in anonymous or low-reciprocity contexts.

Conscientiousness has been linked to moral rule adherence, planning, and long-horizon 
evaluative stability. Individuals high in Conscientiousness exhibit greater behavioural 
regularity and stronger alignment with internalised norms, qualities that translate into 
reduced noise in moral decision-making. Neuroticism predicts greater affective volatility, 
heightened sensitivity to social threat, and increased susceptibility to contextual 
perturbation \cite{Hilbig2013}. Extraversion amplifies responsiveness to social presence and 
increases the weighting of socially salient cues. Openness broadens receptivity to 
contextual novelty, increases tolerance of ambiguity, and enhances exploratory behaviour in 
moral and social domains.

These findings demonstrate that the Big Five traits track the dispositional architecture 
that shapes how agents integrate affective, cognitive, and contextual information into 
evaluative judgments. The BFI therefore directly contributes to estimating the $\beta_C$ 
manifold in the evaluative--topological model. In our experiment, this mapping was essential 
for determining whether the displacement observed in the robot condition reflected 
disposition-specific pathways or a uniform perturbation of the evaluative field.

\subsection{BFI in Social Cognition, SSP, and HRI}

\noindent
Beyond moral psychology, the BFI plays a central role in research on social cognition and 
nonverbal behaviour. Personality traits influence expressive dynamics, gaze patterns, vocal 
modulation, and gesture production—behaviours that constitute the core observational cues in 
Social Signal Processing (SSP). Vinciarelli et al.’s foundational survey \cite{Vinciarelli2009} 
highlights how personality traits can be inferred from multimodal behavioural signatures 
(speech prosody, movement patterns, attention allocation), and how these traits modulate 
social engagement, turn-taking, and responsiveness to social cues.

These insights reinforce the relevance of the BFI in contexts involving robotic presence. 
Traits such as Extraversion and Agreeableness shape social approach tendencies, sensitivity 
to perceived agency, and responsiveness to social affordances—all properties critical in HRI 
scenarios. Banks \cite{Banks2020} demonstrates that personality interacts with perceptions of 
robot trustworthiness, sociality, and intentionality, thereby linking the Big Five to the 
cognitive mechanisms underlying moral or cooperative evaluation of artificial agents.

In the present experiment, these considerations justify the use of the BFI as a means of 
quantifying structural differences in participants' social orientation. Because the 
perturbation introduced by the robot operates at the level of perceived social presence, 
personality traits that modulate such responsiveness play an instrumental role in 
understanding dispositional variation across the sample, and in establishing that the 
displacement effect is indeed field-level rather than trait-dependent.

\subsection{Personality Geometry Within the Evaluative--Topological Framework}

\noindent
Within the evaluative--topological model, personality traits function as geometric 
modifiers of the evaluative field. Agreeableness steepens prosocial attractor basins, 
lowering friction along cooperative trajectories and increasing the salience of altruistic 
outcomes. Conscientiousness stabilises high-level evaluative pathways, introducing strong 
curvature along rule-governed dimensions and reducing susceptibility to contextual noise. 
Neuroticism injects volatility into the evaluative manifold, increasing the amplitude of 
local fluctuations and enhancing the influence of perturbations. Openness expands the 
contextual sensitivity of the evaluative field, enabling broader sampling of informational 
cues. Extraversion intensifies responsiveness to social presence, amplifying the salience 
contributions of agents (human or synthetic) within the perceptual environment.

Taken together, these geometric interpretations allow the BFI traits to be embedded within 
the formalism:
\[
\mathscr{P}(\delta_m) = f(\alpha_E, \beta_C, \gamma_R).
\]
The BFI operationalises key dimensions of $\beta_C$, specifying the metric structure through 
which evaluative trajectories evolve in response to environmental ($\alpha_E$) and 
perturbational ($\gamma_R$) influences. In the experiment, this enabled precise comparison of 
dispositional geometry against the uniform behavioural displacement induced by the robot.

\subsection{BFI, Perturbation, and the Interpretation of Uniform Attenuation}

\noindent
In the experiment underlying this thesis, BFI traits were expected to modulate sensitivity to 
the robotic perturbation, particularly given the relevance of Extraversion, Agreeableness, 
and Neuroticism to social presence, affective reactivity, and contextual susceptibility. Yet 
the empirical results—discussed in Chapter~\ref{chap:experimental_methods}—revealed no such 
moderation. Despite robust dispositional structure uncovered through cluster analysis, all 
groups exhibited the same directional attenuation in prosocial behaviour.

This finding is consistent with evidence from HRI and social cognition that suggests robotic 
presence can shift social affordances at a level that bypasses trait-level predispositions, 
acting instead through global modifications of perceived agency, social monitoring, or norm 
salience \cite{Malle2015,Komatsu2016,Zlotowski2015}. The BFI was crucial in establishing 
this. By providing a structured mapping of personality geometry, the instrument made it 
possible to dissociate trait-level variation from field-level displacement. Without this 
differentiation, the attenuation could have been misattributed to personality differences 
rather than to the synthetic perturbation introduced by the robot.

\subsection{Critiques, Limitations, and Relevance to the Thesis}

\noindent
The Big Five framework is not without its critics. Some theorists argue that it is 
descriptively powerful but theoretically thin, lacking a mechanistic account of trait 
emergence. Others raise concerns about the number of underlying factors, suggesting that 
alternative models (HEXACO, hierarchical factor models) may capture additional variance in 
moral or social behaviour. Still others point to the risk that personality traits are only 
weakly predictive at the level of individual behaviour and depend heavily on situational 
features.

However, for the purposes of the present thesis, these critiques do not undermine the 
instrument's value. The BFI was not employed as a causal explanation of moral behaviour, but 
as a principled means of mapping the dispositional manifold through which perturbations 
propagate. Its high stability, conceptual clarity, and predictive track record make it the 
appropriate tool for modelling $\beta_C$ in a topological framework and for demonstrating 
that the attenuation produced by robotic presence is a field-level displacement rather than 
a trait-level moderation.

\medskip

\noindent
In summary, the Big Five Inventory provides a theoretically grounded and empirically 
validated coordinate system for the dispositional term $\beta_C$. Its integration into the 
evaluative--topological model enables a precise dissociation between trait-level structure 
and field-level perturbation, making it an indispensable tool for interpreting the uniform 
attenuation produced by robotic presence.


\noindent
These dimensions do not function independently; instead, they jointly determine the 
curvature, stability, and topology of the moral field for each participant.

\subsection{Cluster Semantics and BFI Geometry}

\noindent
The cluster analysis in Chapter~\ref{chap:experimental_methods} revealed three 
dispositional attractor structures:

\begin{enumerate}
	\item \textbf{Prosocial–Empathic}: high Agreeableness, high Openness, high EQ; steep 
	affective attractors; strong orientation toward altruistic trajectories.
	\item \textbf{Emotionally Reactive}: high Neuroticism, mixed EQ; unstable gradients; 
	heightened susceptibility to contextual variation.
	\item \textbf{Analytical–Structured}: high Conscientiousness and SQ; rigid gradients; 
	evaluative trajectories shaped by rule-based stability.
\end{enumerate}

\noindent
BFI traits were foundational for revealing this structure. The BFI did not merely 
``measure personality''—it uncovered the dispositional topology against which the 
synthetic perturbation could be interpreted. Crucially, although the clusters display 
meaningful internal geometry, the experimental results showed that all three were 
affected in the same behavioural direction. This demonstrates that the clusters 
represent differentiated \emph{starting positions} within the evaluative manifold, 
not differentiated \emph{responses} to the perturbation.

\subsection{Why BFI Matters}

\noindent
Including the BFI allows:

\begin{itemize}
	\item quantification of dispositional topology,
	\item clustering of heterogeneous evaluative architectures,
	\item grounding normative interpretation in empirically real personality space,
	\item and demonstrating that synthetic moral perturbation operates at the 
	\emph{field level rather than through trait-specific pathways}.
\end{itemize}

\noindent
Without the BFI, the experiment would lack the dimensional granularity required to 
distinguish between dispositional variation and the global topological displacement 
induced by robotic presence.


% -----------------------------------------------------
\section{The Watching-Eye Paradigm: Moral Salience Amplification Through Social Attention}

\noindent
The Watching-Eye effect is one of the most robust phenomena in behavioural science:  
minimal cues of observation—stylised eyes, drawings, or schematic-gaze primes—reliably increase prosocial behaviour, charitable donation, norm compliance, and generosity \cite{HaleyFessler2005,ErnestJones2011,Nettle2013}.

\subsection{The Watching-Eye Paradigm as an Experimental Scaffold}

\noindent
In the present study, the Watching-Eye prime serves three essential functions:

\begin{enumerate}
	\item \textbf{Amplifier of moral salience}: it increases the weight of moral cues in the evaluative field.
	\item \textbf{Probe of susceptibility}: it reveals how strongly each participant’s topology responds to social-evaluative cues.
	\item \textbf{Baseline moral-gradient enhancer}: it provides a moral gradient strong enough to detect attenuation by the robot.
\end{enumerate}

\noindent
The paradigm functions at the level of \emph{perceptual moral salience}, not reflective ethical judgment. It modulates:

\[
\alpha_E \mapsto \alpha_E + \delta \alpha_{\text{eye}},
\]

which increases the steepness of the prosocial gradient—unless perturbed.

\subsection{Watching-Eye Under Synthetic Co-Presence}

\noindent
The experiment revealed a novel effect:

\begin{quote}
	\emph{robotic presence cancels or reverses the moral-salience amplification produced by Watching-Eye cues}.
\end{quote}

\noindent
This establishes the Watching-Eye paradigm as a diagnostic tool for:

\begin{itemize}
	\item measuring topological deformation under synthetic presence,
	\item distinguishing cluster-specific susceptibility,
	\item identifying how $\gamma_R$ interacts with both affective and deliberative pathways.
\end{itemize}

\medskip

\noindent
Formally, the robot introduces a perturbation term:
\[
\gamma_R: \ \alpha_E + \delta \alpha_{\text{eye}} \ \mapsto \ \alpha_E + \delta \alpha_{\text{eye}} - \Delta_{\mathscr{R}},
\]
where \( \Delta_{\mathscr{R}} \) is the amount of salience suppression.

\subsection{Why the Watching-Eye Paradigm Matters}

\noindent
The paradigm is essential because it provides:

\begin{itemize}
	\item a standardized probe of moral salience,
	\item a replicable baseline to study attenuation,
	\item a methodological bridge between moral psychology and HRI,
	\item and the conceptual grounding for interpreting synthetic presence as a \emph{normative disruptor}.
\end{itemize}

\noindent
It transforms the experiment from a simple donation task into a structured test of \emph{moral topology under perturbation}.

\section{Integrative Conclusion: Tools as Windows Into the Moral Topology}
\label{sec:tools_conclusion}

\noindent
The tools introduced in this chapter---the Empathizing Quotient (EQ), the Systemizing Quotient (SQ), the Big Five Inventory (BFI), and the Watching-Eye paradigm---do not function as mere psychometric instruments or isolated experimental devices. 
Within the architecture of this thesis, they constitute a coordinated measurement framework designed to map the \emph{evaluative topology} through which moral cognition becomes behaviour, and to detect the deformation of that topology under synthetic perturbation.

\subsection*{(1) EQ: The Affective Gradient Field}
\noindent
The EQ operationalises the strength, curvature, and accessibility of the \emph{affective gradients} that orient moral appraisal. High-EQ profiles correspond to steep attractor basins toward prosocial behaviour, rapid resonance with morally salient cues, and heightened susceptibility to perturbations that suppress affective weighting.

In the experiment, high-EQ participants belonged primarily to the \textbf{Prosocial--Empathic cluster}, which exhibited the \emph{largest drop} in donation under robotic co-presence. This is precisely what the sentimentalist model predicts: the robot dampens affective activation (\(\delta \mathbf{A}(x;\mathscr{R})<0\)), flattening the empathic landscape that normally channels behaviour into prosocial trajectories.

\subsection*{(2) SQ: Structural Rigidity and Topological Stability}

\noindent
The SQ quantifies the agent’s reliance on structured, rule-like, or causally coherent evaluative processes. High-SQ individuals align with the \textbf{Analytical--Structured cluster}, characterised by rigid gradients and reduced dependence on affective salience.

Consequently, these participants displayed the \emph{least attenuation} under synthetic presence. Their evaluative fields withstand perturbation because the curvature of the moral potential function is dominated by deliberative structure rather than affective resonance. In topological terms: their trajectories maintain directionality even when $\delta \mathbf{A}<0$, because structural components of the field remain intact.

\subsection*{(3) BFI: Personality Geometry as the Substrate of Moral Susceptibility}

\noindent
The BFI provides the multidimensional geometry required to cluster heterogenous evaluative profiles into coherent dispositional regimes. It reveals the latent metric structure that differentiates:

\begin{itemize}
	\item steep affective attractors (high Agreeableness, high Openness),
	\item unstable regions (high Neuroticism),
	\item and rigid gradient fields (high Conscientiousness).
\end{itemize}

\noindent
These geometric properties underpin the three dispositional clusters discovered in the experiment. Without the BFI, the experiment would lack the structural resolution needed to interpret \emph{why} the robot perturbs some topologies dramatically while leaving others nearly invariant.

\subsection*{(4) The Watching-Eye Paradigm: Amplifying Salience to Detect Deformation}

\noindent
The Watching-Eye effect serves as a moral-salience amplifier. It artificially steepens the prosocial gradient, providing a controlled environment in which attenuation or deformation becomes measurable.

The experiment shows that robotic co-presence \emph{cancels or reverses} this amplification. The Watching-Eye cue establishes a high-salience baseline; the robot then reveals the degree to which $\gamma_R$ suppresses, flattens, or redirects moral salience across the personality manifold.

This makes the Watching-Eye paradigm a diagnostic probe for topological deformation under synthetic presence.

\subsection*{(5) Coordinated Function of the Tools in the Experimental Topology}

\noindent
Together, these tools constitute a coherent architecture:

\begin{center}
	\begin{tabular}{cl}
		\textbf{EQ} & measures affective curvature of the field, \\
		\textbf{SQ} & measures structural rigidity of evaluative gradients, \\
		\textbf{BFI} & maps dispositional geometry across personality space, \\
		\textbf{Watching-Eye} & amplifies moral salience to expose perturbation effects. \\
	\end{tabular}
\end{center}

\noindent
Each instrument probes a different dimension of the evaluative field, allowing the experiment to identify not only \emph{whether} synthetic presence perturbs moral behaviour, but \emph{how} and \emph{where} within the moral topology this perturbation manifests.

\subsection*{(6) Ethical Interpretation: Tools as Normative Diagnostics}

\noindent
When interpreted through the ethical frameworks developed in Chapter~\ref{chap:ethics_s}, the tools reveal how synthetic presence interacts with:

\begin{itemize}
	\item \textbf{deontic sensitivity}: Watching-Eye cues activate accountability norms; attenuation reveals disruption of deontic salience.
	\item \textbf{consequentialist gradients}: synthetic presence alters perceived reputational or social payoff structures.
	\item \textbf{virtue-theoretic character}: dispositional clusters encode stable patterns of moral responsiveness, selectively susceptible to $\gamma_R$.
	\item \textbf{sentimentalist vector fields}: the robot acts as a suppressor of affective forces driving prosocial action.
	\item \textbf{contractualist justification}: perturbation alters relational expectations and interpersonal moral justification.
	\item \textbf{particularist salience}: synthetic presence modifies the salience landscape that particularists treat as constitutive of moral perception.
\end{itemize}

\noindent
Thus, the tools chapter does not merely justify instrumentation—it establishes the \emph{measurement logic} required for ethical interpretation of the experiment.

\subsection*{(7) Final Synthesis}

\noindent
The measurement framework developed here reveals a structural picture of moral cognition:

\begin{center}
	\emph{Moral behaviour emerges from a personality-dependent evaluative topology that is selectively deformable under synthetic presence.}
\end{center}

The EQ, SQ, BFI, and Watching-Eye paradigm jointly expose:

\begin{itemize}
	\item the \textbf{affective vectors} that drive prosocial behaviour,
	\item the \textbf{structural gradients} that stabilise or destabilise action selection,
	\item the \textbf{dispositional geometry} that differentiates cluster-level susceptibilities,
	\item and the \textbf{moral-salience amplification} required to detect deformation.
\end{itemize}

\noindent
Most importantly, these tools reveal that:

\begin{quote}
	\textbf{synthetic co-presence perturbs not only behaviour but the evaluative architecture that produces behaviour, and does so in a trait-structured, topologically coherent way.}
\end{quote}

\noindent
This integrative insight prepares the ground for the next chapter. The tools defined here serve as the conceptual and methodological infrastructure through which the experimental design is constructed, the data interpreted, and the broader philosophical implications established.

\begin{center}
	\begin{tcolorbox}[colback=white,colframe=black!60,title=Conceptual Summary]
		The tools of this chapter form the empirical backbone of the thesis.  
		They do not merely measure traits or prime behaviour;  
		they map the geometry, curvature, and salience structure of the evaluative field.  
		Through them, the experiment becomes a principled test of how synthetic presence  
		\emph{deforms the moral topology}, selectively altering the pathways  
		from moral perception to prosocial action.
	\end{tcolorbox}
\end{center}


%%%%%%%%%%%%%%
% OLD CONTENT
%%%%%%%%%%%%%%

\section{The Watching-Eye Effect}

One of the most robust findings in behavioural ethics and social psychology is that subtle cues of observation can increase prosocial behaviour. This phenomenon—commonly referred to as the \emph{watching-eye effect}—demonstrates that even minimal stimuli implying social presence can modulate cooperative or altruistic actions \cite{HaleyFessler2005, BatesonNettleRoberts2006, PfattheicherKeller2015}. Although originally interpreted in terms of reputational concerns, contemporary evidence indicates a multi-component mechanism involving attentional, affective, and interpretive pathways.

\paragraph{Reputational Mechanisms.}
Early accounts emphasised reputational vigilance: cues of observation were posited to activate concerns about social evaluation, thereby increasing norm adherence and generosity \cite{HaleyFessler2005, BatesonNettleRoberts2006}. Even stylised eye images were found to increase cooperation in real-world settings, suggesting that human social cognition is highly sensitive to potential monitoring \cite{BatesonEtAl2013_EyesLittering}. At the Level of Abstraction adopted in this thesis, reputational vigilance can be understood as a deformation of the evaluative landscape: cues implying oversight increase the weight of fairness, compliance, or prosocial norms in action-guiding computations.

\paragraph{Attentional and Perceptual Mechanisms.}
More recent work demonstrates that watching-eye cues also affect the allocation of visual and social attention, shifting perceptual resources toward norm-relevant features \cite{KleckStrenta1980, Emery2000}. Eye cues act as attentional attractors, increasing the salience of one's own behaviour and its alignment with internalised standards or expectations. This attentional modulation modifies the early intuitive gradients that shape moral evaluation, consistent with the topological model presented earlier.

\paragraph{Affective and Self-Conscious Emotion Mechanisms.}
Other studies emphasise the role of self-conscious emotions—such as guilt, embarrassment, or pride—in mediating responses to perceived observation. Eye cues elicit mild increases in affective arousal \cite{PfattheicherKeller2015}, potentially amplifying somatic markers associated with prosocial appraisal. In this sense, watching-eye stimuli operate by perturbing both affective and interpretive components of the evaluative field, thereby increasing the likelihood of prosocial action.

\paragraph{Context Sensitivity and Boundary Conditions.}
Importantly, the watching-eye effect is not uniform across contexts. Its magnitude depends on factors such as prevailing norms \cite{Kawamura2017}, the ambiguity of observational cues, and the ecological validity of the environment. These boundary conditions foreshadow the central empirical question of this thesis: whether the presence of a synthetic agent counts as an observational cue strong enough to elicit similar modulations in moral behaviour.


\section{Why Child-Poster Stimuli Function as Valid Social Cues}

Child-poster images featuring watching eyes are widely used as a minimal and controlled observational cue in donation-based paradigms. Their effectiveness derives from three properties that make them well-suited for experiments requiring precision and reproducibility.

\paragraph{Perceptual Sociality Without Agentic Commitment.}
Child eyes provide a cue that is perceptually social—highly evocative of gaze and attention—yet ontologically unproblematic. Participants do not confuse the poster with an actual agent, but the cue nevertheless activates perceptual mechanisms associated with being observed \cite{HaleyFessler2005}. This makes child-eye stimuli a clean perturbation of attentional and affective gradients without introducing confounds related to mental-state attribution.

\paragraph{Affective Resonance and Care-Related Salience.}
Child-related imagery tends to increase empathic concern and activate care-related motivational systems. Studies of interpersonal gaze show that the perceived innocence or vulnerability of the observer enhances the social salience of eye cues \cite{MasonTatkinMacrae2005}. Within the topological framework of this thesis, child-eye stimuli strengthen the evaluative attractors associated with care, prosociality, and harm avoidance.

\paragraph{Methodological Control.}
Child-eye posters offer high experimental control. Their low-dimensional visual structure avoids the confounds that arise when using real human observers, anthropomorphic agents, or dynamic faces. They therefore serve as a reproducible baseline for assessing how additional or alternative social cues—such as those introduced by a humanoid robot—perturb prosocial behaviour \cite{HaleyFessler2005, BatesonNettleRoberts2006}.


\section{Why Robots May Dilute or Modulate the Watching-Eye Effect}

A central hypothesis of this thesis is that a humanoid robot—despite being perceptually social—may attenuate, distort, or otherwise alter the watching-eye effect. This dilution is not due to reduced salience, but to the \emph{ontological ambiguity} of synthetic agents.

\paragraph{Perceptual Sociality Without Clear Social Ontology.}
Robots are visually social in virtue of their humanoid morphology, but they do not occupy a stable position within the human social ontology. They are neither fully agentic nor fully inert. This indeterminacy can weaken the intuitive mappings between observational cues and reputational or normative expectations. From the perspective of evaluative topology, robots generate conflicting gradients: they signal social presence while simultaneously undermining the interpretive coherence of that presence.

\paragraph{Disrupted Affective and Attentional Gradients.}
The presence of a robot may dampen affective resonance relative to child-eye images. Because the robot lacks a clear moral status, affective systems governing care, empathy, or guilt may be only partially activated. A similar disruption occurs at the attentional level: while robots attract gaze, they may not reliably signal evaluative oversight \cite{ContyGeorgeHietanen2016}. This can flatten or distort the intuitive attractors that normally support prosocial action.

\paragraph{Predictive and Interpretive Uncertainty.}
Mental-state attribution is central to the watching-eye effect. Minimal cues imply that another agent could observe or morally evaluate one's behaviour. With a robot, mental-state attribution becomes unstable: participants may attribute perceptual capacities without attributing evaluative ones. This uncertainty creates a diffuse or bifurcated evaluative field, reducing the force of reputational or care-related attractors and thereby attenuating prosocial tendencies.

\paragraph{Consequences for Evaluative Topology.}
Within the framework of this thesis, robots function as \textit{semiotic perturbators} of the moral field. Their presence shifts the shape of evaluative gradients—sometimes sharpening local attractors, sometimes flattening them, sometimes diverting trajectories altogether. The empirical prediction is thus not a simple decrease in prosociality, but a measurable deformation of the mapping from moral salience to action.


\section{Prosocial Donation Paradigm}

To test these theoretical predictions, this thesis employs a structured donation paradigm widely used in behavioural ethics, moral psychology, and social neuroscience. Donation tasks provide a reproducible, quantifiable measure of prosocial behaviour that reflects practical moral commitment rather than hypothetical endorsement \cite{Moll2002, Decety2004}.

\paragraph{Operational Structure.}
Participants are offered the opportunity to donate part of their experimental compensation to a real charity. Their donation amount serves as a behavioural index of prosocial motivation. Because donations involve a concrete cost, they reveal the strength of evaluative gradients sufficiently strong to influence action.

\paragraph{Integration With Observational Cues.}
The donation task is performed under one of several observational conditions: (i) child-eye stimulus, (ii) humanoid robot presence, or (iii) control condition. By holding all other variables constant, any variation in donation behaviour reflects differences in how observational cues modulate the evaluative topology connecting moral salience to practical action.

\paragraph{Why Donation Is the Appropriate Measure.}
At the chosen Level of Abstraction, moral cognition is defined not by its propositional structure but by its action-guiding function. Donation behaviour captures this directly: it provides a measurable, ecologically relevant manifestation of how evaluative processes culminate in a behavioural output. The paradigm thus serves as a test bed for detecting the subtle, yet theoretically significant, perturbations induced by synthetic social presence.

\paragraph{Expected Perturbation Pattern.}
Based on the architecture articulated in previous sections, the presence of a humanoid robot is predicted to modulate donation behaviour by altering attentional, affective, and interpretive pathways. This modulation is expected to manifest not as random noise but as a coherent deformation of the evaluative topology, consistent with the concept of a \emph{moral refractor}. The empirical chapter demonstrates precisely such patterned perturbation.