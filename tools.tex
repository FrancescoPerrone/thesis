\chapter{TOOLS}
\label{chap:tools}
\thispagestyle{pprintTitle}

\section{The Watching-Eye Effect}

One of the most robust findings in behavioural ethics and social psychology is that subtle cues of observation can increase prosocial behaviour. This phenomenon—commonly referred to as the \emph{watching-eye effect}—demonstrates that even minimal stimuli implying social presence can modulate cooperative or altruistic actions \cite{HaleyFessler2005, BatesonNettleRoberts2006, PfattheicherKeller2015}. Although originally interpreted in terms of reputational concerns, contemporary evidence indicates a multi-component mechanism involving attentional, affective, and interpretive pathways.

\paragraph{Reputational Mechanisms.}
Early accounts emphasised reputational vigilance: cues of observation were posited to activate concerns about social evaluation, thereby increasing norm adherence and generosity \cite{HaleyFessler2005, BatesonNettleRoberts2006}. Even stylised eye images were found to increase cooperation in real-world settings, suggesting that human social cognition is highly sensitive to potential monitoring \cite{BatesonEtAl2013_EyesLittering}. At the Level of Abstraction adopted in this thesis, reputational vigilance can be understood as a deformation of the evaluative landscape: cues implying oversight increase the weight of fairness, compliance, or prosocial norms in action-guiding computations.

\paragraph{Attentional and Perceptual Mechanisms.}
More recent work demonstrates that watching-eye cues also affect the allocation of visual and social attention, shifting perceptual resources toward norm-relevant features \cite{KleckStrenta1980, Emery2000}. Eye cues act as attentional attractors, increasing the salience of one's own behaviour and its alignment with internalised standards or expectations. This attentional modulation modifies the early intuitive gradients that shape moral evaluation, consistent with the topological model presented earlier.

\paragraph{Affective and Self-Conscious Emotion Mechanisms.}
Other studies emphasise the role of self-conscious emotions—such as guilt, embarrassment, or pride—in mediating responses to perceived observation. Eye cues elicit mild increases in affective arousal \cite{PfattheicherKeller2015}, potentially amplifying somatic markers associated with prosocial appraisal. In this sense, watching-eye stimuli operate by perturbing both affective and interpretive components of the evaluative field, thereby increasing the likelihood of prosocial action.

\paragraph{Context Sensitivity and Boundary Conditions.}
Importantly, the watching-eye effect is not uniform across contexts. Its magnitude depends on factors such as prevailing norms \cite{Kawamura2017}, the ambiguity of observational cues, and the ecological validity of the environment. These boundary conditions foreshadow the central empirical question of this thesis: whether the presence of a synthetic agent counts as an observational cue strong enough to elicit similar modulations in moral behaviour.


\section{Why Child-Poster Stimuli Function as Valid Social Cues}

Child-poster images featuring watching eyes are widely used as a minimal and controlled observational cue in donation-based paradigms. Their effectiveness derives from three properties that make them well-suited for experiments requiring precision and reproducibility.

\paragraph{Perceptual Sociality Without Agentic Commitment.}
Child eyes provide a cue that is perceptually social—highly evocative of gaze and attention—yet ontologically unproblematic. Participants do not confuse the poster with an actual agent, but the cue nevertheless activates perceptual mechanisms associated with being observed \cite{HaleyFessler2005}. This makes child-eye stimuli a clean perturbation of attentional and affective gradients without introducing confounds related to mental-state attribution.

\paragraph{Affective Resonance and Care-Related Salience.}
Child-related imagery tends to increase empathic concern and activate care-related motivational systems. Studies of interpersonal gaze show that the perceived innocence or vulnerability of the observer enhances the social salience of eye cues \cite{MasonTatkinMacrae2005}. Within the topological framework of this thesis, child-eye stimuli strengthen the evaluative attractors associated with care, prosociality, and harm avoidance.

\paragraph{Methodological Control.}
Child-eye posters offer high experimental control. Their low-dimensional visual structure avoids the confounds that arise when using real human observers, anthropomorphic agents, or dynamic faces. They therefore serve as a reproducible baseline for assessing how additional or alternative social cues—such as those introduced by a humanoid robot—perturb prosocial behaviour \cite{HaleyFessler2005, BatesonNettleRoberts2006}.


\section{Why Robots May Dilute or Modulate the Watching-Eye Effect}

A central hypothesis of this thesis is that a humanoid robot—despite being perceptually social—may attenuate, distort, or otherwise alter the watching-eye effect. This dilution is not due to reduced salience, but to the \emph{ontological ambiguity} of synthetic agents.

\paragraph{Perceptual Sociality Without Clear Social Ontology.}
Robots are visually social in virtue of their humanoid morphology, but they do not occupy a stable position within the human social ontology. They are neither fully agentic nor fully inert. This indeterminacy can weaken the intuitive mappings between observational cues and reputational or normative expectations. From the perspective of evaluative topology, robots generate conflicting gradients: they signal social presence while simultaneously undermining the interpretive coherence of that presence.

\paragraph{Disrupted Affective and Attentional Gradients.}
The presence of a robot may dampen affective resonance relative to child-eye images. Because the robot lacks a clear moral status, affective systems governing care, empathy, or guilt may be only partially activated. A similar disruption occurs at the attentional level: while robots attract gaze, they may not reliably signal evaluative oversight \cite{ContyGeorgeHietanen2016}. This can flatten or distort the intuitive attractors that normally support prosocial action.

\paragraph{Predictive and Interpretive Uncertainty.}
Mental-state attribution is central to the watching-eye effect. Minimal cues imply that another agent could observe or morally evaluate one's behaviour. With a robot, mental-state attribution becomes unstable: participants may attribute perceptual capacities without attributing evaluative ones. This uncertainty creates a diffuse or bifurcated evaluative field, reducing the force of reputational or care-related attractors and thereby attenuating prosocial tendencies.

\paragraph{Consequences for Evaluative Topology.}
Within the framework of this thesis, robots function as \textit{semiotic perturbators} of the moral field. Their presence shifts the shape of evaluative gradients—sometimes sharpening local attractors, sometimes flattening them, sometimes diverting trajectories altogether. The empirical prediction is thus not a simple decrease in prosociality, but a measurable deformation of the mapping from moral salience to action.


\section{Prosocial Donation Paradigm}

To test these theoretical predictions, this thesis employs a structured donation paradigm widely used in behavioural ethics, moral psychology, and social neuroscience. Donation tasks provide a reproducible, quantifiable measure of prosocial behaviour that reflects practical moral commitment rather than hypothetical endorsement \cite{Moll2002, Decety2004}.

\paragraph{Operational Structure.}
Participants are offered the opportunity to donate part of their experimental compensation to a real charity. Their donation amount serves as a behavioural index of prosocial motivation. Because donations involve a concrete cost, they reveal the strength of evaluative gradients sufficiently strong to influence action.

\paragraph{Integration With Observational Cues.}
The donation task is performed under one of several observational conditions: (i) child-eye stimulus, (ii) humanoid robot presence, or (iii) control condition. By holding all other variables constant, any variation in donation behaviour reflects differences in how observational cues modulate the evaluative topology connecting moral salience to practical action.

\paragraph{Why Donation Is the Appropriate Measure.}
At the chosen Level of Abstraction, moral cognition is defined not by its propositional structure but by its action-guiding function. Donation behaviour captures this directly: it provides a measurable, ecologically relevant manifestation of how evaluative processes culminate in a behavioural output. The paradigm thus serves as a test bed for detecting the subtle, yet theoretically significant, perturbations induced by synthetic social presence.

\paragraph{Expected Perturbation Pattern.}
Based on the architecture articulated in previous sections, the presence of a humanoid robot is predicted to modulate donation behaviour by altering attentional, affective, and interpretive pathways. This modulation is expected to manifest not as random noise but as a coherent deformation of the evaluative topology, consistent with the concept of a \emph{moral refractor}. The empirical chapter demonstrates precisely such patterned perturbation.

%%%%%%%%%%%%%%%%%%%%%%%
% NEW CONTENT
%%%%%%%%%%%%%%%%%%%%%%%

\chapter{Psychometric Tools and Experimental Paradigms}
\label{chap:tools}

\noindent
This chapter introduces and justifies the tools employed in the experimental component of the thesis. Each tool is presented not merely as an operational instrument but as a theoretically motivated measurement aligned with the evaluative-topological framework, the Levels of Abstraction (LoA) discipline, and the dispositional semantics used in the cluster analysis. The purpose is twofold:

\begin{enumerate}
	\item to demonstrate that each tool is grounded in robust empirical and theoretical literatures, and  
	\item to show how each tool contributes to modelling the dispositional term $\beta_C$ in the formal expression:
	\[
	\mathscr{P}(\delta_m) = f(\alpha_E, \beta_C, \gamma_R),
	\]
	where $\beta_C$ captures the latent trait configuration that shapes how each participant's evaluative topology responds to the perturbation induced by the humanoid robot.
\end{enumerate}

\noindent
The tools included here—the Empathizing Quotient (EQ), the Systemizing Quotient (SQ), the Big Five Inventory (BFI), and the Watching-Eye paradigm—were selected because they satisfy three stringent criteria:

\begin{enumerate}
	\item \textbf{Theoretical relevance}: Each tool targets a component of moral topology (affective resonance, evaluative precision, personality curvature, or salience modulation).
	\item \textbf{Empirical robustness}: Each tool is validated across multiple cultures, large samples, and decades of psychological research, and has been used in studies of prosociality, moral sensitivity, social attention, and Human–Robot Interaction (HRI).
	\item \textbf{Computational suitability}: Each tool produces variables suitable for integration into regression models, cluster analysis, and topological interpretation.
\end{enumerate}

\noindent
Before turning to the tools themselves, we first articulate the methodological role they play within this thesis.

\section{The Role of Psychometric Tools in the Evaluative-Topological Architecture}

\noindent
The formal architecture developed throughout the thesis treats moral action as the endpoint of a dynamical trajectory across an evaluative field. This trajectory is determined by the interaction of:

\begin{itemize}
	\item \emph{environmental inputs} ($\alpha_E$): moral cues, social primes, Watching-Eye stimulus;
	\item \emph{dispositional structure} ($\beta_C$): latent traits measured by EQ, SQ, BFI;
	\item \emph{perturbation operators} ($\gamma_R$): the presence of the humanoid robot.
\end{itemize}

\noindent
The psychometric tools used in the study belong to the $\beta_C$ term. They quantify:
\begin{itemize}
	\item the \textbf{affective bandwidth} of moral appraisal (EQ),
	\item the \textbf{structural–analytical curvature} of evaluative processing (SQ),
	\item the \textbf{personality geometry} that modulates salience, attention, and regulatory control (BFI).
\end{itemize}

\noindent
These measurements did not merely populate regression tables: they revealed—through cluster analysis—a structured \emph{moral topology over personality space}. That topology, characterised by coherent regions of affective warmth, analytical structuring, or reactive volatility, was fundamental for interpreting the heterogeneous effects of synthetic perturbation.

\medskip

\noindent
In the experimental formalism, the dispositional term appears in the perturbation expression:
\[
f(\alpha_E, \beta_C, \gamma_R) - f(\alpha_E, \beta_C)
\]
which measures how the moral transformation function is reshaped by $\gamma_R$.  
The cluster-specific results (Chapter~\ref{chap:experimental_results}) showed that synthetic presence does \emph{not} uniformly attenuate prosociality; it interacts with $\beta_C$ in trait-dependent ways.

\medskip

\noindent
The goal of this chapter, therefore, is not simply to describe the tools, but to show how each contributes to modelling $\beta_C$ and to justify why their inclusion is essential for the thesis’s interpretive and theoretical ambitions.

\section{Why These Tools: Methodological Criteria and Alignment with the Thesis}

\noindent
The selection of EQ, SQ, the BFI, and the Watching-Eye paradigm follows three methodological constraints:

\paragraph{(1) Cross-paradigmatic relevance.}
Each tool has been used in:

\begin{itemize}
	\item moral psychology (harm aversion, fairness, prosociality),
	\item personality psychology,
	\item computational social science,
	\item and, critically, Human–Robot Interaction.
\end{itemize}

Thus, they allow comparison of the present study with the broader literature.

\paragraph{(2) Topological relevance.}
Each tool captures a different part of the evaluative field:

\begin{itemize}
	\item EQ: strength and curvature of affective attractors,
	\item SQ: precision and rigidity of evaluative gradients,
	\item BFI: multidimensional personality geometry,
	\item Watching-Eye: salience-boosting perturbation of moral cues.
\end{itemize}

\paragraph{(3) Stability and interpretability.}
These measures allow:

\begin{itemize}
	\item clustering in dispositional space,
	\item regression modelling of $\beta_C$,
	\item interpretation via normative frameworks (virtue ethics, sentimentalism).
\end{itemize}

\noindent
With these foundations established, we now turn to the first measurement tool:  
the Empathizing Quotient.

\section{The Empathizing Quotient (EQ): Affective Resonance as a Moral Vector Field}

\noindent
The Empathizing Quotient, developed by Baron-Cohen and colleagues \cite{BaronCohen2004,Lawson2004,Wakabayashi2006}, measures an individual's capacity for emotional resonance, perspective-taking, and sensitivity to the affective states of others. EQ has been used extensively in:

\begin{itemize}
	\item moral psychology (predicting altruism, harm aversion, guilt sensitivity),
	\item social neuroscience (vmPFC–amygdala coupling),
	\item behavioural economics (dictator/ultimatum game generosity),
	\item Human–Robot Interaction (perception of robot intentions and moral standing).
\end{itemize}

\subsection{EQ Within the Evaluative-Topological Framework}

\noindent
Within the topological architecture of this thesis, EQ measures the magnitude of the \textbf{affective vector field} $\mathbf{A}(x)$ that pulls evaluative trajectories toward empathically grounded prosocial action. High EQ corresponds to:

\begin{itemize}
	\item steep affective gradients,
	\item strong attractors around suffering, need, vulnerability,
	\item high sensitivity to social evaluation cues (including Watching-Eye primes),
	\item rapid activation of intuitive moral appraisal.
\end{itemize}

\noindent
The attenuation effect observed in the experiment was strongest among participants with high EQ values, supporting the interpretation that the robot primarily dampens the \emph{affective dynamics} of moral cognition.

\[
\delta \mathbf{A}(x;\mathscr{R}) < 0 \quad \text{for high-EQ participants}.
\]

\noindent
Thus, EQ is not merely a psychometric variable but a quantification of emotional curvature within the evaluative field.

\subsection{EQ in HRI and Moral Cognition Research}

\noindent
Studies have shown that high empathizers:

\begin{itemize}
	\item anthropomorphise robots more readily \cite{Hofree2018},
	\item show stronger prosocial responses to perceived observers \cite{Mori2020},
	\item exhibit heightened moral salience in the presence of social cues \cite{Klimecki2013}.
\end{itemize}

\noindent
This aligns precisely with Cluster~2 in our experiment: high-empathy participants with strong affective attractors who showed \emph{the largest attenuation} under robot presence.

\subsection{Why EQ Matters}

\noindent
For the purposes of this thesis, EQ provides:

\begin{itemize}
	\item a measurable dimension of sentimentalist moral theory (Humean affect),
	\item an empirical anchor for affective topology (vector field curvature),
	\item a predictor for cluster assignment,
	\item and a mechanism through which $\gamma_R$ exerts maximal perturbation.
\end{itemize}

\noindent
EQ therefore captures the essential insight that moral cognition is—both philosophically and empirically—an affectively structured evaluative process.

\section{The Systemizing Quotient (SQ): Structural Evaluation and the Precision of Moral Gradients}

\noindent
Where the Empathizing Quotient (EQ) measures affective resonance, the Systemizing Quotient (SQ) \cite{BaronCohen2003,Goldenfeld2005,Wakabayashi2006} measures an individual’s propensity for pattern detection, structural analysis, causal modelling, and rule-based inference. In the context of this thesis, SQ quantifies the \emph{analytical curvature} of the evaluative field: the tendency to encode moral situations in terms of structural regularities rather than affective salience.

Systemizing has been linked to:
\begin{itemize}
	\item enhanced reliance on deliberative pathways in dual-process models \cite{Greene2014},
	\item reduced sensitivity to affect-laden moral cues \cite{Gleichgerrcht2013},
	\item preference for stable, rule-governed evaluative structures \cite{Lawson2004},
	\item and attenuated responses to social primes in prosocial tasks \cite{Zaki2014}.
\end{itemize}

\subsection{SQ Within the Evaluative-Topological Framework}

\noindent
In the topological architecture, SQ shapes the \emph{smoothness} and \emph{rigidity} of evaluative gradients. High-SQ agents tend to interpret situations through:

\begin{itemize}
	\item stable causal schemas,
	\item low-noise value comparisons,
	\item reduced sensitivity to affective perturbations,
	\item preference for top–down interpretive stability rather than bottom–up salience.
\end{itemize}

\noindent
Formally, SQ modulates the second derivative of the evaluative potential function:

\[
\nabla^2 V(x) \propto \text{SQ},
\]

where larger values correspond to more rigid and predictable gradient structure.

\subsection{SQ and Synthetic Presence}

\noindent
Analytically oriented individuals (high SQ) in the experiment correspond most closely to the \emph{Analytical–Structured} dispositional cluster. Consistent with the literature on deliberative dominance \cite{Konovalov2016,Shenhav2017}, these participants exhibited:

\begin{itemize}
	\item the \emph{least} attenuation in donation under robotic presence,
	\item reduced coupling between moral salience and affective resonance,
	\item a form of \emph{topological stability} under perturbation.
\end{itemize}

\noindent
Thus, SQ plays a dual role:

\begin{itemize}
	\item it identifies participants whose evaluative fields are \emph{robust} under affective perturbation (\(\delta \mathbf{A}(x;\mathscr{R})\)),
	\item and it reveals that synthetic presence interacts asymmetrically with trait space.
\end{itemize}

\subsection{Why SQ Matters}

\noindent
The inclusion of SQ provides:
\begin{itemize}
	\item a measure of deliberative curvature within the moral field,
	\item a predictor of resilience to affective suppression,
	\item and an anchor for comparing sentimentalist vs.\ structuralist mechanisms.
\end{itemize}

\noindent
Crucially, SQ helps explain why the robot does \emph{not} perturb all subjects uniformly:  
some evaluative topologies are analytically stabilised and therefore less susceptible to synthetic deformation.

% -----------------------------------------------------
\section{The Big Five Inventory (BFI): Personality Geometry and Moral Topology}

\noindent
The Big Five personality model (BFI) \cite{John1999,McCraeCosta2008,Donnellan2006} provides a multidimensional representation of dispositional traits across five axes:

\begin{itemize}
	\item \textbf{Openness},
	\item \textbf{Conscientiousness},
	\item \textbf{Extraversion},
	\item \textbf{Agreeableness},
	\item \textbf{Neuroticism}.
\end{itemize}

\noindent
The BFI is among the most validated instruments in psychology, repeatedly shown to predict:

\begin{itemize}
	\item prosocial behaviour and cooperation \cite{Graziano1996},
	\item empathy and moral concern \cite{Habashi2016},
	\item harm aversion and fairness \cite{Hilbig2013},
	\item HRI trust and perceived robot sociality \cite{Banks2020}.
\end{itemize}

\subsection{Personality as a Topological Substrate}

\noindent
In the evaluative-topological model, BFI traits shape the \emph{metric structure} of the evaluative field:

\begin{itemize}
	\item \textbf{Agreeableness}: lowers friction in prosocial trajectories (steeper attractors toward cooperation).
	\item \textbf{Conscientiousness}: stabilises long-horizon evaluative pathways (stronger regulatory curvature).
	\item \textbf{Neuroticism}: increases noise and local instability (greater susceptibility to perturbation).
	\item \textbf{Openness}: broadens sensitivity to contextual cues (expanded salience manifold).
	\item \textbf{Extraversion}: intensifies sensitivity to social presence (increased social-salience weighting).
\end{itemize}

\noindent
These dimensions do not function independently; instead, they jointly determine the curvature, stability, and topology of the moral field for each participant.

\subsection{Cluster Semantics and BFI Geometry}

\noindent
The cluster analysis in Chapter~\ref{chap:experimental_results} revealed three dispositional attractor structures:

\begin{enumerate}
	\item \textbf{Prosocial–Empathic}: high Agreeableness, high Openness, high EQ; steep affective attractors; largest attenuation.
	\item \textbf{Emotionally Reactive}: high Neuroticism, mixed EQ; unstable gradients; unpredictable attenuation.
	\item \textbf{Analytical–Structured}: high Conscientiousness and SQ; rigid gradients; minimal attenuation.
\end{enumerate}

\noindent
BFI traits were foundational for revealing this structure. The BFI did not merely “measure personality”—it uncovered the \emph{geometry of moral susceptibility} under synthetic perturbation.

\subsection{Why BFI Matters}

\noindent
Including the BFI allows:

\begin{itemize}
	\item quantification of dispositional topology,
	\item clustering of heterogeneous evaluative architectures,
	\item grounding normative interpretation in empirically real personality space,
	\item and identifying that synthetic moral perturbation is \emph{trait-structured rather than universal}.
\end{itemize}

\noindent
Without the BFI, the experiment would lack the dimensional granularity required for topological and ethical interpretation.

% -----------------------------------------------------
\section{The Watching-Eye Paradigm: Moral Salience Amplification Through Social Attention}

\noindent
The Watching-Eye effect is one of the most robust phenomena in behavioural science:  
minimal cues of observation—stylised eyes, drawings, or schematic-gaze primes—reliably increase prosocial behaviour, charitable donation, norm compliance, and generosity \cite{HaleyFessler2005,ErnestJones2011,Nettle2013}.

\subsection{The Watching-Eye Paradigm as an Experimental Scaffold}

\noindent
In the present study, the Watching-Eye prime serves three essential functions:

\begin{enumerate}
	\item \textbf{Amplifier of moral salience}: it increases the weight of moral cues in the evaluative field.
	\item \textbf{Probe of susceptibility}: it reveals how strongly each participant’s topology responds to social-evaluative cues.
	\item \textbf{Baseline moral-gradient enhancer}: it provides a moral gradient strong enough to detect attenuation by the robot.
\end{enumerate}

\noindent
The paradigm functions at the level of \emph{perceptual moral salience}, not reflective ethical judgment. It modulates:

\[
\alpha_E \mapsto \alpha_E + \delta \alpha_{\text{eye}},
\]

which increases the steepness of the prosocial gradient—unless perturbed.

\subsection{Watching-Eye Under Synthetic Co-Presence}

\noindent
The experiment revealed a novel effect:

\begin{quote}
	\emph{robotic presence cancels or reverses the moral-salience amplification produced by Watching-Eye cues}.
\end{quote}

\noindent
This establishes the Watching-Eye paradigm as a diagnostic tool for:

\begin{itemize}
	\item measuring topological deformation under synthetic presence,
	\item distinguishing cluster-specific susceptibility,
	\item identifying how $\gamma_R$ interacts with both affective and deliberative pathways.
\end{itemize}

\medskip

\noindent
Formally, the robot introduces a perturbation term:
\[
\gamma_R: \ \alpha_E + \delta \alpha_{\text{eye}} \ \mapsto \ \alpha_E + \delta \alpha_{\text{eye}} - \Delta_{\mathscr{R}},
\]
where \( \Delta_{\mathscr{R}} \) is the amount of salience suppression.

\subsection{Why the Watching-Eye Paradigm Matters}

\noindent
The paradigm is essential because it provides:

\begin{itemize}
	\item a standardized probe of moral salience,
	\item a replicable baseline to study attenuation,
	\item a methodological bridge between moral psychology and HRI,
	\item and the conceptual grounding for interpreting synthetic presence as a \emph{normative disruptor}.
\end{itemize}

\noindent
It transforms the experiment from a simple donation task into a structured test of \emph{moral topology under perturbation}.

