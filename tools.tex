\chapter{TOOLS}
\label{chap:tools}
\thispagestyle{pprintTitle}

\section{The Watching-Eye Effect}

One of the most robust findings in behavioural ethics and social psychology is that subtle cues of observation can increase prosocial behaviour. This phenomenon—commonly referred to as the \emph{watching-eye effect}—demonstrates that even minimal stimuli implying social presence can modulate cooperative or altruistic actions \cite{HaleyFessler2005, BatesonNettleRoberts2006, PfattheicherKeller2015}. Although originally interpreted in terms of reputational concerns, contemporary evidence indicates a multi-component mechanism involving attentional, affective, and interpretive pathways.

\paragraph{Reputational Mechanisms.}
Early accounts emphasised reputational vigilance: cues of observation were posited to activate concerns about social evaluation, thereby increasing norm adherence and generosity \cite{HaleyFessler2005, BatesonNettleRoberts2006}. Even stylised eye images were found to increase cooperation in real-world settings, suggesting that human social cognition is highly sensitive to potential monitoring \cite{BatesonEtAl2013_EyesLittering}. At the Level of Abstraction adopted in this thesis, reputational vigilance can be understood as a deformation of the evaluative landscape: cues implying oversight increase the weight of fairness, compliance, or prosocial norms in action-guiding computations.

\paragraph{Attentional and Perceptual Mechanisms.}
More recent work demonstrates that watching-eye cues also affect the allocation of visual and social attention, shifting perceptual resources toward norm-relevant features \cite{KleckStrenta1980, Emery2000}. Eye cues act as attentional attractors, increasing the salience of one's own behaviour and its alignment with internalised standards or expectations. This attentional modulation modifies the early intuitive gradients that shape moral evaluation, consistent with the topological model presented earlier.

\paragraph{Affective and Self-Conscious Emotion Mechanisms.}
Other studies emphasise the role of self-conscious emotions—such as guilt, embarrassment, or pride—in mediating responses to perceived observation. Eye cues elicit mild increases in affective arousal \cite{PfattheicherKeller2015}, potentially amplifying somatic markers associated with prosocial appraisal. In this sense, watching-eye stimuli operate by perturbing both affective and interpretive components of the evaluative field, thereby increasing the likelihood of prosocial action.

\paragraph{Context Sensitivity and Boundary Conditions.}
Importantly, the watching-eye effect is not uniform across contexts. Its magnitude depends on factors such as prevailing norms \cite{Kawamura2017}, the ambiguity of observational cues, and the ecological validity of the environment. These boundary conditions foreshadow the central empirical question of this thesis: whether the presence of a synthetic agent counts as an observational cue strong enough to elicit similar modulations in moral behaviour.


\section{Why Child-Poster Stimuli Function as Valid Social Cues}

Child-poster images featuring watching eyes are widely used as a minimal and controlled observational cue in donation-based paradigms. Their effectiveness derives from three properties that make them well-suited for experiments requiring precision and reproducibility.

\paragraph{Perceptual Sociality Without Agentic Commitment.}
Child eyes provide a cue that is perceptually social—highly evocative of gaze and attention—yet ontologically unproblematic. Participants do not confuse the poster with an actual agent, but the cue nevertheless activates perceptual mechanisms associated with being observed \cite{HaleyFessler2005}. This makes child-eye stimuli a clean perturbation of attentional and affective gradients without introducing confounds related to mental-state attribution.

\paragraph{Affective Resonance and Care-Related Salience.}
Child-related imagery tends to increase empathic concern and activate care-related motivational systems. Studies of interpersonal gaze show that the perceived innocence or vulnerability of the observer enhances the social salience of eye cues \cite{MasonTatkinMacrae2005}. Within the topological framework of this thesis, child-eye stimuli strengthen the evaluative attractors associated with care, prosociality, and harm avoidance.

\paragraph{Methodological Control.}
Child-eye posters offer high experimental control. Their low-dimensional visual structure avoids the confounds that arise when using real human observers, anthropomorphic agents, or dynamic faces. They therefore serve as a reproducible baseline for assessing how additional or alternative social cues—such as those introduced by a humanoid robot—perturb prosocial behaviour \cite{HaleyFessler2005, BatesonNettleRoberts2006}.


\section{Why Robots May Dilute or Modulate the Watching-Eye Effect}

A central hypothesis of this thesis is that a humanoid robot—despite being perceptually social—may attenuate, distort, or otherwise alter the watching-eye effect. This dilution is not due to reduced salience, but to the \emph{ontological ambiguity} of synthetic agents.

\paragraph{Perceptual Sociality Without Clear Social Ontology.}
Robots are visually social in virtue of their humanoid morphology, but they do not occupy a stable position within the human social ontology. They are neither fully agentic nor fully inert. This indeterminacy can weaken the intuitive mappings between observational cues and reputational or normative expectations. From the perspective of evaluative topology, robots generate conflicting gradients: they signal social presence while simultaneously undermining the interpretive coherence of that presence.

\paragraph{Disrupted Affective and Attentional Gradients.}
The presence of a robot may dampen affective resonance relative to child-eye images. Because the robot lacks a clear moral status, affective systems governing care, empathy, or guilt may be only partially activated. A similar disruption occurs at the attentional level: while robots attract gaze, they may not reliably signal evaluative oversight \cite{ContyGeorgeHietanen2016}. This can flatten or distort the intuitive attractors that normally support prosocial action.

\paragraph{Predictive and Interpretive Uncertainty.}
Mental-state attribution is central to the watching-eye effect. Minimal cues imply that another agent could observe or morally evaluate one's behaviour. With a robot, mental-state attribution becomes unstable: participants may attribute perceptual capacities without attributing evaluative ones. This uncertainty creates a diffuse or bifurcated evaluative field, reducing the force of reputational or care-related attractors and thereby attenuating prosocial tendencies.

\paragraph{Consequences for Evaluative Topology.}
Within the framework of this thesis, robots function as \textit{semiotic perturbators} of the moral field. Their presence shifts the shape of evaluative gradients—sometimes sharpening local attractors, sometimes flattening them, sometimes diverting trajectories altogether. The empirical prediction is thus not a simple decrease in prosociality, but a measurable deformation of the mapping from moral salience to action.


\section{Prosocial Donation Paradigm}

To test these theoretical predictions, this thesis employs a structured donation paradigm widely used in behavioural ethics, moral psychology, and social neuroscience. Donation tasks provide a reproducible, quantifiable measure of prosocial behaviour that reflects practical moral commitment rather than hypothetical endorsement \cite{Moll2002, Decety2004}.

\paragraph{Operational Structure.}
Participants are offered the opportunity to donate part of their experimental compensation to a real charity. Their donation amount serves as a behavioural index of prosocial motivation. Because donations involve a concrete cost, they reveal the strength of evaluative gradients sufficiently strong to influence action.

\paragraph{Integration With Observational Cues.}
The donation task is performed under one of several observational conditions: (i) child-eye stimulus, (ii) humanoid robot presence, or (iii) control condition. By holding all other variables constant, any variation in donation behaviour reflects differences in how observational cues modulate the evaluative topology connecting moral salience to practical action.

\paragraph{Why Donation Is the Appropriate Measure.}
At the chosen Level of Abstraction, moral cognition is defined not by its propositional structure but by its action-guiding function. Donation behaviour captures this directly: it provides a measurable, ecologically relevant manifestation of how evaluative processes culminate in a behavioural output. The paradigm thus serves as a test bed for detecting the subtle, yet theoretically significant, perturbations induced by synthetic social presence.

\paragraph{Expected Perturbation Pattern.}
Based on the architecture articulated in previous sections, the presence of a humanoid robot is predicted to modulate donation behaviour by altering attentional, affective, and interpretive pathways. This modulation is expected to manifest not as random noise but as a coherent deformation of the evaluative topology, consistent with the concept of a \emph{moral refractor}. The empirical chapter demonstrates precisely such patterned perturbation.
