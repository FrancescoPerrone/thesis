\chapter{Acknowledgements}

There is a peculiar stillness that settles around work completed under the accelerating discipline of contemporary academia—a sense that one has been guided less by patient inquiry than by the unyielding cadence of an institution convinced that thought must keep pace with its deadlines. If these pages read as though composed at a distance, it is only because they carry the faint tension between what might have matured in its own time and what the present era insists must be shaped, finished, and surrendered.

In that unsettled interval I have leaned on those whose presence does not depend on the coherence of my arguments. This thesis is dedicated to my son, Francesco, whose unguarded curiosity offers a quiet antidote to the rushed certainty demanded here; to my mother, Mirella, and my father, Alberto, whose enduring steadiness has outlasted every fluctuation of purpose; and to my wife, Anna, who has carried more than anyone her age should be asked to bear—not only for reasons that cannot be stated within these pages, but because she has been required, time and again, to return to the limits of my own intellect as though they were a place of refuge. Whatever this work may lack in the calm of true gestation, it rests on the grace with which they have all borne its cost.