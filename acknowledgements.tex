\chapter{Acknowledgments}

There is a peculiar stillness that settles around work completed under the accelerating discipline of contemporary academia—a sense that one has been guided less by patient inquiry than by the unhurried rhythm in which ideas ordinarily choose to unfold, and more by the unyielding cadence of an institution convinced that thought must keep pace with its deadlines. If these pages appear composed at a certain distance from themselves, it is because they bear the traces of that tension: the quiet struggle between what might have matured freely and what the present era insists must be shaped, concluded, and surrendered.

In that unsettled interval I have leaned on those whose presence never depended on the coherence of my arguments. This thesis is dedicated to my son, Francesco, who—like an improbable guardian with a light too bright for the corridors I found myself in—used the simple force of his existence to hold back a darkness I could not have held alone. To my mother, Mirella, whose illness and narrowing sight should have been met with my unhurried company rather than the prolonged absences this work imposed; to my father, Alberto, whose quiet vigilance, wisdom, and unwavering care have sustained us in ways that rarely leave a trace on any page; and to Anna, who has carried more than anyone her age should be asked to bear—not only for reasons that cannot be written here, but because she has returned, again and again, to the limits of my own intellect as though they were a place of rest rather than constraint. And to Dr. Herrera Martín, whose presence is not that of a colleague but of someone woven into the inner architecture of my life: through the most unsettled stretches of this work, he guarded the last spark of curiosity when it flickered low, and, with a steadiness all his own, helped the younger version of myself—long buried under years of urgency—to find a way back.

I also acknowledge, with the formality that such matters require, the support of the UK Engineering and Physical Sciences Research Council (EPSRC) through the grant Socially Competent Robots (EP/N035305/1), whose funding partially sustained the research that led to these pages. I thank Alessandro Vinciarelli for having—despite the many difficulties, interruptions, and divergences that marked these years—supported my formal progress through timely extensions and efforts to understand the intentions underlying my work. Whatever distances may have grown around the project, those gestures remain. I further thank Marek Sergot, who first opened this terrain to me during my time at Imperial College and whose early guidance set in motion curiosities that continue, in altered forms, to resonate here.

If this work lacks the calm of a true gestation, it rests nonetheless on the grace with which all of them, in different ways, have borne its cost.