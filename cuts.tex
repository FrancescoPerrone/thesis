\chapter{Cuts}

\textbf{This is all from moral d}

% Setting the font and spacing for the epigraph
\epigraph{\itshape \setstretch{1.2}But one thing is the thought, another thing is the deed, and another thing is the idea of the deed. The wheel of causality doth not roll between them.}{\small{Friedrich Nietzsche, \textit{Thus Spoke Zarathustra} (1883)}}

In here I have moved all content that I have decided not being relevant for the audience of this thesis.

this is alive.

Analysing the concept of \textit{Moral Decision Making} in the context of predicate logic involves interpreting various linguistic elements within a logical framework. 

\begin{itemize}
	\item \textbf{The Word "Decision"}: In predicate logic, "Decision" can be a constant or a variable. 
	\begin{itemize}
		\item As a constant (for a specific decision), it might be represented as \( d \).
		\item As a variable (representing any decision), it could be denoted as \( x \), where \( x \) is a decision.
	\end{itemize}
	
	\item \textbf{The Noun Phrase "Decision Making"}: "Decision Making" can be interpreted as a function in predicate logic. 
	\begin{itemize}
		\item The function \( \text{DecisionMaking}(x) \) represents the output or consequence of making decision \( x \).
	\end{itemize}
	
	\item \textbf{The Adjective "Moral" in "Moral Decision Making"}: "Moral" is a modifier and can be viewed as a predicate.
	\begin{itemize}
		\item The predicate \( \text{Moral}(\text{DecisionMaking}(x)) \) indicates that the decision-making process of \( x \) is of a moral nature.
	\end{itemize}
\end{itemize}

A typical formula connecting these elements might be:

\[ \forall x (\text{Decision}(x) \rightarrow \text{Moral}(\text{DecisionMaking}(x))) \]

This formula can be interpreted as: "For all \( x \), if \( x \) is a decision, then the decision-making process of \( x \) is moral." It employs a universal quantifier (\( \forall \)) to express a general statement about all decisions.

In moral philosophy, these logical structures assist in defining and debating ethical theories and concepts, enabling a rigorous analysis of the nuances of moral decision-making.


The concept of \textit{Moral Decision Making} can be more accurately represented in predicate logic by considering that not all decisions are inherently moral, but rather, they become moral under certain conditions.

Consider the revised approach:

\begin{itemize}
	\item \textbf{Existential Quantification and Conditionality}: The formula should reflect that only some decisions fall under the category of moral decisions, contingent upon specific conditions being met.
\end{itemize}

A more realistic formula would be:

\[ \exists x (C(x) \rightarrow (\text{Decision}(x) \land \text{Moral}(\text{DecisionMaking}(x)))) \]

Here, \( C(x) \) represents the specific conditions under which a decision \( x \) can be considered moral. The formula is interpreted as: "There exists some decision \( x \) such that if the conditions \( C(x) \) are met, then \( x \) is a decision and the decision-making process concerning \( x \) is moral."

This formula acknowledges that morality in decision-making is not a universal attribute of all decisions, but rather a characteristic of certain decisions under specific circumstances. Identifying and analyzing these conditions \( C(x) \) is a key aspect of ethical philosophy and moral reasoning.


I want to precisely narrow down the meaning of the word \textit{Decision}, in the...

In the realm of psychology, behavior is often defined as "the internally coordinated responses of whole living organisms (individuals or groups) to internal or external stimuli, excluding responses more easily understood as developmental changes."~\cite{Levitis2009} 


\textit{From Etymology}


Understanding the etymology of the word morality is even more crucial in our context, where (a) readers are accustomed to a usage of the word morality (and its derived adjective \textit{moral}) that often overflows into adjacent meanings, such as those pertaining to ethical discourse and ethics; and (b) because the principal objective of this project was to investigate machine-detectable cues associated with morally relevant behavior. By examining how moral language has evolved, we can better delineate the conceptual boundaries of morality as a term distinct from ethical deliberation, which is particularly important in the study of Human-Robot Interaction (HRI), where artificial agents affect human moral behavior without being moral agents themselves.

because it allows us to separate its foundational meaning from everyday discourse, which is often shaped by cultural, social, and ideological influences that can obscure or distort its essence. This is important for two main reasons: epistemic precision and historical-philosophical clarity.

\subsection{Epistemic Precision}

Etymology serves as an epistemic tool that helps philosophers clarify concepts by tracing their origins and meanings. The term "morality" originates from the Latin moralitas, which itself derives from mos, moris, meaning "custom" or "habit." This etymology aligns with Aristotle’s concept of ethos (ἦθος), from which the Greek-derived term ethics originates. The distinction between ethics (the philosophical study of what is good) and morality (which historically related to customary social behaviors) provides an essential foundation for philosophical discussions.

Etymology reveals that "morality" was originally tied to customs rather than absolute principles, challenging contemporary interpretations that treat morality as an innate or self-evident framework.
This distinction prevents conceptual drift, ensuring that moral discussions in philosophy remain grounded in rigorous analysis rather than shifting cultural sentiments.

Thus, etymology allows us to investigate whether morality is fundamentally a construct of social norms (as Hume and Nietzsche argue) or if it exists as an objective framework (as Kant and natural law theorists propose).

\subsection{Historical-Philosophical Clarity}

By understanding the historical evolution of "morality," we can detach it from its everyday use, which often introduces ambiguities, biases, and rhetorical manipulations. In contemporary discourse, morality is frequently invoked in political, religious, or subjective ways, leading to:

Moral relativism – where morality is seen as merely a social construct with no objective basis.
Moral absolutism – where morality is dogmatically assumed to be universal without philosophical justification.
Moral emotivism – where moral claims are reduced to expressions of individual sentiment rather than rational inquiry.

For instance:

Nietzsche’s critique of morality in On the Genealogy of Morals is deeply rooted in an etymological investigation, demonstrating how morality transitioned from aristocratic virtue (Greek arete) to Judeo-Christian moral law (based on duty and guilt).
Kant's moral philosophy relies on a distinction between moral law (universal, based on rational duty) and mores (culturally dependent behaviors). Without recognizing this etymological distinction, one might misinterpret Kantian ethics as a cultural rather than a rational enterprise.

By tracing the etymology of morality, we:

Identify shifts in moral discourse from custom-based ethics (mos, moris) to universal moral principles.
Differentiate between philosophical morality and colloquial moral rhetoric, ensuring clarity in ethical debates.
Recognize how language influences moral epistemology, shaping how we define moral duties, rights, and responsibilities.

\section{From Experiment}

During the past decade, new emerging technologies have caused profound changes in the way we communicate and interact~\cite{Pantic2014a}. Some of these changes have affected certain aspects of human behaviour and caused psychiatric disorders~\cite{Xerxa2023}. These technologies have fundamentally altered how we connect with others, potentially exacerbating feelings of loneliness despite increased opportunities for connection. The role that modern technologies—such as mobile communications, digital interaction platforms, and interactive humanoid robots might play in shaping these dynamics is critical, influencing not only interpersonal communications but also moral decision-making in complex social settings~\cite{Allcott2020, Auxier2021, Bail2021, Dwyer2020, Vosoughi2018}. Furthermore, technologies that increase interactive opportunities may not necessarily enhance the quality or \textit{ethical dimensions} of those interactions, which are crucial in scenarios involving moral choices~\cite{Sharkey2010, Vallor2016, Lin2012, Bryson2010}. The constant presence of interactive technologies can lead to a reshaping of social norms and behaviours, which might lead to more engaged or more detached human responses depending on the context and implementation~\cite{Misra, Turkle}.

Foundational insights from studies such as \cite{Xerxa2023} set the stage for a deeper exploration into how contemporary communication technologies, particularly humanoid robots, might amplify or mitigate these effects by altering the quality and nature of social interactions in both visible and subtle ways.
%

This work presents experiments based on the Watching Eye effect, the tendency of people to behave more honestly or more pro-socially when they have the impression of being observed. In particular, the experiments of this work show that the presence of a robot is associated to a lower tendency to donate to a charity despite the presence of a Watching Eye stimulus (the picture of a child portrayed on the brochure of a Non-Governmental Organization providing medical care in poor countries). The tendency to donate was measured in terms of actually donated money and the results show that people donate roughly one and half times as much  when there are no robots (a statistically significant difference). This suggests that, while not necessarily being involved in moral decisions, robots can still be associated to changes in the way people (possibly users) make decisions involving a moral dimension.

%
The \emph{Watching Eye} effect is the tendency of people to behave more honestly or more pro-socially when they feel observed~\cite{Oda2015}, whether such a feeling results from the presence of pictures depicting eyes~\cite{Atran2004}, from the belief in a supernatural being that can see everything~\cite{Bering2005,Shariff2007}, or from any other factors. The goal of this article is to investigate the interplay between the Watching Eye effect and the presence of humanoid robots, a technology expected to play an increasingly more important role in everyday life. In particular, the experiments of this work show that there is an association between the presence of a robot and the observable consequences of the Watching Eye effect.

\section{The Influence of Observational Presence on Human Behavior: Experimental Insights from Human-Robot Interactions}