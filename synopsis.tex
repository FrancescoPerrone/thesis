\chapter{Synopsis}

This thesis investigates whether the mere presence of a humanoid robot can alter the cognitive--affective processes through which morally salient cues are transformed into moral action. Drawing on intuitionist models of moral cognition, it reframes moral judgement as a perceptual and affective process structured by salience, attention, and dispositional architecture rather than explicit reasoning. The theoretical framework is formalised through the evaluative mapping
\[
f(\alpha_E, \beta_C, \gamma_R),
\]
which models how environmental cues, latent psychological structure, and synthetic presence jointly influence behavioural outcomes.

Three hypotheses guide the empirical work: \emph{evaluative deformation}, \emph{synthetic normativity}, and \emph{synthetic perturbation of moral inference}. A controlled experiment examines prosocial donation under a Watching--Eye stimulus, contrasting a human-alone condition with silent robotic co-presence. Across inferential contrasts, cluster-specific analyses, and Bayesian estimation, the results reveal a modest but structured attenuation of prosocial behaviour, concentrated within a Prosocial--Empathic dispositional ecology. The robot does not introduce new norms; instead, its perceptual and ontological ambiguity modulates the salience and affective weight of existing moral cues.

The Discussion situates these findings within Human--Robot Interaction, affective computing, and Floridi’s Levels of Abstraction. NAO is interpreted not as a moral agent but as a \emph{morally relevant informational object} whose presence reorganises the evaluative field within which moral cognition unfolds. The thesis therefore advances an ecological reconceptualisation of Machine Ethics: artificial systems exert morally significant influence not through reasoning or agency, but by reshaping the perceptual and affective scaffolds through which humans register and enact moral meaning. This field-theoretic perspective unifies the empirical, computational, and philosophical contributions, showing that synthetic presence already plays a substantive role in the moral ecology of technologically saturated environments.
